%\documentclass[]{standalone}
%\usepackage{amsmath}
%\usepackage{amsfonts}
%\usepackage{amssymb}
%\usepackage{graphicx}
%\usepackage[table]{xcolor}
%\usepackage{auto-pst-pdf}%for converting pstricks diagrams for compilation in pdflatex
%\usepackage{pst-optexp} %for optics diagrams using pstricks
%\usepackage{wasysym} %for the sun symbol used for light source
%%\usepackage{subfig}
%%\usepackage[para,symbol*]{footmisc}
%\usepackage{tikz}
%%\usepackage{color}
%\usepackage{import}
%\usepackage{pst-solides3d}
%\usepackage{rotating}
%\usepackage{todonotes}
%\renewcommand{\arraystretch}{1.5}
%
%\begin{document}
\begin{pspicture}[](8,6) %start optics diagram (8x6) grid

\pnode(3,6){S1} %labeled nodes are the positions in the (8x6) grid
\pnode(3,5){L1}
\pnode(3,4){Pol}
\pnode(3,3){L2}
\pnode(3,2){P1}
\pnode(3,1){BS1}

\pnode(1,1){M1}
\pnode(1.521,1){Sample}
\pnode(5,1){L3}
\pnode(6,1){F1}
\pnode(7,1){L4}
\pnode(8,1){CCD}

%components in form \name[options](node1)(node1){label on diagram}

\lens[compname=L1,labeloffset=-1](S1)(Pol){L1}
\optplate[compname=Pol,labeloffset=-1](L1)(L2){Pol}
\lens[compname=L2,position=0.5,labeloffset=-1](Pol)(P1){L2}
\pinhole[compname=P1,phwidth=0.1,labeloffset=-1](L2)(BS1){P1}
\mirror[compname=BS1,bsstyle=plate,labelangle=-45](P1)(BS1)(M1){BS}
\mirror[compname=M1,mirrorradius=1,mirrortype=extended, mirrorwidth=1.7](BS1)(M1)(BS1){M1}
\pinhole[position=0.345,phwidth=0.1](BS1)(F1){P2}
\optplate[compname=Pol,labeloffset=-1,position=1.5](L3)(F1){Pol}
\mirror[compname=Sample,mirrortype=extended, mirrorwidth=0.25,labelangle=-80,labeloffset=1.1](M1)(Sample)(M1){G}
\lens[compname=L3,position=0.75,n=1.4](BS1)(F1){L3}
\optplate[compname=F1,position=0.7,optional](L3)(L4){F}
\lens[compname=L4,n=1.55](F1)(CCD){L4}
\optbox[compname=CCD,position=end,labeloffset=0,optboxwidth=1](L4)(CCD){CCD}

\psline{->}(5.75,1)(5.75,1.25)
\rput(5.75,1.45){$r$}
\psline[linestyle=dotted]{}(1,1)(1.521,1)
\rput(1.25,1.175){$\theta$}

\rput(S1){\Large{\sun}} %the sun symbol (not in opt-exp but nice anyways)


%add the beam, in the form \drawwidebeam[options](source){go through this node}{then this one}{then this one}

\addtopsstyle{Beam}{fillstyle=solid, fillcolor=green,opacity=0.05,linestyle}
\drawwidebeam[beamdiv=25, ArrowInside=->,arrowscale=1.3](S1){L1}{L2}{P1}{BS1}{M1}{Sample}{M1}{L3}{L4}{CCD}


\end{pspicture}
%\end{document}