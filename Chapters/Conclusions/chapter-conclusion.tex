\section{Summary of Thesis}
This thesis details original experimental investigations on the excitation of SPPs along metallic diffraction grating surfaces. Broadly, these investigations can be divided in to two main categories: the investigation of SPPs on traditional `crossed' bigratings with novel symmetries, and the introduction of a new type of diffraction grating supporting SPPs, the `zigzag' grating.

The work on the traditional `crossed gratings' is presented in chapters \ref{c:rectangular} \& \ref{c:oblique}, exploring gratings with symmetries that have previously received little attention in the literature. Chapter \ref{c:rectangular} experimentally explores the dispersion and coupling of SPPs on a rectangular bigrating. The groove profiles of such bigratings are described as a Fourier expansion of the constituent gratings, and this is used to explain the experimentally observed scattering and interaction of the supported SPP modes. Surface plasmon band-gaps are seen to occur at the boundaries of the rectangular BZ and are recorded using the novel technique of imaging scatterometry. This measurement technique for acquiring the iso-frequency contours of SPPs in $k$-space is original to this thesis, and detailed in chapter \ref{c:experimentalmethods}. The measured SPP contours at the BZ boundary are reproduced using FEM modelling, showing good agreement with experiment. 

Chapter \ref{c:rectangular} also shows that by increasing the depth of a constituent grating, the SPP travelling in the orthogonal direction can be slowed in group velocity. Experimental results also show that the SPP iso-frequency contour can be shaped by changing the available grating harmonics and hence the strength of scattering amplitudes. This is demonstrated experimentally as a mechanism by which to design anisotropic SPP propagation, which could be useful for the development of SPP surface optics.

Chapter \ref{c:oblique} investigates SPPs on gratings with the lowest symmetry of all the 2D Bravais lattices, oblique bigratings. SPP mediated polarisation conversion is observed on these gratings due to the broken mirror symmetry of the surface. Measurements of the dispersion also show evidence that SPPs may undergo self-coupling. This is the mechanism of a propagating SPP mode being able to resonantly drive another, different, SPP mode. This is a possibility providing the two modes do not travel orthogonally to one another. 

The iso-frequency contours of these SPPs are mapped experimentally using imaging scatterometry, and it is found that band-gaps do not necessarily occur at the conventional definition of the BZ boundary. A discussion of the symmetry considerations on such a grating is used to explain why this is the case, concluding that since the BZ boundary is itself not a contour of high-symmetry for an oblique lattice, there is no condition for the band-gaps to form at this arbitrary boundary. There are, however, isolated high-symmetry points along the BZ boundary, and when the plane of incidence intersects these unique points, it is experimentally observed that SPP band-gaps do still occur.

The final two experimental chapters introduce a new type of diffraction grating: the zigzag grating. These are gratings where a diffractive periodicity has been introduced by `zigzagging' a set of sub-wavelength (non-diffracting) grooves in a metal surface. The electric field of plane polarised light incident on such a zigzag grating will intersect the surface regardless of the polarisation angle, inducing surface charge. This results in SPP excitation on these gratings with either TE or TM polarised light.

Chapter \ref{c:zigzag} details such a zigzag grating that possesses a single mirror plane. The symmetry of this zigzag structure leads to the observation that TE polarised light excites SPPs scattered by odd-order grating vectors, while TM polarised light excites SPPs scattered by even-orders. This separation of SPP diffracted orders by polarisation selectivity is explained using a simple theoretical treatment, and is observed experimentally. 

A second consequence of this symmetric zigzag grating is that the standing surface wave states at the first BZ boundary are found to be degenerate in energy. By modelling the system using the FEM, it is found that the two possible SPP standing waves inhabit identical electromagnetic environments, and so no energy difference exists between them and no bad-gap may form. Experimental observations confirm this lack of SPP band-gap at the first BZ boundary. 

The final results of chapter \ref{c:zigzag} show that SPP band-gaps associated with the short sub-wavelength pitch cause the observed SPP iso-frequency contours to deform. So large is this perturbation that at high frequencies the SPP contours are flat, and the SPP's group velocity is constrained to be parallel to the long-pitch direction, irrespective of incident angle of the coupling light. This, combined with the forbidden band-gaps, makes these zigzag gratings excellent candidates for surface wave collimation devices.

The final experimental results of this thesis are shown chapter \ref{c:azigzag}  and pertain to the SPPs excited along a zigzag grating which possesses no mirror symmetry. This leads to the experimental observation that either TE or TM polarised light may excite the same SPP modes. Band-gaps may now form, as the asymmetric zigzag surface leads to different energetic arrangements of surface charge for the different standing wave states. The anisotropic propagation of SPPs on such a surface observed in chapter \ref{c:zigzag} is also observed for these asymmetric zigzag gratings, which, when combined with the large SPP band-gaps, leads to the formation of a full SPP band-gap, for which SPP propagation is forbidden in all directions.

\section{Future Work}
Throughout this thesis we have demonstrated some novel optical effects related to SPPs on metal gratings, and this provides a wealth of possible future work.

The manipulation of SPP band structures presented using rectangular bigratings in chapters \ref{c:rectangular}, \ref{c:zigzag} and \ref{c:azigzag} may provide useful tools in the design of SPP surface optics, including lenses, collimators and perhaps even negative index devices. Smoothly varying the depths along such gratings would change the effective mode index along the surface, and be used to guide or direct SPPs.

The use of sub-wavelength grooves shaped along their length to provide diffractive coupling to SPPs has a large potential for many interesting future studies. The use of the grating in chapter \ref{c:zigzag} as a collimating device for surface waves would be an excellent avenue for investigation. A schematic of a possible experimental arrangement is shown in figure \ref{fig:experimentforcollimation}. 
\begin{figure}
\begin{center}
\input{selfcollimatorsvg.pdf_tex}
\end{center}
\caption{A possible experimental arrangement for the observation of collimated surface waves using zigzag gratings.\label{fig:experimentforcollimation}}
\end{figure}
In this figure, a SPP point source (red dot) excites and couples to SPPs on a planar metal surface. The use of a near-field optical microscopy (SNOM) tip or a small point scatterer illuminated with a laser could provide near-field coupling into a SPP wave which, in the absence of the grating, will propagate outwards radially. With the addition of a grating similar to the one presented in chapter \ref{c:zigzag}, the allowed momentum states of the SPP are constrained to propagate in a single direction only, causing collimation of the surface wave. The distance between the grating and the point source would control the final (presumably Gaussian) SPP beam width. This collimation could then be observed by measuring the electric field across the surface using SNOM, or by using leaky radiation microscopy \cite{Stein2012} to image the beam.

The potential exists, by altering the zigzag period, the sub-wavelength period or the depth of the sub-wavelength grooves, to allow for the onset of flat SPP contours at a lower frequency than that found in chapter \ref{c:zigzag}, possibly providing a device which can collimate SPP waves over a broad range of frequencies.

Another potential device that could incorporate a zigzag diffraction grating is an SPP mediated light source. Metal-Insulator-Metal (MIM) devices with the top metal surface corrugated to form a diffraction grating have been shown to emit light when driven with a suitable voltage \cite{Theis1983}. This process is mediated by the excitation of SPPs by electrons followed by the out-coupling of these SPPs to light by the diffraction grating. Replacing the top surface diffraction grating with the symmetric zigzag grating described in chapter \ref{c:zigzag} would result in the emission of TE polarised light into the $\pm 1,\pm 3, \pm 5,...$ diffracted orders and also the emission of TM polarised light into the $\pm 2, \pm 4,...$ orders. The device would be a polarisation directing/separating light source. The use of an asymmetric zigzag as detailed in chapter \ref{c:azigzag} would result in the emission of unpolarised light into the diffracted orders.

The asymmetric zigzag in chapter \ref{c:azigzag} also provides some exciting possibilities for other future work. Investigation into the effect of different asymmetric patterns would most likely lead to effects analogous to changing the phase of a $2k_g$ component on a traditional surface-relief grating. If a device could be constructed where the unpolarised light could couple to the SPP standing wave states (and so a region of high density of SPP states) in the upper band edge (as opposed to the preferential coupling to the lower band edge observed in chapter \ref{c:azigzag}), one could envisage a device with great potential in SPP enhanced lasers or SPP enhanced solar cells. In such devices, it is desirable to couple the SPP into a gain medium coating the surface, which favours the upper-band edge of SPP band-gaps since these fields are more `photon-like' extending further into the surround gain medium that the `plasmon-like' lower-band edge. The advantage of using an optimised asymmetric zigzag is that the same SPP modes can be coupled to with any incident polarisation, possibly improving efficiency. The polarisation insensitivity to the generation of the locally TM polarised SPP waves could also prove useful for the efficient coupling of light into optical fibres.
