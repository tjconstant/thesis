\documentclass[oneside,11pt,book]{book}

\usepackage{amsmath}
\usepackage{amsfonts}
\usepackage{amssymb}
\usepackage{graphicx}
\usepackage{auto-pst-pdf}%for converting pstricks diagrams for compilation in pdflatex
\usepackage{pst-optexp} %for optics diagrams using pstricks
\usepackage{wasysym} %for the sun symbol used for light source
%\usepackage{subfig}
%\usepackage[para,symbol*]{footmisc}
\usepackage{tikz}
\usepackage{color}
\usepackage{subcaption}


\begin{document}

\chapter{Optical response of oblique bigratings}
\section{what is an oblique grating, and why do i care?}
An oblique bigrating is formed from two monogratings of differing pitches, crossed at an angle that is not $90^\circ$ nor $60^\circ$. The reciprocal space map for such a grating is shown in figure \ref{}, and constitutes the lowest symmetry lattice set of all the two-dimensional Bravais lattices.
\begin{figure}
\centering % Created by tikzDevice version 0.6.2-92-0ad2792 on 2012-10-24 14:24:45
% !TEX encoding = UTF-8 Unicode
\begin{tikzpicture}[x=1pt,y=1pt]
\definecolor[named]{fillColor}{rgb}{1.00,1.00,1.00}
\path[use as bounding box,fill=fillColor,fill opacity=0.00] (0,0) rectangle (252.94,252.94);
\begin{scope}
\path[clip] (  0.00,  0.00) rectangle (252.94,252.94);
\definecolor[named]{drawColor}{rgb}{0.00,0.00,0.00}

\path[draw=drawColor,line width= 0.4pt,line join=round,line cap=round] ( 24.00, 24.00) --
	(228.94, 24.00) --
	(228.94,228.94) --
	( 24.00,228.94) --
	( 24.00, 24.00);
\end{scope}
\begin{scope}
\path[clip] ( 24.00, 24.00) rectangle (228.94,228.94);
\definecolor[named]{drawColor}{rgb}{0.00,0.00,0.00}

\path[draw=drawColor,line width= 0.4pt,line join=round,line cap=round] (126.47,126.47) --
	(126.47,126.47);
\definecolor[named]{drawColor}{rgb}{1.00,0.00,0.00}
\definecolor[named]{fillColor}{rgb}{1.00,1.00,1.00}

\path[draw=drawColor,line width= 0.8pt,line join=round,line cap=round,fill=fillColor] (126.47,126.47) -- (184.21,126.47);

\path[draw=drawColor,line width= 0.8pt,line join=round,line cap=round] (168.57,117.44) --
	(184.21,126.47) --
	(168.57,135.51);

\path[draw=drawColor,line width= 0.8pt,line join=round,line cap=round,fill=fillColor] (126.47,126.47) -- (104.06,210.13);

\path[draw=drawColor,line width= 0.8pt,line join=round,line cap=round] (116.83,197.36) --
	(104.06,210.13) --
	( 99.38,192.68);
\definecolor[named]{drawColor}{rgb}{0.00,0.00,1.00}

\path[draw=drawColor,line width= 1.2pt,line join=round,line cap=round] ( 19.66,  0.00) --
	( 19.65,  0.01) --
	( 18.93,  0.38) --
	( 18.20,  0.74) --
	( 17.47,  1.08) --
	( 16.74,  1.41) --
	( 15.99,  1.73) --
	( 15.25,  2.04) --
	( 14.49,  2.33) --
	( 13.74,  2.62) --
	( 12.97,  2.88) --
	( 12.20,  3.14) --
	( 11.43,  3.38) --
	( 10.66,  3.60) --
	(  9.88,  3.82) --
	(  9.09,  4.02) --
	(  8.31,  4.20) --
	(  7.52,  4.37) --
	(  6.72,  4.53) --
	(  5.93,  4.68) --
	(  5.13,  4.81) --
	(  4.33,  4.92) --
	(  3.53,  5.03) --
	(  2.72,  5.11) --
	(  1.92,  5.19) --
	(  1.11,  5.25) --
	(  0.31,  5.29) --
	(  0.00,  5.31);
\definecolor[named]{drawColor}{rgb}{0.00,0.00,0.00}

\path[draw=drawColor,line width= 0.4pt,dash pattern=on 1pt off 3pt ,line join=round,line cap=round] ( 28.31,  0.00) --
	( 27.84,  0.34) --
	( 27.12,  0.86) --
	( 26.38,  1.36) --
	( 25.64,  1.85) --
	( 24.89,  2.32) --
	( 24.13,  2.78) --
	( 23.36,  3.23) --
	( 22.59,  3.67) --
	( 21.80,  4.09) --
	( 21.01,  4.50) --
	( 20.22,  4.89) --
	( 19.41,  5.27) --
	( 18.60,  5.64) --
	( 17.79,  5.99) --
	( 16.96,  6.33) --
	( 16.13,  6.65) --
	( 15.30,  6.96) --
	( 14.46,  7.26) --
	( 13.62,  7.53) --
	( 12.77,  7.80) --
	( 11.91,  8.05) --
	( 11.06,  8.28) --
	( 10.19,  8.50) --
	(  9.33,  8.71) --
	(  8.46,  8.90) --
	(  7.59,  9.07) --
	(  6.71,  9.23) --
	(  5.83,  9.37) --
	(  4.95,  9.50) --
	(  4.07,  9.61) --
	(  3.19,  9.71) --
	(  2.30,  9.79) --
	(  1.42,  9.86) --
	(  0.53,  9.91) --
	(  0.00,  9.93);
\definecolor[named]{drawColor}{rgb}{0.00,0.00,1.00}

\path[draw=drawColor,line width= 1.2pt,line join=round,line cap=round] ( 21.86, 42.81) --
	( 21.85, 43.62) --
	( 21.83, 44.43) --
	( 21.80, 45.24) --
	( 21.75, 46.04) --
	( 21.68, 46.85) --
	( 21.61, 47.65) --
	( 21.52, 48.46) --
	( 21.41, 49.26) --
	( 21.29, 50.06) --
	( 21.16, 50.86) --
	( 21.01, 51.65) --
	( 20.85, 52.44) --
	( 20.67, 53.23) --
	( 20.48, 54.02) --
	( 20.28, 54.80) --
	( 20.06, 55.58) --
	( 19.83, 56.36) --
	( 19.59, 57.13) --
	( 19.33, 57.89) --
	( 19.06, 58.65) --
	( 18.78, 59.41) --
	( 18.48, 60.16) --
	( 18.17, 60.91) --
	( 17.85, 61.65) --
	( 17.51, 62.39) --
	( 17.16, 63.12) --
	( 16.80, 63.84) --
	( 16.42, 64.56) --
	( 16.04, 65.27) --
	( 15.64, 65.97) --
	( 15.23, 66.66) --
	( 14.80, 67.35) --
	( 14.37, 68.03) --
	( 13.92, 68.71) --
	( 13.46, 69.37) --
	( 12.99, 70.03) --
	( 12.51, 70.68) --
	( 12.02, 71.32) --
	( 11.51, 71.95) --
	( 11.00, 72.57) --
	( 10.47, 73.19) --
	(  9.93, 73.79) --
	(  9.39, 74.39) --
	(  8.83, 74.97) --
	(  8.26, 75.55) --
	(  7.68, 76.11) --
	(  7.09, 76.67) --
	(  6.50, 77.21) --
	(  5.89, 77.75) --
	(  5.27, 78.27) --
	(  4.65, 78.78) --
	(  4.02, 79.29) --
	(  3.37, 79.78) --
	(  2.72, 80.26) --
	(  2.06, 80.72) --
	(  1.39, 81.18) --
	(  0.72, 81.62) --
	(  0.04, 82.06) --
	(  0.00, 82.08);

\path[draw=drawColor,line width= 1.2pt,line join=round,line cap=round] (  0.00,  3.55) --
	(  0.04,  3.57) --
	(  0.72,  4.00) --
	(  1.39,  4.44) --
	(  2.06,  4.90) --
	(  2.72,  5.37) --
	(  3.37,  5.85) --
	(  4.02,  6.34) --
	(  4.65,  6.84) --
	(  5.27,  7.35) --
	(  5.89,  7.88) --
	(  6.50,  8.41) --
	(  7.09,  8.96) --
	(  7.68,  9.51) --
	(  8.26, 10.08) --
	(  8.83, 10.65) --
	(  9.39, 11.24) --
	(  9.93, 11.83) --
	( 10.47, 12.44) --
	( 11.00, 13.05) --
	( 11.51, 13.67) --
	( 12.02, 14.30) --
	( 12.51, 14.95) --
	( 12.99, 15.59) --
	( 13.46, 16.25) --
	( 13.92, 16.92) --
	( 14.37, 17.59) --
	( 14.80, 18.27) --
	( 15.23, 18.96) --
	( 15.64, 19.66) --
	( 16.04, 20.36) --
	( 16.42, 21.07) --
	( 16.80, 21.79) --
	( 17.16, 22.51) --
	( 17.51, 23.24) --
	( 17.85, 23.97) --
	( 18.17, 24.72) --
	( 18.48, 25.46) --
	( 18.78, 26.21) --
	( 19.06, 26.97) --
	( 19.33, 27.73) --
	( 19.59, 28.50) --
	( 19.83, 29.27) --
	( 20.06, 30.04) --
	( 20.28, 30.82) --
	( 20.48, 31.61) --
	( 20.67, 32.39) --
	( 20.85, 33.18) --
	( 21.01, 33.97) --
	( 21.16, 34.77) --
	( 21.29, 35.57) --
	( 21.41, 36.37) --
	( 21.52, 37.17) --
	( 21.61, 37.97) --
	( 21.68, 38.78) --
	( 21.75, 39.58) --
	( 21.80, 40.39) --
	( 21.83, 41.20) --
	( 21.85, 42.00) --
	( 21.86, 42.81);
\definecolor[named]{drawColor}{rgb}{0.00,0.00,0.00}

\path[draw=drawColor,line width= 0.4pt,dash pattern=on 1pt off 3pt ,line join=round,line cap=round] ( 26.48, 42.81) --
	( 26.47, 43.70) --
	( 26.45, 44.59) --
	( 26.41, 45.48) --
	( 26.36, 46.37) --
	( 26.29, 47.25) --
	( 26.20, 48.14) --
	( 26.10, 49.02) --
	( 25.98, 49.90) --
	( 25.85, 50.78) --
	( 25.70, 51.66) --
	( 25.54, 52.54) --
	( 25.36, 53.41) --
	( 25.17, 54.27) --
	( 24.96, 55.14) --
	( 24.74, 56.00) --
	( 24.50, 56.86) --
	( 24.25, 57.71) --
	( 23.98, 58.56) --
	( 23.70, 59.40) --
	( 23.40, 60.24) --
	( 23.09, 61.07) --
	( 22.76, 61.90) --
	( 22.42, 62.72) --
	( 22.06, 63.53) --
	( 21.69, 64.34) --
	( 21.31, 65.15) --
	( 20.91, 65.94) --
	( 20.50, 66.73) --
	( 20.08, 67.51) --
	( 19.64, 68.28) --
	( 19.18, 69.05) --
	( 18.72, 69.81) --
	( 18.24, 70.56) --
	( 17.75, 71.30) --
	( 17.24, 72.03) --
	( 16.72, 72.75) --
	( 16.19, 73.47) --
	( 15.65, 74.17) --
	( 15.10, 74.87) --
	( 14.53, 75.55) --
	( 13.95, 76.23) --
	( 13.36, 76.89) --
	( 12.76, 77.55) --
	( 12.14, 78.19) --
	( 11.52, 78.82) --
	( 10.88, 79.44) --
	( 10.24, 80.05) --
	(  9.58, 80.65) --
	(  8.91, 81.24) --
	(  8.23, 81.82) --
	(  7.55, 82.38) --
	(  6.85, 82.93) --
	(  6.14, 83.47) --
	(  5.43, 84.00) --
	(  4.70, 84.52) --
	(  3.97, 85.02) --
	(  3.22, 85.51) --
	(  2.47, 85.98) --
	(  1.71, 86.44) --
	(  0.95, 86.89) --
	(  0.17, 87.33) --
	(  0.00, 87.42);

\path[draw=drawColor,line width= 0.4pt,dash pattern=on 1pt off 3pt ,line join=round,line cap=round] (  3.04,  0.00) --
	(  3.22,  0.12) --
	(  3.97,  0.61) --
	(  4.70,  1.11) --
	(  5.43,  1.62) --
	(  6.14,  2.15) --
	(  6.85,  2.69) --
	(  7.55,  3.24) --
	(  8.23,  3.81) --
	(  8.91,  4.38) --
	(  9.58,  4.97) --
	( 10.24,  5.57) --
	( 10.88,  6.18) --
	( 11.52,  6.80) --
	( 12.14,  7.44) --
	( 12.76,  8.08) --
	( 13.36,  8.73) --
	( 13.95,  9.40) --
	( 14.53, 10.07) --
	( 15.10, 10.76) --
	( 15.65, 11.45) --
	( 16.19, 12.16) --
	( 16.72, 12.87) --
	( 17.24, 13.60) --
	( 17.75, 14.33) --
	( 18.24, 15.07) --
	( 18.72, 15.82) --
	( 19.18, 16.58) --
	( 19.64, 17.34) --
	( 20.08, 18.11) --
	( 20.50, 18.90) --
	( 20.91, 19.68) --
	( 21.31, 20.48) --
	( 21.69, 21.28) --
	( 22.06, 22.09) --
	( 22.42, 22.91) --
	( 22.76, 23.73) --
	( 23.09, 24.55) --
	( 23.40, 25.39) --
	( 23.70, 26.22) --
	( 23.98, 27.07) --
	( 24.25, 27.92) --
	( 24.50, 28.77) --
	( 24.74, 29.62) --
	( 24.96, 30.49) --
	( 25.17, 31.35) --
	( 25.36, 32.22) --
	( 25.54, 33.09) --
	( 25.70, 33.96) --
	( 25.85, 34.84) --
	( 25.98, 35.72) --
	( 26.10, 36.60) --
	( 26.20, 37.49) --
	( 26.29, 38.37) --
	( 26.36, 39.26) --
	( 26.41, 40.15) --
	( 26.45, 41.03) --
	( 26.47, 41.92) --
	( 26.48, 42.81);

\path[draw=drawColor,line width= 0.4pt,dash pattern=on 1pt off 3pt ,line join=round,line cap=round] (  4.06,126.47) --
	(  4.06,127.36) --
	(  4.03,128.25) --
	(  3.99,129.14) --
	(  3.94,130.03) --
	(  3.87,130.91) --
	(  3.78,131.80) --
	(  3.68,132.68) --
	(  3.57,133.56) --
	(  3.44,134.44) --
	(  3.29,135.32) --
	(  3.13,136.20) --
	(  2.95,137.07) --
	(  2.75,137.93) --
	(  2.55,138.80) --
	(  2.32,139.66) --
	(  2.09,140.52) --
	(  1.83,141.37) --
	(  1.56,142.22) --
	(  1.28,143.06) --
	(  0.98,143.90) --
	(  0.67,144.73) --
	(  0.34,145.56) --
	(  0.00,146.38) --
	(  0.00,146.39);

\path[draw=drawColor,line width= 0.4pt,dash pattern=on 1pt off 3pt ,line join=round,line cap=round] (  0.00,106.56) --
	(  0.00,106.57) --
	(  0.34,107.39) --
	(  0.67,108.21) --
	(  0.98,109.05) --
	(  1.28,109.88) --
	(  1.56,110.73) --
	(  1.83,111.58) --
	(  2.09,112.43) --
	(  2.32,113.28) --
	(  2.55,114.15) --
	(  2.75,115.01) --
	(  2.95,115.88) --
	(  3.13,116.75) --
	(  3.29,117.62) --
	(  3.44,118.50) --
	(  3.57,119.38) --
	(  3.68,120.26) --
	(  3.78,121.15) --
	(  3.87,122.03) --
	(  3.94,122.92) --
	(  3.99,123.81) --
	(  4.03,124.69) --
	(  4.06,125.58) --
	(  4.06,126.47);
\definecolor[named]{drawColor}{rgb}{0.00,0.00,1.00}

\path[draw=drawColor,line width= 1.2pt,line join=round,line cap=round] ( 77.40,  0.00) --
	( 77.39,  0.01) --
	( 76.67,  0.38) --
	( 75.95,  0.74) --
	( 75.21,  1.08) --
	( 74.48,  1.41) --
	( 73.74,  1.73) --
	( 72.99,  2.04) --
	( 72.23,  2.33) --
	( 71.48,  2.62) --
	( 70.71,  2.88) --
	( 69.95,  3.14) --
	( 69.17,  3.38) --
	( 68.40,  3.60) --
	( 67.62,  3.82) --
	( 66.83,  4.02) --
	( 66.05,  4.20) --
	( 65.26,  4.37) --
	( 64.46,  4.53) --
	( 63.67,  4.68) --
	( 62.87,  4.81) --
	( 62.07,  4.92) --
	( 61.27,  5.03) --
	( 60.47,  5.11) --
	( 59.66,  5.19) --
	( 58.85,  5.25) --
	( 58.05,  5.29) --
	( 57.24,  5.33) --
	( 56.43,  5.34) --
	( 55.62,  5.35) --
	( 54.81,  5.34) --
	( 54.01,  5.31) --
	( 53.20,  5.27) --
	( 52.39,  5.22) --
	( 51.59,  5.15) --
	( 50.78,  5.07) --
	( 49.98,  4.98) --
	( 49.18,  4.87) --
	( 48.38,  4.74) --
	( 47.58,  4.61) --
	( 46.79,  4.45) --
	( 46.00,  4.29) --
	( 45.21,  4.11) --
	( 44.42,  3.92) --
	( 43.64,  3.71) --
	( 42.86,  3.49) --
	( 42.09,  3.26) --
	( 41.32,  3.01) --
	( 40.55,  2.75) --
	( 39.79,  2.48) --
	( 39.04,  2.19) --
	( 38.29,  1.89) --
	( 37.54,  1.58) --
	( 36.80,  1.25) --
	( 36.07,  0.91) --
	( 35.34,  0.56) --
	( 34.62,  0.19) --
	( 34.25,  0.00);
\definecolor[named]{drawColor}{rgb}{0.00,0.00,0.00}

\path[draw=drawColor,line width= 0.4pt,dash pattern=on 1pt off 3pt ,line join=round,line cap=round] ( 86.05,  0.00) --
	( 85.58,  0.34) --
	( 84.86,  0.86) --
	( 84.12,  1.36) --
	( 83.38,  1.85) --
	( 82.63,  2.32) --
	( 81.87,  2.78) --
	( 81.10,  3.23) --
	( 80.33,  3.67) --
	( 79.54,  4.09) --
	( 78.75,  4.50) --
	( 77.96,  4.89) --
	( 77.15,  5.27) --
	( 76.34,  5.64) --
	( 75.53,  5.99) --
	( 74.70,  6.33) --
	( 73.88,  6.65) --
	( 73.04,  6.96) --
	( 72.20,  7.26) --
	( 71.36,  7.53) --
	( 70.51,  7.80) --
	( 69.65,  8.05) --
	( 68.80,  8.28) --
	( 67.94,  8.50) --
	( 67.07,  8.71) --
	( 66.20,  8.90) --
	( 65.33,  9.07) --
	( 64.45,  9.23) --
	( 63.58,  9.37) --
	( 62.70,  9.50) --
	( 61.81,  9.61) --
	( 60.93,  9.71) --
	( 60.04,  9.79) --
	( 59.16,  9.86) --
	( 58.27,  9.91) --
	( 57.38,  9.94) --
	( 56.49,  9.96) --
	( 55.60,  9.97) --
	( 54.71,  9.95) --
	( 53.82,  9.93) --
	( 52.94,  9.88) --
	( 52.05,  9.83) --
	( 51.16,  9.75) --
	( 50.28,  9.66) --
	( 49.39,  9.56) --
	( 48.51,  9.44) --
	( 47.63,  9.30) --
	( 46.76,  9.15) --
	( 45.88,  8.99) --
	( 45.01,  8.80) --
	( 44.15,  8.61) --
	( 43.28,  8.39) --
	( 42.42,  8.17) --
	( 41.57,  7.93) --
	( 40.71,  7.67) --
	( 39.87,  7.40) --
	( 39.03,  7.11) --
	( 38.19,  6.81) --
	( 37.36,  6.49) --
	( 36.53,  6.16) --
	( 35.71,  5.82) --
	( 34.90,  5.46) --
	( 34.09,  5.09) --
	( 33.29,  4.70) --
	( 32.50,  4.30) --
	( 31.71,  3.88) --
	( 30.93,  3.45) --
	( 30.16,  3.01) --
	( 29.40,  2.55) --
	( 28.64,  2.09) --
	( 27.89,  1.60) --
	( 27.16,  1.11) --
	( 26.43,  0.60) --
	( 25.70,  0.08) --
	( 25.60,  0.00);
\definecolor[named]{fillColor}{rgb}{0.00,0.00,0.00}

\path[fill=fillColor] ( 33.41, 42.81) circle (  2.25);
\definecolor[named]{drawColor}{rgb}{0.00,0.00,1.00}

\path[draw=drawColor,line width= 1.2pt,line join=round,line cap=round] ( 79.60, 42.81) --
	( 79.60, 43.62) --
	( 79.57, 44.43) --
	( 79.54, 45.24) --
	( 79.49, 46.04) --
	( 79.43, 46.85) --
	( 79.35, 47.65) --
	( 79.26, 48.46) --
	( 79.15, 49.26) --
	( 79.03, 50.06) --
	( 78.90, 50.86) --
	( 78.75, 51.65) --
	( 78.59, 52.44) --
	( 78.41, 53.23) --
	( 78.22, 54.02) --
	( 78.02, 54.80) --
	( 77.80, 55.58) --
	( 77.57, 56.36) --
	( 77.33, 57.13) --
	( 77.07, 57.89) --
	( 76.80, 58.65) --
	( 76.52, 59.41) --
	( 76.22, 60.16) --
	( 75.91, 60.91) --
	( 75.59, 61.65) --
	( 75.25, 62.39) --
	( 74.90, 63.12) --
	( 74.54, 63.84) --
	( 74.17, 64.56) --
	( 73.78, 65.27) --
	( 73.38, 65.97) --
	( 72.97, 66.66) --
	( 72.54, 67.35) --
	( 72.11, 68.03) --
	( 71.66, 68.71) --
	( 71.20, 69.37) --
	( 70.73, 70.03) --
	( 70.25, 70.68) --
	( 69.76, 71.32) --
	( 69.25, 71.95) --
	( 68.74, 72.57) --
	( 68.21, 73.19) --
	( 67.67, 73.79) --
	( 67.13, 74.39) --
	( 66.57, 74.97) --
	( 66.00, 75.55) --
	( 65.42, 76.11) --
	( 64.84, 76.67) --
	( 64.24, 77.21) --
	( 63.63, 77.75) --
	( 63.01, 78.27) --
	( 62.39, 78.78) --
	( 61.76, 79.29) --
	( 61.11, 79.78) --
	( 60.46, 80.26) --
	( 59.80, 80.72) --
	( 59.14, 81.18) --
	( 58.46, 81.62) --
	( 57.78, 82.06) --
	( 57.09, 82.48) --
	( 56.39, 82.89) --
	( 55.68, 83.28) --
	( 54.97, 83.67) --
	( 54.25, 84.04) --
	( 53.53, 84.40) --
	( 52.80, 84.74) --
	( 52.06, 85.07) --
	( 51.32, 85.39) --
	( 50.57, 85.70) --
	( 49.82, 85.99) --
	( 49.06, 86.28) --
	( 48.30, 86.54) --
	( 47.53, 86.80) --
	( 46.76, 87.04) --
	( 45.98, 87.26) --
	( 45.20, 87.48) --
	( 44.42, 87.68) --
	( 43.63, 87.86) --
	( 42.84, 88.03) --
	( 42.05, 88.19) --
	( 41.25, 88.34) --
	( 40.45, 88.47) --
	( 39.65, 88.58) --
	( 38.85, 88.69) --
	( 38.05, 88.77) --
	( 37.24, 88.85) --
	( 36.44, 88.91) --
	( 35.63, 88.95) --
	( 34.82, 88.99) --
	( 34.01, 89.00) --
	( 33.21, 89.01) --
	( 32.40, 89.00) --
	( 31.59, 88.97) --
	( 30.78, 88.93) --
	( 29.97, 88.88) --
	( 29.17, 88.81) --
	( 28.36, 88.73) --
	( 27.56, 88.64) --
	( 26.76, 88.53) --
	( 25.96, 88.40) --
	( 25.16, 88.27) --
	( 24.37, 88.11) --
	( 23.58, 87.95) --
	( 22.79, 87.77) --
	( 22.01, 87.58) --
	( 21.22, 87.37) --
	( 20.45, 87.15) --
	( 19.67, 86.92) --
	( 18.90, 86.67) --
	( 18.14, 86.41) --
	( 17.38, 86.14) --
	( 16.62, 85.85) --
	( 15.87, 85.55) --
	( 15.12, 85.24) --
	( 14.38, 84.91) --
	( 13.65, 84.57) --
	( 12.92, 84.22) --
	( 12.20, 83.85) --
	( 11.49, 83.48) --
	( 10.78, 83.09) --
	( 10.08, 82.68) --
	(  9.38, 82.27) --
	(  8.70, 81.84) --
	(  8.02, 81.40) --
	(  7.35, 80.95) --
	(  6.68, 80.49) --
	(  6.03, 80.02) --
	(  5.38, 79.53) --
	(  4.74, 79.04) --
	(  4.11, 78.53) --
	(  3.49, 78.01) --
	(  2.88, 77.48) --
	(  2.28, 76.94) --
	(  1.68, 76.39) --
	(  1.10, 75.83) --
	(  0.53, 75.26) --
	(  0.00, 74.72);

\path[draw=drawColor,line width= 1.2pt,line join=round,line cap=round] (  0.00, 10.91) --
	(  0.53, 10.36) --
	(  1.10,  9.79) --
	(  1.68,  9.23) --
	(  2.28,  8.68) --
	(  2.88,  8.14) --
	(  3.49,  7.61) --
	(  4.11,  7.10) --
	(  4.74,  6.59) --
	(  5.38,  6.09) --
	(  6.03,  5.61) --
	(  6.68,  5.13) --
	(  7.35,  4.67) --
	(  8.02,  4.22) --
	(  8.70,  3.78) --
	(  9.38,  3.36) --
	( 10.08,  2.94) --
	( 10.78,  2.54) --
	( 11.49,  2.15) --
	( 12.20,  1.77) --
	( 12.92,  1.41) --
	( 13.65,  1.06) --
	( 14.38,  0.72) --
	( 15.12,  0.39) --
	( 15.87,  0.08) --
	( 16.06,  0.00);

\path[draw=drawColor,line width= 1.2pt,line join=round,line cap=round] ( 50.76,  0.00) --
	( 51.32,  0.23) --
	( 52.06,  0.55) --
	( 52.80,  0.88) --
	( 53.53,  1.23) --
	( 54.25,  1.59) --
	( 54.97,  1.96) --
	( 55.68,  2.34) --
	( 56.39,  2.74) --
	( 57.09,  3.15) --
	( 57.78,  3.57) --
	( 58.46,  4.00) --
	( 59.14,  4.44) --
	( 59.80,  4.90) --
	( 60.46,  5.37) --
	( 61.11,  5.85) --
	( 61.76,  6.34) --
	( 62.39,  6.84) --
	( 63.01,  7.35) --
	( 63.63,  7.88) --
	( 64.24,  8.41) --
	( 64.84,  8.96) --
	( 65.42,  9.51) --
	( 66.00, 10.08) --
	( 66.57, 10.65) --
	( 67.13, 11.24) --
	( 67.67, 11.83) --
	( 68.21, 12.44) --
	( 68.74, 13.05) --
	( 69.25, 13.67) --
	( 69.76, 14.30) --
	( 70.25, 14.95) --
	( 70.73, 15.59) --
	( 71.20, 16.25) --
	( 71.66, 16.92) --
	( 72.11, 17.59) --
	( 72.54, 18.27) --
	( 72.97, 18.96) --
	( 73.38, 19.66) --
	( 73.78, 20.36) --
	( 74.17, 21.07) --
	( 74.54, 21.79) --
	( 74.90, 22.51) --
	( 75.25, 23.24) --
	( 75.59, 23.97) --
	( 75.91, 24.72) --
	( 76.22, 25.46) --
	( 76.52, 26.21) --
	( 76.80, 26.97) --
	( 77.07, 27.73) --
	( 77.33, 28.50) --
	( 77.57, 29.27) --
	( 77.80, 30.04) --
	( 78.02, 30.82) --
	( 78.22, 31.61) --
	( 78.41, 32.39) --
	( 78.59, 33.18) --
	( 78.75, 33.97) --
	( 78.90, 34.77) --
	( 79.03, 35.57) --
	( 79.15, 36.37) --
	( 79.26, 37.17) --
	( 79.35, 37.97) --
	( 79.43, 38.78) --
	( 79.49, 39.58) --
	( 79.54, 40.39) --
	( 79.57, 41.20) --
	( 79.60, 42.00) --
	( 79.60, 42.81);
\definecolor[named]{drawColor}{rgb}{0.00,0.00,0.00}

\path[draw=drawColor,line width= 0.4pt,dash pattern=on 1pt off 3pt ,line join=round,line cap=round] ( 84.22, 42.81) --
	( 84.21, 43.70) --
	( 84.19, 44.59) --
	( 84.15, 45.48) --
	( 84.10, 46.37) --
	( 84.03, 47.25) --
	( 83.94, 48.14) --
	( 83.84, 49.02) --
	( 83.72, 49.90) --
	( 83.59, 50.78) --
	( 83.45, 51.66) --
	( 83.28, 52.54) --
	( 83.11, 53.41) --
	( 82.91, 54.27) --
	( 82.70, 55.14) --
	( 82.48, 56.00) --
	( 82.24, 56.86) --
	( 81.99, 57.71) --
	( 81.72, 58.56) --
	( 81.44, 59.40) --
	( 81.14, 60.24) --
	( 80.83, 61.07) --
	( 80.50, 61.90) --
	( 80.16, 62.72) --
	( 79.80, 63.53) --
	( 79.43, 64.34) --
	( 79.05, 65.15) --
	( 78.65, 65.94) --
	( 78.24, 66.73) --
	( 77.82, 67.51) --
	( 77.38, 68.28) --
	( 76.92, 69.05) --
	( 76.46, 69.81) --
	( 75.98, 70.56) --
	( 75.49, 71.30) --
	( 74.98, 72.03) --
	( 74.46, 72.75) --
	( 73.93, 73.47) --
	( 73.39, 74.17) --
	( 72.84, 74.87) --
	( 72.27, 75.55) --
	( 71.69, 76.23) --
	( 71.10, 76.89) --
	( 70.50, 77.55) --
	( 69.88, 78.19) --
	( 69.26, 78.82) --
	( 68.62, 79.44) --
	( 67.98, 80.05) --
	( 67.32, 80.65) --
	( 66.65, 81.24) --
	( 65.98, 81.82) --
	( 65.29, 82.38) --
	( 64.59, 82.93) --
	( 63.88, 83.47) --
	( 63.17, 84.00) --
	( 62.44, 84.52) --
	( 61.71, 85.02) --
	( 60.96, 85.51) --
	( 60.21, 85.98) --
	( 59.45, 86.44) --
	( 58.69, 86.89) --
	( 57.91, 87.33) --
	( 57.13, 87.75) --
	( 56.34, 88.16) --
	( 55.54, 88.55) --
	( 54.74, 88.93) --
	( 53.93, 89.30) --
	( 53.11, 89.65) --
	( 52.29, 89.99) --
	( 51.46, 90.31) --
	( 50.62, 90.62) --
	( 49.79, 90.92) --
	( 48.94, 91.19) --
	( 48.09, 91.46) --
	( 47.24, 91.71) --
	( 46.38, 91.94) --
	( 45.52, 92.16) --
	( 44.65, 92.37) --
	( 43.78, 92.56) --
	( 42.91, 92.73) --
	( 42.04, 92.89) --
	( 41.16, 93.03) --
	( 40.28, 93.16) --
	( 39.40, 93.27) --
	( 38.51, 93.37) --
	( 37.63, 93.45) --
	( 36.74, 93.52) --
	( 35.85, 93.57) --
	( 34.96, 93.60) --
	( 34.07, 93.62) --
	( 33.19, 93.63) --
	( 32.30, 93.61) --
	( 31.41, 93.59) --
	( 30.52, 93.54) --
	( 29.63, 93.49) --
	( 28.75, 93.41) --
	( 27.86, 93.32) --
	( 26.98, 93.22) --
	( 26.10, 93.10) --
	( 25.22, 92.96) --
	( 24.34, 92.81) --
	( 23.47, 92.65) --
	( 22.60, 92.46) --
	( 21.73, 92.27) --
	( 20.87, 92.05) --
	( 20.01, 91.83) --
	( 19.15, 91.59) --
	( 18.30, 91.33) --
	( 17.45, 91.06) --
	( 16.61, 90.77) --
	( 15.77, 90.47) --
	( 14.94, 90.15) --
	( 14.12, 89.82) --
	( 13.30, 89.48) --
	( 12.48, 89.12) --
	( 11.68, 88.75) --
	( 10.87, 88.36) --
	( 10.08, 87.96) --
	(  9.29, 87.54) --
	(  8.52, 87.11) --
	(  7.74, 86.67) --
	(  6.98, 86.21) --
	(  6.22, 85.75) --
	(  5.48, 85.26) --
	(  4.74, 84.77) --
	(  4.01, 84.26) --
	(  3.29, 83.74) --
	(  2.58, 83.21) --
	(  1.87, 82.66) --
	(  1.18, 82.10) --
	(  0.50, 81.53) --
	(  0.00, 81.10);

\path[draw=drawColor,line width= 0.4pt,dash pattern=on 1pt off 3pt ,line join=round,line cap=round] (  0.00,  4.53) --
	(  0.50,  4.09) --
	(  1.18,  3.52) --
	(  1.87,  2.97) --
	(  2.58,  2.42) --
	(  3.29,  1.89) --
	(  4.01,  1.37) --
	(  4.74,  0.86) --
	(  5.48,  0.36) --
	(  6.04,  0.00);

\path[draw=drawColor,line width= 0.4pt,dash pattern=on 1pt off 3pt ,line join=round,line cap=round] ( 60.78,  0.00) --
	( 60.96,  0.12) --
	( 61.71,  0.61) --
	( 62.44,  1.11) --
	( 63.17,  1.62) --
	( 63.88,  2.15) --
	( 64.59,  2.69) --
	( 65.29,  3.24) --
	( 65.98,  3.81) --
	( 66.65,  4.38) --
	( 67.32,  4.97) --
	( 67.98,  5.57) --
	( 68.62,  6.18) --
	( 69.26,  6.80) --
	( 69.88,  7.44) --
	( 70.50,  8.08) --
	( 71.10,  8.73) --
	( 71.69,  9.40) --
	( 72.27, 10.07) --
	( 72.84, 10.76) --
	( 73.39, 11.45) --
	( 73.93, 12.16) --
	( 74.46, 12.87) --
	( 74.98, 13.60) --
	( 75.49, 14.33) --
	( 75.98, 15.07) --
	( 76.46, 15.82) --
	( 76.92, 16.58) --
	( 77.38, 17.34) --
	( 77.82, 18.11) --
	( 78.24, 18.90) --
	( 78.65, 19.68) --
	( 79.05, 20.48) --
	( 79.43, 21.28) --
	( 79.80, 22.09) --
	( 80.16, 22.91) --
	( 80.50, 23.73) --
	( 80.83, 24.55) --
	( 81.14, 25.39) --
	( 81.44, 26.22) --
	( 81.72, 27.07) --
	( 81.99, 27.92) --
	( 82.24, 28.77) --
	( 82.48, 29.62) --
	( 82.70, 30.49) --
	( 82.91, 31.35) --
	( 83.11, 32.22) --
	( 83.28, 33.09) --
	( 83.45, 33.96) --
	( 83.59, 34.84) --
	( 83.72, 35.72) --
	( 83.84, 36.60) --
	( 83.94, 37.49) --
	( 84.03, 38.37) --
	( 84.10, 39.26) --
	( 84.15, 40.15) --
	( 84.19, 41.03) --
	( 84.21, 41.92) --
	( 84.22, 42.81);

\path[fill=fillColor] ( 10.99,126.47) circle (  2.25);
\definecolor[named]{drawColor}{rgb}{0.00,0.00,1.00}

\path[draw=drawColor,line width= 1.2pt,line join=round,line cap=round] ( 57.19,126.47) --
	( 57.18,127.28) --
	( 57.16,128.09) --
	( 57.12,128.90) --
	( 57.07,129.70) --
	( 57.01,130.51) --
	( 56.93,131.31) --
	( 56.84,132.12) --
	( 56.73,132.92) --
	( 56.61,133.72) --
	( 56.48,134.52) --
	( 56.33,135.31) --
	( 56.17,136.10) --
	( 56.00,136.89) --
	( 55.81,137.68) --
	( 55.60,138.46) --
	( 55.39,139.24) --
	( 55.16,140.02) --
	( 54.91,140.79) --
	( 54.66,141.55) --
	( 54.38,142.31) --
	( 54.10,143.07) --
	( 53.80,143.82) --
	( 53.49,144.57) --
	( 53.17,145.31) --
	( 52.83,146.05) --
	( 52.49,146.78) --
	( 52.12,147.50) --
	( 51.75,148.22) --
	( 51.36,148.93) --
	( 50.96,149.63) --
	( 50.55,150.32) --
	( 50.13,151.01) --
	( 49.69,151.69) --
	( 49.25,152.37) --
	( 48.79,153.03) --
	( 48.32,153.69) --
	( 47.83,154.34) --
	( 47.34,154.98) --
	( 46.84,155.61) --
	( 46.32,156.23) --
	( 45.79,156.85) --
	( 45.26,157.45) --
	( 44.71,158.05) --
	( 44.15,158.63) --
	( 43.58,159.21) --
	( 43.01,159.77) --
	( 42.42,160.33) --
	( 41.82,160.87) --
	( 41.21,161.41) --
	( 40.60,161.93) --
	( 39.97,162.44) --
	( 39.34,162.95) --
	( 38.70,163.44) --
	( 38.05,163.92) --
	( 37.39,164.38) --
	( 36.72,164.84) --
	( 36.04,165.28) --
	( 35.36,165.72) --
	( 34.67,166.14) --
	( 33.97,166.55) --
	( 33.27,166.94) --
	( 32.56,167.33) --
	( 31.84,167.70) --
	( 31.11,168.06) --
	( 30.38,168.40) --
	( 29.64,168.73) --
	( 28.90,169.05) --
	( 28.15,169.36) --
	( 27.40,169.65) --
	( 26.64,169.94) --
	( 25.88,170.20) --
	( 25.11,170.46) --
	( 24.34,170.70) --
	( 23.56,170.92) --
	( 22.78,171.14) --
	( 22.00,171.34) --
	( 21.21,171.52) --
	( 20.42,171.69) --
	( 19.63,171.85) --
	( 18.84,172.00) --
	( 18.04,172.13) --
	( 17.24,172.24) --
	( 16.44,172.35) --
	( 15.63,172.43) --
	( 14.83,172.51) --
	( 14.02,172.57) --
	( 13.21,172.61) --
	( 12.41,172.65) --
	( 11.60,172.66) --
	( 10.79,172.67) --
	(  9.98,172.66) --
	(  9.17,172.63) --
	(  8.36,172.59) --
	(  7.56,172.54) --
	(  6.75,172.47) --
	(  5.95,172.39) --
	(  5.15,172.30) --
	(  4.34,172.19) --
	(  3.54,172.06) --
	(  2.75,171.93) --
	(  1.95,171.77) --
	(  1.16,171.61) --
	(  0.37,171.43) --
	(  0.00,171.34);

\path[draw=drawColor,line width= 1.2pt,line join=round,line cap=round] (  0.00, 81.61) --
	(  0.37, 81.51) --
	(  1.16, 81.34) --
	(  1.95, 81.17) --
	(  2.75, 81.02) --
	(  3.54, 80.88) --
	(  4.34, 80.76) --
	(  5.15, 80.65) --
	(  5.95, 80.55) --
	(  6.75, 80.47) --
	(  7.56, 80.41) --
	(  8.36, 80.35) --
	(  9.17, 80.31) --
	(  9.98, 80.29) --
	( 10.79, 80.28) --
	( 11.60, 80.28) --
	( 12.41, 80.30) --
	( 13.21, 80.33) --
	( 14.02, 80.38) --
	( 14.83, 80.44) --
	( 15.63, 80.51) --
	( 16.44, 80.60) --
	( 17.24, 80.70) --
	( 18.04, 80.82) --
	( 18.84, 80.95) --
	( 19.63, 81.09) --
	( 20.42, 81.25) --
	( 21.21, 81.42) --
	( 22.00, 81.61) --
	( 22.78, 81.81) --
	( 23.56, 82.02) --
	( 24.34, 82.25) --
	( 25.11, 82.49) --
	( 25.88, 82.74) --
	( 26.64, 83.01) --
	( 27.40, 83.29) --
	( 28.15, 83.58) --
	( 28.90, 83.89) --
	( 29.64, 84.21) --
	( 30.38, 84.54) --
	( 31.11, 84.89) --
	( 31.84, 85.25) --
	( 32.56, 85.62) --
	( 33.27, 86.00) --
	( 33.97, 86.40) --
	( 34.67, 86.81) --
	( 35.36, 87.23) --
	( 36.04, 87.66) --
	( 36.72, 88.10) --
	( 37.39, 88.56) --
	( 38.05, 89.03) --
	( 38.70, 89.51) --
	( 39.34, 90.00) --
	( 39.97, 90.50) --
	( 40.60, 91.01) --
	( 41.21, 91.54) --
	( 41.82, 92.07) --
	( 42.42, 92.62) --
	( 43.01, 93.17) --
	( 43.58, 93.74) --
	( 44.15, 94.31) --
	( 44.71, 94.90) --
	( 45.26, 95.49) --
	( 45.79, 96.10) --
	( 46.32, 96.71) --
	( 46.84, 97.33) --
	( 47.34, 97.96) --
	( 47.83, 98.61) --
	( 48.32, 99.25) --
	( 48.79, 99.91) --
	( 49.25,100.58) --
	( 49.69,101.25) --
	( 50.13,101.93) --
	( 50.55,102.62) --
	( 50.96,103.32) --
	( 51.36,104.02) --
	( 51.75,104.73) --
	( 52.12,105.45) --
	( 52.49,106.17) --
	( 52.83,106.90) --
	( 53.17,107.63) --
	( 53.49,108.38) --
	( 53.80,109.12) --
	( 54.10,109.87) --
	( 54.38,110.63) --
	( 54.66,111.39) --
	( 54.91,112.16) --
	( 55.16,112.93) --
	( 55.39,113.70) --
	( 55.60,114.48) --
	( 55.81,115.27) --
	( 56.00,116.05) --
	( 56.17,116.84) --
	( 56.33,117.63) --
	( 56.48,118.43) --
	( 56.61,119.23) --
	( 56.73,120.03) --
	( 56.84,120.83) --
	( 56.93,121.63) --
	( 57.01,122.44) --
	( 57.07,123.24) --
	( 57.12,124.05) --
	( 57.16,124.86) --
	( 57.18,125.66) --
	( 57.19,126.47);
\definecolor[named]{drawColor}{rgb}{0.00,0.00,0.00}

\path[draw=drawColor,line width= 0.4pt,dash pattern=on 1pt off 3pt ,line join=round,line cap=round] ( 61.81,126.47) --
	( 61.80,127.36) --
	( 61.77,128.25) --
	( 61.74,129.14) --
	( 61.68,130.03) --
	( 61.61,130.91) --
	( 61.53,131.80) --
	( 61.42,132.68) --
	( 61.31,133.56) --
	( 61.18,134.44) --
	( 61.03,135.32) --
	( 60.87,136.20) --
	( 60.69,137.07) --
	( 60.50,137.93) --
	( 60.29,138.80) --
	( 60.06,139.66) --
	( 59.83,140.52) --
	( 59.57,141.37) --
	( 59.30,142.22) --
	( 59.02,143.06) --
	( 58.72,143.90) --
	( 58.41,144.73) --
	( 58.08,145.56) --
	( 57.74,146.38) --
	( 57.39,147.19) --
	( 57.02,148.00) --
	( 56.63,148.81) --
	( 56.24,149.60) --
	( 55.82,150.39) --
	( 55.40,151.17) --
	( 54.96,151.94) --
	( 54.51,152.71) --
	( 54.04,153.47) --
	( 53.56,154.22) --
	( 53.07,154.96) --
	( 52.57,155.69) --
	( 52.05,156.41) --
	( 51.52,157.13) --
	( 50.98,157.83) --
	( 50.42,158.53) --
	( 49.85,159.21) --
	( 49.27,159.89) --
	( 48.68,160.55) --
	( 48.08,161.21) --
	( 47.47,161.85) --
	( 46.84,162.48) --
	( 46.21,163.10) --
	( 45.56,163.71) --
	( 44.90,164.31) --
	( 44.24,164.90) --
	( 43.56,165.48) --
	( 42.87,166.04) --
	( 42.17,166.59) --
	( 41.47,167.13) --
	( 40.75,167.66) --
	( 40.03,168.18) --
	( 39.29,168.68) --
	( 38.55,169.17) --
	( 37.80,169.64) --
	( 37.04,170.10) --
	( 36.27,170.55) --
	( 35.49,170.99) --
	( 34.71,171.41) --
	( 33.92,171.82) --
	( 33.12,172.21) --
	( 32.32,172.59) --
	( 31.51,172.96) --
	( 30.69,173.31) --
	( 29.87,173.65) --
	( 29.04,173.97) --
	( 28.21,174.28) --
	( 27.37,174.58) --
	( 26.52,174.85) --
	( 25.68,175.12) --
	( 24.82,175.37) --
	( 23.96,175.60) --
	( 23.10,175.82) --
	( 22.24,176.03) --
	( 21.37,176.22) --
	( 20.50,176.39) --
	( 19.62,176.55) --
	( 18.74,176.69) --
	( 17.86,176.82) --
	( 16.98,176.93) --
	( 16.10,177.03) --
	( 15.21,177.11) --
	( 14.32,177.18) --
	( 13.44,177.23) --
	( 12.55,177.26) --
	( 11.66,177.28) --
	( 10.77,177.29) --
	(  9.88,177.27) --
	(  8.99,177.25) --
	(  8.10,177.20) --
	(  7.21,177.15) --
	(  6.33,177.07) --
	(  5.44,176.98) --
	(  4.56,176.88) --
	(  3.68,176.76) --
	(  2.80,176.62) --
	(  1.92,176.47) --
	(  1.05,176.31) --
	(  0.18,176.12) --
	(  0.00,176.08);

\path[draw=drawColor,line width= 0.4pt,dash pattern=on 1pt off 3pt ,line join=round,line cap=round] (  0.00, 76.86) --
	(  0.18, 76.82) --
	(  1.05, 76.64) --
	(  1.92, 76.47) --
	(  2.80, 76.32) --
	(  3.68, 76.19) --
	(  4.56, 76.07) --
	(  5.44, 75.96) --
	(  6.33, 75.87) --
	(  7.21, 75.80) --
	(  8.10, 75.74) --
	(  8.99, 75.70) --
	(  9.88, 75.67) --
	( 10.77, 75.66) --
	( 11.66, 75.66) --
	( 12.55, 75.68) --
	( 13.44, 75.72) --
	( 14.32, 75.77) --
	( 15.21, 75.83) --
	( 16.10, 75.92) --
	( 16.98, 76.01) --
	( 17.86, 76.12) --
	( 18.74, 76.25) --
	( 19.62, 76.40) --
	( 20.50, 76.55) --
	( 21.37, 76.73) --
	( 22.24, 76.92) --
	( 23.10, 77.12) --
	( 23.96, 77.34) --
	( 24.82, 77.58) --
	( 25.68, 77.83) --
	( 26.52, 78.09) --
	( 27.37, 78.37) --
	( 28.21, 78.66) --
	( 29.04, 78.97) --
	( 29.87, 79.30) --
	( 30.69, 79.63) --
	( 31.51, 79.99) --
	( 32.32, 80.35) --
	( 33.12, 80.73) --
	( 33.92, 81.13) --
	( 34.71, 81.53) --
	( 35.49, 81.96) --
	( 36.27, 82.39) --
	( 37.04, 82.84) --
	( 37.80, 83.30) --
	( 38.55, 83.78) --
	( 39.29, 84.27) --
	( 40.03, 84.77) --
	( 40.75, 85.28) --
	( 41.47, 85.81) --
	( 42.17, 86.35) --
	( 42.87, 86.90) --
	( 43.56, 87.47) --
	( 44.24, 88.04) --
	( 44.90, 88.63) --
	( 45.56, 89.23) --
	( 46.21, 89.84) --
	( 46.84, 90.46) --
	( 47.47, 91.10) --
	( 48.08, 91.74) --
	( 48.68, 92.39) --
	( 49.27, 93.06) --
	( 49.85, 93.73) --
	( 50.42, 94.42) --
	( 50.98, 95.11) --
	( 51.52, 95.82) --
	( 52.05, 96.53) --
	( 52.57, 97.26) --
	( 53.07, 97.99) --
	( 53.56, 98.73) --
	( 54.04, 99.48) --
	( 54.51,100.24) --
	( 54.96,101.00) --
	( 55.40,101.77) --
	( 55.82,102.56) --
	( 56.24,103.34) --
	( 56.63,104.14) --
	( 57.02,104.94) --
	( 57.39,105.75) --
	( 57.74,106.57) --
	( 58.08,107.39) --
	( 58.41,108.21) --
	( 58.72,109.05) --
	( 59.02,109.88) --
	( 59.30,110.73) --
	( 59.57,111.58) --
	( 59.83,112.43) --
	( 60.06,113.28) --
	( 60.29,114.15) --
	( 60.50,115.01) --
	( 60.69,115.88) --
	( 60.87,116.75) --
	( 61.03,117.62) --
	( 61.18,118.50) --
	( 61.31,119.38) --
	( 61.42,120.26) --
	( 61.53,121.15) --
	( 61.61,122.03) --
	( 61.68,122.92) --
	( 61.74,123.81) --
	( 61.77,124.69) --
	( 61.80,125.58) --
	( 61.81,126.47);
\definecolor[named]{drawColor}{rgb}{0.00,0.00,1.00}

\path[draw=drawColor,line width= 1.2pt,line join=round,line cap=round] ( 34.77,210.13) --
	( 34.76,210.94) --
	( 34.74,211.75) --
	( 34.71,212.56) --
	( 34.66,213.36) --
	( 34.59,214.17) --
	( 34.51,214.97) --
	( 34.42,215.78) --
	( 34.32,216.58) --
	( 34.20,217.38) --
	( 34.06,218.18) --
	( 33.92,218.97) --
	( 33.75,219.76) --
	( 33.58,220.55) --
	( 33.39,221.34) --
	( 33.19,222.12) --
	( 32.97,222.90) --
	( 32.74,223.68) --
	( 32.50,224.45) --
	( 32.24,225.21) --
	( 31.97,225.97) --
	( 31.68,226.73) --
	( 31.39,227.48) --
	( 31.08,228.23) --
	( 30.75,228.97) --
	( 30.42,229.71) --
	( 30.07,230.44) --
	( 29.71,231.16) --
	( 29.33,231.88) --
	( 28.95,232.59) --
	( 28.55,233.29) --
	( 28.14,233.98) --
	( 27.71,234.67) --
	( 27.28,235.35) --
	( 26.83,236.03) --
	( 26.37,236.69) --
	( 25.90,237.35) --
	( 25.42,238.00) --
	( 24.92,238.64) --
	( 24.42,239.27) --
	( 23.90,239.89) --
	( 23.38,240.51) --
	( 22.84,241.11) --
	( 22.29,241.71) --
	( 21.74,242.29) --
	( 21.17,242.87) --
	( 20.59,243.43) --
	( 20.00,243.99) --
	( 19.40,244.53) --
	( 18.80,245.07) --
	( 18.18,245.59) --
	( 17.56,246.10) --
	( 16.92,246.61) --
	( 16.28,247.10) --
	( 15.63,247.58) --
	( 14.97,248.04) --
	( 14.30,248.50) --
	( 13.63,248.94) --
	( 12.94,249.38) --
	( 12.25,249.80) --
	( 11.55,250.21) --
	( 10.85,250.60) --
	( 10.14,250.99) --
	(  9.42,251.36) --
	(  8.70,251.72) --
	(  7.96,252.06) --
	(  7.23,252.39) --
	(  6.49,252.71) --
	(  5.92,252.94);

\path[draw=drawColor,line width= 1.2pt,line join=round,line cap=round] (  0.00,165.37) --
	(  0.37,165.47) --
	(  1.15,165.68) --
	(  1.92,165.91) --
	(  2.70,166.15) --
	(  3.46,166.40) --
	(  4.23,166.67) --
	(  4.98,166.95) --
	(  5.74,167.24) --
	(  6.49,167.55) --
	(  7.23,167.87) --
	(  7.96,168.20) --
	(  8.70,168.55) --
	(  9.42,168.91) --
	( 10.14,169.28) --
	( 10.85,169.66) --
	( 11.55,170.06) --
	( 12.25,170.47) --
	( 12.94,170.89) --
	( 13.63,171.32) --
	( 14.30,171.76) --
	( 14.97,172.22) --
	( 15.63,172.69) --
	( 16.28,173.17) --
	( 16.92,173.66) --
	( 17.56,174.16) --
	( 18.18,174.67) --
	( 18.80,175.20) --
	( 19.40,175.73) --
	( 20.00,176.28) --
	( 20.59,176.83) --
	( 21.17,177.40) --
	( 21.74,177.97) --
	( 22.29,178.56) --
	( 22.84,179.15) --
	( 23.38,179.76) --
	( 23.90,180.37) --
	( 24.42,180.99) --
	( 24.92,181.62) --
	( 25.42,182.27) --
	( 25.90,182.91) --
	( 26.37,183.57) --
	( 26.83,184.24) --
	( 27.28,184.91) --
	( 27.71,185.59) --
	( 28.14,186.28) --
	( 28.55,186.98) --
	( 28.95,187.68) --
	( 29.33,188.39) --
	( 29.71,189.11) --
	( 30.07,189.83) --
	( 30.42,190.56) --
	( 30.75,191.29) --
	( 31.08,192.04) --
	( 31.39,192.78) --
	( 31.68,193.53) --
	( 31.97,194.29) --
	( 32.24,195.05) --
	( 32.50,195.82) --
	( 32.74,196.59) --
	( 32.97,197.36) --
	( 33.19,198.14) --
	( 33.39,198.93) --
	( 33.58,199.71) --
	( 33.75,200.50) --
	( 33.92,201.29) --
	( 34.06,202.09) --
	( 34.20,202.89) --
	( 34.32,203.69) --
	( 34.42,204.49) --
	( 34.51,205.29) --
	( 34.59,206.10) --
	( 34.66,206.90) --
	( 34.71,207.71) --
	( 34.74,208.52) --
	( 34.76,209.32) --
	( 34.77,210.13);
\definecolor[named]{drawColor}{rgb}{0.00,0.00,0.00}

\path[draw=drawColor,line width= 0.4pt,dash pattern=on 1pt off 3pt ,line join=round,line cap=round] ( 39.39,210.13) --
	( 39.38,211.02) --
	( 39.36,211.91) --
	( 39.32,212.80) --
	( 39.26,213.69) --
	( 39.19,214.57) --
	( 39.11,215.46) --
	( 39.01,216.34) --
	( 38.89,217.22) --
	( 38.76,218.10) --
	( 38.61,218.98) --
	( 38.45,219.86) --
	( 38.27,220.73) --
	( 38.08,221.59) --
	( 37.87,222.46) --
	( 37.65,223.32) --
	( 37.41,224.18) --
	( 37.16,225.03) --
	( 36.89,225.88) --
	( 36.60,226.72) --
	( 36.31,227.56) --
	( 36.00,228.39) --
	( 35.67,229.22) --
	( 35.33,230.04) --
	( 34.97,230.85) --
	( 34.60,231.66) --
	( 34.22,232.47) --
	( 33.82,233.26) --
	( 33.41,234.05) --
	( 32.98,234.83) --
	( 32.54,235.60) --
	( 32.09,236.37) --
	( 31.63,237.13) --
	( 31.15,237.88) --
	( 30.65,238.62) --
	( 30.15,239.35) --
	( 29.63,240.07) --
	( 29.10,240.79) --
	( 28.56,241.49) --
	( 28.00,242.19) --
	( 27.44,242.87) --
	( 26.86,243.55) --
	( 26.27,244.21) --
	( 25.67,244.87) --
	( 25.05,245.51) --
	( 24.43,246.14) --
	( 23.79,246.76) --
	( 23.14,247.37) --
	( 22.49,247.97) --
	( 21.82,248.56) --
	( 21.14,249.14) --
	( 20.45,249.70) --
	( 19.76,250.25) --
	( 19.05,250.79) --
	( 18.33,251.32) --
	( 17.61,251.84) --
	( 16.87,252.34) --
	( 16.13,252.83) --
	( 15.94,252.94);

\path[draw=drawColor,line width= 0.4pt,dash pattern=on 1pt off 3pt ,line join=round,line cap=round] (  0.00,160.62) --
	(  0.69,160.78) --
	(  1.55,161.00) --
	(  2.40,161.24) --
	(  3.26,161.49) --
	(  4.11,161.75) --
	(  4.95,162.03) --
	(  5.79,162.32) --
	(  6.63,162.63) --
	(  7.45,162.96) --
	(  8.28,163.29) --
	(  9.09,163.65) --
	(  9.90,164.01) --
	( 10.71,164.39) --
	( 11.50,164.79) --
	( 12.29,165.19) --
	( 13.08,165.62) --
	( 13.85,166.05) --
	( 14.62,166.50) --
	( 15.38,166.96) --
	( 16.13,167.44) --
	( 16.87,167.93) --
	( 17.61,168.43) --
	( 18.33,168.94) --
	( 19.05,169.47) --
	( 19.76,170.01) --
	( 20.45,170.56) --
	( 21.14,171.13) --
	( 21.82,171.70) --
	( 22.49,172.29) --
	( 23.14,172.89) --
	( 23.79,173.50) --
	( 24.43,174.12) --
	( 25.05,174.76) --
	( 25.67,175.40) --
	( 26.27,176.05) --
	( 26.86,176.72) --
	( 27.44,177.39) --
	( 28.00,178.08) --
	( 28.56,178.77) --
	( 29.10,179.48) --
	( 29.63,180.19) --
	( 30.15,180.92) --
	( 30.65,181.65) --
	( 31.15,182.39) --
	( 31.63,183.14) --
	( 32.09,183.90) --
	( 32.54,184.66) --
	( 32.98,185.43) --
	( 33.41,186.22) --
	( 33.82,187.00) --
	( 34.22,187.80) --
	( 34.60,188.60) --
	( 34.97,189.41) --
	( 35.33,190.23) --
	( 35.67,191.05) --
	( 36.00,191.87) --
	( 36.31,192.71) --
	( 36.60,193.54) --
	( 36.89,194.39) --
	( 37.16,195.24) --
	( 37.41,196.09) --
	( 37.65,196.94) --
	( 37.87,197.81) --
	( 38.08,198.67) --
	( 38.27,199.54) --
	( 38.45,200.41) --
	( 38.61,201.28) --
	( 38.76,202.16) --
	( 38.89,203.04) --
	( 39.01,203.92) --
	( 39.11,204.81) --
	( 39.19,205.69) --
	( 39.26,206.58) --
	( 39.32,207.47) --
	( 39.36,208.35) --
	( 39.38,209.24) --
	( 39.39,210.13);
\definecolor[named]{drawColor}{rgb}{0.00,0.00,1.00}

\path[draw=drawColor,line width= 1.2pt,line join=round,line cap=round] (135.14,  0.00) --
	(135.13,  0.01) --
	(134.41,  0.38) --
	(133.69,  0.74) --
	(132.96,  1.08) --
	(132.22,  1.41) --
	(131.48,  1.73) --
	(130.73,  2.04) --
	(129.97,  2.33) --
	(129.22,  2.62) --
	(128.45,  2.88) --
	(127.69,  3.14) --
	(126.91,  3.38) --
	(126.14,  3.60) --
	(125.36,  3.82) --
	(124.57,  4.02) --
	(123.79,  4.20) --
	(123.00,  4.37) --
	(122.21,  4.53) --
	(121.41,  4.68) --
	(120.61,  4.81) --
	(119.81,  4.92) --
	(119.01,  5.03) --
	(118.21,  5.11) --
	(117.40,  5.19) --
	(116.59,  5.25) --
	(115.79,  5.29) --
	(114.98,  5.33) --
	(114.17,  5.34) --
	(113.36,  5.35) --
	(112.55,  5.34) --
	(111.75,  5.31) --
	(110.94,  5.27) --
	(110.13,  5.22) --
	(109.33,  5.15) --
	(108.52,  5.07) --
	(107.72,  4.98) --
	(106.92,  4.87) --
	(106.12,  4.74) --
	(105.32,  4.61) --
	(104.53,  4.45) --
	(103.74,  4.29) --
	(102.95,  4.11) --
	(102.16,  3.92) --
	(101.38,  3.71) --
	(100.60,  3.49) --
	( 99.83,  3.26) --
	( 99.06,  3.01) --
	( 98.29,  2.75) --
	( 97.53,  2.48) --
	( 96.78,  2.19) --
	( 96.03,  1.89) --
	( 95.28,  1.58) --
	( 94.54,  1.25) --
	( 93.81,  0.91) --
	( 93.08,  0.56) --
	( 92.36,  0.19) --
	( 91.99,  0.00);
\definecolor[named]{drawColor}{rgb}{0.00,0.00,0.00}

\path[draw=drawColor,line width= 0.4pt,dash pattern=on 1pt off 3pt ,line join=round,line cap=round] (143.79,  0.00) --
	(143.32,  0.34) --
	(142.60,  0.86) --
	(141.87,  1.36) --
	(141.12,  1.85) --
	(140.37,  2.32) --
	(139.61,  2.78) --
	(138.84,  3.23) --
	(138.07,  3.67) --
	(137.29,  4.09) --
	(136.50,  4.50) --
	(135.70,  4.89) --
	(134.89,  5.27) --
	(134.08,  5.64) --
	(133.27,  5.99) --
	(132.44,  6.33) --
	(131.62,  6.65) --
	(130.78,  6.96) --
	(129.94,  7.26) --
	(129.10,  7.53) --
	(128.25,  7.80) --
	(127.40,  8.05) --
	(126.54,  8.28) --
	(125.68,  8.50) --
	(124.81,  8.71) --
	(123.94,  8.90) --
	(123.07,  9.07) --
	(122.19,  9.23) --
	(121.32,  9.37) --
	(120.44,  9.50) --
	(119.55,  9.61) --
	(118.67,  9.71) --
	(117.78,  9.79) --
	(116.90,  9.86) --
	(116.01,  9.91) --
	(115.12,  9.94) --
	(114.23,  9.96) --
	(113.34,  9.97) --
	(112.45,  9.95) --
	(111.56,  9.93) --
	(110.68,  9.88) --
	(109.79,  9.83) --
	(108.90,  9.75) --
	(108.02,  9.66) --
	(107.13,  9.56) --
	(106.25,  9.44) --
	(105.37,  9.30) --
	(104.50,  9.15) --
	(103.62,  8.99) --
	(102.75,  8.80) --
	(101.89,  8.61) --
	(101.02,  8.39) --
	(100.16,  8.17) --
	( 99.31,  7.93) --
	( 98.46,  7.67) --
	( 97.61,  7.40) --
	( 96.77,  7.11) --
	( 95.93,  6.81) --
	( 95.10,  6.49) --
	( 94.27,  6.16) --
	( 93.45,  5.82) --
	( 92.64,  5.46) --
	( 91.83,  5.09) --
	( 91.03,  4.70) --
	( 90.24,  4.30) --
	( 89.45,  3.88) --
	( 88.67,  3.45) --
	( 87.90,  3.01) --
	( 87.14,  2.55) --
	( 86.38,  2.09) --
	( 85.64,  1.60) --
	( 84.90,  1.11) --
	( 84.17,  0.60) --
	( 83.45,  0.08) --
	( 83.34,  0.00);

\path[fill=fillColor] ( 91.15, 42.81) circle (  2.25);
\definecolor[named]{drawColor}{rgb}{0.00,0.00,1.00}

\path[draw=drawColor,line width= 1.2pt,line join=round,line cap=round] (137.34, 42.81) --
	(137.34, 43.62) --
	(137.31, 44.43) --
	(137.28, 45.24) --
	(137.23, 46.04) --
	(137.17, 46.85) --
	(137.09, 47.65) --
	(137.00, 48.46) --
	(136.89, 49.26) --
	(136.77, 50.06) --
	(136.64, 50.86) --
	(136.49, 51.65) --
	(136.33, 52.44) --
	(136.15, 53.23) --
	(135.96, 54.02) --
	(135.76, 54.80) --
	(135.54, 55.58) --
	(135.31, 56.36) --
	(135.07, 57.13) --
	(134.81, 57.89) --
	(134.54, 58.65) --
	(134.26, 59.41) --
	(133.96, 60.16) --
	(133.65, 60.91) --
	(133.33, 61.65) --
	(132.99, 62.39) --
	(132.64, 63.12) --
	(132.28, 63.84) --
	(131.91, 64.56) --
	(131.52, 65.27) --
	(131.12, 65.97) --
	(130.71, 66.66) --
	(130.29, 67.35) --
	(129.85, 68.03) --
	(129.40, 68.71) --
	(128.94, 69.37) --
	(128.47, 70.03) --
	(127.99, 70.68) --
	(127.50, 71.32) --
	(126.99, 71.95) --
	(126.48, 72.57) --
	(125.95, 73.19) --
	(125.41, 73.79) --
	(124.87, 74.39) --
	(124.31, 74.97) --
	(123.74, 75.55) --
	(123.16, 76.11) --
	(122.58, 76.67) --
	(121.98, 77.21) --
	(121.37, 77.75) --
	(120.76, 78.27) --
	(120.13, 78.78) --
	(119.50, 79.29) --
	(118.85, 79.78) --
	(118.20, 80.26) --
	(117.54, 80.72) --
	(116.88, 81.18) --
	(116.20, 81.62) --
	(115.52, 82.06) --
	(114.83, 82.48) --
	(114.13, 82.89) --
	(113.42, 83.28) --
	(112.71, 83.67) --
	(111.99, 84.04) --
	(111.27, 84.40) --
	(110.54, 84.74) --
	(109.80, 85.07) --
	(109.06, 85.39) --
	(108.31, 85.70) --
	(107.56, 85.99) --
	(106.80, 86.28) --
	(106.04, 86.54) --
	(105.27, 86.80) --
	(104.50, 87.04) --
	(103.72, 87.26) --
	(102.94, 87.48) --
	(102.16, 87.68) --
	(101.37, 87.86) --
	(100.58, 88.03) --
	( 99.79, 88.19) --
	( 98.99, 88.34) --
	( 98.20, 88.47) --
	( 97.40, 88.58) --
	( 96.59, 88.69) --
	( 95.79, 88.77) --
	( 94.98, 88.85) --
	( 94.18, 88.91) --
	( 93.37, 88.95) --
	( 92.56, 88.99) --
	( 91.75, 89.00) --
	( 90.95, 89.01) --
	( 90.14, 89.00) --
	( 89.33, 88.97) --
	( 88.52, 88.93) --
	( 87.72, 88.88) --
	( 86.91, 88.81) --
	( 86.11, 88.73) --
	( 85.30, 88.64) --
	( 84.50, 88.53) --
	( 83.70, 88.40) --
	( 82.91, 88.27) --
	( 82.11, 88.11) --
	( 81.32, 87.95) --
	( 80.53, 87.77) --
	( 79.75, 87.58) --
	( 78.96, 87.37) --
	( 78.19, 87.15) --
	( 77.41, 86.92) --
	( 76.64, 86.67) --
	( 75.88, 86.41) --
	( 75.12, 86.14) --
	( 74.36, 85.85) --
	( 73.61, 85.55) --
	( 72.87, 85.24) --
	( 72.13, 84.91) --
	( 71.39, 84.57) --
	( 70.66, 84.22) --
	( 69.94, 83.85) --
	( 69.23, 83.48) --
	( 68.52, 83.09) --
	( 67.82, 82.68) --
	( 67.12, 82.27) --
	( 66.44, 81.84) --
	( 65.76, 81.40) --
	( 65.09, 80.95) --
	( 64.42, 80.49) --
	( 63.77, 80.02) --
	( 63.12, 79.53) --
	( 62.48, 79.04) --
	( 61.85, 78.53) --
	( 61.23, 78.01) --
	( 60.62, 77.48) --
	( 60.02, 76.94) --
	( 59.43, 76.39) --
	( 58.84, 75.83) --
	( 58.27, 75.26) --
	( 57.71, 74.68) --
	( 57.15, 74.09) --
	( 56.61, 73.49) --
	( 56.08, 72.88) --
	( 55.56, 72.26) --
	( 55.05, 71.64) --
	( 54.55, 71.00) --
	( 54.06, 70.36) --
	( 53.59, 69.70) --
	( 53.12, 69.04) --
	( 52.67, 68.37) --
	( 52.23, 67.69) --
	( 51.80, 67.01) --
	( 51.38, 66.32) --
	( 50.97, 65.62) --
	( 50.58, 64.91) --
	( 50.20, 64.20) --
	( 49.83, 63.48) --
	( 49.48, 62.75) --
	( 49.14, 62.02) --
	( 48.81, 61.28) --
	( 48.49, 60.54) --
	( 48.19, 59.79) --
	( 47.89, 59.03) --
	( 47.62, 58.27) --
	( 47.35, 57.51) --
	( 47.10, 56.74) --
	( 46.87, 55.97) --
	( 46.64, 55.19) --
	( 46.43, 54.41) --
	( 46.24, 53.63) --
	( 46.05, 52.84) --
	( 45.89, 52.05) --
	( 45.73, 51.25) --
	( 45.59, 50.46) --
	( 45.46, 49.66) --
	( 45.35, 48.86) --
	( 45.25, 48.06) --
	( 45.17, 47.25) --
	( 45.10, 46.45) --
	( 45.04, 45.64) --
	( 45.00, 44.83) --
	( 44.97, 44.03) --
	( 44.96, 43.22) --
	( 44.96, 42.41) --
	( 44.97, 41.60) --
	( 45.00, 40.79) --
	( 45.04, 39.98) --
	( 45.10, 39.18) --
	( 45.17, 38.37) --
	( 45.25, 37.57) --
	( 45.35, 36.77) --
	( 45.46, 35.97) --
	( 45.59, 35.17) --
	( 45.73, 34.37) --
	( 45.89, 33.58) --
	( 46.05, 32.79) --
	( 46.24, 32.00) --
	( 46.43, 31.21) --
	( 46.64, 30.43) --
	( 46.87, 29.66) --
	( 47.10, 28.88) --
	( 47.35, 28.12) --
	( 47.62, 27.35) --
	( 47.89, 26.59) --
	( 48.19, 25.84) --
	( 48.49, 25.09) --
	( 48.81, 24.34) --
	( 49.14, 23.61) --
	( 49.48, 22.87) --
	( 49.83, 22.15) --
	( 50.20, 21.43) --
	( 50.58, 20.71) --
	( 50.97, 20.01) --
	( 51.38, 19.31) --
	( 51.80, 18.62) --
	( 52.23, 17.93) --
	( 52.67, 17.25) --
	( 53.12, 16.58) --
	( 53.59, 15.92) --
	( 54.06, 15.27) --
	( 54.55, 14.62) --
	( 55.05, 13.99) --
	( 55.56, 13.36) --
	( 56.08, 12.74) --
	( 56.61, 12.13) --
	( 57.15, 11.53) --
	( 57.71, 10.94) --
	( 58.27, 10.36) --
	( 58.84,  9.79) --
	( 59.43,  9.23) --
	( 60.02,  8.68) --
	( 60.62,  8.14) --
	( 61.23,  7.61) --
	( 61.85,  7.10) --
	( 62.48,  6.59) --
	( 63.12,  6.09) --
	( 63.77,  5.61) --
	( 64.42,  5.13) --
	( 65.09,  4.67) --
	( 65.76,  4.22) --
	( 66.44,  3.78) --
	( 67.12,  3.36) --
	( 67.82,  2.94) --
	( 68.52,  2.54) --
	( 69.23,  2.15) --
	( 69.94,  1.77) --
	( 70.66,  1.41) --
	( 71.39,  1.06) --
	( 72.13,  0.72) --
	( 72.87,  0.39) --
	( 73.61,  0.08) --
	( 73.80,  0.00);

\path[draw=drawColor,line width= 1.2pt,line join=round,line cap=round] (108.50,  0.00) --
	(109.06,  0.23) --
	(109.80,  0.55) --
	(110.54,  0.88) --
	(111.27,  1.23) --
	(111.99,  1.59) --
	(112.71,  1.96) --
	(113.42,  2.34) --
	(114.13,  2.74) --
	(114.83,  3.15) --
	(115.52,  3.57) --
	(116.20,  4.00) --
	(116.88,  4.44) --
	(117.54,  4.90) --
	(118.20,  5.37) --
	(118.85,  5.85) --
	(119.50,  6.34) --
	(120.13,  6.84) --
	(120.76,  7.35) --
	(121.37,  7.88) --
	(121.98,  8.41) --
	(122.58,  8.96) --
	(123.16,  9.51) --
	(123.74, 10.08) --
	(124.31, 10.65) --
	(124.87, 11.24) --
	(125.41, 11.83) --
	(125.95, 12.44) --
	(126.48, 13.05) --
	(126.99, 13.67) --
	(127.50, 14.30) --
	(127.99, 14.95) --
	(128.47, 15.59) --
	(128.94, 16.25) --
	(129.40, 16.92) --
	(129.85, 17.59) --
	(130.29, 18.27) --
	(130.71, 18.96) --
	(131.12, 19.66) --
	(131.52, 20.36) --
	(131.91, 21.07) --
	(132.28, 21.79) --
	(132.64, 22.51) --
	(132.99, 23.24) --
	(133.33, 23.97) --
	(133.65, 24.72) --
	(133.96, 25.46) --
	(134.26, 26.21) --
	(134.54, 26.97) --
	(134.81, 27.73) --
	(135.07, 28.50) --
	(135.31, 29.27) --
	(135.54, 30.04) --
	(135.76, 30.82) --
	(135.96, 31.61) --
	(136.15, 32.39) --
	(136.33, 33.18) --
	(136.49, 33.97) --
	(136.64, 34.77) --
	(136.77, 35.57) --
	(136.89, 36.37) --
	(137.00, 37.17) --
	(137.09, 37.97) --
	(137.17, 38.78) --
	(137.23, 39.58) --
	(137.28, 40.39) --
	(137.31, 41.20) --
	(137.34, 42.00) --
	(137.34, 42.81);
\definecolor[named]{drawColor}{rgb}{0.00,0.00,0.00}

\path[draw=drawColor,line width= 0.4pt,dash pattern=on 1pt off 3pt ,line join=round,line cap=round] (141.96, 42.81) --
	(141.95, 43.70) --
	(141.93, 44.59) --
	(141.89, 45.48) --
	(141.84, 46.37) --
	(141.77, 47.25) --
	(141.68, 48.14) --
	(141.58, 49.02) --
	(141.47, 49.90) --
	(141.33, 50.78) --
	(141.19, 51.66) --
	(141.02, 52.54) --
	(140.85, 53.41) --
	(140.65, 54.27) --
	(140.44, 55.14) --
	(140.22, 56.00) --
	(139.98, 56.86) --
	(139.73, 57.71) --
	(139.46, 58.56) --
	(139.18, 59.40) --
	(138.88, 60.24) --
	(138.57, 61.07) --
	(138.24, 61.90) --
	(137.90, 62.72) --
	(137.55, 63.53) --
	(137.18, 64.34) --
	(136.79, 65.15) --
	(136.39, 65.94) --
	(135.98, 66.73) --
	(135.56, 67.51) --
	(135.12, 68.28) --
	(134.67, 69.05) --
	(134.20, 69.81) --
	(133.72, 70.56) --
	(133.23, 71.30) --
	(132.72, 72.03) --
	(132.21, 72.75) --
	(131.68, 73.47) --
	(131.13, 74.17) --
	(130.58, 74.87) --
	(130.01, 75.55) --
	(129.43, 76.23) --
	(128.84, 76.89) --
	(128.24, 77.55) --
	(127.63, 78.19) --
	(127.00, 78.82) --
	(126.37, 79.44) --
	(125.72, 80.05) --
	(125.06, 80.65) --
	(124.39, 81.24) --
	(123.72, 81.82) --
	(123.03, 82.38) --
	(122.33, 82.93) --
	(121.62, 83.47) --
	(120.91, 84.00) --
	(120.18, 84.52) --
	(119.45, 85.02) --
	(118.71, 85.51) --
	(117.95, 85.98) --
	(117.19, 86.44) --
	(116.43, 86.89) --
	(115.65, 87.33) --
	(114.87, 87.75) --
	(114.08, 88.16) --
	(113.28, 88.55) --
	(112.48, 88.93) --
	(111.67, 89.30) --
	(110.85, 89.65) --
	(110.03, 89.99) --
	(109.20, 90.31) --
	(108.37, 90.62) --
	(107.53, 90.92) --
	(106.68, 91.19) --
	(105.83, 91.46) --
	(104.98, 91.71) --
	(104.12, 91.94) --
	(103.26, 92.16) --
	(102.39, 92.37) --
	(101.52, 92.56) --
	(100.65, 92.73) --
	( 99.78, 92.89) --
	( 98.90, 93.03) --
	( 98.02, 93.16) --
	( 97.14, 93.27) --
	( 96.25, 93.37) --
	( 95.37, 93.45) --
	( 94.48, 93.52) --
	( 93.59, 93.57) --
	( 92.70, 93.60) --
	( 91.82, 93.62) --
	( 90.93, 93.63) --
	( 90.04, 93.61) --
	( 89.15, 93.59) --
	( 88.26, 93.54) --
	( 87.37, 93.49) --
	( 86.49, 93.41) --
	( 85.60, 93.32) --
	( 84.72, 93.22) --
	( 83.84, 93.10) --
	( 82.96, 92.96) --
	( 82.08, 92.81) --
	( 81.21, 92.65) --
	( 80.34, 92.46) --
	( 79.47, 92.27) --
	( 78.61, 92.05) --
	( 77.75, 91.83) --
	( 76.89, 91.59) --
	( 76.04, 91.33) --
	( 75.19, 91.06) --
	( 74.35, 90.77) --
	( 73.51, 90.47) --
	( 72.68, 90.15) --
	( 71.86, 89.82) --
	( 71.04, 89.48) --
	( 70.22, 89.12) --
	( 69.42, 88.75) --
	( 68.62, 88.36) --
	( 67.82, 87.96) --
	( 67.04, 87.54) --
	( 66.26, 87.11) --
	( 65.48, 86.67) --
	( 64.72, 86.21) --
	( 63.97, 85.75) --
	( 63.22, 85.26) --
	( 62.48, 84.77) --
	( 61.75, 84.26) --
	( 61.03, 83.74) --
	( 60.32, 83.21) --
	( 59.62, 82.66) --
	( 58.92, 82.10) --
	( 58.24, 81.53) --
	( 57.57, 80.95) --
	( 56.91, 80.36) --
	( 56.25, 79.75) --
	( 55.61, 79.13) --
	( 54.98, 78.51) --
	( 54.36, 77.87) --
	( 53.76, 77.22) --
	( 53.16, 76.56) --
	( 52.57, 75.89) --
	( 52.00, 75.21) --
	( 51.44, 74.52) --
	( 50.89, 73.82) --
	( 50.35, 73.11) --
	( 49.83, 72.39) --
	( 49.32, 71.66) --
	( 48.82, 70.93) --
	( 48.34, 70.18) --
	( 47.86, 69.43) --
	( 47.40, 68.67) --
	( 46.96, 67.90) --
	( 46.53, 67.12) --
	( 46.11, 66.34) --
	( 45.70, 65.54) --
	( 45.31, 64.75) --
	( 44.93, 63.94) --
	( 44.57, 63.13) --
	( 44.22, 62.31) --
	( 43.89, 61.49) --
	( 43.57, 60.66) --
	( 43.26, 59.82) --
	( 42.97, 58.98) --
	( 42.70, 58.13) --
	( 42.44, 57.28) --
	( 42.19, 56.43) --
	( 41.96, 55.57) --
	( 41.75, 54.71) --
	( 41.55, 53.84) --
	( 41.36, 52.97) --
	( 41.19, 52.10) --
	( 41.03, 51.22) --
	( 40.90, 50.34) --
	( 40.77, 49.46) --
	( 40.66, 48.58) --
	( 40.57, 47.70) --
	( 40.49, 46.81) --
	( 40.43, 45.92) --
	( 40.38, 45.04) --
	( 40.35, 44.15) --
	( 40.34, 43.26) --
	( 40.34, 42.37) --
	( 40.35, 41.48) --
	( 40.38, 40.59) --
	( 40.43, 39.70) --
	( 40.49, 38.81) --
	( 40.57, 37.93) --
	( 40.66, 37.04) --
	( 40.77, 36.16) --
	( 40.90, 35.28) --
	( 41.03, 34.40) --
	( 41.19, 33.53) --
	( 41.36, 32.65) --
	( 41.55, 31.78) --
	( 41.75, 30.92) --
	( 41.96, 30.05) --
	( 42.19, 29.20) --
	( 42.44, 28.34) --
	( 42.70, 27.49) --
	( 42.97, 26.65) --
	( 43.26, 25.80) --
	( 43.57, 24.97) --
	( 43.89, 24.14) --
	( 44.22, 23.32) --
	( 44.57, 22.50) --
	( 44.93, 21.69) --
	( 45.31, 20.88) --
	( 45.70, 20.08) --
	( 46.11, 19.29) --
	( 46.53, 18.50) --
	( 46.96, 17.73) --
	( 47.40, 16.96) --
	( 47.86, 16.20) --
	( 48.34, 15.44) --
	( 48.82, 14.70) --
	( 49.32, 13.96) --
	( 49.83, 13.23) --
	( 50.35, 12.51) --
	( 50.89, 11.81) --
	( 51.44, 11.11) --
	( 52.00, 10.42) --
	( 52.57,  9.74) --
	( 53.16,  9.07) --
	( 53.76,  8.41) --
	( 54.36,  7.76) --
	( 54.98,  7.12) --
	( 55.61,  6.49) --
	( 56.25,  5.87) --
	( 56.91,  5.27) --
	( 57.57,  4.68) --
	( 58.24,  4.09) --
	( 58.92,  3.52) --
	( 59.62,  2.97) --
	( 60.32,  2.42) --
	( 61.03,  1.89) --
	( 61.75,  1.37) --
	( 62.48,  0.86) --
	( 63.22,  0.36) --
	( 63.78,  0.00);

\path[draw=drawColor,line width= 0.4pt,dash pattern=on 1pt off 3pt ,line join=round,line cap=round] (118.52,  0.00) --
	(118.71,  0.12) --
	(119.45,  0.61) --
	(120.18,  1.11) --
	(120.91,  1.62) --
	(121.62,  2.15) --
	(122.33,  2.69) --
	(123.03,  3.24) --
	(123.72,  3.81) --
	(124.39,  4.38) --
	(125.06,  4.97) --
	(125.72,  5.57) --
	(126.37,  6.18) --
	(127.00,  6.80) --
	(127.63,  7.44) --
	(128.24,  8.08) --
	(128.84,  8.73) --
	(129.43,  9.40) --
	(130.01, 10.07) --
	(130.58, 10.76) --
	(131.13, 11.45) --
	(131.68, 12.16) --
	(132.21, 12.87) --
	(132.72, 13.60) --
	(133.23, 14.33) --
	(133.72, 15.07) --
	(134.20, 15.82) --
	(134.67, 16.58) --
	(135.12, 17.34) --
	(135.56, 18.11) --
	(135.98, 18.90) --
	(136.39, 19.68) --
	(136.79, 20.48) --
	(137.18, 21.28) --
	(137.55, 22.09) --
	(137.90, 22.91) --
	(138.24, 23.73) --
	(138.57, 24.55) --
	(138.88, 25.39) --
	(139.18, 26.22) --
	(139.46, 27.07) --
	(139.73, 27.92) --
	(139.98, 28.77) --
	(140.22, 29.62) --
	(140.44, 30.49) --
	(140.65, 31.35) --
	(140.85, 32.22) --
	(141.02, 33.09) --
	(141.19, 33.96) --
	(141.33, 34.84) --
	(141.47, 35.72) --
	(141.58, 36.60) --
	(141.68, 37.49) --
	(141.77, 38.37) --
	(141.84, 39.26) --
	(141.89, 40.15) --
	(141.93, 41.03) --
	(141.95, 41.92) --
	(141.96, 42.81);

\path[fill=fillColor] ( 68.73,126.47) circle (  2.25);
\definecolor[named]{drawColor}{rgb}{0.00,0.00,1.00}

\path[draw=drawColor,line width= 1.2pt,line join=round,line cap=round] (114.93,126.47) --
	(114.92,127.28) --
	(114.90,128.09) --
	(114.86,128.90) --
	(114.81,129.70) --
	(114.75,130.51) --
	(114.67,131.31) --
	(114.58,132.12) --
	(114.47,132.92) --
	(114.35,133.72) --
	(114.22,134.52) --
	(114.07,135.31) --
	(113.91,136.10) --
	(113.74,136.89) --
	(113.55,137.68) --
	(113.34,138.46) --
	(113.13,139.24) --
	(112.90,140.02) --
	(112.65,140.79) --
	(112.40,141.55) --
	(112.13,142.31) --
	(111.84,143.07) --
	(111.54,143.82) --
	(111.23,144.57) --
	(110.91,145.31) --
	(110.57,146.05) --
	(110.23,146.78) --
	(109.86,147.50) --
	(109.49,148.22) --
	(109.10,148.93) --
	(108.70,149.63) --
	(108.29,150.32) --
	(107.87,151.01) --
	(107.43,151.69) --
	(106.99,152.37) --
	(106.53,153.03) --
	(106.06,153.69) --
	(105.57,154.34) --
	(105.08,154.98) --
	(104.58,155.61) --
	(104.06,156.23) --
	(103.53,156.85) --
	(103.00,157.45) --
	(102.45,158.05) --
	(101.89,158.63) --
	(101.32,159.21) --
	(100.75,159.77) --
	(100.16,160.33) --
	( 99.56,160.87) --
	( 98.96,161.41) --
	( 98.34,161.93) --
	( 97.71,162.44) --
	( 97.08,162.95) --
	( 96.44,163.44) --
	( 95.79,163.92) --
	( 95.13,164.38) --
	( 94.46,164.84) --
	( 93.78,165.28) --
	( 93.10,165.72) --
	( 92.41,166.14) --
	( 91.71,166.55) --
	( 91.01,166.94) --
	( 90.30,167.33) --
	( 89.58,167.70) --
	( 88.85,168.06) --
	( 88.12,168.40) --
	( 87.39,168.73) --
	( 86.64,169.05) --
	( 85.89,169.36) --
	( 85.14,169.65) --
	( 84.38,169.94) --
	( 83.62,170.20) --
	( 82.85,170.46) --
	( 82.08,170.70) --
	( 81.30,170.92) --
	( 80.53,171.14) --
	( 79.74,171.34) --
	( 78.95,171.52) --
	( 78.16,171.69) --
	( 77.37,171.85) --
	( 76.58,172.00) --
	( 75.78,172.13) --
	( 74.98,172.24) --
	( 74.18,172.35) --
	( 73.37,172.43) --
	( 72.57,172.51) --
	( 71.76,172.57) --
	( 70.95,172.61) --
	( 70.15,172.65) --
	( 69.34,172.66) --
	( 68.53,172.67) --
	( 67.72,172.66) --
	( 66.91,172.63) --
	( 66.11,172.59) --
	( 65.30,172.54) --
	( 64.49,172.47) --
	( 63.69,172.39) --
	( 62.89,172.30) --
	( 62.08,172.19) --
	( 61.29,172.06) --
	( 60.49,171.93) --
	( 59.69,171.77) --
	( 58.90,171.61) --
	( 58.11,171.43) --
	( 57.33,171.24) --
	( 56.55,171.03) --
	( 55.77,170.81) --
	( 55.00,170.58) --
	( 54.23,170.33) --
	( 53.46,170.07) --
	( 52.70,169.80) --
	( 51.94,169.51) --
	( 51.19,169.21) --
	( 50.45,168.90) --
	( 49.71,168.57) --
	( 48.98,168.23) --
	( 48.25,167.88) --
	( 47.53,167.51) --
	( 46.81,167.14) --
	( 46.10,166.75) --
	( 45.40,166.34) --
	( 44.71,165.93) --
	( 44.02,165.50) --
	( 43.34,165.06) --
	( 42.67,164.61) --
	( 42.01,164.15) --
	( 41.35,163.68) --
	( 40.70,163.19) --
	( 40.07,162.70) --
	( 39.44,162.19) --
	( 38.82,161.67) --
	( 38.20,161.14) --
	( 37.60,160.60) --
	( 37.01,160.05) --
	( 36.43,159.49) --
	( 35.85,158.92) --
	( 35.29,158.34) --
	( 34.74,157.75) --
	( 34.20,157.15) --
	( 33.66,156.54) --
	( 33.14,155.92) --
	( 32.63,155.30) --
	( 32.13,154.66) --
	( 31.65,154.02) --
	( 31.17,153.36) --
	( 30.71,152.70) --
	( 30.25,152.03) --
	( 29.81,151.35) --
	( 29.38,150.67) --
	( 28.96,149.98) --
	( 28.56,149.28) --
	( 28.17,148.57) --
	( 27.78,147.86) --
	( 27.42,147.14) --
	( 27.06,146.41) --
	( 26.72,145.68) --
	( 26.39,144.94) --
	( 26.07,144.20) --
	( 25.77,143.45) --
	( 25.48,142.69) --
	( 25.20,141.93) --
	( 24.94,141.17) --
	( 24.69,140.40) --
	( 24.45,139.63) --
	( 24.23,138.85) --
	( 24.02,138.07) --
	( 23.82,137.29) --
	( 23.64,136.50) --
	( 23.47,135.71) --
	( 23.31,134.91) --
	( 23.17,134.12) --
	( 23.05,133.32) --
	( 22.93,132.52) --
	( 22.84,131.72) --
	( 22.75,130.91) --
	( 22.68,130.11) --
	( 22.62,129.30) --
	( 22.58,128.49) --
	( 22.55,127.69) --
	( 22.54,126.88) --
	( 22.54,126.07) --
	( 22.55,125.26) --
	( 22.58,124.45) --
	( 22.62,123.64) --
	( 22.68,122.84) --
	( 22.75,122.03) --
	( 22.84,121.23) --
	( 22.93,120.43) --
	( 23.05,119.63) --
	( 23.17,118.83) --
	( 23.31,118.03) --
	( 23.47,117.24) --
	( 23.64,116.45) --
	( 23.82,115.66) --
	( 24.02,114.87) --
	( 24.23,114.09) --
	( 24.45,113.32) --
	( 24.69,112.54) --
	( 24.94,111.78) --
	( 25.20,111.01) --
	( 25.48,110.25) --
	( 25.77,109.50) --
	( 26.07,108.75) --
	( 26.39,108.00) --
	( 26.72,107.27) --
	( 27.06,106.53) --
	( 27.42,105.81) --
	( 27.78,105.09) --
	( 28.17,104.37) --
	( 28.56,103.67) --
	( 28.96,102.97) --
	( 29.38,102.28) --
	( 29.81,101.59) --
	( 30.25,100.91) --
	( 30.71,100.24) --
	( 31.17, 99.58) --
	( 31.65, 98.93) --
	( 32.13, 98.28) --
	( 32.63, 97.65) --
	( 33.14, 97.02) --
	( 33.66, 96.40) --
	( 34.20, 95.79) --
	( 34.74, 95.19) --
	( 35.29, 94.60) --
	( 35.85, 94.02) --
	( 36.43, 93.45) --
	( 37.01, 92.89) --
	( 37.60, 92.34) --
	( 38.20, 91.80) --
	( 38.82, 91.27) --
	( 39.44, 90.76) --
	( 40.07, 90.25) --
	( 40.70, 89.75) --
	( 41.35, 89.27) --
	( 42.01, 88.79) --
	( 42.67, 88.33) --
	( 43.34, 87.88) --
	( 44.02, 87.44) --
	( 44.71, 87.02) --
	( 45.40, 86.60) --
	( 46.10, 86.20) --
	( 46.81, 85.81) --
	( 47.53, 85.43) --
	( 48.25, 85.07) --
	( 48.98, 84.72) --
	( 49.71, 84.38) --
	( 50.45, 84.05) --
	( 51.19, 83.74) --
	( 51.94, 83.44) --
	( 52.70, 83.15) --
	( 53.46, 82.87) --
	( 54.23, 82.61) --
	( 55.00, 82.37) --
	( 55.77, 82.13) --
	( 56.55, 81.91) --
	( 57.33, 81.71) --
	( 58.11, 81.51) --
	( 58.90, 81.34) --
	( 59.69, 81.17) --
	( 60.49, 81.02) --
	( 61.29, 80.88) --
	( 62.08, 80.76) --
	( 62.89, 80.65) --
	( 63.69, 80.55) --
	( 64.49, 80.47) --
	( 65.30, 80.41) --
	( 66.11, 80.35) --
	( 66.91, 80.31) --
	( 67.72, 80.29) --
	( 68.53, 80.28) --
	( 69.34, 80.28) --
	( 70.15, 80.30) --
	( 70.95, 80.33) --
	( 71.76, 80.38) --
	( 72.57, 80.44) --
	( 73.37, 80.51) --
	( 74.18, 80.60) --
	( 74.98, 80.70) --
	( 75.78, 80.82) --
	( 76.58, 80.95) --
	( 77.37, 81.09) --
	( 78.16, 81.25) --
	( 78.95, 81.42) --
	( 79.74, 81.61) --
	( 80.53, 81.81) --
	( 81.30, 82.02) --
	( 82.08, 82.25) --
	( 82.85, 82.49) --
	( 83.62, 82.74) --
	( 84.38, 83.01) --
	( 85.14, 83.29) --
	( 85.89, 83.58) --
	( 86.64, 83.89) --
	( 87.39, 84.21) --
	( 88.12, 84.54) --
	( 88.85, 84.89) --
	( 89.58, 85.25) --
	( 90.30, 85.62) --
	( 91.01, 86.00) --
	( 91.71, 86.40) --
	( 92.41, 86.81) --
	( 93.10, 87.23) --
	( 93.78, 87.66) --
	( 94.46, 88.10) --
	( 95.13, 88.56) --
	( 95.79, 89.03) --
	( 96.44, 89.51) --
	( 97.08, 90.00) --
	( 97.71, 90.50) --
	( 98.34, 91.01) --
	( 98.96, 91.54) --
	( 99.56, 92.07) --
	(100.16, 92.62) --
	(100.75, 93.17) --
	(101.32, 93.74) --
	(101.89, 94.31) --
	(102.45, 94.90) --
	(103.00, 95.49) --
	(103.53, 96.10) --
	(104.06, 96.71) --
	(104.58, 97.33) --
	(105.08, 97.96) --
	(105.57, 98.61) --
	(106.06, 99.25) --
	(106.53, 99.91) --
	(106.99,100.58) --
	(107.43,101.25) --
	(107.87,101.93) --
	(108.29,102.62) --
	(108.70,103.32) --
	(109.10,104.02) --
	(109.49,104.73) --
	(109.86,105.45) --
	(110.23,106.17) --
	(110.57,106.90) --
	(110.91,107.63) --
	(111.23,108.38) --
	(111.54,109.12) --
	(111.84,109.87) --
	(112.13,110.63) --
	(112.40,111.39) --
	(112.65,112.16) --
	(112.90,112.93) --
	(113.13,113.70) --
	(113.34,114.48) --
	(113.55,115.27) --
	(113.74,116.05) --
	(113.91,116.84) --
	(114.07,117.63) --
	(114.22,118.43) --
	(114.35,119.23) --
	(114.47,120.03) --
	(114.58,120.83) --
	(114.67,121.63) --
	(114.75,122.44) --
	(114.81,123.24) --
	(114.86,124.05) --
	(114.90,124.86) --
	(114.92,125.66) --
	(114.93,126.47);
\definecolor[named]{drawColor}{rgb}{0.00,0.00,0.00}

\path[draw=drawColor,line width= 0.4pt,dash pattern=on 1pt off 3pt ,line join=round,line cap=round] (119.55,126.47) --
	(119.54,127.36) --
	(119.52,128.25) --
	(119.48,129.14) --
	(119.42,130.03) --
	(119.35,130.91) --
	(119.27,131.80) --
	(119.17,132.68) --
	(119.05,133.56) --
	(118.92,134.44) --
	(118.77,135.32) --
	(118.61,136.20) --
	(118.43,137.07) --
	(118.24,137.93) --
	(118.03,138.80) --
	(117.81,139.66) --
	(117.57,140.52) --
	(117.31,141.37) --
	(117.05,142.22) --
	(116.76,143.06) --
	(116.46,143.90) --
	(116.15,144.73) --
	(115.83,145.56) --
	(115.48,146.38) --
	(115.13,147.19) --
	(114.76,148.00) --
	(114.38,148.81) --
	(113.98,149.60) --
	(113.57,150.39) --
	(113.14,151.17) --
	(112.70,151.94) --
	(112.25,152.71) --
	(111.78,153.47) --
	(111.30,154.22) --
	(110.81,154.96) --
	(110.31,155.69) --
	(109.79,156.41) --
	(109.26,157.13) --
	(108.72,157.83) --
	(108.16,158.53) --
	(107.59,159.21) --
	(107.02,159.89) --
	(106.42,160.55) --
	(105.82,161.21) --
	(105.21,161.85) --
	(104.58,162.48) --
	(103.95,163.10) --
	(103.30,163.71) --
	(102.64,164.31) --
	(101.98,164.90) --
	(101.30,165.48) --
	(100.61,166.04) --
	( 99.91,166.59) --
	( 99.21,167.13) --
	( 98.49,167.66) --
	( 97.77,168.18) --
	( 97.03,168.68) --
	( 96.29,169.17) --
	( 95.54,169.64) --
	( 94.78,170.10) --
	( 94.01,170.55) --
	( 93.24,170.99) --
	( 92.45,171.41) --
	( 91.66,171.82) --
	( 90.87,172.21) --
	( 90.06,172.59) --
	( 89.25,172.96) --
	( 88.43,173.31) --
	( 87.61,173.65) --
	( 86.78,173.97) --
	( 85.95,174.28) --
	( 85.11,174.58) --
	( 84.26,174.85) --
	( 83.42,175.12) --
	( 82.56,175.37) --
	( 81.70,175.60) --
	( 80.84,175.82) --
	( 79.98,176.03) --
	( 79.11,176.22) --
	( 78.24,176.39) --
	( 77.36,176.55) --
	( 76.48,176.69) --
	( 75.60,176.82) --
	( 74.72,176.93) --
	( 73.84,177.03) --
	( 72.95,177.11) --
	( 72.06,177.18) --
	( 71.18,177.23) --
	( 70.29,177.26) --
	( 69.40,177.28) --
	( 68.51,177.29) --
	( 67.62,177.27) --
	( 66.73,177.25) --
	( 65.84,177.20) --
	( 64.96,177.15) --
	( 64.07,177.07) --
	( 63.18,176.98) --
	( 62.30,176.88) --
	( 61.42,176.76) --
	( 60.54,176.62) --
	( 59.66,176.47) --
	( 58.79,176.31) --
	( 57.92,176.12) --
	( 57.05,175.93) --
	( 56.19,175.71) --
	( 55.33,175.49) --
	( 54.47,175.25) --
	( 53.62,174.99) --
	( 52.78,174.72) --
	( 51.93,174.43) --
	( 51.10,174.13) --
	( 50.27,173.81) --
	( 49.44,173.48) --
	( 48.62,173.14) --
	( 47.81,172.78) --
	( 47.00,172.41) --
	( 46.20,172.02) --
	( 45.41,171.62) --
	( 44.62,171.20) --
	( 43.84,170.77) --
	( 43.07,170.33) --
	( 42.30,169.87) --
	( 41.55,169.41) --
	( 40.80,168.92) --
	( 40.06,168.43) --
	( 39.33,167.92) --
	( 38.61,167.40) --
	( 37.90,166.87) --
	( 37.20,166.32) --
	( 36.51,165.76) --
	( 35.82,165.19) --
	( 35.15,164.61) --
	( 34.49,164.02) --
	( 33.84,163.41) --
	( 33.20,162.79) --
	( 32.57,162.17) --
	( 31.95,161.53) --
	( 31.34,160.88) --
	( 30.74,160.22) --
	( 30.16,159.55) --
	( 29.58,158.87) --
	( 29.02,158.18) --
	( 28.47,157.48) --
	( 27.94,156.77) --
	( 27.41,156.05) --
	( 26.90,155.32) --
	( 26.40,154.59) --
	( 25.92,153.84) --
	( 25.45,153.09) --
	( 24.99,152.33) --
	( 24.54,151.56) --
	( 24.11,150.78) --
	( 23.69,150.00) --
	( 23.29,149.20) --
	( 22.89,148.41) --
	( 22.52,147.60) --
	( 22.15,146.79) --
	( 21.81,145.97) --
	( 21.47,145.15) --
	( 21.15,144.32) --
	( 20.85,143.48) --
	( 20.56,142.64) --
	( 20.28,141.79) --
	( 20.02,140.94) --
	( 19.78,140.09) --
	( 19.54,139.23) --
	( 19.33,138.37) --
	( 19.13,137.50) --
	( 18.94,136.63) --
	( 18.77,135.76) --
	( 18.62,134.88) --
	( 18.48,134.00) --
	( 18.35,133.12) --
	( 18.25,132.24) --
	( 18.15,131.36) --
	( 18.07,130.47) --
	( 18.01,129.58) --
	( 17.97,128.70) --
	( 17.93,127.81) --
	( 17.92,126.92) --
	( 17.92,126.03) --
	( 17.93,125.14) --
	( 17.97,124.25) --
	( 18.01,123.36) --
	( 18.07,122.47) --
	( 18.15,121.59) --
	( 18.25,120.70) --
	( 18.35,119.82) --
	( 18.48,118.94) --
	( 18.62,118.06) --
	( 18.77,117.19) --
	( 18.94,116.31) --
	( 19.13,115.44) --
	( 19.33,114.58) --
	( 19.54,113.71) --
	( 19.78,112.86) --
	( 20.02,112.00) --
	( 20.28,111.15) --
	( 20.56,110.31) --
	( 20.85,109.46) --
	( 21.15,108.63) --
	( 21.47,107.80) --
	( 21.81,106.98) --
	( 22.15,106.16) --
	( 22.52,105.35) --
	( 22.89,104.54) --
	( 23.29,103.74) --
	( 23.69,102.95) --
	( 24.11,102.16) --
	( 24.54,101.39) --
	( 24.99,100.62) --
	( 25.45, 99.86) --
	( 25.92, 99.10) --
	( 26.40, 98.36) --
	( 26.90, 97.62) --
	( 27.41, 96.89) --
	( 27.94, 96.17) --
	( 28.47, 95.47) --
	( 29.02, 94.77) --
	( 29.58, 94.08) --
	( 30.16, 93.40) --
	( 30.74, 92.73) --
	( 31.34, 92.07) --
	( 31.95, 91.42) --
	( 32.57, 90.78) --
	( 33.20, 90.15) --
	( 33.84, 89.53) --
	( 34.49, 88.93) --
	( 35.15, 88.34) --
	( 35.82, 87.75) --
	( 36.51, 87.18) --
	( 37.20, 86.63) --
	( 37.90, 86.08) --
	( 38.61, 85.55) --
	( 39.33, 85.03) --
	( 40.06, 84.52) --
	( 40.80, 84.02) --
	( 41.55, 83.54) --
	( 42.30, 83.07) --
	( 43.07, 82.61) --
	( 43.84, 82.17) --
	( 44.62, 81.74) --
	( 45.41, 81.33) --
	( 46.20, 80.93) --
	( 47.00, 80.54) --
	( 47.81, 80.17) --
	( 48.62, 79.81) --
	( 49.44, 79.46) --
	( 50.27, 79.13) --
	( 51.10, 78.82) --
	( 51.93, 78.51) --
	( 52.78, 78.23) --
	( 53.62, 77.96) --
	( 54.47, 77.70) --
	( 55.33, 77.46) --
	( 56.19, 77.23) --
	( 57.05, 77.02) --
	( 57.92, 76.82) --
	( 58.79, 76.64) --
	( 59.66, 76.47) --
	( 60.54, 76.32) --
	( 61.42, 76.19) --
	( 62.30, 76.07) --
	( 63.18, 75.96) --
	( 64.07, 75.87) --
	( 64.96, 75.80) --
	( 65.84, 75.74) --
	( 66.73, 75.70) --
	( 67.62, 75.67) --
	( 68.51, 75.66) --
	( 69.40, 75.66) --
	( 70.29, 75.68) --
	( 71.18, 75.72) --
	( 72.06, 75.77) --
	( 72.95, 75.83) --
	( 73.84, 75.92) --
	( 74.72, 76.01) --
	( 75.60, 76.12) --
	( 76.48, 76.25) --
	( 77.36, 76.40) --
	( 78.24, 76.55) --
	( 79.11, 76.73) --
	( 79.98, 76.92) --
	( 80.84, 77.12) --
	( 81.70, 77.34) --
	( 82.56, 77.58) --
	( 83.42, 77.83) --
	( 84.26, 78.09) --
	( 85.11, 78.37) --
	( 85.95, 78.66) --
	( 86.78, 78.97) --
	( 87.61, 79.30) --
	( 88.43, 79.63) --
	( 89.25, 79.99) --
	( 90.06, 80.35) --
	( 90.87, 80.73) --
	( 91.66, 81.13) --
	( 92.45, 81.53) --
	( 93.24, 81.96) --
	( 94.01, 82.39) --
	( 94.78, 82.84) --
	( 95.54, 83.30) --
	( 96.29, 83.78) --
	( 97.03, 84.27) --
	( 97.77, 84.77) --
	( 98.49, 85.28) --
	( 99.21, 85.81) --
	( 99.91, 86.35) --
	(100.61, 86.90) --
	(101.30, 87.47) --
	(101.98, 88.04) --
	(102.64, 88.63) --
	(103.30, 89.23) --
	(103.95, 89.84) --
	(104.58, 90.46) --
	(105.21, 91.10) --
	(105.82, 91.74) --
	(106.42, 92.39) --
	(107.02, 93.06) --
	(107.59, 93.73) --
	(108.16, 94.42) --
	(108.72, 95.11) --
	(109.26, 95.82) --
	(109.79, 96.53) --
	(110.31, 97.26) --
	(110.81, 97.99) --
	(111.30, 98.73) --
	(111.78, 99.48) --
	(112.25,100.24) --
	(112.70,101.00) --
	(113.14,101.77) --
	(113.57,102.56) --
	(113.98,103.34) --
	(114.38,104.14) --
	(114.76,104.94) --
	(115.13,105.75) --
	(115.48,106.57) --
	(115.83,107.39) --
	(116.15,108.21) --
	(116.46,109.05) --
	(116.76,109.88) --
	(117.05,110.73) --
	(117.31,111.58) --
	(117.57,112.43) --
	(117.81,113.28) --
	(118.03,114.15) --
	(118.24,115.01) --
	(118.43,115.88) --
	(118.61,116.75) --
	(118.77,117.62) --
	(118.92,118.50) --
	(119.05,119.38) --
	(119.17,120.26) --
	(119.27,121.15) --
	(119.35,122.03) --
	(119.42,122.92) --
	(119.48,123.81) --
	(119.52,124.69) --
	(119.54,125.58) --
	(119.55,126.47);

\path[fill=fillColor] ( 46.32,210.13) circle (  2.25);
\definecolor[named]{drawColor}{rgb}{0.00,0.00,1.00}

\path[draw=drawColor,line width= 1.2pt,line join=round,line cap=round] ( 92.51,210.13) --
	( 92.50,210.94) --
	( 92.48,211.75) --
	( 92.45,212.56) --
	( 92.40,213.36) --
	( 92.33,214.17) --
	( 92.26,214.97) --
	( 92.16,215.78) --
	( 92.06,216.58) --
	( 91.94,217.38) --
	( 91.80,218.18) --
	( 91.66,218.97) --
	( 91.49,219.76) --
	( 91.32,220.55) --
	( 91.13,221.34) --
	( 90.93,222.12) --
	( 90.71,222.90) --
	( 90.48,223.68) --
	( 90.24,224.45) --
	( 89.98,225.21) --
	( 89.71,225.97) --
	( 89.42,226.73) --
	( 89.13,227.48) --
	( 88.82,228.23) --
	( 88.49,228.97) --
	( 88.16,229.71) --
	( 87.81,230.44) --
	( 87.45,231.16) --
	( 87.07,231.88) --
	( 86.69,232.59) --
	( 86.29,233.29) --
	( 85.88,233.98) --
	( 85.45,234.67) --
	( 85.02,235.35) --
	( 84.57,236.03) --
	( 84.11,236.69) --
	( 83.64,237.35) --
	( 83.16,238.00) --
	( 82.66,238.64) --
	( 82.16,239.27) --
	( 81.64,239.89) --
	( 81.12,240.51) --
	( 80.58,241.11) --
	( 80.03,241.71) --
	( 79.48,242.29) --
	( 78.91,242.87) --
	( 78.33,243.43) --
	( 77.74,243.99) --
	( 77.15,244.53) --
	( 76.54,245.07) --
	( 75.92,245.59) --
	( 75.30,246.10) --
	( 74.66,246.61) --
	( 74.02,247.10) --
	( 73.37,247.58) --
	( 72.71,248.04) --
	( 72.04,248.50) --
	( 71.37,248.94) --
	( 70.68,249.38) --
	( 69.99,249.80) --
	( 69.30,250.21) --
	( 68.59,250.60) --
	( 67.88,250.99) --
	( 67.16,251.36) --
	( 66.44,251.72) --
	( 65.71,252.06) --
	( 64.97,252.39) --
	( 64.23,252.71) --
	( 63.66,252.94);

\path[draw=drawColor,line width= 1.2pt,line join=round,line cap=round] ( 28.97,252.94) --
	( 28.78,252.87) --
	( 28.03,252.56) --
	( 27.29,252.23) --
	( 26.56,251.89) --
	( 25.83,251.54) --
	( 25.11,251.17) --
	( 24.39,250.80) --
	( 23.69,250.41) --
	( 22.98,250.00) --
	( 22.29,249.59) --
	( 21.60,249.16) --
	( 20.92,248.72) --
	( 20.25,248.27) --
	( 19.59,247.81) --
	( 18.93,247.34) --
	( 18.29,246.85) --
	( 17.65,246.36) --
	( 17.02,245.85) --
	( 16.40,245.33) --
	( 15.79,244.80) --
	( 15.18,244.26) --
	( 14.59,243.71) --
	( 14.01,243.15) --
	( 13.44,242.58) --
	( 12.87,242.00) --
	( 12.32,241.41) --
	( 11.78,240.81) --
	( 11.25,240.20) --
	( 10.73,239.58) --
	( 10.22,238.96) --
	(  9.72,238.32) --
	(  9.23,237.68) --
	(  8.75,237.02) --
	(  8.29,236.36) --
	(  7.84,235.69) --
	(  7.39,235.01) --
	(  6.96,234.33) --
	(  6.55,233.64) --
	(  6.14,232.94) --
	(  5.75,232.23) --
	(  5.37,231.52) --
	(  5.00,230.80) --
	(  4.64,230.07) --
	(  4.30,229.34) --
	(  3.97,228.60) --
	(  3.66,227.86) --
	(  3.35,227.11) --
	(  3.06,226.35) --
	(  2.78,225.59) --
	(  2.52,224.83) --
	(  2.27,224.06) --
	(  2.03,223.29) --
	(  1.81,222.51) --
	(  1.60,221.73) --
	(  1.40,220.95) --
	(  1.22,220.16) --
	(  1.05,219.37) --
	(  0.90,218.57) --
	(  0.76,217.78) --
	(  0.63,216.98) --
	(  0.52,216.18) --
	(  0.42,215.38) --
	(  0.33,214.57) --
	(  0.26,213.77) --
	(  0.21,212.96) --
	(  0.16,212.15) --
	(  0.14,211.35) --
	(  0.12,210.54) --
	(  0.12,209.73) --
	(  0.14,208.92) --
	(  0.16,208.11) --
	(  0.21,207.30) --
	(  0.26,206.50) --
	(  0.33,205.69) --
	(  0.42,204.89) --
	(  0.52,204.09) --
	(  0.63,203.29) --
	(  0.76,202.49) --
	(  0.90,201.69) --
	(  1.05,200.90) --
	(  1.22,200.11) --
	(  1.40,199.32) --
	(  1.60,198.53) --
	(  1.81,197.75) --
	(  2.03,196.98) --
	(  2.27,196.20) --
	(  2.52,195.44) --
	(  2.78,194.67) --
	(  3.06,193.91) --
	(  3.35,193.16) --
	(  3.66,192.41) --
	(  3.97,191.66) --
	(  4.30,190.93) --
	(  4.64,190.19) --
	(  5.00,189.47) --
	(  5.37,188.75) --
	(  5.75,188.03) --
	(  6.14,187.33) --
	(  6.55,186.63) --
	(  6.96,185.94) --
	(  7.39,185.25) --
	(  7.84,184.57) --
	(  8.29,183.90) --
	(  8.75,183.24) --
	(  9.23,182.59) --
	(  9.72,181.94) --
	( 10.22,181.31) --
	( 10.73,180.68) --
	( 11.25,180.06) --
	( 11.78,179.45) --
	( 12.32,178.85) --
	( 12.87,178.26) --
	( 13.44,177.68) --
	( 14.01,177.11) --
	( 14.59,176.55) --
	( 15.18,176.00) --
	( 15.79,175.46) --
	( 16.40,174.93) --
	( 17.02,174.42) --
	( 17.65,173.91) --
	( 18.29,173.41) --
	( 18.93,172.93) --
	( 19.59,172.45) --
	( 20.25,171.99) --
	( 20.92,171.54) --
	( 21.60,171.10) --
	( 22.29,170.68) --
	( 22.98,170.26) --
	( 23.69,169.86) --
	( 24.39,169.47) --
	( 25.11,169.09) --
	( 25.83,168.73) --
	( 26.56,168.38) --
	( 27.29,168.04) --
	( 28.03,167.71) --
	( 28.78,167.40) --
	( 29.53,167.10) --
	( 30.28,166.81) --
	( 31.04,166.53) --
	( 31.81,166.27) --
	( 32.58,166.03) --
	( 33.35,165.79) --
	( 34.13,165.57) --
	( 34.91,165.37) --
	( 35.70,165.17) --
	( 36.49,165.00) --
	( 37.28,164.83) --
	( 38.07,164.68) --
	( 38.87,164.54) --
	( 39.67,164.42) --
	( 40.47,164.31) --
	( 41.27,164.21) --
	( 42.08,164.13) --
	( 42.88,164.07) --
	( 43.69,164.01) --
	( 44.50,163.97) --
	( 45.30,163.95) --
	( 46.11,163.94) --
	( 46.92,163.94) --
	( 47.73,163.96) --
	( 48.54,163.99) --
	( 49.34,164.04) --
	( 50.15,164.10) --
	( 50.96,164.17) --
	( 51.76,164.26) --
	( 52.56,164.36) --
	( 53.36,164.48) --
	( 54.16,164.61) --
	( 54.96,164.75) --
	( 55.75,164.91) --
	( 56.54,165.08) --
	( 57.32,165.27) --
	( 58.11,165.47) --
	( 58.89,165.68) --
	( 59.66,165.91) --
	( 60.44,166.15) --
	( 61.20,166.40) --
	( 61.97,166.67) --
	( 62.73,166.95) --
	( 63.48,167.24) --
	( 64.23,167.55) --
	( 64.97,167.87) --
	( 65.71,168.20) --
	( 66.44,168.55) --
	( 67.16,168.91) --
	( 67.88,169.28) --
	( 68.59,169.66) --
	( 69.30,170.06) --
	( 69.99,170.47) --
	( 70.68,170.89) --
	( 71.37,171.32) --
	( 72.04,171.76) --
	( 72.71,172.22) --
	( 73.37,172.69) --
	( 74.02,173.17) --
	( 74.66,173.66) --
	( 75.30,174.16) --
	( 75.92,174.67) --
	( 76.54,175.20) --
	( 77.15,175.73) --
	( 77.74,176.28) --
	( 78.33,176.83) --
	( 78.91,177.40) --
	( 79.48,177.97) --
	( 80.03,178.56) --
	( 80.58,179.15) --
	( 81.12,179.76) --
	( 81.64,180.37) --
	( 82.16,180.99) --
	( 82.66,181.62) --
	( 83.16,182.27) --
	( 83.64,182.91) --
	( 84.11,183.57) --
	( 84.57,184.24) --
	( 85.02,184.91) --
	( 85.45,185.59) --
	( 85.88,186.28) --
	( 86.29,186.98) --
	( 86.69,187.68) --
	( 87.07,188.39) --
	( 87.45,189.11) --
	( 87.81,189.83) --
	( 88.16,190.56) --
	( 88.49,191.29) --
	( 88.82,192.04) --
	( 89.13,192.78) --
	( 89.42,193.53) --
	( 89.71,194.29) --
	( 89.98,195.05) --
	( 90.24,195.82) --
	( 90.48,196.59) --
	( 90.71,197.36) --
	( 90.93,198.14) --
	( 91.13,198.93) --
	( 91.32,199.71) --
	( 91.49,200.50) --
	( 91.66,201.29) --
	( 91.80,202.09) --
	( 91.94,202.89) --
	( 92.06,203.69) --
	( 92.16,204.49) --
	( 92.26,205.29) --
	( 92.33,206.10) --
	( 92.40,206.90) --
	( 92.45,207.71) --
	( 92.48,208.52) --
	( 92.50,209.32) --
	( 92.51,210.13);
\definecolor[named]{drawColor}{rgb}{0.00,0.00,0.00}

\path[draw=drawColor,line width= 0.4pt,dash pattern=on 1pt off 3pt ,line join=round,line cap=round] ( 97.13,210.13) --
	( 97.12,211.02) --
	( 97.10,211.91) --
	( 97.06,212.80) --
	( 97.01,213.69) --
	( 96.94,214.57) --
	( 96.85,215.46) --
	( 96.75,216.34) --
	( 96.63,217.22) --
	( 96.50,218.10) --
	( 96.35,218.98) --
	( 96.19,219.86) --
	( 96.01,220.73) --
	( 95.82,221.59) --
	( 95.61,222.46) --
	( 95.39,223.32) --
	( 95.15,224.18) --
	( 94.90,225.03) --
	( 94.63,225.88) --
	( 94.35,226.72) --
	( 94.05,227.56) --
	( 93.74,228.39) --
	( 93.41,229.22) --
	( 93.07,230.04) --
	( 92.71,230.85) --
	( 92.34,231.66) --
	( 91.96,232.47) --
	( 91.56,233.26) --
	( 91.15,234.05) --
	( 90.72,234.83) --
	( 90.28,235.60) --
	( 89.83,236.37) --
	( 89.37,237.13) --
	( 88.89,237.88) --
	( 88.40,238.62) --
	( 87.89,239.35) --
	( 87.37,240.07) --
	( 86.84,240.79) --
	( 86.30,241.49) --
	( 85.74,242.19) --
	( 85.18,242.87) --
	( 84.60,243.55) --
	( 84.01,244.21) --
	( 83.41,244.87) --
	( 82.79,245.51) --
	( 82.17,246.14) --
	( 81.53,246.76) --
	( 80.89,247.37) --
	( 80.23,247.97) --
	( 79.56,248.56) --
	( 78.88,249.14) --
	( 78.20,249.70) --
	( 77.50,250.25) --
	( 76.79,250.79) --
	( 76.08,251.32) --
	( 75.35,251.84) --
	( 74.62,252.34) --
	( 73.87,252.83) --
	( 73.68,252.94);

\path[draw=drawColor,line width= 0.4pt,dash pattern=on 1pt off 3pt ,line join=round,line cap=round] ( 18.95,252.94) --
	( 18.39,252.58) --
	( 17.65,252.09) --
	( 16.92,251.58) --
	( 16.20,251.06) --
	( 15.48,250.53) --
	( 14.78,249.98) --
	( 14.09,249.42) --
	( 13.41,248.85) --
	( 12.73,248.27) --
	( 12.07,247.68) --
	( 11.42,247.07) --
	( 10.78,246.45) --
	( 10.15,245.83) --
	(  9.53,245.19) --
	(  8.92,244.54) --
	(  8.33,243.88) --
	(  7.74,243.21) --
	(  7.17,242.53) --
	(  6.61,241.84) --
	(  6.06,241.14) --
	(  5.52,240.43) --
	(  5.00,239.71) --
	(  4.49,238.98) --
	(  3.99,238.25) --
	(  3.50,237.50) --
	(  3.03,236.75) --
	(  2.57,235.99) --
	(  2.12,235.22) --
	(  1.69,234.44) --
	(  1.27,233.66) --
	(  0.87,232.86) --
	(  0.48,232.07) --
	(  0.10,231.26) --
	(  0.00,231.03);

\path[draw=drawColor,line width= 0.4pt,dash pattern=on 1pt off 3pt ,line join=round,line cap=round] (  0.00,189.23) --
	(  0.10,189.01) --
	(  0.48,188.20) --
	(  0.87,187.40) --
	(  1.27,186.61) --
	(  1.69,185.82) --
	(  2.12,185.05) --
	(  2.57,184.28) --
	(  3.03,183.52) --
	(  3.50,182.76) --
	(  3.99,182.02) --
	(  4.49,181.28) --
	(  5.00,180.55) --
	(  5.52,179.83) --
	(  6.06,179.13) --
	(  6.61,178.43) --
	(  7.17,177.74) --
	(  7.74,177.06) --
	(  8.33,176.39) --
	(  8.92,175.73) --
	(  9.53,175.08) --
	( 10.15,174.44) --
	( 10.78,173.81) --
	( 11.42,173.19) --
	( 12.07,172.59) --
	( 12.73,172.00) --
	( 13.41,171.41) --
	( 14.09,170.84) --
	( 14.78,170.29) --
	( 15.48,169.74) --
	( 16.20,169.21) --
	( 16.92,168.69) --
	( 17.65,168.18) --
	( 18.39,167.68) --
	( 19.13,167.20) --
	( 19.89,166.73) --
	( 20.65,166.27) --
	( 21.42,165.83) --
	( 22.20,165.40) --
	( 22.99,164.99) --
	( 23.78,164.59) --
	( 24.58,164.20) --
	( 25.39,163.83) --
	( 26.20,163.47) --
	( 27.02,163.12) --
	( 27.85,162.79) --
	( 28.68,162.48) --
	( 29.52,162.17) --
	( 30.36,161.89) --
	( 31.21,161.62) --
	( 32.06,161.36) --
	( 32.91,161.12) --
	( 33.77,160.89) --
	( 34.64,160.68) --
	( 35.50,160.48) --
	( 36.37,160.30) --
	( 37.25,160.13) --
	( 38.12,159.98) --
	( 39.00,159.85) --
	( 39.88,159.73) --
	( 40.77,159.62) --
	( 41.65,159.53) --
	( 42.54,159.46) --
	( 43.43,159.40) --
	( 44.31,159.36) --
	( 45.20,159.33) --
	( 46.09,159.32) --
	( 46.98,159.32) --
	( 47.87,159.34) --
	( 48.76,159.38) --
	( 49.65,159.43) --
	( 50.53,159.49) --
	( 51.42,159.58) --
	( 52.30,159.67) --
	( 53.19,159.78) --
	( 54.07,159.91) --
	( 54.94,160.06) --
	( 55.82,160.21) --
	( 56.69,160.39) --
	( 57.56,160.58) --
	( 58.43,160.78) --
	( 59.29,161.00) --
	( 60.15,161.24) --
	( 61.00,161.49) --
	( 61.85,161.75) --
	( 62.69,162.03) --
	( 63.53,162.32) --
	( 64.37,162.63) --
	( 65.19,162.96) --
	( 66.02,163.29) --
	( 66.83,163.65) --
	( 67.64,164.01) --
	( 68.45,164.39) --
	( 69.25,164.79) --
	( 70.04,165.19) --
	( 70.82,165.62) --
	( 71.59,166.05) --
	( 72.36,166.50) --
	( 73.12,166.96) --
	( 73.87,167.44) --
	( 74.62,167.93) --
	( 75.35,168.43) --
	( 76.08,168.94) --
	( 76.79,169.47) --
	( 77.50,170.01) --
	( 78.20,170.56) --
	( 78.88,171.13) --
	( 79.56,171.70) --
	( 80.23,172.29) --
	( 80.89,172.89) --
	( 81.53,173.50) --
	( 82.17,174.12) --
	( 82.79,174.76) --
	( 83.41,175.40) --
	( 84.01,176.05) --
	( 84.60,176.72) --
	( 85.18,177.39) --
	( 85.74,178.08) --
	( 86.30,178.77) --
	( 86.84,179.48) --
	( 87.37,180.19) --
	( 87.89,180.92) --
	( 88.40,181.65) --
	( 88.89,182.39) --
	( 89.37,183.14) --
	( 89.83,183.90) --
	( 90.28,184.66) --
	( 90.72,185.43) --
	( 91.15,186.22) --
	( 91.56,187.00) --
	( 91.96,187.80) --
	( 92.34,188.60) --
	( 92.71,189.41) --
	( 93.07,190.23) --
	( 93.41,191.05) --
	( 93.74,191.87) --
	( 94.05,192.71) --
	( 94.35,193.54) --
	( 94.63,194.39) --
	( 94.90,195.24) --
	( 95.15,196.09) --
	( 95.39,196.94) --
	( 95.61,197.81) --
	( 95.82,198.67) --
	( 96.01,199.54) --
	( 96.19,200.41) --
	( 96.35,201.28) --
	( 96.50,202.16) --
	( 96.63,203.04) --
	( 96.75,203.92) --
	( 96.85,204.81) --
	( 96.94,205.69) --
	( 97.01,206.58) --
	( 97.06,207.47) --
	( 97.10,208.35) --
	( 97.12,209.24) --
	( 97.13,210.13);
\definecolor[named]{drawColor}{rgb}{0.00,0.00,1.00}

\path[draw=drawColor,line width= 1.2pt,line join=round,line cap=round] (  2.33,252.94) --
	(  2.69,252.75) --
	(  3.41,252.39) --
	(  4.14,252.04) --
	(  4.88,251.70) --
	(  5.62,251.37) --
	(  6.36,251.06) --
	(  7.11,250.76) --
	(  7.87,250.47) --
	(  8.63,250.19) --
	(  9.39,249.93) --
	( 10.16,249.69) --
	( 10.94,249.45) --
	( 11.71,249.23) --
	( 12.50,249.03) --
	( 13.28,248.83) --
	( 14.07,248.66) --
	( 14.86,248.49) --
	( 15.66,248.34) --
	( 16.45,248.20) --
	( 17.25,248.08) --
	( 18.05,247.97) --
	( 18.86,247.87) --
	( 19.66,247.79) --
	( 20.47,247.73) --
	( 21.27,247.67) --
	( 22.08,247.63) --
	( 22.89,247.61) --
	( 23.70,247.60) --
	( 24.50,247.60) --
	( 25.31,247.62) --
	( 26.12,247.65) --
	( 26.93,247.70) --
	( 27.73,247.76) --
	( 28.54,247.83) --
	( 29.34,247.92) --
	( 30.15,248.02) --
	( 30.95,248.14) --
	( 31.74,248.27) --
	( 32.54,248.41) --
	( 33.33,248.57) --
	( 34.12,248.74) --
	( 34.91,248.93) --
	( 35.69,249.13) --
	( 36.47,249.34) --
	( 37.25,249.57) --
	( 38.02,249.81) --
	( 38.79,250.06) --
	( 39.55,250.33) --
	( 40.31,250.61) --
	( 41.06,250.90) --
	( 41.81,251.21) --
	( 42.55,251.53) --
	( 43.29,251.86) --
	( 44.02,252.21) --
	( 44.74,252.57) --
	( 45.46,252.94) --
	( 45.47,252.94);
\definecolor[named]{drawColor}{rgb}{0.00,0.00,0.00}

\path[draw=drawColor,line width= 0.4pt,dash pattern=on 1pt off 3pt ,line join=round,line cap=round] (  0.00,248.95) --
	(  0.57,248.65) --
	(  1.37,248.25) --
	(  2.17,247.86) --
	(  2.97,247.49) --
	(  3.79,247.13) --
	(  4.61,246.78) --
	(  5.43,246.45) --
	(  6.26,246.14) --
	(  7.10,245.83) --
	(  7.94,245.55) --
	(  8.79,245.28) --
	(  9.64,245.02) --
	( 10.50,244.78) --
	( 11.36,244.55) --
	( 12.22,244.34) --
	( 13.09,244.14) --
	( 13.96,243.96) --
	( 14.83,243.79) --
	( 15.71,243.64) --
	( 16.59,243.51) --
	( 17.47,243.39) --
	( 18.35,243.28) --
	( 19.24,243.19) --
	( 20.12,243.12) --
	( 21.01,243.06) --
	( 21.90,243.02) --
	( 22.79,242.99) --
	( 23.68,242.98) --
	( 24.57,242.98) --
	( 25.45,243.00) --
	( 26.34,243.04) --
	( 27.23,243.09) --
	( 28.12,243.15) --
	( 29.00,243.24) --
	( 29.89,243.33) --
	( 30.77,243.44) --
	( 31.65,243.57) --
	( 32.53,243.72) --
	( 33.40,243.87) --
	( 34.27,244.05) --
	( 35.14,244.24) --
	( 36.01,244.44) --
	( 36.87,244.66) --
	( 37.73,244.90) --
	( 38.58,245.15) --
	( 39.43,245.41) --
	( 40.28,245.69) --
	( 41.12,245.98) --
	( 41.95,246.29) --
	( 42.78,246.62) --
	( 43.60,246.95) --
	( 44.42,247.31) --
	( 45.23,247.67) --
	( 46.03,248.05) --
	( 46.83,248.45) --
	( 47.62,248.85) --
	( 48.40,249.28) --
	( 49.18,249.71) --
	( 49.94,250.16) --
	( 50.70,250.62) --
	( 51.46,251.10) --
	( 52.20,251.59) --
	( 52.93,252.09) --
	( 53.66,252.60) --
	( 54.12,252.94);
\definecolor[named]{drawColor}{rgb}{0.00,0.00,1.00}

\path[draw=drawColor,line width= 1.2pt,line join=round,line cap=round] (192.88,  0.00) --
	(192.87,  0.01) --
	(192.15,  0.38) --
	(191.43,  0.74) --
	(190.70,  1.08) --
	(189.96,  1.41) --
	(189.22,  1.73) --
	(188.47,  2.04) --
	(187.72,  2.33) --
	(186.96,  2.62) --
	(186.19,  2.88) --
	(185.43,  3.14) --
	(184.65,  3.38) --
	(183.88,  3.60) --
	(183.10,  3.82) --
	(182.32,  4.02) --
	(181.53,  4.20) --
	(180.74,  4.37) --
	(179.95,  4.53) --
	(179.15,  4.68) --
	(178.35,  4.81) --
	(177.55,  4.92) --
	(176.75,  5.03) --
	(175.95,  5.11) --
	(175.14,  5.19) --
	(174.34,  5.25) --
	(173.53,  5.29) --
	(172.72,  5.33) --
	(171.91,  5.34) --
	(171.10,  5.35) --
	(170.30,  5.34) --
	(169.49,  5.31) --
	(168.68,  5.27) --
	(167.87,  5.22) --
	(167.07,  5.15) --
	(166.26,  5.07) --
	(165.46,  4.98) --
	(164.66,  4.87) --
	(163.86,  4.74) --
	(163.06,  4.61) --
	(162.27,  4.45) --
	(161.48,  4.29) --
	(160.69,  4.11) --
	(159.90,  3.92) --
	(159.12,  3.71) --
	(158.34,  3.49) --
	(157.57,  3.26) --
	(156.80,  3.01) --
	(156.03,  2.75) --
	(155.27,  2.48) --
	(154.52,  2.19) --
	(153.77,  1.89) --
	(153.02,  1.58) --
	(152.28,  1.25) --
	(151.55,  0.91) --
	(150.82,  0.56) --
	(150.10,  0.19) --
	(149.74,  0.00);
\definecolor[named]{drawColor}{rgb}{0.00,0.00,0.00}

\path[draw=drawColor,line width= 0.4pt,dash pattern=on 1pt off 3pt ,line join=round,line cap=round] (201.53,  0.00) --
	(201.07,  0.34) --
	(200.34,  0.86) --
	(199.61,  1.36) --
	(198.86,  1.85) --
	(198.11,  2.32) --
	(197.35,  2.78) --
	(196.58,  3.23) --
	(195.81,  3.67) --
	(195.03,  4.09) --
	(194.24,  4.50) --
	(193.44,  4.89) --
	(192.64,  5.27) --
	(191.82,  5.64) --
	(191.01,  5.99) --
	(190.19,  6.33) --
	(189.36,  6.65) --
	(188.52,  6.96) --
	(187.68,  7.26) --
	(186.84,  7.53) --
	(185.99,  7.80) --
	(185.14,  8.05) --
	(184.28,  8.28) --
	(183.42,  8.50) --
	(182.55,  8.71) --
	(181.68,  8.90) --
	(180.81,  9.07) --
	(179.93,  9.23) --
	(179.06,  9.37) --
	(178.18,  9.50) --
	(177.29,  9.61) --
	(176.41,  9.71) --
	(175.53,  9.79) --
	(174.64,  9.86) --
	(173.75,  9.91) --
	(172.86,  9.94) --
	(171.97,  9.96) --
	(171.08,  9.97) --
	(170.19,  9.95) --
	(169.31,  9.93) --
	(168.42,  9.88) --
	(167.53,  9.83) --
	(166.64,  9.75) --
	(165.76,  9.66) --
	(164.88,  9.56) --
	(163.99,  9.44) --
	(163.12,  9.30) --
	(162.24,  9.15) --
	(161.37,  8.99) --
	(160.49,  8.80) --
	(159.63,  8.61) --
	(158.76,  8.39) --
	(157.90,  8.17) --
	(157.05,  7.93) --
	(156.20,  7.67) --
	(155.35,  7.40) --
	(154.51,  7.11) --
	(153.67,  6.81) --
	(152.84,  6.49) --
	(152.01,  6.16) --
	(151.19,  5.82) --
	(150.38,  5.46) --
	(149.57,  5.09) --
	(148.77,  4.70) --
	(147.98,  4.30) --
	(147.19,  3.88) --
	(146.41,  3.45) --
	(145.64,  3.01) --
	(144.88,  2.55) --
	(144.12,  2.09) --
	(143.38,  1.60) --
	(142.64,  1.11) --
	(141.91,  0.60) --
	(141.19,  0.08) --
	(141.08,  0.00);

\path[fill=fillColor] (148.89, 42.81) circle (  2.25);
\definecolor[named]{drawColor}{rgb}{0.00,0.00,1.00}

\path[draw=drawColor,line width= 1.2pt,line join=round,line cap=round] (195.08, 42.81) --
	(195.08, 43.62) --
	(195.06, 44.43) --
	(195.02, 45.24) --
	(194.97, 46.04) --
	(194.91, 46.85) --
	(194.83, 47.65) --
	(194.74, 48.46) --
	(194.63, 49.26) --
	(194.51, 50.06) --
	(194.38, 50.86) --
	(194.23, 51.65) --
	(194.07, 52.44) --
	(193.89, 53.23) --
	(193.70, 54.02) --
	(193.50, 54.80) --
	(193.28, 55.58) --
	(193.05, 56.36) --
	(192.81, 57.13) --
	(192.55, 57.89) --
	(192.28, 58.65) --
	(192.00, 59.41) --
	(191.70, 60.16) --
	(191.39, 60.91) --
	(191.07, 61.65) --
	(190.73, 62.39) --
	(190.38, 63.12) --
	(190.02, 63.84) --
	(189.65, 64.56) --
	(189.26, 65.27) --
	(188.86, 65.97) --
	(188.45, 66.66) --
	(188.03, 67.35) --
	(187.59, 68.03) --
	(187.14, 68.71) --
	(186.68, 69.37) --
	(186.21, 70.03) --
	(185.73, 70.68) --
	(185.24, 71.32) --
	(184.73, 71.95) --
	(184.22, 72.57) --
	(183.69, 73.19) --
	(183.16, 73.79) --
	(182.61, 74.39) --
	(182.05, 74.97) --
	(181.48, 75.55) --
	(180.90, 76.11) --
	(180.32, 76.67) --
	(179.72, 77.21) --
	(179.11, 77.75) --
	(178.50, 78.27) --
	(177.87, 78.78) --
	(177.24, 79.29) --
	(176.59, 79.78) --
	(175.94, 80.26) --
	(175.28, 80.72) --
	(174.62, 81.18) --
	(173.94, 81.62) --
	(173.26, 82.06) --
	(172.57, 82.48) --
	(171.87, 82.89) --
	(171.16, 83.28) --
	(170.45, 83.67) --
	(169.74, 84.04) --
	(169.01, 84.40) --
	(168.28, 84.74) --
	(167.54, 85.07) --
	(166.80, 85.39) --
	(166.05, 85.70) --
	(165.30, 85.99) --
	(164.54, 86.28) --
	(163.78, 86.54) --
	(163.01, 86.80) --
	(162.24, 87.04) --
	(161.46, 87.26) --
	(160.68, 87.48) --
	(159.90, 87.68) --
	(159.11, 87.86) --
	(158.32, 88.03) --
	(157.53, 88.19) --
	(156.73, 88.34) --
	(155.94, 88.47) --
	(155.14, 88.58) --
	(154.33, 88.69) --
	(153.53, 88.77) --
	(152.73, 88.85) --
	(151.92, 88.91) --
	(151.11, 88.95) --
	(150.30, 88.99) --
	(149.50, 89.00) --
	(148.69, 89.01) --
	(147.88, 89.00) --
	(147.07, 88.97) --
	(146.26, 88.93) --
	(145.46, 88.88) --
	(144.65, 88.81) --
	(143.85, 88.73) --
	(143.04, 88.64) --
	(142.24, 88.53) --
	(141.44, 88.40) --
	(140.65, 88.27) --
	(139.85, 88.11) --
	(139.06, 87.95) --
	(138.27, 87.77) --
	(137.49, 87.58) --
	(136.71, 87.37) --
	(135.93, 87.15) --
	(135.15, 86.92) --
	(134.38, 86.67) --
	(133.62, 86.41) --
	(132.86, 86.14) --
	(132.10, 85.85) --
	(131.35, 85.55) --
	(130.61, 85.24) --
	(129.87, 84.91) --
	(129.13, 84.57) --
	(128.40, 84.22) --
	(127.68, 83.85) --
	(126.97, 83.48) --
	(126.26, 83.09) --
	(125.56, 82.68) --
	(124.86, 82.27) --
	(124.18, 81.84) --
	(123.50, 81.40) --
	(122.83, 80.95) --
	(122.16, 80.49) --
	(121.51, 80.02) --
	(120.86, 79.53) --
	(120.22, 79.04) --
	(119.59, 78.53) --
	(118.97, 78.01) --
	(118.36, 77.48) --
	(117.76, 76.94) --
	(117.17, 76.39) --
	(116.58, 75.83) --
	(116.01, 75.26) --
	(115.45, 74.68) --
	(114.90, 74.09) --
	(114.35, 73.49) --
	(113.82, 72.88) --
	(113.30, 72.26) --
	(112.79, 71.64) --
	(112.29, 71.00) --
	(111.80, 70.36) --
	(111.33, 69.70) --
	(110.86, 69.04) --
	(110.41, 68.37) --
	(109.97, 67.69) --
	(109.54, 67.01) --
	(109.12, 66.32) --
	(108.72, 65.62) --
	(108.32, 64.91) --
	(107.94, 64.20) --
	(107.57, 63.48) --
	(107.22, 62.75) --
	(106.88, 62.02) --
	(106.55, 61.28) --
	(106.23, 60.54) --
	(105.93, 59.79) --
	(105.64, 59.03) --
	(105.36, 58.27) --
	(105.09, 57.51) --
	(104.84, 56.74) --
	(104.61, 55.97) --
	(104.38, 55.19) --
	(104.17, 54.41) --
	(103.98, 53.63) --
	(103.80, 52.84) --
	(103.63, 52.05) --
	(103.47, 51.25) --
	(103.33, 50.46) --
	(103.20, 49.66) --
	(103.09, 48.86) --
	(102.99, 48.06) --
	(102.91, 47.25) --
	(102.84, 46.45) --
	(102.78, 45.64) --
	(102.74, 44.83) --
	(102.71, 44.03) --
	(102.70, 43.22) --
	(102.70, 42.41) --
	(102.71, 41.60) --
	(102.74, 40.79) --
	(102.78, 39.98) --
	(102.84, 39.18) --
	(102.91, 38.37) --
	(102.99, 37.57) --
	(103.09, 36.77) --
	(103.20, 35.97) --
	(103.33, 35.17) --
	(103.47, 34.37) --
	(103.63, 33.58) --
	(103.80, 32.79) --
	(103.98, 32.00) --
	(104.17, 31.21) --
	(104.38, 30.43) --
	(104.61, 29.66) --
	(104.84, 28.88) --
	(105.09, 28.12) --
	(105.36, 27.35) --
	(105.64, 26.59) --
	(105.93, 25.84) --
	(106.23, 25.09) --
	(106.55, 24.34) --
	(106.88, 23.61) --
	(107.22, 22.87) --
	(107.57, 22.15) --
	(107.94, 21.43) --
	(108.32, 20.71) --
	(108.72, 20.01) --
	(109.12, 19.31) --
	(109.54, 18.62) --
	(109.97, 17.93) --
	(110.41, 17.25) --
	(110.86, 16.58) --
	(111.33, 15.92) --
	(111.80, 15.27) --
	(112.29, 14.62) --
	(112.79, 13.99) --
	(113.30, 13.36) --
	(113.82, 12.74) --
	(114.35, 12.13) --
	(114.90, 11.53) --
	(115.45, 10.94) --
	(116.01, 10.36) --
	(116.58,  9.79) --
	(117.17,  9.23) --
	(117.76,  8.68) --
	(118.36,  8.14) --
	(118.97,  7.61) --
	(119.59,  7.10) --
	(120.22,  6.59) --
	(120.86,  6.09) --
	(121.51,  5.61) --
	(122.16,  5.13) --
	(122.83,  4.67) --
	(123.50,  4.22) --
	(124.18,  3.78) --
	(124.86,  3.36) --
	(125.56,  2.94) --
	(126.26,  2.54) --
	(126.97,  2.15) --
	(127.68,  1.77) --
	(128.40,  1.41) --
	(129.13,  1.06) --
	(129.87,  0.72) --
	(130.61,  0.39) --
	(131.35,  0.08) --
	(131.54,  0.00);

\path[draw=drawColor,line width= 1.2pt,line join=round,line cap=round] (166.24,  0.00) --
	(166.80,  0.23) --
	(167.54,  0.55) --
	(168.28,  0.88) --
	(169.01,  1.23) --
	(169.74,  1.59) --
	(170.45,  1.96) --
	(171.16,  2.34) --
	(171.87,  2.74) --
	(172.57,  3.15) --
	(173.26,  3.57) --
	(173.94,  4.00) --
	(174.62,  4.44) --
	(175.28,  4.90) --
	(175.94,  5.37) --
	(176.59,  5.85) --
	(177.24,  6.34) --
	(177.87,  6.84) --
	(178.50,  7.35) --
	(179.11,  7.88) --
	(179.72,  8.41) --
	(180.32,  8.96) --
	(180.90,  9.51) --
	(181.48, 10.08) --
	(182.05, 10.65) --
	(182.61, 11.24) --
	(183.16, 11.83) --
	(183.69, 12.44) --
	(184.22, 13.05) --
	(184.73, 13.67) --
	(185.24, 14.30) --
	(185.73, 14.95) --
	(186.21, 15.59) --
	(186.68, 16.25) --
	(187.14, 16.92) --
	(187.59, 17.59) --
	(188.03, 18.27) --
	(188.45, 18.96) --
	(188.86, 19.66) --
	(189.26, 20.36) --
	(189.65, 21.07) --
	(190.02, 21.79) --
	(190.38, 22.51) --
	(190.73, 23.24) --
	(191.07, 23.97) --
	(191.39, 24.72) --
	(191.70, 25.46) --
	(192.00, 26.21) --
	(192.28, 26.97) --
	(192.55, 27.73) --
	(192.81, 28.50) --
	(193.05, 29.27) --
	(193.28, 30.04) --
	(193.50, 30.82) --
	(193.70, 31.61) --
	(193.89, 32.39) --
	(194.07, 33.18) --
	(194.23, 33.97) --
	(194.38, 34.77) --
	(194.51, 35.57) --
	(194.63, 36.37) --
	(194.74, 37.17) --
	(194.83, 37.97) --
	(194.91, 38.78) --
	(194.97, 39.58) --
	(195.02, 40.39) --
	(195.06, 41.20) --
	(195.08, 42.00) --
	(195.08, 42.81);
\definecolor[named]{drawColor}{rgb}{0.00,0.00,0.00}

\path[draw=drawColor,line width= 0.4pt,dash pattern=on 1pt off 3pt ,line join=round,line cap=round] (199.70, 42.81) --
	(199.70, 43.70) --
	(199.67, 44.59) --
	(199.63, 45.48) --
	(199.58, 46.37) --
	(199.51, 47.25) --
	(199.42, 48.14) --
	(199.32, 49.02) --
	(199.21, 49.90) --
	(199.07, 50.78) --
	(198.93, 51.66) --
	(198.76, 52.54) --
	(198.59, 53.41) --
	(198.39, 54.27) --
	(198.19, 55.14) --
	(197.96, 56.00) --
	(197.72, 56.86) --
	(197.47, 57.71) --
	(197.20, 58.56) --
	(196.92, 59.40) --
	(196.62, 60.24) --
	(196.31, 61.07) --
	(195.98, 61.90) --
	(195.64, 62.72) --
	(195.29, 63.53) --
	(194.92, 64.34) --
	(194.53, 65.15) --
	(194.13, 65.94) --
	(193.72, 66.73) --
	(193.30, 67.51) --
	(192.86, 68.28) --
	(192.41, 69.05) --
	(191.94, 69.81) --
	(191.46, 70.56) --
	(190.97, 71.30) --
	(190.46, 72.03) --
	(189.95, 72.75) --
	(189.42, 73.47) --
	(188.87, 74.17) --
	(188.32, 74.87) --
	(187.75, 75.55) --
	(187.17, 76.23) --
	(186.58, 76.89) --
	(185.98, 77.55) --
	(185.37, 78.19) --
	(184.74, 78.82) --
	(184.11, 79.44) --
	(183.46, 80.05) --
	(182.80, 80.65) --
	(182.13, 81.24) --
	(181.46, 81.82) --
	(180.77, 82.38) --
	(180.07, 82.93) --
	(179.37, 83.47) --
	(178.65, 84.00) --
	(177.92, 84.52) --
	(177.19, 85.02) --
	(176.45, 85.51) --
	(175.70, 85.98) --
	(174.94, 86.44) --
	(174.17, 86.89) --
	(173.39, 87.33) --
	(172.61, 87.75) --
	(171.82, 88.16) --
	(171.02, 88.55) --
	(170.22, 88.93) --
	(169.41, 89.30) --
	(168.59, 89.65) --
	(167.77, 89.99) --
	(166.94, 90.31) --
	(166.11, 90.62) --
	(165.27, 90.92) --
	(164.42, 91.19) --
	(163.57, 91.46) --
	(162.72, 91.71) --
	(161.86, 91.94) --
	(161.00, 92.16) --
	(160.13, 92.37) --
	(159.27, 92.56) --
	(158.39, 92.73) --
	(157.52, 92.89) --
	(156.64, 93.03) --
	(155.76, 93.16) --
	(154.88, 93.27) --
	(153.99, 93.37) --
	(153.11, 93.45) --
	(152.22, 93.52) --
	(151.33, 93.57) --
	(150.45, 93.60) --
	(149.56, 93.62) --
	(148.67, 93.63) --
	(147.78, 93.61) --
	(146.89, 93.59) --
	(146.00, 93.54) --
	(145.11, 93.49) --
	(144.23, 93.41) --
	(143.34, 93.32) --
	(142.46, 93.22) --
	(141.58, 93.10) --
	(140.70, 92.96) --
	(139.82, 92.81) --
	(138.95, 92.65) --
	(138.08, 92.46) --
	(137.21, 92.27) --
	(136.35, 92.05) --
	(135.49, 91.83) --
	(134.63, 91.59) --
	(133.78, 91.33) --
	(132.93, 91.06) --
	(132.09, 90.77) --
	(131.25, 90.47) --
	(130.42, 90.15) --
	(129.60, 89.82) --
	(128.78, 89.48) --
	(127.96, 89.12) --
	(127.16, 88.75) --
	(126.36, 88.36) --
	(125.56, 87.96) --
	(124.78, 87.54) --
	(124.00, 87.11) --
	(123.23, 86.67) --
	(122.46, 86.21) --
	(121.71, 85.75) --
	(120.96, 85.26) --
	(120.22, 84.77) --
	(119.49, 84.26) --
	(118.77, 83.74) --
	(118.06, 83.21) --
	(117.36, 82.66) --
	(116.66, 82.10) --
	(115.98, 81.53) --
	(115.31, 80.95) --
	(114.65, 80.36) --
	(113.99, 79.75) --
	(113.35, 79.13) --
	(112.72, 78.51) --
	(112.10, 77.87) --
	(111.50, 77.22) --
	(110.90, 76.56) --
	(110.31, 75.89) --
	(109.74, 75.21) --
	(109.18, 74.52) --
	(108.63, 73.82) --
	(108.10, 73.11) --
	(107.57, 72.39) --
	(107.06, 71.66) --
	(106.56, 70.93) --
	(106.08, 70.18) --
	(105.60, 69.43) --
	(105.14, 68.67) --
	(104.70, 67.90) --
	(104.27, 67.12) --
	(103.85, 66.34) --
	(103.44, 65.54) --
	(103.05, 64.75) --
	(102.68, 63.94) --
	(102.31, 63.13) --
	(101.96, 62.31) --
	(101.63, 61.49) --
	(101.31, 60.66) --
	(101.01, 59.82) --
	(100.72, 58.98) --
	(100.44, 58.13) --
	(100.18, 57.28) --
	( 99.93, 56.43) --
	( 99.70, 55.57) --
	( 99.49, 54.71) --
	( 99.29, 53.84) --
	( 99.10, 52.97) --
	( 98.93, 52.10) --
	( 98.78, 51.22) --
	( 98.64, 50.34) --
	( 98.51, 49.46) --
	( 98.40, 48.58) --
	( 98.31, 47.70) --
	( 98.23, 46.81) --
	( 98.17, 45.92) --
	( 98.12, 45.04) --
	( 98.09, 44.15) --
	( 98.08, 43.26) --
	( 98.08, 42.37) --
	( 98.09, 41.48) --
	( 98.12, 40.59) --
	( 98.17, 39.70) --
	( 98.23, 38.81) --
	( 98.31, 37.93) --
	( 98.40, 37.04) --
	( 98.51, 36.16) --
	( 98.64, 35.28) --
	( 98.78, 34.40) --
	( 98.93, 33.53) --
	( 99.10, 32.65) --
	( 99.29, 31.78) --
	( 99.49, 30.92) --
	( 99.70, 30.05) --
	( 99.93, 29.20) --
	(100.18, 28.34) --
	(100.44, 27.49) --
	(100.72, 26.65) --
	(101.01, 25.80) --
	(101.31, 24.97) --
	(101.63, 24.14) --
	(101.96, 23.32) --
	(102.31, 22.50) --
	(102.68, 21.69) --
	(103.05, 20.88) --
	(103.44, 20.08) --
	(103.85, 19.29) --
	(104.27, 18.50) --
	(104.70, 17.73) --
	(105.14, 16.96) --
	(105.60, 16.20) --
	(106.08, 15.44) --
	(106.56, 14.70) --
	(107.06, 13.96) --
	(107.57, 13.23) --
	(108.10, 12.51) --
	(108.63, 11.81) --
	(109.18, 11.11) --
	(109.74, 10.42) --
	(110.31,  9.74) --
	(110.90,  9.07) --
	(111.50,  8.41) --
	(112.10,  7.76) --
	(112.72,  7.12) --
	(113.35,  6.49) --
	(113.99,  5.87) --
	(114.65,  5.27) --
	(115.31,  4.68) --
	(115.98,  4.09) --
	(116.66,  3.52) --
	(117.36,  2.97) --
	(118.06,  2.42) --
	(118.77,  1.89) --
	(119.49,  1.37) --
	(120.22,  0.86) --
	(120.96,  0.36) --
	(121.52,  0.00);

\path[draw=drawColor,line width= 0.4pt,dash pattern=on 1pt off 3pt ,line join=round,line cap=round] (176.26,  0.00) --
	(176.45,  0.12) --
	(177.19,  0.61) --
	(177.92,  1.11) --
	(178.65,  1.62) --
	(179.37,  2.15) --
	(180.07,  2.69) --
	(180.77,  3.24) --
	(181.46,  3.81) --
	(182.13,  4.38) --
	(182.80,  4.97) --
	(183.46,  5.57) --
	(184.11,  6.18) --
	(184.74,  6.80) --
	(185.37,  7.44) --
	(185.98,  8.08) --
	(186.58,  8.73) --
	(187.17,  9.40) --
	(187.75, 10.07) --
	(188.32, 10.76) --
	(188.87, 11.45) --
	(189.42, 12.16) --
	(189.95, 12.87) --
	(190.46, 13.60) --
	(190.97, 14.33) --
	(191.46, 15.07) --
	(191.94, 15.82) --
	(192.41, 16.58) --
	(192.86, 17.34) --
	(193.30, 18.11) --
	(193.72, 18.90) --
	(194.13, 19.68) --
	(194.53, 20.48) --
	(194.92, 21.28) --
	(195.29, 22.09) --
	(195.64, 22.91) --
	(195.98, 23.73) --
	(196.31, 24.55) --
	(196.62, 25.39) --
	(196.92, 26.22) --
	(197.20, 27.07) --
	(197.47, 27.92) --
	(197.72, 28.77) --
	(197.96, 29.62) --
	(198.19, 30.49) --
	(198.39, 31.35) --
	(198.59, 32.22) --
	(198.76, 33.09) --
	(198.93, 33.96) --
	(199.07, 34.84) --
	(199.21, 35.72) --
	(199.32, 36.60) --
	(199.42, 37.49) --
	(199.51, 38.37) --
	(199.58, 39.26) --
	(199.63, 40.15) --
	(199.67, 41.03) --
	(199.70, 41.92) --
	(199.70, 42.81);

\path[fill=fillColor] (126.47,126.47) circle (  2.25);
\definecolor[named]{drawColor}{rgb}{0.00,0.00,1.00}

\path[draw=drawColor,line width= 1.2pt,line join=round,line cap=round] (172.67,126.47) --
	(172.66,127.28) --
	(172.64,128.09) --
	(172.60,128.90) --
	(172.55,129.70) --
	(172.49,130.51) --
	(172.41,131.31) --
	(172.32,132.12) --
	(172.22,132.92) --
	(172.10,133.72) --
	(171.96,134.52) --
	(171.81,135.31) --
	(171.65,136.10) --
	(171.48,136.89) --
	(171.29,137.68) --
	(171.08,138.46) --
	(170.87,139.24) --
	(170.64,140.02) --
	(170.39,140.79) --
	(170.14,141.55) --
	(169.87,142.31) --
	(169.58,143.07) --
	(169.29,143.82) --
	(168.97,144.57) --
	(168.65,145.31) --
	(168.32,146.05) --
	(167.97,146.78) --
	(167.60,147.50) --
	(167.23,148.22) --
	(166.84,148.93) --
	(166.44,149.63) --
	(166.03,150.32) --
	(165.61,151.01) --
	(165.17,151.69) --
	(164.73,152.37) --
	(164.27,153.03) --
	(163.80,153.69) --
	(163.32,154.34) --
	(162.82,154.98) --
	(162.32,155.61) --
	(161.80,156.23) --
	(161.28,156.85) --
	(160.74,157.45) --
	(160.19,158.05) --
	(159.63,158.63) --
	(159.07,159.21) --
	(158.49,159.77) --
	(157.90,160.33) --
	(157.30,160.87) --
	(156.70,161.41) --
	(156.08,161.93) --
	(155.45,162.44) --
	(154.82,162.95) --
	(154.18,163.44) --
	(153.53,163.92) --
	(152.87,164.38) --
	(152.20,164.84) --
	(151.52,165.28) --
	(150.84,165.72) --
	(150.15,166.14) --
	(149.45,166.55) --
	(148.75,166.94) --
	(148.04,167.33) --
	(147.32,167.70) --
	(146.59,168.06) --
	(145.86,168.40) --
	(145.13,168.73) --
	(144.38,169.05) --
	(143.64,169.36) --
	(142.88,169.65) --
	(142.12,169.94) --
	(141.36,170.20) --
	(140.59,170.46) --
	(139.82,170.70) --
	(139.05,170.92) --
	(138.27,171.14) --
	(137.48,171.34) --
	(136.70,171.52) --
	(135.91,171.69) --
	(135.11,171.85) --
	(134.32,172.00) --
	(133.52,172.13) --
	(132.72,172.24) --
	(131.92,172.35) --
	(131.11,172.43) --
	(130.31,172.51) --
	(129.50,172.57) --
	(128.70,172.61) --
	(127.89,172.65) --
	(127.08,172.66) --
	(126.27,172.67) --
	(125.46,172.66) --
	(124.65,172.63) --
	(123.85,172.59) --
	(123.04,172.54) --
	(122.23,172.47) --
	(121.43,172.39) --
	(120.63,172.30) --
	(119.83,172.19) --
	(119.03,172.06) --
	(118.23,171.93) --
	(117.44,171.77) --
	(116.64,171.61) --
	(115.86,171.43) --
	(115.07,171.24) --
	(114.29,171.03) --
	(113.51,170.81) --
	(112.74,170.58) --
	(111.97,170.33) --
	(111.20,170.07) --
	(110.44,169.80) --
	(109.69,169.51) --
	(108.93,169.21) --
	(108.19,168.90) --
	(107.45,168.57) --
	(106.72,168.23) --
	(105.99,167.88) --
	(105.27,167.51) --
	(104.55,167.14) --
	(103.84,166.75) --
	(103.14,166.34) --
	(102.45,165.93) --
	(101.76,165.50) --
	(101.08,165.06) --
	(100.41,164.61) --
	( 99.75,164.15) --
	( 99.09,163.68) --
	( 98.44,163.19) --
	( 97.81,162.70) --
	( 97.18,162.19) --
	( 96.56,161.67) --
	( 95.94,161.14) --
	( 95.34,160.60) --
	( 94.75,160.05) --
	( 94.17,159.49) --
	( 93.59,158.92) --
	( 93.03,158.34) --
	( 92.48,157.75) --
	( 91.94,157.15) --
	( 91.40,156.54) --
	( 90.88,155.92) --
	( 90.37,155.30) --
	( 89.87,154.66) --
	( 89.39,154.02) --
	( 88.91,153.36) --
	( 88.45,152.70) --
	( 87.99,152.03) --
	( 87.55,151.35) --
	( 87.12,150.67) --
	( 86.70,149.98) --
	( 86.30,149.28) --
	( 85.91,148.57) --
	( 85.53,147.86) --
	( 85.16,147.14) --
	( 84.80,146.41) --
	( 84.46,145.68) --
	( 84.13,144.94) --
	( 83.81,144.20) --
	( 83.51,143.45) --
	( 83.22,142.69) --
	( 82.94,141.93) --
	( 82.68,141.17) --
	( 82.43,140.40) --
	( 82.19,139.63) --
	( 81.97,138.85) --
	( 81.76,138.07) --
	( 81.56,137.29) --
	( 81.38,136.50) --
	( 81.21,135.71) --
	( 81.06,134.91) --
	( 80.91,134.12) --
	( 80.79,133.32) --
	( 80.67,132.52) --
	( 80.58,131.72) --
	( 80.49,130.91) --
	( 80.42,130.11) --
	( 80.36,129.30) --
	( 80.32,128.49) --
	( 80.29,127.69) --
	( 80.28,126.88) --
	( 80.28,126.07) --
	( 80.29,125.26) --
	( 80.32,124.45) --
	( 80.36,123.64) --
	( 80.42,122.84) --
	( 80.49,122.03) --
	( 80.58,121.23) --
	( 80.67,120.43) --
	( 80.79,119.63) --
	( 80.91,118.83) --
	( 81.06,118.03) --
	( 81.21,117.24) --
	( 81.38,116.45) --
	( 81.56,115.66) --
	( 81.76,114.87) --
	( 81.97,114.09) --
	( 82.19,113.32) --
	( 82.43,112.54) --
	( 82.68,111.78) --
	( 82.94,111.01) --
	( 83.22,110.25) --
	( 83.51,109.50) --
	( 83.81,108.75) --
	( 84.13,108.00) --
	( 84.46,107.27) --
	( 84.80,106.53) --
	( 85.16,105.81) --
	( 85.53,105.09) --
	( 85.91,104.37) --
	( 86.30,103.67) --
	( 86.70,102.97) --
	( 87.12,102.28) --
	( 87.55,101.59) --
	( 87.99,100.91) --
	( 88.45,100.24) --
	( 88.91, 99.58) --
	( 89.39, 98.93) --
	( 89.87, 98.28) --
	( 90.37, 97.65) --
	( 90.88, 97.02) --
	( 91.40, 96.40) --
	( 91.94, 95.79) --
	( 92.48, 95.19) --
	( 93.03, 94.60) --
	( 93.59, 94.02) --
	( 94.17, 93.45) --
	( 94.75, 92.89) --
	( 95.34, 92.34) --
	( 95.94, 91.80) --
	( 96.56, 91.27) --
	( 97.18, 90.76) --
	( 97.81, 90.25) --
	( 98.44, 89.75) --
	( 99.09, 89.27) --
	( 99.75, 88.79) --
	(100.41, 88.33) --
	(101.08, 87.88) --
	(101.76, 87.44) --
	(102.45, 87.02) --
	(103.14, 86.60) --
	(103.84, 86.20) --
	(104.55, 85.81) --
	(105.27, 85.43) --
	(105.99, 85.07) --
	(106.72, 84.72) --
	(107.45, 84.38) --
	(108.19, 84.05) --
	(108.93, 83.74) --
	(109.69, 83.44) --
	(110.44, 83.15) --
	(111.20, 82.87) --
	(111.97, 82.61) --
	(112.74, 82.37) --
	(113.51, 82.13) --
	(114.29, 81.91) --
	(115.07, 81.71) --
	(115.86, 81.51) --
	(116.64, 81.34) --
	(117.44, 81.17) --
	(118.23, 81.02) --
	(119.03, 80.88) --
	(119.83, 80.76) --
	(120.63, 80.65) --
	(121.43, 80.55) --
	(122.23, 80.47) --
	(123.04, 80.41) --
	(123.85, 80.35) --
	(124.65, 80.31) --
	(125.46, 80.29) --
	(126.27, 80.28) --
	(127.08, 80.28) --
	(127.89, 80.30) --
	(128.70, 80.33) --
	(129.50, 80.38) --
	(130.31, 80.44) --
	(131.11, 80.51) --
	(131.92, 80.60) --
	(132.72, 80.70) --
	(133.52, 80.82) --
	(134.32, 80.95) --
	(135.11, 81.09) --
	(135.91, 81.25) --
	(136.70, 81.42) --
	(137.48, 81.61) --
	(138.27, 81.81) --
	(139.05, 82.02) --
	(139.82, 82.25) --
	(140.59, 82.49) --
	(141.36, 82.74) --
	(142.12, 83.01) --
	(142.88, 83.29) --
	(143.64, 83.58) --
	(144.38, 83.89) --
	(145.13, 84.21) --
	(145.86, 84.54) --
	(146.59, 84.89) --
	(147.32, 85.25) --
	(148.04, 85.62) --
	(148.75, 86.00) --
	(149.45, 86.40) --
	(150.15, 86.81) --
	(150.84, 87.23) --
	(151.52, 87.66) --
	(152.20, 88.10) --
	(152.87, 88.56) --
	(153.53, 89.03) --
	(154.18, 89.51) --
	(154.82, 90.00) --
	(155.45, 90.50) --
	(156.08, 91.01) --
	(156.70, 91.54) --
	(157.30, 92.07) --
	(157.90, 92.62) --
	(158.49, 93.17) --
	(159.07, 93.74) --
	(159.63, 94.31) --
	(160.19, 94.90) --
	(160.74, 95.49) --
	(161.28, 96.10) --
	(161.80, 96.71) --
	(162.32, 97.33) --
	(162.82, 97.96) --
	(163.32, 98.61) --
	(163.80, 99.25) --
	(164.27, 99.91) --
	(164.73,100.58) --
	(165.17,101.25) --
	(165.61,101.93) --
	(166.03,102.62) --
	(166.44,103.32) --
	(166.84,104.02) --
	(167.23,104.73) --
	(167.60,105.45) --
	(167.97,106.17) --
	(168.32,106.90) --
	(168.65,107.63) --
	(168.97,108.38) --
	(169.29,109.12) --
	(169.58,109.87) --
	(169.87,110.63) --
	(170.14,111.39) --
	(170.39,112.16) --
	(170.64,112.93) --
	(170.87,113.70) --
	(171.08,114.48) --
	(171.29,115.27) --
	(171.48,116.05) --
	(171.65,116.84) --
	(171.81,117.63) --
	(171.96,118.43) --
	(172.10,119.23) --
	(172.22,120.03) --
	(172.32,120.83) --
	(172.41,121.63) --
	(172.49,122.44) --
	(172.55,123.24) --
	(172.60,124.05) --
	(172.64,124.86) --
	(172.66,125.66) --
	(172.67,126.47);
\definecolor[named]{drawColor}{rgb}{0.00,0.00,0.00}

\path[draw=drawColor,line width= 0.4pt,dash pattern=on 1pt off 3pt ,line join=round,line cap=round] (177.29,126.47) --
	(177.28,127.36) --
	(177.26,128.25) --
	(177.22,129.14) --
	(177.16,130.03) --
	(177.09,130.91) --
	(177.01,131.80) --
	(176.91,132.68) --
	(176.79,133.56) --
	(176.66,134.44) --
	(176.51,135.32) --
	(176.35,136.20) --
	(176.17,137.07) --
	(175.98,137.93) --
	(175.77,138.80) --
	(175.55,139.66) --
	(175.31,140.52) --
	(175.05,141.37) --
	(174.79,142.22) --
	(174.50,143.06) --
	(174.21,143.90) --
	(173.89,144.73) --
	(173.57,145.56) --
	(173.23,146.38) --
	(172.87,147.19) --
	(172.50,148.00) --
	(172.12,148.81) --
	(171.72,149.60) --
	(171.31,150.39) --
	(170.88,151.17) --
	(170.44,151.94) --
	(169.99,152.71) --
	(169.52,153.47) --
	(169.04,154.22) --
	(168.55,154.96) --
	(168.05,155.69) --
	(167.53,156.41) --
	(167.00,157.13) --
	(166.46,157.83) --
	(165.90,158.53) --
	(165.34,159.21) --
	(164.76,159.89) --
	(164.17,160.55) --
	(163.56,161.21) --
	(162.95,161.85) --
	(162.33,162.48) --
	(161.69,163.10) --
	(161.04,163.71) --
	(160.39,164.31) --
	(159.72,164.90) --
	(159.04,165.48) --
	(158.35,166.04) --
	(157.66,166.59) --
	(156.95,167.13) --
	(156.23,167.66) --
	(155.51,168.18) --
	(154.77,168.68) --
	(154.03,169.17) --
	(153.28,169.64) --
	(152.52,170.10) --
	(151.75,170.55) --
	(150.98,170.99) --
	(150.19,171.41) --
	(149.40,171.82) --
	(148.61,172.21) --
	(147.80,172.59) --
	(146.99,172.96) --
	(146.17,173.31) --
	(145.35,173.65) --
	(144.52,173.97) --
	(143.69,174.28) --
	(142.85,174.58) --
	(142.01,174.85) --
	(141.16,175.12) --
	(140.30,175.37) --
	(139.45,175.60) --
	(138.58,175.82) --
	(137.72,176.03) --
	(136.85,176.22) --
	(135.98,176.39) --
	(135.10,176.55) --
	(134.22,176.69) --
	(133.34,176.82) --
	(132.46,176.93) --
	(131.58,177.03) --
	(130.69,177.11) --
	(129.81,177.18) --
	(128.92,177.23) --
	(128.03,177.26) --
	(127.14,177.28) --
	(126.25,177.29) --
	(125.36,177.27) --
	(124.47,177.25) --
	(123.58,177.20) --
	(122.70,177.15) --
	(121.81,177.07) --
	(120.93,176.98) --
	(120.04,176.88) --
	(119.16,176.76) --
	(118.28,176.62) --
	(117.41,176.47) --
	(116.53,176.31) --
	(115.66,176.12) --
	(114.79,175.93) --
	(113.93,175.71) --
	(113.07,175.49) --
	(112.21,175.25) --
	(111.36,174.99) --
	(110.52,174.72) --
	(109.67,174.43) --
	(108.84,174.13) --
	(108.01,173.81) --
	(107.18,173.48) --
	(106.36,173.14) --
	(105.55,172.78) --
	(104.74,172.41) --
	(103.94,172.02) --
	(103.15,171.62) --
	(102.36,171.20) --
	(101.58,170.77) --
	(100.81,170.33) --
	(100.05,169.87) --
	( 99.29,169.41) --
	( 98.54,168.92) --
	( 97.80,168.43) --
	( 97.07,167.92) --
	( 96.35,167.40) --
	( 95.64,166.87) --
	( 94.94,166.32) --
	( 94.25,165.76) --
	( 93.56,165.19) --
	( 92.89,164.61) --
	( 92.23,164.02) --
	( 91.58,163.41) --
	( 90.94,162.79) --
	( 90.31,162.17) --
	( 89.69,161.53) --
	( 89.08,160.88) --
	( 88.48,160.22) --
	( 87.90,159.55) --
	( 87.32,158.87) --
	( 86.76,158.18) --
	( 86.22,157.48) --
	( 85.68,156.77) --
	( 85.15,156.05) --
	( 84.64,155.32) --
	( 84.14,154.59) --
	( 83.66,153.84) --
	( 83.19,153.09) --
	( 82.73,152.33) --
	( 82.28,151.56) --
	( 81.85,150.78) --
	( 81.43,150.00) --
	( 81.03,149.20) --
	( 80.64,148.41) --
	( 80.26,147.60) --
	( 79.90,146.79) --
	( 79.55,145.97) --
	( 79.21,145.15) --
	( 78.89,144.32) --
	( 78.59,143.48) --
	( 78.30,142.64) --
	( 78.02,141.79) --
	( 77.76,140.94) --
	( 77.52,140.09) --
	( 77.29,139.23) --
	( 77.07,138.37) --
	( 76.87,137.50) --
	( 76.68,136.63) --
	( 76.51,135.76) --
	( 76.36,134.88) --
	( 76.22,134.00) --
	( 76.10,133.12) --
	( 75.99,132.24) --
	( 75.89,131.36) --
	( 75.82,130.47) --
	( 75.75,129.58) --
	( 75.71,128.70) --
	( 75.68,127.81) --
	( 75.66,126.92) --
	( 75.66,126.03) --
	( 75.68,125.14) --
	( 75.71,124.25) --
	( 75.75,123.36) --
	( 75.82,122.47) --
	( 75.89,121.59) --
	( 75.99,120.70) --
	( 76.10,119.82) --
	( 76.22,118.94) --
	( 76.36,118.06) --
	( 76.51,117.19) --
	( 76.68,116.31) --
	( 76.87,115.44) --
	( 77.07,114.58) --
	( 77.29,113.71) --
	( 77.52,112.86) --
	( 77.76,112.00) --
	( 78.02,111.15) --
	( 78.30,110.31) --
	( 78.59,109.46) --
	( 78.89,108.63) --
	( 79.21,107.80) --
	( 79.55,106.98) --
	( 79.90,106.16) --
	( 80.26,105.35) --
	( 80.64,104.54) --
	( 81.03,103.74) --
	( 81.43,102.95) --
	( 81.85,102.16) --
	( 82.28,101.39) --
	( 82.73,100.62) --
	( 83.19, 99.86) --
	( 83.66, 99.10) --
	( 84.14, 98.36) --
	( 84.64, 97.62) --
	( 85.15, 96.89) --
	( 85.68, 96.17) --
	( 86.22, 95.47) --
	( 86.76, 94.77) --
	( 87.32, 94.08) --
	( 87.90, 93.40) --
	( 88.48, 92.73) --
	( 89.08, 92.07) --
	( 89.69, 91.42) --
	( 90.31, 90.78) --
	( 90.94, 90.15) --
	( 91.58, 89.53) --
	( 92.23, 88.93) --
	( 92.89, 88.34) --
	( 93.56, 87.75) --
	( 94.25, 87.18) --
	( 94.94, 86.63) --
	( 95.64, 86.08) --
	( 96.35, 85.55) --
	( 97.07, 85.03) --
	( 97.80, 84.52) --
	( 98.54, 84.02) --
	( 99.29, 83.54) --
	(100.05, 83.07) --
	(100.81, 82.61) --
	(101.58, 82.17) --
	(102.36, 81.74) --
	(103.15, 81.33) --
	(103.94, 80.93) --
	(104.74, 80.54) --
	(105.55, 80.17) --
	(106.36, 79.81) --
	(107.18, 79.46) --
	(108.01, 79.13) --
	(108.84, 78.82) --
	(109.67, 78.51) --
	(110.52, 78.23) --
	(111.36, 77.96) --
	(112.21, 77.70) --
	(113.07, 77.46) --
	(113.93, 77.23) --
	(114.79, 77.02) --
	(115.66, 76.82) --
	(116.53, 76.64) --
	(117.41, 76.47) --
	(118.28, 76.32) --
	(119.16, 76.19) --
	(120.04, 76.07) --
	(120.93, 75.96) --
	(121.81, 75.87) --
	(122.70, 75.80) --
	(123.58, 75.74) --
	(124.47, 75.70) --
	(125.36, 75.67) --
	(126.25, 75.66) --
	(127.14, 75.66) --
	(128.03, 75.68) --
	(128.92, 75.72) --
	(129.81, 75.77) --
	(130.69, 75.83) --
	(131.58, 75.92) --
	(132.46, 76.01) --
	(133.34, 76.12) --
	(134.22, 76.25) --
	(135.10, 76.40) --
	(135.98, 76.55) --
	(136.85, 76.73) --
	(137.72, 76.92) --
	(138.58, 77.12) --
	(139.45, 77.34) --
	(140.30, 77.58) --
	(141.16, 77.83) --
	(142.01, 78.09) --
	(142.85, 78.37) --
	(143.69, 78.66) --
	(144.52, 78.97) --
	(145.35, 79.30) --
	(146.17, 79.63) --
	(146.99, 79.99) --
	(147.80, 80.35) --
	(148.61, 80.73) --
	(149.40, 81.13) --
	(150.19, 81.53) --
	(150.98, 81.96) --
	(151.75, 82.39) --
	(152.52, 82.84) --
	(153.28, 83.30) --
	(154.03, 83.78) --
	(154.77, 84.27) --
	(155.51, 84.77) --
	(156.23, 85.28) --
	(156.95, 85.81) --
	(157.66, 86.35) --
	(158.35, 86.90) --
	(159.04, 87.47) --
	(159.72, 88.04) --
	(160.39, 88.63) --
	(161.04, 89.23) --
	(161.69, 89.84) --
	(162.33, 90.46) --
	(162.95, 91.10) --
	(163.56, 91.74) --
	(164.17, 92.39) --
	(164.76, 93.06) --
	(165.34, 93.73) --
	(165.90, 94.42) --
	(166.46, 95.11) --
	(167.00, 95.82) --
	(167.53, 96.53) --
	(168.05, 97.26) --
	(168.55, 97.99) --
	(169.04, 98.73) --
	(169.52, 99.48) --
	(169.99,100.24) --
	(170.44,101.00) --
	(170.88,101.77) --
	(171.31,102.56) --
	(171.72,103.34) --
	(172.12,104.14) --
	(172.50,104.94) --
	(172.87,105.75) --
	(173.23,106.57) --
	(173.57,107.39) --
	(173.89,108.21) --
	(174.21,109.05) --
	(174.50,109.88) --
	(174.79,110.73) --
	(175.05,111.58) --
	(175.31,112.43) --
	(175.55,113.28) --
	(175.77,114.15) --
	(175.98,115.01) --
	(176.17,115.88) --
	(176.35,116.75) --
	(176.51,117.62) --
	(176.66,118.50) --
	(176.79,119.38) --
	(176.91,120.26) --
	(177.01,121.15) --
	(177.09,122.03) --
	(177.16,122.92) --
	(177.22,123.81) --
	(177.26,124.69) --
	(177.28,125.58) --
	(177.29,126.47);

\path[fill=fillColor] (104.06,210.13) circle (  2.25);
\definecolor[named]{drawColor}{rgb}{0.00,0.00,1.00}

\path[draw=drawColor,line width= 1.2pt,line join=round,line cap=round] (150.25,210.13) --
	(150.24,210.94) --
	(150.22,211.75) --
	(150.19,212.56) --
	(150.14,213.36) --
	(150.07,214.17) --
	(150.00,214.97) --
	(149.90,215.78) --
	(149.80,216.58) --
	(149.68,217.38) --
	(149.55,218.18) --
	(149.40,218.97) --
	(149.24,219.76) --
	(149.06,220.55) --
	(148.87,221.34) --
	(148.67,222.12) --
	(148.45,222.90) --
	(148.22,223.68) --
	(147.98,224.45) --
	(147.72,225.21) --
	(147.45,225.97) --
	(147.17,226.73) --
	(146.87,227.48) --
	(146.56,228.23) --
	(146.24,228.97) --
	(145.90,229.71) --
	(145.55,230.44) --
	(145.19,231.16) --
	(144.81,231.88) --
	(144.43,232.59) --
	(144.03,233.29) --
	(143.62,233.98) --
	(143.19,234.67) --
	(142.76,235.35) --
	(142.31,236.03) --
	(141.85,236.69) --
	(141.38,237.35) --
	(140.90,238.00) --
	(140.41,238.64) --
	(139.90,239.27) --
	(139.39,239.89) --
	(138.86,240.51) --
	(138.32,241.11) --
	(137.77,241.71) --
	(137.22,242.29) --
	(136.65,242.87) --
	(136.07,243.43) --
	(135.48,243.99) --
	(134.89,244.53) --
	(134.28,245.07) --
	(133.66,245.59) --
	(133.04,246.10) --
	(132.40,246.61) --
	(131.76,247.10) --
	(131.11,247.58) --
	(130.45,248.04) --
	(129.78,248.50) --
	(129.11,248.94) --
	(128.42,249.38) --
	(127.73,249.80) --
	(127.04,250.21) --
	(126.33,250.60) --
	(125.62,250.99) --
	(124.90,251.36) --
	(124.18,251.72) --
	(123.45,252.06) --
	(122.71,252.39) --
	(121.97,252.71) --
	(121.40,252.94);

\path[draw=drawColor,line width= 1.2pt,line join=round,line cap=round] ( 86.71,252.94) --
	( 86.52,252.87) --
	( 85.77,252.56) --
	( 85.03,252.23) --
	( 84.30,251.89) --
	( 83.57,251.54) --
	( 82.85,251.17) --
	( 82.14,250.80) --
	( 81.43,250.41) --
	( 80.73,250.00) --
	( 80.03,249.59) --
	( 79.34,249.16) --
	( 78.67,248.72) --
	( 77.99,248.27) --
	( 77.33,247.81) --
	( 76.67,247.34) --
	( 76.03,246.85) --
	( 75.39,246.36) --
	( 74.76,245.85) --
	( 74.14,245.33) --
	( 73.53,244.80) --
	( 72.93,244.26) --
	( 72.33,243.71) --
	( 71.75,243.15) --
	( 71.18,242.58) --
	( 70.61,242.00) --
	( 70.06,241.41) --
	( 69.52,240.81) --
	( 68.99,240.20) --
	( 68.47,239.58) --
	( 67.96,238.96) --
	( 67.46,238.32) --
	( 66.97,237.68) --
	( 66.49,237.02) --
	( 66.03,236.36) --
	( 65.58,235.69) --
	( 65.13,235.01) --
	( 64.71,234.33) --
	( 64.29,233.64) --
	( 63.88,232.94) --
	( 63.49,232.23) --
	( 63.11,231.52) --
	( 62.74,230.80) --
	( 62.39,230.07) --
	( 62.04,229.34) --
	( 61.71,228.60) --
	( 61.40,227.86) --
	( 61.09,227.11) --
	( 60.80,226.35) --
	( 60.53,225.59) --
	( 60.26,224.83) --
	( 60.01,224.06) --
	( 59.77,223.29) --
	( 59.55,222.51) --
	( 59.34,221.73) --
	( 59.14,220.95) --
	( 58.96,220.16) --
	( 58.79,219.37) --
	( 58.64,218.57) --
	( 58.50,217.78) --
	( 58.37,216.98) --
	( 58.26,216.18) --
	( 58.16,215.38) --
	( 58.07,214.57) --
	( 58.00,213.77) --
	( 57.95,212.96) --
	( 57.91,212.15) --
	( 57.88,211.35) --
	( 57.86,210.54) --
	( 57.86,209.73) --
	( 57.88,208.92) --
	( 57.91,208.11) --
	( 57.95,207.30) --
	( 58.00,206.50) --
	( 58.07,205.69) --
	( 58.16,204.89) --
	( 58.26,204.09) --
	( 58.37,203.29) --
	( 58.50,202.49) --
	( 58.64,201.69) --
	( 58.79,200.90) --
	( 58.96,200.11) --
	( 59.14,199.32) --
	( 59.34,198.53) --
	( 59.55,197.75) --
	( 59.77,196.98) --
	( 60.01,196.20) --
	( 60.26,195.44) --
	( 60.53,194.67) --
	( 60.80,193.91) --
	( 61.09,193.16) --
	( 61.40,192.41) --
	( 61.71,191.66) --
	( 62.04,190.93) --
	( 62.39,190.19) --
	( 62.74,189.47) --
	( 63.11,188.75) --
	( 63.49,188.03) --
	( 63.88,187.33) --
	( 64.29,186.63) --
	( 64.71,185.94) --
	( 65.13,185.25) --
	( 65.58,184.57) --
	( 66.03,183.90) --
	( 66.49,183.24) --
	( 66.97,182.59) --
	( 67.46,181.94) --
	( 67.96,181.31) --
	( 68.47,180.68) --
	( 68.99,180.06) --
	( 69.52,179.45) --
	( 70.06,178.85) --
	( 70.61,178.26) --
	( 71.18,177.68) --
	( 71.75,177.11) --
	( 72.33,176.55) --
	( 72.93,176.00) --
	( 73.53,175.46) --
	( 74.14,174.93) --
	( 74.76,174.42) --
	( 75.39,173.91) --
	( 76.03,173.41) --
	( 76.67,172.93) --
	( 77.33,172.45) --
	( 77.99,171.99) --
	( 78.67,171.54) --
	( 79.34,171.10) --
	( 80.03,170.68) --
	( 80.73,170.26) --
	( 81.43,169.86) --
	( 82.14,169.47) --
	( 82.85,169.09) --
	( 83.57,168.73) --
	( 84.30,168.38) --
	( 85.03,168.04) --
	( 85.77,167.71) --
	( 86.52,167.40) --
	( 87.27,167.10) --
	( 88.02,166.81) --
	( 88.79,166.53) --
	( 89.55,166.27) --
	( 90.32,166.03) --
	( 91.09,165.79) --
	( 91.87,165.57) --
	( 92.65,165.37) --
	( 93.44,165.17) --
	( 94.23,165.00) --
	( 95.02,164.83) --
	( 95.81,164.68) --
	( 96.61,164.54) --
	( 97.41,164.42) --
	( 98.21,164.31) --
	( 99.01,164.21) --
	( 99.82,164.13) --
	(100.62,164.07) --
	(101.43,164.01) --
	(102.24,163.97) --
	(103.05,163.95) --
	(103.85,163.94) --
	(104.66,163.94) --
	(105.47,163.96) --
	(106.28,163.99) --
	(107.09,164.04) --
	(107.89,164.10) --
	(108.70,164.17) --
	(109.50,164.26) --
	(110.30,164.36) --
	(111.10,164.48) --
	(111.90,164.61) --
	(112.70,164.75) --
	(113.49,164.91) --
	(114.28,165.08) --
	(115.07,165.27) --
	(115.85,165.47) --
	(116.63,165.68) --
	(117.41,165.91) --
	(118.18,166.15) --
	(118.94,166.40) --
	(119.71,166.67) --
	(120.47,166.95) --
	(121.22,167.24) --
	(121.97,167.55) --
	(122.71,167.87) --
	(123.45,168.20) --
	(124.18,168.55) --
	(124.90,168.91) --
	(125.62,169.28) --
	(126.33,169.66) --
	(127.04,170.06) --
	(127.73,170.47) --
	(128.42,170.89) --
	(129.11,171.32) --
	(129.78,171.76) --
	(130.45,172.22) --
	(131.11,172.69) --
	(131.76,173.17) --
	(132.40,173.66) --
	(133.04,174.16) --
	(133.66,174.67) --
	(134.28,175.20) --
	(134.89,175.73) --
	(135.48,176.28) --
	(136.07,176.83) --
	(136.65,177.40) --
	(137.22,177.97) --
	(137.77,178.56) --
	(138.32,179.15) --
	(138.86,179.76) --
	(139.39,180.37) --
	(139.90,180.99) --
	(140.41,181.62) --
	(140.90,182.27) --
	(141.38,182.91) --
	(141.85,183.57) --
	(142.31,184.24) --
	(142.76,184.91) --
	(143.19,185.59) --
	(143.62,186.28) --
	(144.03,186.98) --
	(144.43,187.68) --
	(144.81,188.39) --
	(145.19,189.11) --
	(145.55,189.83) --
	(145.90,190.56) --
	(146.24,191.29) --
	(146.56,192.04) --
	(146.87,192.78) --
	(147.17,193.53) --
	(147.45,194.29) --
	(147.72,195.05) --
	(147.98,195.82) --
	(148.22,196.59) --
	(148.45,197.36) --
	(148.67,198.14) --
	(148.87,198.93) --
	(149.06,199.71) --
	(149.24,200.50) --
	(149.40,201.29) --
	(149.55,202.09) --
	(149.68,202.89) --
	(149.80,203.69) --
	(149.90,204.49) --
	(150.00,205.29) --
	(150.07,206.10) --
	(150.14,206.90) --
	(150.19,207.71) --
	(150.22,208.52) --
	(150.24,209.32) --
	(150.25,210.13);
\definecolor[named]{drawColor}{rgb}{0.00,0.00,0.00}

\path[draw=drawColor,line width= 0.4pt,dash pattern=on 1pt off 3pt ,line join=round,line cap=round] (154.87,210.13) --
	(154.86,211.02) --
	(154.84,211.91) --
	(154.80,212.80) --
	(154.75,213.69) --
	(154.68,214.57) --
	(154.59,215.46) --
	(154.49,216.34) --
	(154.37,217.22) --
	(154.24,218.10) --
	(154.09,218.98) --
	(153.93,219.86) --
	(153.75,220.73) --
	(153.56,221.59) --
	(153.35,222.46) --
	(153.13,223.32) --
	(152.89,224.18) --
	(152.64,225.03) --
	(152.37,225.88) --
	(152.09,226.72) --
	(151.79,227.56) --
	(151.48,228.39) --
	(151.15,229.22) --
	(150.81,230.04) --
	(150.45,230.85) --
	(150.08,231.66) --
	(149.70,232.47) --
	(149.30,233.26) --
	(148.89,234.05) --
	(148.46,234.83) --
	(148.03,235.60) --
	(147.57,236.37) --
	(147.11,237.13) --
	(146.63,237.88) --
	(146.14,238.62) --
	(145.63,239.35) --
	(145.11,240.07) --
	(144.58,240.79) --
	(144.04,241.49) --
	(143.49,242.19) --
	(142.92,242.87) --
	(142.34,243.55) --
	(141.75,244.21) --
	(141.15,244.87) --
	(140.53,245.51) --
	(139.91,246.14) --
	(139.27,246.76) --
	(138.63,247.37) --
	(137.97,247.97) --
	(137.30,248.56) --
	(136.62,249.14) --
	(135.94,249.70) --
	(135.24,250.25) --
	(134.53,250.79) --
	(133.82,251.32) --
	(133.09,251.84) --
	(132.36,252.34) --
	(131.61,252.83) --
	(131.42,252.94);

\path[draw=drawColor,line width= 0.4pt,dash pattern=on 1pt off 3pt ,line join=round,line cap=round] ( 76.69,252.94) --
	( 76.13,252.58) --
	( 75.39,252.09) --
	( 74.66,251.58) --
	( 73.94,251.06) --
	( 73.23,250.53) --
	( 72.52,249.98) --
	( 71.83,249.42) --
	( 71.15,248.85) --
	( 70.48,248.27) --
	( 69.81,247.68) --
	( 69.16,247.07) --
	( 68.52,246.45) --
	( 67.89,245.83) --
	( 67.27,245.19) --
	( 66.66,244.54) --
	( 66.07,243.88) --
	( 65.48,243.21) --
	( 64.91,242.53) --
	( 64.35,241.84) --
	( 63.80,241.14) --
	( 63.26,240.43) --
	( 62.74,239.71) --
	( 62.23,238.98) --
	( 61.73,238.25) --
	( 61.24,237.50) --
	( 60.77,236.75) --
	( 60.31,235.99) --
	( 59.87,235.22) --
	( 59.43,234.44) --
	( 59.01,233.66) --
	( 58.61,232.86) --
	( 58.22,232.07) --
	( 57.84,231.26) --
	( 57.48,230.45) --
	( 57.13,229.63) --
	( 56.80,228.81) --
	( 56.48,227.98) --
	( 56.17,227.14) --
	( 55.88,226.30) --
	( 55.61,225.45) --
	( 55.35,224.60) --
	( 55.10,223.75) --
	( 54.87,222.89) --
	( 54.65,222.03) --
	( 54.45,221.16) --
	( 54.27,220.29) --
	( 54.10,219.42) --
	( 53.94,218.54) --
	( 53.80,217.66) --
	( 53.68,216.78) --
	( 53.57,215.90) --
	( 53.48,215.02) --
	( 53.40,214.13) --
	( 53.34,213.24) --
	( 53.29,212.36) --
	( 53.26,211.47) --
	( 53.24,210.58) --
	( 53.24,209.69) --
	( 53.26,208.80) --
	( 53.29,207.91) --
	( 53.34,207.02) --
	( 53.40,206.13) --
	( 53.48,205.25) --
	( 53.57,204.36) --
	( 53.68,203.48) --
	( 53.80,202.60) --
	( 53.94,201.72) --
	( 54.10,200.85) --
	( 54.27,199.97) --
	( 54.45,199.10) --
	( 54.65,198.24) --
	( 54.87,197.37) --
	( 55.10,196.52) --
	( 55.35,195.66) --
	( 55.61,194.81) --
	( 55.88,193.97) --
	( 56.17,193.12) --
	( 56.48,192.29) --
	( 56.80,191.46) --
	( 57.13,190.64) --
	( 57.48,189.82) --
	( 57.84,189.01) --
	( 58.22,188.20) --
	( 58.61,187.40) --
	( 59.01,186.61) --
	( 59.43,185.82) --
	( 59.87,185.05) --
	( 60.31,184.28) --
	( 60.77,183.52) --
	( 61.24,182.76) --
	( 61.73,182.02) --
	( 62.23,181.28) --
	( 62.74,180.55) --
	( 63.26,179.83) --
	( 63.80,179.13) --
	( 64.35,178.43) --
	( 64.91,177.74) --
	( 65.48,177.06) --
	( 66.07,176.39) --
	( 66.66,175.73) --
	( 67.27,175.08) --
	( 67.89,174.44) --
	( 68.52,173.81) --
	( 69.16,173.19) --
	( 69.81,172.59) --
	( 70.48,172.00) --
	( 71.15,171.41) --
	( 71.83,170.84) --
	( 72.52,170.29) --
	( 73.23,169.74) --
	( 73.94,169.21) --
	( 74.66,168.69) --
	( 75.39,168.18) --
	( 76.13,167.68) --
	( 76.87,167.20) --
	( 77.63,166.73) --
	( 78.39,166.27) --
	( 79.16,165.83) --
	( 79.94,165.40) --
	( 80.73,164.99) --
	( 81.52,164.59) --
	( 82.32,164.20) --
	( 83.13,163.83) --
	( 83.94,163.47) --
	( 84.76,163.12) --
	( 85.59,162.79) --
	( 86.42,162.48) --
	( 87.26,162.17) --
	( 88.10,161.89) --
	( 88.95,161.62) --
	( 89.80,161.36) --
	( 90.65,161.12) --
	( 91.51,160.89) --
	( 92.38,160.68) --
	( 93.24,160.48) --
	( 94.12,160.30) --
	( 94.99,160.13) --
	( 95.87,159.98) --
	( 96.74,159.85) --
	( 97.63,159.73) --
	( 98.51,159.62) --
	( 99.39,159.53) --
	(100.28,159.46) --
	(101.17,159.40) --
	(102.06,159.36) --
	(102.94,159.33) --
	(103.83,159.32) --
	(104.72,159.32) --
	(105.61,159.34) --
	(106.50,159.38) --
	(107.39,159.43) --
	(108.28,159.49) --
	(109.16,159.58) --
	(110.05,159.67) --
	(110.93,159.78) --
	(111.81,159.91) --
	(112.69,160.06) --
	(113.56,160.21) --
	(114.43,160.39) --
	(115.30,160.58) --
	(116.17,160.78) --
	(117.03,161.00) --
	(117.89,161.24) --
	(118.74,161.49) --
	(119.59,161.75) --
	(120.43,162.03) --
	(121.27,162.32) --
	(122.11,162.63) --
	(122.94,162.96) --
	(123.76,163.29) --
	(124.57,163.65) --
	(125.39,164.01) --
	(126.19,164.39) --
	(126.99,164.79) --
	(127.78,165.19) --
	(128.56,165.62) --
	(129.33,166.05) --
	(130.10,166.50) --
	(130.86,166.96) --
	(131.61,167.44) --
	(132.36,167.93) --
	(133.09,168.43) --
	(133.82,168.94) --
	(134.53,169.47) --
	(135.24,170.01) --
	(135.94,170.56) --
	(136.62,171.13) --
	(137.30,171.70) --
	(137.97,172.29) --
	(138.63,172.89) --
	(139.27,173.50) --
	(139.91,174.12) --
	(140.53,174.76) --
	(141.15,175.40) --
	(141.75,176.05) --
	(142.34,176.72) --
	(142.92,177.39) --
	(143.49,178.08) --
	(144.04,178.77) --
	(144.58,179.48) --
	(145.11,180.19) --
	(145.63,180.92) --
	(146.14,181.65) --
	(146.63,182.39) --
	(147.11,183.14) --
	(147.57,183.90) --
	(148.03,184.66) --
	(148.46,185.43) --
	(148.89,186.22) --
	(149.30,187.00) --
	(149.70,187.80) --
	(150.08,188.60) --
	(150.45,189.41) --
	(150.81,190.23) --
	(151.15,191.05) --
	(151.48,191.87) --
	(151.79,192.71) --
	(152.09,193.54) --
	(152.37,194.39) --
	(152.64,195.24) --
	(152.89,196.09) --
	(153.13,196.94) --
	(153.35,197.81) --
	(153.56,198.67) --
	(153.75,199.54) --
	(153.93,200.41) --
	(154.09,201.28) --
	(154.24,202.16) --
	(154.37,203.04) --
	(154.49,203.92) --
	(154.59,204.81) --
	(154.68,205.69) --
	(154.75,206.58) --
	(154.80,207.47) --
	(154.84,208.35) --
	(154.86,209.24) --
	(154.87,210.13);
\definecolor[named]{drawColor}{rgb}{0.00,0.00,1.00}

\path[draw=drawColor,line width= 1.2pt,line join=round,line cap=round] ( 60.07,252.94) --
	( 60.43,252.75) --
	( 61.15,252.39) --
	( 61.88,252.04) --
	( 62.62,251.70) --
	( 63.36,251.37) --
	( 64.10,251.06) --
	( 64.85,250.76) --
	( 65.61,250.47) --
	( 66.37,250.19) --
	( 67.13,249.93) --
	( 67.90,249.69) --
	( 68.68,249.45) --
	( 69.46,249.23) --
	( 70.24,249.03) --
	( 71.02,248.83) --
	( 71.81,248.66) --
	( 72.60,248.49) --
	( 73.40,248.34) --
	( 74.19,248.20) --
	( 74.99,248.08) --
	( 75.79,247.97) --
	( 76.60,247.87) --
	( 77.40,247.79) --
	( 78.21,247.73) --
	( 79.01,247.67) --
	( 79.82,247.63) --
	( 80.63,247.61) --
	( 81.44,247.60) --
	( 82.25,247.60) --
	( 83.05,247.62) --
	( 83.86,247.65) --
	( 84.67,247.70) --
	( 85.48,247.76) --
	( 86.28,247.83) --
	( 87.08,247.92) --
	( 87.89,248.02) --
	( 88.69,248.14) --
	( 89.48,248.27) --
	( 90.28,248.41) --
	( 91.07,248.57) --
	( 91.86,248.74) --
	( 92.65,248.93) --
	( 93.43,249.13) --
	( 94.21,249.34) --
	( 94.99,249.57) --
	( 95.76,249.81) --
	( 96.53,250.06) --
	( 97.29,250.33) --
	( 98.05,250.61) --
	( 98.80,250.90) --
	( 99.55,251.21) --
	(100.29,251.53) --
	(101.03,251.86) --
	(101.76,252.21) --
	(102.49,252.57) --
	(103.20,252.94) --
	(103.21,252.94);
\definecolor[named]{drawColor}{rgb}{0.00,0.00,0.00}

\path[draw=drawColor,line width= 0.4pt,dash pattern=on 1pt off 3pt ,line join=round,line cap=round] ( 51.42,252.94) --
	( 51.52,252.87) --
	( 52.24,252.35) --
	( 52.97,251.84) --
	( 53.71,251.34) --
	( 54.46,250.86) --
	( 55.21,250.39) --
	( 55.98,249.93) --
	( 56.75,249.49) --
	( 57.53,249.06) --
	( 58.31,248.65) --
	( 59.11,248.25) --
	( 59.91,247.86) --
	( 60.71,247.49) --
	( 61.53,247.13) --
	( 62.35,246.78) --
	( 63.17,246.45) --
	( 64.00,246.14) --
	( 64.84,245.83) --
	( 65.68,245.55) --
	( 66.53,245.28) --
	( 67.38,245.02) --
	( 68.24,244.78) --
	( 69.10,244.55) --
	( 69.96,244.34) --
	( 70.83,244.14) --
	( 71.70,243.96) --
	( 72.57,243.79) --
	( 73.45,243.64) --
	( 74.33,243.51) --
	( 75.21,243.39) --
	( 76.09,243.28) --
	( 76.98,243.19) --
	( 77.86,243.12) --
	( 78.75,243.06) --
	( 79.64,243.02) --
	( 80.53,242.99) --
	( 81.42,242.98) --
	( 82.31,242.98) --
	( 83.20,243.00) --
	( 84.08,243.04) --
	( 84.97,243.09) --
	( 85.86,243.15) --
	( 86.74,243.24) --
	( 87.63,243.33) --
	( 88.51,243.44) --
	( 89.39,243.57) --
	( 90.27,243.72) --
	( 91.14,243.87) --
	( 92.02,244.05) --
	( 92.88,244.24) --
	( 93.75,244.44) --
	( 94.61,244.66) --
	( 95.47,244.90) --
	( 96.32,245.15) --
	( 97.17,245.41) --
	( 98.02,245.69) --
	( 98.86,245.98) --
	( 99.69,246.29) --
	(100.52,246.62) --
	(101.34,246.95) --
	(102.16,247.31) --
	(102.97,247.67) --
	(103.77,248.05) --
	(104.57,248.45) --
	(105.36,248.85) --
	(106.14,249.28) --
	(106.92,249.71) --
	(107.69,250.16) --
	(108.45,250.62) --
	(109.20,251.10) --
	(109.94,251.59) --
	(110.67,252.09) --
	(111.40,252.60) --
	(111.86,252.94);
\definecolor[named]{drawColor}{rgb}{0.00,0.00,1.00}

\path[draw=drawColor,line width= 1.2pt,line join=round,line cap=round] (250.62,  0.00) --
	(250.61,  0.01) --
	(249.89,  0.38) --
	(249.17,  0.74) --
	(248.44,  1.08) --
	(247.70,  1.41) --
	(246.96,  1.73) --
	(246.21,  2.04) --
	(245.46,  2.33) --
	(244.70,  2.62) --
	(243.94,  2.88) --
	(243.17,  3.14) --
	(242.40,  3.38) --
	(241.62,  3.60) --
	(240.84,  3.82) --
	(240.06,  4.02) --
	(239.27,  4.20) --
	(238.48,  4.37) --
	(237.69,  4.53) --
	(236.89,  4.68) --
	(236.09,  4.81) --
	(235.29,  4.92) --
	(234.49,  5.03) --
	(233.69,  5.11) --
	(232.88,  5.19) --
	(232.08,  5.25) --
	(231.27,  5.29) --
	(230.46,  5.33) --
	(229.65,  5.34) --
	(228.84,  5.35) --
	(228.04,  5.34) --
	(227.23,  5.31) --
	(226.42,  5.27) --
	(225.61,  5.22) --
	(224.81,  5.15) --
	(224.00,  5.07) --
	(223.20,  4.98) --
	(222.40,  4.87) --
	(221.60,  4.74) --
	(220.80,  4.61) --
	(220.01,  4.45) --
	(219.22,  4.29) --
	(218.43,  4.11) --
	(217.64,  3.92) --
	(216.86,  3.71) --
	(216.08,  3.49) --
	(215.31,  3.26) --
	(214.54,  3.01) --
	(213.78,  2.75) --
	(213.02,  2.48) --
	(212.26,  2.19) --
	(211.51,  1.89) --
	(210.76,  1.58) --
	(210.02,  1.25) --
	(209.29,  0.91) --
	(208.56,  0.56) --
	(207.84,  0.19) --
	(207.48,  0.00);
\definecolor[named]{drawColor}{rgb}{0.00,0.00,0.00}

\path[draw=drawColor,line width= 0.4pt,dash pattern=on 1pt off 3pt ,line join=round,line cap=round] (252.94,  3.99) --
	(252.77,  4.09) --
	(251.98,  4.50) --
	(251.18,  4.89) --
	(250.38,  5.27) --
	(249.57,  5.64) --
	(248.75,  5.99) --
	(247.93,  6.33) --
	(247.10,  6.65) --
	(246.26,  6.96) --
	(245.42,  7.26) --
	(244.58,  7.53) --
	(243.73,  7.80) --
	(242.88,  8.05) --
	(242.02,  8.28) --
	(241.16,  8.50) --
	(240.29,  8.71) --
	(239.42,  8.90) --
	(238.55,  9.07) --
	(237.68,  9.23) --
	(236.80,  9.37) --
	(235.92,  9.50) --
	(235.04,  9.61) --
	(234.15,  9.71) --
	(233.27,  9.79) --
	(232.38,  9.86) --
	(231.49,  9.91) --
	(230.60,  9.94) --
	(229.71,  9.96) --
	(228.82,  9.97) --
	(227.93,  9.95) --
	(227.05,  9.93) --
	(226.16,  9.88) --
	(225.27,  9.83) --
	(224.38,  9.75) --
	(223.50,  9.66) --
	(222.62,  9.56) --
	(221.73,  9.44) --
	(220.86,  9.30) --
	(219.98,  9.15) --
	(219.11,  8.99) --
	(218.24,  8.80) --
	(217.37,  8.61) --
	(216.50,  8.39) --
	(215.64,  8.17) --
	(214.79,  7.93) --
	(213.94,  7.67) --
	(213.09,  7.40) --
	(212.25,  7.11) --
	(211.41,  6.81) --
	(210.58,  6.49) --
	(209.75,  6.16) --
	(208.94,  5.82) --
	(208.12,  5.46) --
	(207.31,  5.09) --
	(206.51,  4.70) --
	(205.72,  4.30) --
	(204.93,  3.88) --
	(204.15,  3.45) --
	(203.38,  3.01) --
	(202.62,  2.55) --
	(201.86,  2.09) --
	(201.12,  1.60) --
	(200.38,  1.11) --
	(199.65,  0.60) --
	(198.93,  0.08) --
	(198.82,  0.00);

\path[fill=fillColor] (206.63, 42.81) circle (  2.25);
\definecolor[named]{drawColor}{rgb}{0.00,0.00,1.00}

\path[draw=drawColor,line width= 1.2pt,line join=round,line cap=round] (252.82, 42.81) --
	(252.82, 43.62) --
	(252.80, 44.43) --
	(252.76, 45.24) --
	(252.71, 46.04) --
	(252.65, 46.85) --
	(252.57, 47.65) --
	(252.48, 48.46) --
	(252.37, 49.26) --
	(252.25, 50.06) --
	(252.12, 50.86) --
	(251.97, 51.65) --
	(251.81, 52.44) --
	(251.63, 53.23) --
	(251.45, 54.02) --
	(251.24, 54.80) --
	(251.03, 55.58) --
	(250.80, 56.36) --
	(250.55, 57.13) --
	(250.29, 57.89) --
	(250.02, 58.65) --
	(249.74, 59.41) --
	(249.44, 60.16) --
	(249.13, 60.91) --
	(248.81, 61.65) --
	(248.47, 62.39) --
	(248.12, 63.12) --
	(247.76, 63.84) --
	(247.39, 64.56) --
	(247.00, 65.27) --
	(246.60, 65.97) --
	(246.19, 66.66) --
	(245.77, 67.35) --
	(245.33, 68.03) --
	(244.88, 68.71) --
	(244.43, 69.37) --
	(243.95, 70.03) --
	(243.47, 70.68) --
	(242.98, 71.32) --
	(242.47, 71.95) --
	(241.96, 72.57) --
	(241.43, 73.19) --
	(240.90, 73.79) --
	(240.35, 74.39) --
	(239.79, 74.97) --
	(239.22, 75.55) --
	(238.65, 76.11) --
	(238.06, 76.67) --
	(237.46, 77.21) --
	(236.85, 77.75) --
	(236.24, 78.27) --
	(235.61, 78.78) --
	(234.98, 79.29) --
	(234.34, 79.78) --
	(233.68, 80.26) --
	(233.02, 80.72) --
	(232.36, 81.18) --
	(231.68, 81.62) --
	(231.00, 82.06) --
	(230.31, 82.48) --
	(229.61, 82.89) --
	(228.91, 83.28) --
	(228.19, 83.67) --
	(227.48, 84.04) --
	(226.75, 84.40) --
	(226.02, 84.74) --
	(225.28, 85.07) --
	(224.54, 85.39) --
	(223.79, 85.70) --
	(223.04, 85.99) --
	(222.28, 86.28) --
	(221.52, 86.54) --
	(220.75, 86.80) --
	(219.98, 87.04) --
	(219.20, 87.26) --
	(218.42, 87.48) --
	(217.64, 87.68) --
	(216.85, 87.86) --
	(216.06, 88.03) --
	(215.27, 88.19) --
	(214.47, 88.34) --
	(213.68, 88.47) --
	(212.88, 88.58) --
	(212.07, 88.69) --
	(211.27, 88.77) --
	(210.47, 88.85) --
	(209.66, 88.91) --
	(208.85, 88.95) --
	(208.04, 88.99) --
	(207.24, 89.00) --
	(206.43, 89.01) --
	(205.62, 89.00) --
	(204.81, 88.97) --
	(204.00, 88.93) --
	(203.20, 88.88) --
	(202.39, 88.81) --
	(201.59, 88.73) --
	(200.78, 88.64) --
	(199.98, 88.53) --
	(199.18, 88.40) --
	(198.39, 88.27) --
	(197.59, 88.11) --
	(196.80, 87.95) --
	(196.01, 87.77) --
	(195.23, 87.58) --
	(194.45, 87.37) --
	(193.67, 87.15) --
	(192.89, 86.92) --
	(192.12, 86.67) --
	(191.36, 86.41) --
	(190.60, 86.14) --
	(189.84, 85.85) --
	(189.09, 85.55) --
	(188.35, 85.24) --
	(187.61, 84.91) --
	(186.87, 84.57) --
	(186.15, 84.22) --
	(185.42, 83.85) --
	(184.71, 83.48) --
	(184.00, 83.09) --
	(183.30, 82.68) --
	(182.61, 82.27) --
	(181.92, 81.84) --
	(181.24, 81.40) --
	(180.57, 80.95) --
	(179.90, 80.49) --
	(179.25, 80.02) --
	(178.60, 79.53) --
	(177.96, 79.04) --
	(177.33, 78.53) --
	(176.71, 78.01) --
	(176.10, 77.48) --
	(175.50, 76.94) --
	(174.91, 76.39) --
	(174.32, 75.83) --
	(173.75, 75.26) --
	(173.19, 74.68) --
	(172.64, 74.09) --
	(172.09, 73.49) --
	(171.56, 72.88) --
	(171.04, 72.26) --
	(170.53, 71.64) --
	(170.03, 71.00) --
	(169.54, 70.36) --
	(169.07, 69.70) --
	(168.60, 69.04) --
	(168.15, 68.37) --
	(167.71, 67.69) --
	(167.28, 67.01) --
	(166.86, 66.32) --
	(166.46, 65.62) --
	(166.06, 64.91) --
	(165.68, 64.20) --
	(165.32, 63.48) --
	(164.96, 62.75) --
	(164.62, 62.02) --
	(164.29, 61.28) --
	(163.97, 60.54) --
	(163.67, 59.79) --
	(163.38, 59.03) --
	(163.10, 58.27) --
	(162.84, 57.51) --
	(162.58, 56.74) --
	(162.35, 55.97) --
	(162.12, 55.19) --
	(161.91, 54.41) --
	(161.72, 53.63) --
	(161.54, 52.84) --
	(161.37, 52.05) --
	(161.21, 51.25) --
	(161.07, 50.46) --
	(160.95, 49.66) --
	(160.83, 48.86) --
	(160.73, 48.06) --
	(160.65, 47.25) --
	(160.58, 46.45) --
	(160.52, 45.64) --
	(160.48, 44.83) --
	(160.45, 44.03) --
	(160.44, 43.22) --
	(160.44, 42.41) --
	(160.45, 41.60) --
	(160.48, 40.79) --
	(160.52, 39.98) --
	(160.58, 39.18) --
	(160.65, 38.37) --
	(160.73, 37.57) --
	(160.83, 36.77) --
	(160.95, 35.97) --
	(161.07, 35.17) --
	(161.21, 34.37) --
	(161.37, 33.58) --
	(161.54, 32.79) --
	(161.72, 32.00) --
	(161.91, 31.21) --
	(162.12, 30.43) --
	(162.35, 29.66) --
	(162.58, 28.88) --
	(162.84, 28.12) --
	(163.10, 27.35) --
	(163.38, 26.59) --
	(163.67, 25.84) --
	(163.97, 25.09) --
	(164.29, 24.34) --
	(164.62, 23.61) --
	(164.96, 22.87) --
	(165.32, 22.15) --
	(165.68, 21.43) --
	(166.06, 20.71) --
	(166.46, 20.01) --
	(166.86, 19.31) --
	(167.28, 18.62) --
	(167.71, 17.93) --
	(168.15, 17.25) --
	(168.60, 16.58) --
	(169.07, 15.92) --
	(169.54, 15.27) --
	(170.03, 14.62) --
	(170.53, 13.99) --
	(171.04, 13.36) --
	(171.56, 12.74) --
	(172.09, 12.13) --
	(172.64, 11.53) --
	(173.19, 10.94) --
	(173.75, 10.36) --
	(174.32,  9.79) --
	(174.91,  9.23) --
	(175.50,  8.68) --
	(176.10,  8.14) --
	(176.71,  7.61) --
	(177.33,  7.10) --
	(177.96,  6.59) --
	(178.60,  6.09) --
	(179.25,  5.61) --
	(179.90,  5.13) --
	(180.57,  4.67) --
	(181.24,  4.22) --
	(181.92,  3.78) --
	(182.61,  3.36) --
	(183.30,  2.94) --
	(184.00,  2.54) --
	(184.71,  2.15) --
	(185.42,  1.77) --
	(186.15,  1.41) --
	(186.87,  1.06) --
	(187.61,  0.72) --
	(188.35,  0.39) --
	(189.09,  0.08) --
	(189.28,  0.00);

\path[draw=drawColor,line width= 1.2pt,line join=round,line cap=round] (223.98,  0.00) --
	(224.54,  0.23) --
	(225.28,  0.55) --
	(226.02,  0.88) --
	(226.75,  1.23) --
	(227.48,  1.59) --
	(228.19,  1.96) --
	(228.91,  2.34) --
	(229.61,  2.74) --
	(230.31,  3.15) --
	(231.00,  3.57) --
	(231.68,  4.00) --
	(232.36,  4.44) --
	(233.02,  4.90) --
	(233.68,  5.37) --
	(234.34,  5.85) --
	(234.98,  6.34) --
	(235.61,  6.84) --
	(236.24,  7.35) --
	(236.85,  7.88) --
	(237.46,  8.41) --
	(238.06,  8.96) --
	(238.65,  9.51) --
	(239.22, 10.08) --
	(239.79, 10.65) --
	(240.35, 11.24) --
	(240.90, 11.83) --
	(241.43, 12.44) --
	(241.96, 13.05) --
	(242.47, 13.67) --
	(242.98, 14.30) --
	(243.47, 14.95) --
	(243.95, 15.59) --
	(244.43, 16.25) --
	(244.88, 16.92) --
	(245.33, 17.59) --
	(245.77, 18.27) --
	(246.19, 18.96) --
	(246.60, 19.66) --
	(247.00, 20.36) --
	(247.39, 21.07) --
	(247.76, 21.79) --
	(248.12, 22.51) --
	(248.47, 23.24) --
	(248.81, 23.97) --
	(249.13, 24.72) --
	(249.44, 25.46) --
	(249.74, 26.21) --
	(250.02, 26.97) --
	(250.29, 27.73) --
	(250.55, 28.50) --
	(250.80, 29.27) --
	(251.03, 30.04) --
	(251.24, 30.82) --
	(251.45, 31.61) --
	(251.63, 32.39) --
	(251.81, 33.18) --
	(251.97, 33.97) --
	(252.12, 34.77) --
	(252.25, 35.57) --
	(252.37, 36.37) --
	(252.48, 37.17) --
	(252.57, 37.97) --
	(252.65, 38.78) --
	(252.71, 39.58) --
	(252.76, 40.39) --
	(252.80, 41.20) --
	(252.82, 42.00) --
	(252.82, 42.81);
\definecolor[named]{drawColor}{rgb}{0.00,0.00,0.00}

\path[draw=drawColor,line width= 0.4pt,dash pattern=on 1pt off 3pt ,line join=round,line cap=round] (252.94, 63.71) --
	(252.66, 64.34) --
	(252.27, 65.15) --
	(251.88, 65.94) --
	(251.46, 66.73) --
	(251.04, 67.51) --
	(250.60, 68.28) --
	(250.15, 69.05) --
	(249.68, 69.81) --
	(249.20, 70.56) --
	(248.71, 71.30) --
	(248.21, 72.03) --
	(247.69, 72.75) --
	(247.16, 73.47) --
	(246.61, 74.17) --
	(246.06, 74.87) --
	(245.49, 75.55) --
	(244.91, 76.23) --
	(244.32, 76.89) --
	(243.72, 77.55) --
	(243.11, 78.19) --
	(242.48, 78.82) --
	(241.85, 79.44) --
	(241.20, 80.05) --
	(240.54, 80.65) --
	(239.88, 81.24) --
	(239.20, 81.82) --
	(238.51, 82.38) --
	(237.81, 82.93) --
	(237.11, 83.47) --
	(236.39, 84.00) --
	(235.66, 84.52) --
	(234.93, 85.02) --
	(234.19, 85.51) --
	(233.44, 85.98) --
	(232.68, 86.44) --
	(231.91, 86.89) --
	(231.13, 87.33) --
	(230.35, 87.75) --
	(229.56, 88.16) --
	(228.76, 88.55) --
	(227.96, 88.93) --
	(227.15, 89.30) --
	(226.33, 89.65) --
	(225.51, 89.99) --
	(224.68, 90.31) --
	(223.85, 90.62) --
	(223.01, 90.92) --
	(222.16, 91.19) --
	(221.31, 91.46) --
	(220.46, 91.71) --
	(219.60, 91.94) --
	(218.74, 92.16) --
	(217.88, 92.37) --
	(217.01, 92.56) --
	(216.13, 92.73) --
	(215.26, 92.89) --
	(214.38, 93.03) --
	(213.50, 93.16) --
	(212.62, 93.27) --
	(211.74, 93.37) --
	(210.85, 93.45) --
	(209.96, 93.52) --
	(209.07, 93.57) --
	(208.19, 93.60) --
	(207.30, 93.62) --
	(206.41, 93.63) --
	(205.52, 93.61) --
	(204.63, 93.59) --
	(203.74, 93.54) --
	(202.85, 93.49) --
	(201.97, 93.41) --
	(201.08, 93.32) --
	(200.20, 93.22) --
	(199.32, 93.10) --
	(198.44, 92.96) --
	(197.56, 92.81) --
	(196.69, 92.65) --
	(195.82, 92.46) --
	(194.95, 92.27) --
	(194.09, 92.05) --
	(193.23, 91.83) --
	(192.37, 91.59) --
	(191.52, 91.33) --
	(190.67, 91.06) --
	(189.83, 90.77) --
	(189.00, 90.47) --
	(188.16, 90.15) --
	(187.34, 89.82) --
	(186.52, 89.48) --
	(185.70, 89.12) --
	(184.90, 88.75) --
	(184.10, 88.36) --
	(183.30, 87.96) --
	(182.52, 87.54) --
	(181.74, 87.11) --
	(180.97, 86.67) --
	(180.20, 86.21) --
	(179.45, 85.75) --
	(178.70, 85.26) --
	(177.96, 84.77) --
	(177.23, 84.26) --
	(176.51, 83.74) --
	(175.80, 83.21) --
	(175.10, 82.66) --
	(174.40, 82.10) --
	(173.72, 81.53) --
	(173.05, 80.95) --
	(172.39, 80.36) --
	(171.73, 79.75) --
	(171.09, 79.13) --
	(170.46, 78.51) --
	(169.84, 77.87) --
	(169.24, 77.22) --
	(168.64, 76.56) --
	(168.06, 75.89) --
	(167.48, 75.21) --
	(166.92, 74.52) --
	(166.37, 73.82) --
	(165.84, 73.11) --
	(165.31, 72.39) --
	(164.80, 71.66) --
	(164.30, 70.93) --
	(163.82, 70.18) --
	(163.34, 69.43) --
	(162.89, 68.67) --
	(162.44, 67.90) --
	(162.01, 67.12) --
	(161.59, 66.34) --
	(161.18, 65.54) --
	(160.79, 64.75) --
	(160.42, 63.94) --
	(160.05, 63.13) --
	(159.70, 62.31) --
	(159.37, 61.49) --
	(159.05, 60.66) --
	(158.75, 59.82) --
	(158.46, 58.98) --
	(158.18, 58.13) --
	(157.92, 57.28) --
	(157.67, 56.43) --
	(157.44, 55.57) --
	(157.23, 54.71) --
	(157.03, 53.84) --
	(156.84, 52.97) --
	(156.67, 52.10) --
	(156.52, 51.22) --
	(156.38, 50.34) --
	(156.25, 49.46) --
	(156.14, 48.58) --
	(156.05, 47.70) --
	(155.97, 46.81) --
	(155.91, 45.92) --
	(155.86, 45.04) --
	(155.83, 44.15) --
	(155.82, 43.26) --
	(155.82, 42.37) --
	(155.83, 41.48) --
	(155.86, 40.59) --
	(155.91, 39.70) --
	(155.97, 38.81) --
	(156.05, 37.93) --
	(156.14, 37.04) --
	(156.25, 36.16) --
	(156.38, 35.28) --
	(156.52, 34.40) --
	(156.67, 33.53) --
	(156.84, 32.65) --
	(157.03, 31.78) --
	(157.23, 30.92) --
	(157.44, 30.05) --
	(157.67, 29.20) --
	(157.92, 28.34) --
	(158.18, 27.49) --
	(158.46, 26.65) --
	(158.75, 25.80) --
	(159.05, 24.97) --
	(159.37, 24.14) --
	(159.70, 23.32) --
	(160.05, 22.50) --
	(160.42, 21.69) --
	(160.79, 20.88) --
	(161.18, 20.08) --
	(161.59, 19.29) --
	(162.01, 18.50) --
	(162.44, 17.73) --
	(162.89, 16.96) --
	(163.34, 16.20) --
	(163.82, 15.44) --
	(164.30, 14.70) --
	(164.80, 13.96) --
	(165.31, 13.23) --
	(165.84, 12.51) --
	(166.37, 11.81) --
	(166.92, 11.11) --
	(167.48, 10.42) --
	(168.06,  9.74) --
	(168.64,  9.07) --
	(169.24,  8.41) --
	(169.84,  7.76) --
	(170.46,  7.12) --
	(171.09,  6.49) --
	(171.73,  5.87) --
	(172.39,  5.27) --
	(173.05,  4.68) --
	(173.72,  4.09) --
	(174.40,  3.52) --
	(175.10,  2.97) --
	(175.80,  2.42) --
	(176.51,  1.89) --
	(177.23,  1.37) --
	(177.96,  0.86) --
	(178.70,  0.36) --
	(179.26,  0.00);

\path[draw=drawColor,line width= 0.4pt,dash pattern=on 1pt off 3pt ,line join=round,line cap=round] (234.00,  0.00) --
	(234.19,  0.12) --
	(234.93,  0.61) --
	(235.66,  1.11) --
	(236.39,  1.62) --
	(237.11,  2.15) --
	(237.81,  2.69) --
	(238.51,  3.24) --
	(239.20,  3.81) --
	(239.88,  4.38) --
	(240.54,  4.97) --
	(241.20,  5.57) --
	(241.85,  6.18) --
	(242.48,  6.80) --
	(243.11,  7.44) --
	(243.72,  8.08) --
	(244.32,  8.73) --
	(244.91,  9.40) --
	(245.49, 10.07) --
	(246.06, 10.76) --
	(246.61, 11.45) --
	(247.16, 12.16) --
	(247.69, 12.87) --
	(248.21, 13.60) --
	(248.71, 14.33) --
	(249.20, 15.07) --
	(249.68, 15.82) --
	(250.15, 16.58) --
	(250.60, 17.34) --
	(251.04, 18.11) --
	(251.46, 18.90) --
	(251.88, 19.68) --
	(252.27, 20.48) --
	(252.66, 21.28) --
	(252.94, 21.91);

\path[fill=fillColor] (184.21,126.47) circle (  2.25);
\definecolor[named]{drawColor}{rgb}{0.00,0.00,1.00}

\path[draw=drawColor,line width= 1.2pt,line join=round,line cap=round] (230.41,126.47) --
	(230.40,127.28) --
	(230.38,128.09) --
	(230.34,128.90) --
	(230.30,129.70) --
	(230.23,130.51) --
	(230.15,131.31) --
	(230.06,132.12) --
	(229.96,132.92) --
	(229.84,133.72) --
	(229.70,134.52) --
	(229.55,135.31) --
	(229.39,136.10) --
	(229.22,136.89) --
	(229.03,137.68) --
	(228.83,138.46) --
	(228.61,139.24) --
	(228.38,140.02) --
	(228.13,140.79) --
	(227.88,141.55) --
	(227.61,142.31) --
	(227.32,143.07) --
	(227.03,143.82) --
	(226.72,144.57) --
	(226.39,145.31) --
	(226.06,146.05) --
	(225.71,146.78) --
	(225.35,147.50) --
	(224.97,148.22) --
	(224.58,148.93) --
	(224.19,149.63) --
	(223.77,150.32) --
	(223.35,151.01) --
	(222.92,151.69) --
	(222.47,152.37) --
	(222.01,153.03) --
	(221.54,153.69) --
	(221.06,154.34) --
	(220.56,154.98) --
	(220.06,155.61) --
	(219.54,156.23) --
	(219.02,156.85) --
	(218.48,157.45) --
	(217.93,158.05) --
	(217.37,158.63) --
	(216.81,159.21) --
	(216.23,159.77) --
	(215.64,160.33) --
	(215.04,160.87) --
	(214.44,161.41) --
	(213.82,161.93) --
	(213.20,162.44) --
	(212.56,162.95) --
	(211.92,163.44) --
	(211.27,163.92) --
	(210.61,164.38) --
	(209.94,164.84) --
	(209.27,165.28) --
	(208.58,165.72) --
	(207.89,166.14) --
	(207.19,166.55) --
	(206.49,166.94) --
	(205.78,167.33) --
	(205.06,167.70) --
	(204.33,168.06) --
	(203.60,168.40) --
	(202.87,168.73) --
	(202.12,169.05) --
	(201.38,169.36) --
	(200.62,169.65) --
	(199.87,169.94) --
	(199.10,170.20) --
	(198.33,170.46) --
	(197.56,170.70) --
	(196.79,170.92) --
	(196.01,171.14) --
	(195.22,171.34) --
	(194.44,171.52) --
	(193.65,171.69) --
	(192.85,171.85) --
	(192.06,172.00) --
	(191.26,172.13) --
	(190.46,172.24) --
	(189.66,172.35) --
	(188.85,172.43) --
	(188.05,172.51) --
	(187.24,172.57) --
	(186.44,172.61) --
	(185.63,172.65) --
	(184.82,172.66) --
	(184.01,172.67) --
	(183.20,172.66) --
	(182.39,172.63) --
	(181.59,172.59) --
	(180.78,172.54) --
	(179.97,172.47) --
	(179.17,172.39) --
	(178.37,172.30) --
	(177.57,172.19) --
	(176.77,172.06) --
	(175.97,171.93) --
	(175.18,171.77) --
	(174.38,171.61) --
	(173.60,171.43) --
	(172.81,171.24) --
	(172.03,171.03) --
	(171.25,170.81) --
	(170.48,170.58) --
	(169.71,170.33) --
	(168.94,170.07) --
	(168.18,169.80) --
	(167.43,169.51) --
	(166.68,169.21) --
	(165.93,168.90) --
	(165.19,168.57) --
	(164.46,168.23) --
	(163.73,167.88) --
	(163.01,167.51) --
	(162.29,167.14) --
	(161.58,166.75) --
	(160.88,166.34) --
	(160.19,165.93) --
	(159.50,165.50) --
	(158.82,165.06) --
	(158.15,164.61) --
	(157.49,164.15) --
	(156.83,163.68) --
	(156.19,163.19) --
	(155.55,162.70) --
	(154.92,162.19) --
	(154.30,161.67) --
	(153.69,161.14) --
	(153.08,160.60) --
	(152.49,160.05) --
	(151.91,159.49) --
	(151.33,158.92) --
	(150.77,158.34) --
	(150.22,157.75) --
	(149.68,157.15) --
	(149.15,156.54) --
	(148.62,155.92) --
	(148.11,155.30) --
	(147.62,154.66) --
	(147.13,154.02) --
	(146.65,153.36) --
	(146.19,152.70) --
	(145.73,152.03) --
	(145.29,151.35) --
	(144.86,150.67) --
	(144.45,149.98) --
	(144.04,149.28) --
	(143.65,148.57) --
	(143.27,147.86) --
	(142.90,147.14) --
	(142.54,146.41) --
	(142.20,145.68) --
	(141.87,144.94) --
	(141.55,144.20) --
	(141.25,143.45) --
	(140.96,142.69) --
	(140.68,141.93) --
	(140.42,141.17) --
	(140.17,140.40) --
	(139.93,139.63) --
	(139.71,138.85) --
	(139.50,138.07) --
	(139.30,137.29) --
	(139.12,136.50) --
	(138.95,135.71) --
	(138.80,134.91) --
	(138.66,134.12) --
	(138.53,133.32) --
	(138.42,132.52) --
	(138.32,131.72) --
	(138.23,130.91) --
	(138.16,130.11) --
	(138.10,129.30) --
	(138.06,128.49) --
	(138.03,127.69) --
	(138.02,126.88) --
	(138.02,126.07) --
	(138.03,125.26) --
	(138.06,124.45) --
	(138.10,123.64) --
	(138.16,122.84) --
	(138.23,122.03) --
	(138.32,121.23) --
	(138.42,120.43) --
	(138.53,119.63) --
	(138.66,118.83) --
	(138.80,118.03) --
	(138.95,117.24) --
	(139.12,116.45) --
	(139.30,115.66) --
	(139.50,114.87) --
	(139.71,114.09) --
	(139.93,113.32) --
	(140.17,112.54) --
	(140.42,111.78) --
	(140.68,111.01) --
	(140.96,110.25) --
	(141.25,109.50) --
	(141.55,108.75) --
	(141.87,108.00) --
	(142.20,107.27) --
	(142.54,106.53) --
	(142.90,105.81) --
	(143.27,105.09) --
	(143.65,104.37) --
	(144.04,103.67) --
	(144.45,102.97) --
	(144.86,102.28) --
	(145.29,101.59) --
	(145.73,100.91) --
	(146.19,100.24) --
	(146.65, 99.58) --
	(147.13, 98.93) --
	(147.62, 98.28) --
	(148.11, 97.65) --
	(148.62, 97.02) --
	(149.15, 96.40) --
	(149.68, 95.79) --
	(150.22, 95.19) --
	(150.77, 94.60) --
	(151.33, 94.02) --
	(151.91, 93.45) --
	(152.49, 92.89) --
	(153.08, 92.34) --
	(153.69, 91.80) --
	(154.30, 91.27) --
	(154.92, 90.76) --
	(155.55, 90.25) --
	(156.19, 89.75) --
	(156.83, 89.27) --
	(157.49, 88.79) --
	(158.15, 88.33) --
	(158.82, 87.88) --
	(159.50, 87.44) --
	(160.19, 87.02) --
	(160.88, 86.60) --
	(161.58, 86.20) --
	(162.29, 85.81) --
	(163.01, 85.43) --
	(163.73, 85.07) --
	(164.46, 84.72) --
	(165.19, 84.38) --
	(165.93, 84.05) --
	(166.68, 83.74) --
	(167.43, 83.44) --
	(168.18, 83.15) --
	(168.94, 82.87) --
	(169.71, 82.61) --
	(170.48, 82.37) --
	(171.25, 82.13) --
	(172.03, 81.91) --
	(172.81, 81.71) --
	(173.60, 81.51) --
	(174.38, 81.34) --
	(175.18, 81.17) --
	(175.97, 81.02) --
	(176.77, 80.88) --
	(177.57, 80.76) --
	(178.37, 80.65) --
	(179.17, 80.55) --
	(179.97, 80.47) --
	(180.78, 80.41) --
	(181.59, 80.35) --
	(182.39, 80.31) --
	(183.20, 80.29) --
	(184.01, 80.28) --
	(184.82, 80.28) --
	(185.63, 80.30) --
	(186.44, 80.33) --
	(187.24, 80.38) --
	(188.05, 80.44) --
	(188.85, 80.51) --
	(189.66, 80.60) --
	(190.46, 80.70) --
	(191.26, 80.82) --
	(192.06, 80.95) --
	(192.85, 81.09) --
	(193.65, 81.25) --
	(194.44, 81.42) --
	(195.22, 81.61) --
	(196.01, 81.81) --
	(196.79, 82.02) --
	(197.56, 82.25) --
	(198.33, 82.49) --
	(199.10, 82.74) --
	(199.87, 83.01) --
	(200.62, 83.29) --
	(201.38, 83.58) --
	(202.12, 83.89) --
	(202.87, 84.21) --
	(203.60, 84.54) --
	(204.33, 84.89) --
	(205.06, 85.25) --
	(205.78, 85.62) --
	(206.49, 86.00) --
	(207.19, 86.40) --
	(207.89, 86.81) --
	(208.58, 87.23) --
	(209.27, 87.66) --
	(209.94, 88.10) --
	(210.61, 88.56) --
	(211.27, 89.03) --
	(211.92, 89.51) --
	(212.56, 90.00) --
	(213.20, 90.50) --
	(213.82, 91.01) --
	(214.44, 91.54) --
	(215.04, 92.07) --
	(215.64, 92.62) --
	(216.23, 93.17) --
	(216.81, 93.74) --
	(217.37, 94.31) --
	(217.93, 94.90) --
	(218.48, 95.49) --
	(219.02, 96.10) --
	(219.54, 96.71) --
	(220.06, 97.33) --
	(220.56, 97.96) --
	(221.06, 98.61) --
	(221.54, 99.25) --
	(222.01, 99.91) --
	(222.47,100.58) --
	(222.92,101.25) --
	(223.35,101.93) --
	(223.77,102.62) --
	(224.19,103.32) --
	(224.58,104.02) --
	(224.97,104.73) --
	(225.35,105.45) --
	(225.71,106.17) --
	(226.06,106.90) --
	(226.39,107.63) --
	(226.72,108.38) --
	(227.03,109.12) --
	(227.32,109.87) --
	(227.61,110.63) --
	(227.88,111.39) --
	(228.13,112.16) --
	(228.38,112.93) --
	(228.61,113.70) --
	(228.83,114.48) --
	(229.03,115.27) --
	(229.22,116.05) --
	(229.39,116.84) --
	(229.55,117.63) --
	(229.70,118.43) --
	(229.84,119.23) --
	(229.96,120.03) --
	(230.06,120.83) --
	(230.15,121.63) --
	(230.23,122.44) --
	(230.30,123.24) --
	(230.34,124.05) --
	(230.38,124.86) --
	(230.40,125.66) --
	(230.41,126.47);
\definecolor[named]{drawColor}{rgb}{0.00,0.00,0.00}

\path[draw=drawColor,line width= 0.4pt,dash pattern=on 1pt off 3pt ,line join=round,line cap=round] (235.03,126.47) --
	(235.02,127.36) --
	(235.00,128.25) --
	(234.96,129.14) --
	(234.90,130.03) --
	(234.83,130.91) --
	(234.75,131.80) --
	(234.65,132.68) --
	(234.53,133.56) --
	(234.40,134.44) --
	(234.25,135.32) --
	(234.09,136.20) --
	(233.91,137.07) --
	(233.72,137.93) --
	(233.51,138.80) --
	(233.29,139.66) --
	(233.05,140.52) --
	(232.80,141.37) --
	(232.53,142.22) --
	(232.24,143.06) --
	(231.95,143.90) --
	(231.63,144.73) --
	(231.31,145.56) --
	(230.97,146.38) --
	(230.61,147.19) --
	(230.24,148.00) --
	(229.86,148.81) --
	(229.46,149.60) --
	(229.05,150.39) --
	(228.62,151.17) --
	(228.18,151.94) --
	(227.73,152.71) --
	(227.26,153.47) --
	(226.79,154.22) --
	(226.29,154.96) --
	(225.79,155.69) --
	(225.27,156.41) --
	(224.74,157.13) --
	(224.20,157.83) --
	(223.64,158.53) --
	(223.08,159.21) --
	(222.50,159.89) --
	(221.91,160.55) --
	(221.30,161.21) --
	(220.69,161.85) --
	(220.07,162.48) --
	(219.43,163.10) --
	(218.78,163.71) --
	(218.13,164.31) --
	(217.46,164.90) --
	(216.78,165.48) --
	(216.09,166.04) --
	(215.40,166.59) --
	(214.69,167.13) --
	(213.97,167.66) --
	(213.25,168.18) --
	(212.51,168.68) --
	(211.77,169.17) --
	(211.02,169.64) --
	(210.26,170.10) --
	(209.49,170.55) --
	(208.72,170.99) --
	(207.93,171.41) --
	(207.14,171.82) --
	(206.35,172.21) --
	(205.54,172.59) --
	(204.73,172.96) --
	(203.92,173.31) --
	(203.09,173.65) --
	(202.26,173.97) --
	(201.43,174.28) --
	(200.59,174.58) --
	(199.75,174.85) --
	(198.90,175.12) --
	(198.04,175.37) --
	(197.19,175.60) --
	(196.32,175.82) --
	(195.46,176.03) --
	(194.59,176.22) --
	(193.72,176.39) --
	(192.84,176.55) --
	(191.96,176.69) --
	(191.08,176.82) --
	(190.20,176.93) --
	(189.32,177.03) --
	(188.43,177.11) --
	(187.55,177.18) --
	(186.66,177.23) --
	(185.77,177.26) --
	(184.88,177.28) --
	(183.99,177.29) --
	(183.10,177.27) --
	(182.21,177.25) --
	(181.32,177.20) --
	(180.44,177.15) --
	(179.55,177.07) --
	(178.67,176.98) --
	(177.78,176.88) --
	(176.90,176.76) --
	(176.02,176.62) --
	(175.15,176.47) --
	(174.27,176.31) --
	(173.40,176.12) --
	(172.53,175.93) --
	(171.67,175.71) --
	(170.81,175.49) --
	(169.96,175.25) --
	(169.10,174.99) --
	(168.26,174.72) --
	(167.42,174.43) --
	(166.58,174.13) --
	(165.75,173.81) --
	(164.92,173.48) --
	(164.10,173.14) --
	(163.29,172.78) --
	(162.48,172.41) --
	(161.68,172.02) --
	(160.89,171.62) --
	(160.10,171.20) --
	(159.32,170.77) --
	(158.55,170.33) --
	(157.79,169.87) --
	(157.03,169.41) --
	(156.28,168.92) --
	(155.54,168.43) --
	(154.81,167.92) --
	(154.09,167.40) --
	(153.38,166.87) --
	(152.68,166.32) --
	(151.99,165.76) --
	(151.31,165.19) --
	(150.63,164.61) --
	(149.97,164.02) --
	(149.32,163.41) --
	(148.68,162.79) --
	(148.05,162.17) --
	(147.43,161.53) --
	(146.82,160.88) --
	(146.22,160.22) --
	(145.64,159.55) --
	(145.07,158.87) --
	(144.50,158.18) --
	(143.96,157.48) --
	(143.42,156.77) --
	(142.90,156.05) --
	(142.38,155.32) --
	(141.89,154.59) --
	(141.40,153.84) --
	(140.93,153.09) --
	(140.47,152.33) --
	(140.02,151.56) --
	(139.59,150.78) --
	(139.17,150.00) --
	(138.77,149.20) --
	(138.38,148.41) --
	(138.00,147.60) --
	(137.64,146.79) --
	(137.29,145.97) --
	(136.95,145.15) --
	(136.63,144.32) --
	(136.33,143.48) --
	(136.04,142.64) --
	(135.76,141.79) --
	(135.50,140.94) --
	(135.26,140.09) --
	(135.03,139.23) --
	(134.81,138.37) --
	(134.61,137.50) --
	(134.42,136.63) --
	(134.25,135.76) --
	(134.10,134.88) --
	(133.96,134.00) --
	(133.84,133.12) --
	(133.73,132.24) --
	(133.63,131.36) --
	(133.56,130.47) --
	(133.49,129.58) --
	(133.45,128.70) --
	(133.42,127.81) --
	(133.40,126.92) --
	(133.40,126.03) --
	(133.42,125.14) --
	(133.45,124.25) --
	(133.49,123.36) --
	(133.56,122.47) --
	(133.63,121.59) --
	(133.73,120.70) --
	(133.84,119.82) --
	(133.96,118.94) --
	(134.10,118.06) --
	(134.25,117.19) --
	(134.42,116.31) --
	(134.61,115.44) --
	(134.81,114.58) --
	(135.03,113.71) --
	(135.26,112.86) --
	(135.50,112.00) --
	(135.76,111.15) --
	(136.04,110.31) --
	(136.33,109.46) --
	(136.63,108.63) --
	(136.95,107.80) --
	(137.29,106.98) --
	(137.64,106.16) --
	(138.00,105.35) --
	(138.38,104.54) --
	(138.77,103.74) --
	(139.17,102.95) --
	(139.59,102.16) --
	(140.02,101.39) --
	(140.47,100.62) --
	(140.93, 99.86) --
	(141.40, 99.10) --
	(141.89, 98.36) --
	(142.38, 97.62) --
	(142.90, 96.89) --
	(143.42, 96.17) --
	(143.96, 95.47) --
	(144.50, 94.77) --
	(145.07, 94.08) --
	(145.64, 93.40) --
	(146.22, 92.73) --
	(146.82, 92.07) --
	(147.43, 91.42) --
	(148.05, 90.78) --
	(148.68, 90.15) --
	(149.32, 89.53) --
	(149.97, 88.93) --
	(150.63, 88.34) --
	(151.31, 87.75) --
	(151.99, 87.18) --
	(152.68, 86.63) --
	(153.38, 86.08) --
	(154.09, 85.55) --
	(154.81, 85.03) --
	(155.54, 84.52) --
	(156.28, 84.02) --
	(157.03, 83.54) --
	(157.79, 83.07) --
	(158.55, 82.61) --
	(159.32, 82.17) --
	(160.10, 81.74) --
	(160.89, 81.33) --
	(161.68, 80.93) --
	(162.48, 80.54) --
	(163.29, 80.17) --
	(164.10, 79.81) --
	(164.92, 79.46) --
	(165.75, 79.13) --
	(166.58, 78.82) --
	(167.42, 78.51) --
	(168.26, 78.23) --
	(169.10, 77.96) --
	(169.96, 77.70) --
	(170.81, 77.46) --
	(171.67, 77.23) --
	(172.53, 77.02) --
	(173.40, 76.82) --
	(174.27, 76.64) --
	(175.15, 76.47) --
	(176.02, 76.32) --
	(176.90, 76.19) --
	(177.78, 76.07) --
	(178.67, 75.96) --
	(179.55, 75.87) --
	(180.44, 75.80) --
	(181.32, 75.74) --
	(182.21, 75.70) --
	(183.10, 75.67) --
	(183.99, 75.66) --
	(184.88, 75.66) --
	(185.77, 75.68) --
	(186.66, 75.72) --
	(187.55, 75.77) --
	(188.43, 75.83) --
	(189.32, 75.92) --
	(190.20, 76.01) --
	(191.08, 76.12) --
	(191.96, 76.25) --
	(192.84, 76.40) --
	(193.72, 76.55) --
	(194.59, 76.73) --
	(195.46, 76.92) --
	(196.32, 77.12) --
	(197.19, 77.34) --
	(198.04, 77.58) --
	(198.90, 77.83) --
	(199.75, 78.09) --
	(200.59, 78.37) --
	(201.43, 78.66) --
	(202.26, 78.97) --
	(203.09, 79.30) --
	(203.92, 79.63) --
	(204.73, 79.99) --
	(205.54, 80.35) --
	(206.35, 80.73) --
	(207.14, 81.13) --
	(207.93, 81.53) --
	(208.72, 81.96) --
	(209.49, 82.39) --
	(210.26, 82.84) --
	(211.02, 83.30) --
	(211.77, 83.78) --
	(212.51, 84.27) --
	(213.25, 84.77) --
	(213.97, 85.28) --
	(214.69, 85.81) --
	(215.40, 86.35) --
	(216.09, 86.90) --
	(216.78, 87.47) --
	(217.46, 88.04) --
	(218.13, 88.63) --
	(218.78, 89.23) --
	(219.43, 89.84) --
	(220.07, 90.46) --
	(220.69, 91.10) --
	(221.30, 91.74) --
	(221.91, 92.39) --
	(222.50, 93.06) --
	(223.08, 93.73) --
	(223.64, 94.42) --
	(224.20, 95.11) --
	(224.74, 95.82) --
	(225.27, 96.53) --
	(225.79, 97.26) --
	(226.29, 97.99) --
	(226.79, 98.73) --
	(227.26, 99.48) --
	(227.73,100.24) --
	(228.18,101.00) --
	(228.62,101.77) --
	(229.05,102.56) --
	(229.46,103.34) --
	(229.86,104.14) --
	(230.24,104.94) --
	(230.61,105.75) --
	(230.97,106.57) --
	(231.31,107.39) --
	(231.63,108.21) --
	(231.95,109.05) --
	(232.24,109.88) --
	(232.53,110.73) --
	(232.80,111.58) --
	(233.05,112.43) --
	(233.29,113.28) --
	(233.51,114.15) --
	(233.72,115.01) --
	(233.91,115.88) --
	(234.09,116.75) --
	(234.25,117.62) --
	(234.40,118.50) --
	(234.53,119.38) --
	(234.65,120.26) --
	(234.75,121.15) --
	(234.83,122.03) --
	(234.90,122.92) --
	(234.96,123.81) --
	(235.00,124.69) --
	(235.02,125.58) --
	(235.03,126.47);

\path[fill=fillColor] (161.80,210.13) circle (  2.25);
\definecolor[named]{drawColor}{rgb}{0.00,0.00,1.00}

\path[draw=drawColor,line width= 1.2pt,line join=round,line cap=round] (207.99,210.13) --
	(207.98,210.94) --
	(207.96,211.75) --
	(207.93,212.56) --
	(207.88,213.36) --
	(207.81,214.17) --
	(207.74,214.97) --
	(207.65,215.78) --
	(207.54,216.58) --
	(207.42,217.38) --
	(207.29,218.18) --
	(207.14,218.97) --
	(206.98,219.76) --
	(206.80,220.55) --
	(206.61,221.34) --
	(206.41,222.12) --
	(206.19,222.90) --
	(205.96,223.68) --
	(205.72,224.45) --
	(205.46,225.21) --
	(205.19,225.97) --
	(204.91,226.73) --
	(204.61,227.48) --
	(204.30,228.23) --
	(203.98,228.97) --
	(203.64,229.71) --
	(203.29,230.44) --
	(202.93,231.16) --
	(202.55,231.88) --
	(202.17,232.59) --
	(201.77,233.29) --
	(201.36,233.98) --
	(200.93,234.67) --
	(200.50,235.35) --
	(200.05,236.03) --
	(199.59,236.69) --
	(199.12,237.35) --
	(198.64,238.00) --
	(198.15,238.64) --
	(197.64,239.27) --
	(197.13,239.89) --
	(196.60,240.51) --
	(196.06,241.11) --
	(195.52,241.71) --
	(194.96,242.29) --
	(194.39,242.87) --
	(193.81,243.43) --
	(193.22,243.99) --
	(192.63,244.53) --
	(192.02,245.07) --
	(191.40,245.59) --
	(190.78,246.10) --
	(190.14,246.61) --
	(189.50,247.10) --
	(188.85,247.58) --
	(188.19,248.04) --
	(187.52,248.50) --
	(186.85,248.94) --
	(186.17,249.38) --
	(185.48,249.80) --
	(184.78,250.21) --
	(184.07,250.60) --
	(183.36,250.99) --
	(182.64,251.36) --
	(181.92,251.72) --
	(181.19,252.06) --
	(180.45,252.39) --
	(179.71,252.71) --
	(179.14,252.94);

\path[draw=drawColor,line width= 1.2pt,line join=round,line cap=round] (144.45,252.94) --
	(144.26,252.87) --
	(143.51,252.56) --
	(142.77,252.23) --
	(142.04,251.89) --
	(141.31,251.54) --
	(140.59,251.17) --
	(139.88,250.80) --
	(139.17,250.41) --
	(138.47,250.00) --
	(137.77,249.59) --
	(137.09,249.16) --
	(136.41,248.72) --
	(135.73,248.27) --
	(135.07,247.81) --
	(134.42,247.34) --
	(133.77,246.85) --
	(133.13,246.36) --
	(132.50,245.85) --
	(131.88,245.33) --
	(131.27,244.80) --
	(130.67,244.26) --
	(130.07,243.71) --
	(129.49,243.15) --
	(128.92,242.58) --
	(128.36,242.00) --
	(127.80,241.41) --
	(127.26,240.81) --
	(126.73,240.20) --
	(126.21,239.58) --
	(125.70,238.96) --
	(125.20,238.32) --
	(124.71,237.68) --
	(124.23,237.02) --
	(123.77,236.36) --
	(123.32,235.69) --
	(122.88,235.01) --
	(122.45,234.33) --
	(122.03,233.64) --
	(121.62,232.94) --
	(121.23,232.23) --
	(120.85,231.52) --
	(120.48,230.80) --
	(120.13,230.07) --
	(119.78,229.34) --
	(119.45,228.60) --
	(119.14,227.86) --
	(118.83,227.11) --
	(118.54,226.35) --
	(118.27,225.59) --
	(118.00,224.83) --
	(117.75,224.06) --
	(117.51,223.29) --
	(117.29,222.51) --
	(117.08,221.73) --
	(116.89,220.95) --
	(116.70,220.16) --
	(116.53,219.37) --
	(116.38,218.57) --
	(116.24,217.78) --
	(116.11,216.98) --
	(116.00,216.18) --
	(115.90,215.38) --
	(115.82,214.57) --
	(115.74,213.77) --
	(115.69,212.96) --
	(115.65,212.15) --
	(115.62,211.35) --
	(115.60,210.54) --
	(115.60,209.73) --
	(115.62,208.92) --
	(115.65,208.11) --
	(115.69,207.30) --
	(115.74,206.50) --
	(115.82,205.69) --
	(115.90,204.89) --
	(116.00,204.09) --
	(116.11,203.29) --
	(116.24,202.49) --
	(116.38,201.69) --
	(116.53,200.90) --
	(116.70,200.11) --
	(116.89,199.32) --
	(117.08,198.53) --
	(117.29,197.75) --
	(117.51,196.98) --
	(117.75,196.20) --
	(118.00,195.44) --
	(118.27,194.67) --
	(118.54,193.91) --
	(118.83,193.16) --
	(119.14,192.41) --
	(119.45,191.66) --
	(119.78,190.93) --
	(120.13,190.19) --
	(120.48,189.47) --
	(120.85,188.75) --
	(121.23,188.03) --
	(121.62,187.33) --
	(122.03,186.63) --
	(122.45,185.94) --
	(122.88,185.25) --
	(123.32,184.57) --
	(123.77,183.90) --
	(124.23,183.24) --
	(124.71,182.59) --
	(125.20,181.94) --
	(125.70,181.31) --
	(126.21,180.68) --
	(126.73,180.06) --
	(127.26,179.45) --
	(127.80,178.85) --
	(128.36,178.26) --
	(128.92,177.68) --
	(129.49,177.11) --
	(130.07,176.55) --
	(130.67,176.00) --
	(131.27,175.46) --
	(131.88,174.93) --
	(132.50,174.42) --
	(133.13,173.91) --
	(133.77,173.41) --
	(134.42,172.93) --
	(135.07,172.45) --
	(135.73,171.99) --
	(136.41,171.54) --
	(137.09,171.10) --
	(137.77,170.68) --
	(138.47,170.26) --
	(139.17,169.86) --
	(139.88,169.47) --
	(140.59,169.09) --
	(141.31,168.73) --
	(142.04,168.38) --
	(142.77,168.04) --
	(143.51,167.71) --
	(144.26,167.40) --
	(145.01,167.10) --
	(145.77,166.81) --
	(146.53,166.53) --
	(147.29,166.27) --
	(148.06,166.03) --
	(148.83,165.79) --
	(149.61,165.57) --
	(150.39,165.37) --
	(151.18,165.17) --
	(151.97,165.00) --
	(152.76,164.83) --
	(153.55,164.68) --
	(154.35,164.54) --
	(155.15,164.42) --
	(155.95,164.31) --
	(156.75,164.21) --
	(157.56,164.13) --
	(158.36,164.07) --
	(159.17,164.01) --
	(159.98,163.97) --
	(160.79,163.95) --
	(161.59,163.94) --
	(162.40,163.94) --
	(163.21,163.96) --
	(164.02,163.99) --
	(164.83,164.04) --
	(165.63,164.10) --
	(166.44,164.17) --
	(167.24,164.26) --
	(168.04,164.36) --
	(168.84,164.48) --
	(169.64,164.61) --
	(170.44,164.75) --
	(171.23,164.91) --
	(172.02,165.08) --
	(172.81,165.27) --
	(173.59,165.47) --
	(174.37,165.68) --
	(175.15,165.91) --
	(175.92,166.15) --
	(176.69,166.40) --
	(177.45,166.67) --
	(178.21,166.95) --
	(178.96,167.24) --
	(179.71,167.55) --
	(180.45,167.87) --
	(181.19,168.20) --
	(181.92,168.55) --
	(182.64,168.91) --
	(183.36,169.28) --
	(184.07,169.66) --
	(184.78,170.06) --
	(185.48,170.47) --
	(186.17,170.89) --
	(186.85,171.32) --
	(187.52,171.76) --
	(188.19,172.22) --
	(188.85,172.69) --
	(189.50,173.17) --
	(190.14,173.66) --
	(190.78,174.16) --
	(191.40,174.67) --
	(192.02,175.20) --
	(192.63,175.73) --
	(193.22,176.28) --
	(193.81,176.83) --
	(194.39,177.40) --
	(194.96,177.97) --
	(195.52,178.56) --
	(196.06,179.15) --
	(196.60,179.76) --
	(197.13,180.37) --
	(197.64,180.99) --
	(198.15,181.62) --
	(198.64,182.27) --
	(199.12,182.91) --
	(199.59,183.57) --
	(200.05,184.24) --
	(200.50,184.91) --
	(200.93,185.59) --
	(201.36,186.28) --
	(201.77,186.98) --
	(202.17,187.68) --
	(202.55,188.39) --
	(202.93,189.11) --
	(203.29,189.83) --
	(203.64,190.56) --
	(203.98,191.29) --
	(204.30,192.04) --
	(204.61,192.78) --
	(204.91,193.53) --
	(205.19,194.29) --
	(205.46,195.05) --
	(205.72,195.82) --
	(205.96,196.59) --
	(206.19,197.36) --
	(206.41,198.14) --
	(206.61,198.93) --
	(206.80,199.71) --
	(206.98,200.50) --
	(207.14,201.29) --
	(207.29,202.09) --
	(207.42,202.89) --
	(207.54,203.69) --
	(207.65,204.49) --
	(207.74,205.29) --
	(207.81,206.10) --
	(207.88,206.90) --
	(207.93,207.71) --
	(207.96,208.52) --
	(207.98,209.32) --
	(207.99,210.13);
\definecolor[named]{drawColor}{rgb}{0.00,0.00,0.00}

\path[draw=drawColor,line width= 0.4pt,dash pattern=on 1pt off 3pt ,line join=round,line cap=round] (212.61,210.13) --
	(212.60,211.02) --
	(212.58,211.91) --
	(212.54,212.80) --
	(212.49,213.69) --
	(212.42,214.57) --
	(212.33,215.46) --
	(212.23,216.34) --
	(212.11,217.22) --
	(211.98,218.10) --
	(211.83,218.98) --
	(211.67,219.86) --
	(211.49,220.73) --
	(211.30,221.59) --
	(211.09,222.46) --
	(210.87,223.32) --
	(210.63,224.18) --
	(210.38,225.03) --
	(210.11,225.88) --
	(209.83,226.72) --
	(209.53,227.56) --
	(209.22,228.39) --
	(208.89,229.22) --
	(208.55,230.04) --
	(208.19,230.85) --
	(207.82,231.66) --
	(207.44,232.47) --
	(207.04,233.26) --
	(206.63,234.05) --
	(206.21,234.83) --
	(205.77,235.60) --
	(205.31,236.37) --
	(204.85,237.13) --
	(204.37,237.88) --
	(203.88,238.62) --
	(203.37,239.35) --
	(202.85,240.07) --
	(202.32,240.79) --
	(201.78,241.49) --
	(201.23,242.19) --
	(200.66,242.87) --
	(200.08,243.55) --
	(199.49,244.21) --
	(198.89,244.87) --
	(198.27,245.51) --
	(197.65,246.14) --
	(197.01,246.76) --
	(196.37,247.37) --
	(195.71,247.97) --
	(195.04,248.56) --
	(194.36,249.14) --
	(193.68,249.70) --
	(192.98,250.25) --
	(192.27,250.79) --
	(191.56,251.32) --
	(190.83,251.84) --
	(190.10,252.34) --
	(189.35,252.83) --
	(189.17,252.94);

\path[draw=drawColor,line width= 0.4pt,dash pattern=on 1pt off 3pt ,line join=round,line cap=round] (134.43,252.94) --
	(133.87,252.58) --
	(133.13,252.09) --
	(132.40,251.58) --
	(131.68,251.06) --
	(130.97,250.53) --
	(130.26,249.98) --
	(129.57,249.42) --
	(128.89,248.85) --
	(128.22,248.27) --
	(127.55,247.68) --
	(126.90,247.07) --
	(126.26,246.45) --
	(125.63,245.83) --
	(125.01,245.19) --
	(124.40,244.54) --
	(123.81,243.88) --
	(123.22,243.21) --
	(122.65,242.53) --
	(122.09,241.84) --
	(121.54,241.14) --
	(121.00,240.43) --
	(120.48,239.71) --
	(119.97,238.98) --
	(119.47,238.25) --
	(118.98,237.50) --
	(118.51,236.75) --
	(118.05,235.99) --
	(117.61,235.22) --
	(117.17,234.44) --
	(116.76,233.66) --
	(116.35,232.86) --
	(115.96,232.07) --
	(115.58,231.26) --
	(115.22,230.45) --
	(114.87,229.63) --
	(114.54,228.81) --
	(114.22,227.98) --
	(113.91,227.14) --
	(113.62,226.30) --
	(113.35,225.45) --
	(113.09,224.60) --
	(112.84,223.75) --
	(112.61,222.89) --
	(112.39,222.03) --
	(112.19,221.16) --
	(112.01,220.29) --
	(111.84,219.42) --
	(111.68,218.54) --
	(111.54,217.66) --
	(111.42,216.78) --
	(111.31,215.90) --
	(111.22,215.02) --
	(111.14,214.13) --
	(111.08,213.24) --
	(111.03,212.36) --
	(111.00,211.47) --
	(110.98,210.58) --
	(110.98,209.69) --
	(111.00,208.80) --
	(111.03,207.91) --
	(111.08,207.02) --
	(111.14,206.13) --
	(111.22,205.25) --
	(111.31,204.36) --
	(111.42,203.48) --
	(111.54,202.60) --
	(111.68,201.72) --
	(111.84,200.85) --
	(112.01,199.97) --
	(112.19,199.10) --
	(112.39,198.24) --
	(112.61,197.37) --
	(112.84,196.52) --
	(113.09,195.66) --
	(113.35,194.81) --
	(113.62,193.97) --
	(113.91,193.12) --
	(114.22,192.29) --
	(114.54,191.46) --
	(114.87,190.64) --
	(115.22,189.82) --
	(115.58,189.01) --
	(115.96,188.20) --
	(116.35,187.40) --
	(116.76,186.61) --
	(117.17,185.82) --
	(117.61,185.05) --
	(118.05,184.28) --
	(118.51,183.52) --
	(118.98,182.76) --
	(119.47,182.02) --
	(119.97,181.28) --
	(120.48,180.55) --
	(121.00,179.83) --
	(121.54,179.13) --
	(122.09,178.43) --
	(122.65,177.74) --
	(123.22,177.06) --
	(123.81,176.39) --
	(124.40,175.73) --
	(125.01,175.08) --
	(125.63,174.44) --
	(126.26,173.81) --
	(126.90,173.19) --
	(127.55,172.59) --
	(128.22,172.00) --
	(128.89,171.41) --
	(129.57,170.84) --
	(130.26,170.29) --
	(130.97,169.74) --
	(131.68,169.21) --
	(132.40,168.69) --
	(133.13,168.18) --
	(133.87,167.68) --
	(134.61,167.20) --
	(135.37,166.73) --
	(136.13,166.27) --
	(136.90,165.83) --
	(137.68,165.40) --
	(138.47,164.99) --
	(139.26,164.59) --
	(140.06,164.20) --
	(140.87,163.83) --
	(141.69,163.47) --
	(142.50,163.12) --
	(143.33,162.79) --
	(144.16,162.48) --
	(145.00,162.17) --
	(145.84,161.89) --
	(146.69,161.62) --
	(147.54,161.36) --
	(148.39,161.12) --
	(149.25,160.89) --
	(150.12,160.68) --
	(150.99,160.48) --
	(151.86,160.30) --
	(152.73,160.13) --
	(153.61,159.98) --
	(154.48,159.85) --
	(155.37,159.73) --
	(156.25,159.62) --
	(157.13,159.53) --
	(158.02,159.46) --
	(158.91,159.40) --
	(159.80,159.36) --
	(160.69,159.33) --
	(161.57,159.32) --
	(162.46,159.32) --
	(163.35,159.34) --
	(164.24,159.38) --
	(165.13,159.43) --
	(166.02,159.49) --
	(166.90,159.58) --
	(167.79,159.67) --
	(168.67,159.78) --
	(169.55,159.91) --
	(170.43,160.06) --
	(171.30,160.21) --
	(172.17,160.39) --
	(173.04,160.58) --
	(173.91,160.78) --
	(174.77,161.00) --
	(175.63,161.24) --
	(176.48,161.49) --
	(177.33,161.75) --
	(178.17,162.03) --
	(179.01,162.32) --
	(179.85,162.63) --
	(180.68,162.96) --
	(181.50,163.29) --
	(182.32,163.65) --
	(183.13,164.01) --
	(183.93,164.39) --
	(184.73,164.79) --
	(185.52,165.19) --
	(186.30,165.62) --
	(187.08,166.05) --
	(187.84,166.50) --
	(188.60,166.96) --
	(189.35,167.44) --
	(190.10,167.93) --
	(190.83,168.43) --
	(191.56,168.94) --
	(192.27,169.47) --
	(192.98,170.01) --
	(193.68,170.56) --
	(194.36,171.13) --
	(195.04,171.70) --
	(195.71,172.29) --
	(196.37,172.89) --
	(197.01,173.50) --
	(197.65,174.12) --
	(198.27,174.76) --
	(198.89,175.40) --
	(199.49,176.05) --
	(200.08,176.72) --
	(200.66,177.39) --
	(201.23,178.08) --
	(201.78,178.77) --
	(202.32,179.48) --
	(202.85,180.19) --
	(203.37,180.92) --
	(203.88,181.65) --
	(204.37,182.39) --
	(204.85,183.14) --
	(205.31,183.90) --
	(205.77,184.66) --
	(206.21,185.43) --
	(206.63,186.22) --
	(207.04,187.00) --
	(207.44,187.80) --
	(207.82,188.60) --
	(208.19,189.41) --
	(208.55,190.23) --
	(208.89,191.05) --
	(209.22,191.87) --
	(209.53,192.71) --
	(209.83,193.54) --
	(210.11,194.39) --
	(210.38,195.24) --
	(210.63,196.09) --
	(210.87,196.94) --
	(211.09,197.81) --
	(211.30,198.67) --
	(211.49,199.54) --
	(211.67,200.41) --
	(211.83,201.28) --
	(211.98,202.16) --
	(212.11,203.04) --
	(212.23,203.92) --
	(212.33,204.81) --
	(212.42,205.69) --
	(212.49,206.58) --
	(212.54,207.47) --
	(212.58,208.35) --
	(212.60,209.24) --
	(212.61,210.13);
\definecolor[named]{drawColor}{rgb}{0.00,0.00,1.00}

\path[draw=drawColor,line width= 1.2pt,line join=round,line cap=round] (117.81,252.94) --
	(118.17,252.75) --
	(118.90,252.39) --
	(119.62,252.04) --
	(120.36,251.70) --
	(121.10,251.37) --
	(121.84,251.06) --
	(122.59,250.76) --
	(123.35,250.47) --
	(124.11,250.19) --
	(124.87,249.93) --
	(125.64,249.69) --
	(126.42,249.45) --
	(127.20,249.23) --
	(127.98,249.03) --
	(128.76,248.83) --
	(129.55,248.66) --
	(130.34,248.49) --
	(131.14,248.34) --
	(131.93,248.20) --
	(132.73,248.08) --
	(133.53,247.97) --
	(134.34,247.87) --
	(135.14,247.79) --
	(135.95,247.73) --
	(136.75,247.67) --
	(137.56,247.63) --
	(138.37,247.61) --
	(139.18,247.60) --
	(139.99,247.60) --
	(140.79,247.62) --
	(141.60,247.65) --
	(142.41,247.70) --
	(143.22,247.76) --
	(144.02,247.83) --
	(144.82,247.92) --
	(145.63,248.02) --
	(146.43,248.14) --
	(147.22,248.27) --
	(148.02,248.41) --
	(148.81,248.57) --
	(149.60,248.74) --
	(150.39,248.93) --
	(151.17,249.13) --
	(151.95,249.34) --
	(152.73,249.57) --
	(153.50,249.81) --
	(154.27,250.06) --
	(155.03,250.33) --
	(155.79,250.61) --
	(156.54,250.90) --
	(157.29,251.21) --
	(158.03,251.53) --
	(158.77,251.86) --
	(159.50,252.21) --
	(160.23,252.57) --
	(160.94,252.94) --
	(160.95,252.94);
\definecolor[named]{drawColor}{rgb}{0.00,0.00,0.00}

\path[draw=drawColor,line width= 0.4pt,dash pattern=on 1pt off 3pt ,line join=round,line cap=round] (109.16,252.94) --
	(109.26,252.87) --
	(109.98,252.35) --
	(110.71,251.84) --
	(111.45,251.34) --
	(112.20,250.86) --
	(112.95,250.39) --
	(113.72,249.93) --
	(114.49,249.49) --
	(115.27,249.06) --
	(116.05,248.65) --
	(116.85,248.25) --
	(117.65,247.86) --
	(118.45,247.49) --
	(119.27,247.13) --
	(120.09,246.78) --
	(120.91,246.45) --
	(121.75,246.14) --
	(122.58,245.83) --
	(123.42,245.55) --
	(124.27,245.28) --
	(125.12,245.02) --
	(125.98,244.78) --
	(126.84,244.55) --
	(127.70,244.34) --
	(128.57,244.14) --
	(129.44,243.96) --
	(130.31,243.79) --
	(131.19,243.64) --
	(132.07,243.51) --
	(132.95,243.39) --
	(133.83,243.28) --
	(134.72,243.19) --
	(135.60,243.12) --
	(136.49,243.06) --
	(137.38,243.02) --
	(138.27,242.99) --
	(139.16,242.98) --
	(140.05,242.98) --
	(140.94,243.00) --
	(141.82,243.04) --
	(142.71,243.09) --
	(143.60,243.15) --
	(144.49,243.24) --
	(145.37,243.33) --
	(146.25,243.44) --
	(147.13,243.57) --
	(148.01,243.72) --
	(148.88,243.87) --
	(149.76,244.05) --
	(150.63,244.24) --
	(151.49,244.44) --
	(152.35,244.66) --
	(153.21,244.90) --
	(154.06,245.15) --
	(154.91,245.41) --
	(155.76,245.69) --
	(156.60,245.98) --
	(157.43,246.29) --
	(158.26,246.62) --
	(159.08,246.95) --
	(159.90,247.31) --
	(160.71,247.67) --
	(161.51,248.05) --
	(162.31,248.45) --
	(163.10,248.85) --
	(163.88,249.28) --
	(164.66,249.71) --
	(165.43,250.16) --
	(166.19,250.62) --
	(166.94,251.10) --
	(167.68,251.59) --
	(168.41,252.09) --
	(169.14,252.60) --
	(169.60,252.94);
\definecolor[named]{drawColor}{rgb}{0.00,0.00,1.00}

\path[draw=drawColor,line width= 1.2pt,line join=round,line cap=round] (252.94, 87.57) --
	(252.19, 87.37) --
	(251.41, 87.15) --
	(250.63, 86.92) --
	(249.87, 86.67) --
	(249.10, 86.41) --
	(248.34, 86.14) --
	(247.58, 85.85) --
	(246.83, 85.55) --
	(246.09, 85.24) --
	(245.35, 84.91) --
	(244.61, 84.57) --
	(243.89, 84.22) --
	(243.16, 83.85) --
	(242.45, 83.48) --
	(241.74, 83.09) --
	(241.04, 82.68) --
	(240.35, 82.27) --
	(239.66, 81.84) --
	(238.98, 81.40) --
	(238.31, 80.95) --
	(237.64, 80.49) --
	(236.99, 80.02) --
	(236.34, 79.53) --
	(235.70, 79.04) --
	(235.07, 78.53) --
	(234.45, 78.01) --
	(233.84, 77.48) --
	(233.24, 76.94) --
	(232.65, 76.39) --
	(232.07, 75.83) --
	(231.49, 75.26) --
	(230.93, 74.68) --
	(230.38, 74.09) --
	(229.83, 73.49) --
	(229.30, 72.88) --
	(228.78, 72.26) --
	(228.27, 71.64) --
	(227.77, 71.00) --
	(227.29, 70.36) --
	(226.81, 69.70) --
	(226.34, 69.04) --
	(225.89, 68.37) --
	(225.45, 67.69) --
	(225.02, 67.01) --
	(224.60, 66.32) --
	(224.20, 65.62) --
	(223.80, 64.91) --
	(223.42, 64.20) --
	(223.06, 63.48) --
	(222.70, 62.75) --
	(222.36, 62.02) --
	(222.03, 61.28) --
	(221.71, 60.54) --
	(221.41, 59.79) --
	(221.12, 59.03) --
	(220.84, 58.27) --
	(220.58, 57.51) --
	(220.33, 56.74) --
	(220.09, 55.97) --
	(219.87, 55.19) --
	(219.66, 54.41) --
	(219.46, 53.63) --
	(219.28, 52.84) --
	(219.11, 52.05) --
	(218.95, 51.25) --
	(218.81, 50.46) --
	(218.69, 49.66) --
	(218.57, 48.86) --
	(218.47, 48.06) --
	(218.39, 47.25) --
	(218.32, 46.45) --
	(218.26, 45.64) --
	(218.22, 44.83) --
	(218.19, 44.03) --
	(218.18, 43.22) --
	(218.18, 42.41) --
	(218.19, 41.60) --
	(218.22, 40.79) --
	(218.26, 39.98) --
	(218.32, 39.18) --
	(218.39, 38.37) --
	(218.47, 37.57) --
	(218.57, 36.77) --
	(218.69, 35.97) --
	(218.81, 35.17) --
	(218.95, 34.37) --
	(219.11, 33.58) --
	(219.28, 32.79) --
	(219.46, 32.00) --
	(219.66, 31.21) --
	(219.87, 30.43) --
	(220.09, 29.66) --
	(220.33, 28.88) --
	(220.58, 28.12) --
	(220.84, 27.35) --
	(221.12, 26.59) --
	(221.41, 25.84) --
	(221.71, 25.09) --
	(222.03, 24.34) --
	(222.36, 23.61) --
	(222.70, 22.87) --
	(223.06, 22.15) --
	(223.42, 21.43) --
	(223.80, 20.71) --
	(224.20, 20.01) --
	(224.60, 19.31) --
	(225.02, 18.62) --
	(225.45, 17.93) --
	(225.89, 17.25) --
	(226.34, 16.58) --
	(226.81, 15.92) --
	(227.29, 15.27) --
	(227.77, 14.62) --
	(228.27, 13.99) --
	(228.78, 13.36) --
	(229.30, 12.74) --
	(229.83, 12.13) --
	(230.38, 11.53) --
	(230.93, 10.94) --
	(231.49, 10.36) --
	(232.07,  9.79) --
	(232.65,  9.23) --
	(233.24,  8.68) --
	(233.84,  8.14) --
	(234.45,  7.61) --
	(235.07,  7.10) --
	(235.70,  6.59) --
	(236.34,  6.09) --
	(236.99,  5.61) --
	(237.64,  5.13) --
	(238.31,  4.67) --
	(238.98,  4.22) --
	(239.66,  3.78) --
	(240.35,  3.36) --
	(241.04,  2.94) --
	(241.74,  2.54) --
	(242.45,  2.15) --
	(243.16,  1.77) --
	(243.89,  1.41) --
	(244.61,  1.06) --
	(245.35,  0.72) --
	(246.09,  0.39) --
	(246.83,  0.08) --
	(247.02,  0.00);
\definecolor[named]{drawColor}{rgb}{0.00,0.00,0.00}

\path[draw=drawColor,line width= 0.4pt,dash pattern=on 1pt off 3pt ,line join=round,line cap=round] (252.94, 92.32) --
	(252.69, 92.27) --
	(251.83, 92.05) --
	(250.97, 91.83) --
	(250.11, 91.59) --
	(249.26, 91.33) --
	(248.41, 91.06) --
	(247.57, 90.77) --
	(246.74, 90.47) --
	(245.90, 90.15) --
	(245.08, 89.82) --
	(244.26, 89.48) --
	(243.45, 89.12) --
	(242.64, 88.75) --
	(241.84, 88.36) --
	(241.04, 87.96) --
	(240.26, 87.54) --
	(239.48, 87.11) --
	(238.71, 86.67) --
	(237.94, 86.21) --
	(237.19, 85.75) --
	(236.44, 85.26) --
	(235.70, 84.77) --
	(234.97, 84.26) --
	(234.25, 83.74) --
	(233.54, 83.21) --
	(232.84, 82.66) --
	(232.15, 82.10) --
	(231.46, 81.53) --
	(230.79, 80.95) --
	(230.13, 80.36) --
	(229.48, 79.75) --
	(228.83, 79.13) --
	(228.20, 78.51) --
	(227.59, 77.87) --
	(226.98, 77.22) --
	(226.38, 76.56) --
	(225.80, 75.89) --
	(225.22, 75.21) --
	(224.66, 74.52) --
	(224.11, 73.82) --
	(223.58, 73.11) --
	(223.05, 72.39) --
	(222.54, 71.66) --
	(222.04, 70.93) --
	(221.56, 70.18) --
	(221.08, 69.43) --
	(220.63, 68.67) --
	(220.18, 67.90) --
	(219.75, 67.12) --
	(219.33, 66.34) --
	(218.92, 65.54) --
	(218.53, 64.75) --
	(218.16, 63.94) --
	(217.79, 63.13) --
	(217.45, 62.31) --
	(217.11, 61.49) --
	(216.79, 60.66) --
	(216.49, 59.82) --
	(216.20, 58.98) --
	(215.92, 58.13) --
	(215.66, 57.28) --
	(215.41, 56.43) --
	(215.18, 55.57) --
	(214.97, 54.71) --
	(214.77, 53.84) --
	(214.58, 52.97) --
	(214.41, 52.10) --
	(214.26, 51.22) --
	(214.12, 50.34) --
	(213.99, 49.46) --
	(213.88, 48.58) --
	(213.79, 47.70) --
	(213.71, 46.81) --
	(213.65, 45.92) --
	(213.60, 45.04) --
	(213.57, 44.15) --
	(213.56, 43.26) --
	(213.56, 42.37) --
	(213.57, 41.48) --
	(213.60, 40.59) --
	(213.65, 39.70) --
	(213.71, 38.81) --
	(213.79, 37.93) --
	(213.88, 37.04) --
	(213.99, 36.16) --
	(214.12, 35.28) --
	(214.26, 34.40) --
	(214.41, 33.53) --
	(214.58, 32.65) --
	(214.77, 31.78) --
	(214.97, 30.92) --
	(215.18, 30.05) --
	(215.41, 29.20) --
	(215.66, 28.34) --
	(215.92, 27.49) --
	(216.20, 26.65) --
	(216.49, 25.80) --
	(216.79, 24.97) --
	(217.11, 24.14) --
	(217.45, 23.32) --
	(217.79, 22.50) --
	(218.16, 21.69) --
	(218.53, 20.88) --
	(218.92, 20.08) --
	(219.33, 19.29) --
	(219.75, 18.50) --
	(220.18, 17.73) --
	(220.63, 16.96) --
	(221.08, 16.20) --
	(221.56, 15.44) --
	(222.04, 14.70) --
	(222.54, 13.96) --
	(223.05, 13.23) --
	(223.58, 12.51) --
	(224.11, 11.81) --
	(224.66, 11.11) --
	(225.22, 10.42) --
	(225.80,  9.74) --
	(226.38,  9.07) --
	(226.98,  8.41) --
	(227.59,  7.76) --
	(228.20,  7.12) --
	(228.83,  6.49) --
	(229.48,  5.87) --
	(230.13,  5.27) --
	(230.79,  4.68) --
	(231.46,  4.09) --
	(232.15,  3.52) --
	(232.84,  2.97) --
	(233.54,  2.42) --
	(234.25,  1.89) --
	(234.97,  1.37) --
	(235.70,  0.86) --
	(236.44,  0.36) --
	(237.00,  0.00);

\path[fill=fillColor] (241.95,126.47) circle (  2.25);
\definecolor[named]{drawColor}{rgb}{0.00,0.00,1.00}

\path[draw=drawColor,line width= 1.2pt,line join=round,line cap=round] (252.94,171.34) --
	(252.18,171.52) --
	(251.39,171.69) --
	(250.59,171.85) --
	(249.80,172.00) --
	(249.00,172.13) --
	(248.20,172.24) --
	(247.40,172.35) --
	(246.60,172.43) --
	(245.79,172.51) --
	(244.98,172.57) --
	(244.18,172.61) --
	(243.37,172.65) --
	(242.56,172.66) --
	(241.75,172.67) --
	(240.94,172.66) --
	(240.14,172.63) --
	(239.33,172.59) --
	(238.52,172.54) --
	(237.72,172.47) --
	(236.91,172.39) --
	(236.11,172.30) --
	(235.31,172.19) --
	(234.51,172.06) --
	(233.71,171.93) --
	(232.92,171.77) --
	(232.13,171.61) --
	(231.34,171.43) --
	(230.55,171.24) --
	(229.77,171.03) --
	(228.99,170.81) --
	(228.22,170.58) --
	(227.45,170.33) --
	(226.68,170.07) --
	(225.92,169.80) --
	(225.17,169.51) --
	(224.42,169.21) --
	(223.67,168.90) --
	(222.93,168.57) --
	(222.20,168.23) --
	(221.47,167.88) --
	(220.75,167.51) --
	(220.03,167.14) --
	(219.32,166.75) --
	(218.62,166.34) --
	(217.93,165.93) --
	(217.24,165.50) --
	(216.56,165.06) --
	(215.89,164.61) --
	(215.23,164.15) --
	(214.57,163.68) --
	(213.93,163.19) --
	(213.29,162.70) --
	(212.66,162.19) --
	(212.04,161.67) --
	(211.43,161.14) --
	(210.82,160.60) --
	(210.23,160.05) --
	(209.65,159.49) --
	(209.08,158.92) --
	(208.51,158.34) --
	(207.96,157.75) --
	(207.42,157.15) --
	(206.89,156.54) --
	(206.37,155.92) --
	(205.86,155.30) --
	(205.36,154.66) --
	(204.87,154.02) --
	(204.39,153.36) --
	(203.93,152.70) --
	(203.47,152.03) --
	(203.03,151.35) --
	(202.60,150.67) --
	(202.19,149.98) --
	(201.78,149.28) --
	(201.39,148.57) --
	(201.01,147.86) --
	(200.64,147.14) --
	(200.28,146.41) --
	(199.94,145.68) --
	(199.61,144.94) --
	(199.29,144.20) --
	(198.99,143.45) --
	(198.70,142.69) --
	(198.42,141.93) --
	(198.16,141.17) --
	(197.91,140.40) --
	(197.67,139.63) --
	(197.45,138.85) --
	(197.24,138.07) --
	(197.04,137.29) --
	(196.86,136.50) --
	(196.69,135.71) --
	(196.54,134.91) --
	(196.40,134.12) --
	(196.27,133.32) --
	(196.16,132.52) --
	(196.06,131.72) --
	(195.97,130.91) --
	(195.90,130.11) --
	(195.85,129.30) --
	(195.80,128.49) --
	(195.78,127.69) --
	(195.76,126.88) --
	(195.76,126.07) --
	(195.78,125.26) --
	(195.80,124.45) --
	(195.85,123.64) --
	(195.90,122.84) --
	(195.97,122.03) --
	(196.06,121.23) --
	(196.16,120.43) --
	(196.27,119.63) --
	(196.40,118.83) --
	(196.54,118.03) --
	(196.69,117.24) --
	(196.86,116.45) --
	(197.04,115.66) --
	(197.24,114.87) --
	(197.45,114.09) --
	(197.67,113.32) --
	(197.91,112.54) --
	(198.16,111.78) --
	(198.42,111.01) --
	(198.70,110.25) --
	(198.99,109.50) --
	(199.29,108.75) --
	(199.61,108.00) --
	(199.94,107.27) --
	(200.28,106.53) --
	(200.64,105.81) --
	(201.01,105.09) --
	(201.39,104.37) --
	(201.78,103.67) --
	(202.19,102.97) --
	(202.60,102.28) --
	(203.03,101.59) --
	(203.47,100.91) --
	(203.93,100.24) --
	(204.39, 99.58) --
	(204.87, 98.93) --
	(205.36, 98.28) --
	(205.86, 97.65) --
	(206.37, 97.02) --
	(206.89, 96.40) --
	(207.42, 95.79) --
	(207.96, 95.19) --
	(208.51, 94.60) --
	(209.08, 94.02) --
	(209.65, 93.45) --
	(210.23, 92.89) --
	(210.82, 92.34) --
	(211.43, 91.80) --
	(212.04, 91.27) --
	(212.66, 90.76) --
	(213.29, 90.25) --
	(213.93, 89.75) --
	(214.57, 89.27) --
	(215.23, 88.79) --
	(215.89, 88.33) --
	(216.56, 87.88) --
	(217.24, 87.44) --
	(217.93, 87.02) --
	(218.62, 86.60) --
	(219.32, 86.20) --
	(220.03, 85.81) --
	(220.75, 85.43) --
	(221.47, 85.07) --
	(222.20, 84.72) --
	(222.93, 84.38) --
	(223.67, 84.05) --
	(224.42, 83.74) --
	(225.17, 83.44) --
	(225.92, 83.15) --
	(226.68, 82.87) --
	(227.45, 82.61) --
	(228.22, 82.37) --
	(228.99, 82.13) --
	(229.77, 81.91) --
	(230.55, 81.71) --
	(231.34, 81.51) --
	(232.13, 81.34) --
	(232.92, 81.17) --
	(233.71, 81.02) --
	(234.51, 80.88) --
	(235.31, 80.76) --
	(236.11, 80.65) --
	(236.91, 80.55) --
	(237.72, 80.47) --
	(238.52, 80.41) --
	(239.33, 80.35) --
	(240.14, 80.31) --
	(240.94, 80.29) --
	(241.75, 80.28) --
	(242.56, 80.28) --
	(243.37, 80.30) --
	(244.18, 80.33) --
	(244.98, 80.38) --
	(245.79, 80.44) --
	(246.60, 80.51) --
	(247.40, 80.60) --
	(248.20, 80.70) --
	(249.00, 80.82) --
	(249.80, 80.95) --
	(250.59, 81.09) --
	(251.39, 81.25) --
	(252.18, 81.42) --
	(252.94, 81.60);
\definecolor[named]{drawColor}{rgb}{0.00,0.00,0.00}

\path[draw=drawColor,line width= 0.4pt,dash pattern=on 1pt off 3pt ,line join=round,line cap=round] (252.94,176.08) --
	(252.33,176.22) --
	(251.46,176.39) --
	(250.58,176.55) --
	(249.71,176.69) --
	(248.83,176.82) --
	(247.94,176.93) --
	(247.06,177.03) --
	(246.17,177.11) --
	(245.29,177.18) --
	(244.40,177.23) --
	(243.51,177.26) --
	(242.62,177.28) --
	(241.73,177.29) --
	(240.84,177.27) --
	(239.95,177.25) --
	(239.07,177.20) --
	(238.18,177.15) --
	(237.29,177.07) --
	(236.41,176.98) --
	(235.52,176.88) --
	(234.64,176.76) --
	(233.76,176.62) --
	(232.89,176.47) --
	(232.01,176.31) --
	(231.14,176.12) --
	(230.28,175.93) --
	(229.41,175.71) --
	(228.55,175.49) --
	(227.70,175.25) --
	(226.84,174.99) --
	(226.00,174.72) --
	(225.16,174.43) --
	(224.32,174.13) --
	(223.49,173.81) --
	(222.66,173.48) --
	(221.84,173.14) --
	(221.03,172.78) --
	(220.22,172.41) --
	(219.42,172.02) --
	(218.63,171.62) --
	(217.84,171.20) --
	(217.06,170.77) --
	(216.29,170.33) --
	(215.53,169.87) --
	(214.77,169.41) --
	(214.02,168.92) --
	(213.29,168.43) --
	(212.56,167.92) --
	(211.83,167.40) --
	(211.12,166.87) --
	(210.42,166.32) --
	(209.73,165.76) --
	(209.05,165.19) --
	(208.37,164.61) --
	(207.71,164.02) --
	(207.06,163.41) --
	(206.42,162.79) --
	(205.79,162.17) --
	(205.17,161.53) --
	(204.56,160.88) --
	(203.96,160.22) --
	(203.38,159.55) --
	(202.81,158.87) --
	(202.25,158.18) --
	(201.70,157.48) --
	(201.16,156.77) --
	(200.64,156.05) --
	(200.12,155.32) --
	(199.63,154.59) --
	(199.14,153.84) --
	(198.67,153.09) --
	(198.21,152.33) --
	(197.76,151.56) --
	(197.33,150.78) --
	(196.91,150.00) --
	(196.51,149.20) --
	(196.12,148.41) --
	(195.74,147.60) --
	(195.38,146.79) --
	(195.03,145.97) --
	(194.69,145.15) --
	(194.38,144.32) --
	(194.07,143.48) --
	(193.78,142.64) --
	(193.50,141.79) --
	(193.24,140.94) --
	(193.00,140.09) --
	(192.77,139.23) --
	(192.55,138.37) --
	(192.35,137.50) --
	(192.17,136.63) --
	(192.00,135.76) --
	(191.84,134.88) --
	(191.70,134.00) --
	(191.58,133.12) --
	(191.47,132.24) --
	(191.37,131.36) --
	(191.30,130.47) --
	(191.23,129.58) --
	(191.19,128.70) --
	(191.16,127.81) --
	(191.14,126.92) --
	(191.14,126.03) --
	(191.16,125.14) --
	(191.19,124.25) --
	(191.23,123.36) --
	(191.30,122.47) --
	(191.37,121.59) --
	(191.47,120.70) --
	(191.58,119.82) --
	(191.70,118.94) --
	(191.84,118.06) --
	(192.00,117.19) --
	(192.17,116.31) --
	(192.35,115.44) --
	(192.55,114.58) --
	(192.77,113.71) --
	(193.00,112.86) --
	(193.24,112.00) --
	(193.50,111.15) --
	(193.78,110.31) --
	(194.07,109.46) --
	(194.38,108.63) --
	(194.69,107.80) --
	(195.03,106.98) --
	(195.38,106.16) --
	(195.74,105.35) --
	(196.12,104.54) --
	(196.51,103.74) --
	(196.91,102.95) --
	(197.33,102.16) --
	(197.76,101.39) --
	(198.21,100.62) --
	(198.67, 99.86) --
	(199.14, 99.10) --
	(199.63, 98.36) --
	(200.12, 97.62) --
	(200.64, 96.89) --
	(201.16, 96.17) --
	(201.70, 95.47) --
	(202.25, 94.77) --
	(202.81, 94.08) --
	(203.38, 93.40) --
	(203.96, 92.73) --
	(204.56, 92.07) --
	(205.17, 91.42) --
	(205.79, 90.78) --
	(206.42, 90.15) --
	(207.06, 89.53) --
	(207.71, 88.93) --
	(208.37, 88.34) --
	(209.05, 87.75) --
	(209.73, 87.18) --
	(210.42, 86.63) --
	(211.12, 86.08) --
	(211.83, 85.55) --
	(212.56, 85.03) --
	(213.29, 84.52) --
	(214.02, 84.02) --
	(214.77, 83.54) --
	(215.53, 83.07) --
	(216.29, 82.61) --
	(217.06, 82.17) --
	(217.84, 81.74) --
	(218.63, 81.33) --
	(219.42, 80.93) --
	(220.22, 80.54) --
	(221.03, 80.17) --
	(221.84, 79.81) --
	(222.66, 79.46) --
	(223.49, 79.13) --
	(224.32, 78.82) --
	(225.16, 78.51) --
	(226.00, 78.23) --
	(226.84, 77.96) --
	(227.70, 77.70) --
	(228.55, 77.46) --
	(229.41, 77.23) --
	(230.28, 77.02) --
	(231.14, 76.82) --
	(232.01, 76.64) --
	(232.89, 76.47) --
	(233.76, 76.32) --
	(234.64, 76.19) --
	(235.52, 76.07) --
	(236.41, 75.96) --
	(237.29, 75.87) --
	(238.18, 75.80) --
	(239.07, 75.74) --
	(239.95, 75.70) --
	(240.84, 75.67) --
	(241.73, 75.66) --
	(242.62, 75.66) --
	(243.51, 75.68) --
	(244.40, 75.72) --
	(245.29, 75.77) --
	(246.17, 75.83) --
	(247.06, 75.92) --
	(247.94, 76.01) --
	(248.83, 76.12) --
	(249.71, 76.25) --
	(250.58, 76.40) --
	(251.46, 76.55) --
	(252.33, 76.73) --
	(252.94, 76.86);

\path[fill=fillColor] (219.54,210.13) circle (  2.25);
\definecolor[named]{drawColor}{rgb}{0.00,0.00,1.00}

\path[draw=drawColor,line width= 1.2pt,line join=round,line cap=round] (252.94,242.03) --
	(252.70,242.29) --
	(252.13,242.87) --
	(251.55,243.43) --
	(250.97,243.99) --
	(250.37,244.53) --
	(249.76,245.07) --
	(249.14,245.59) --
	(248.52,246.10) --
	(247.89,246.61) --
	(247.24,247.10) --
	(246.59,247.58) --
	(245.93,248.04) --
	(245.26,248.50) --
	(244.59,248.94) --
	(243.91,249.38) --
	(243.22,249.80) --
	(242.52,250.21) --
	(241.81,250.60) --
	(241.10,250.99) --
	(240.38,251.36) --
	(239.66,251.72) --
	(238.93,252.06) --
	(238.19,252.39) --
	(237.45,252.71) --
	(236.89,252.94);

\path[draw=drawColor,line width= 1.2pt,line join=round,line cap=round] (202.19,252.94) --
	(202.00,252.87) --
	(201.25,252.56) --
	(200.51,252.23) --
	(199.78,251.89) --
	(199.05,251.54) --
	(198.33,251.17) --
	(197.62,250.80) --
	(196.91,250.41) --
	(196.21,250.00) --
	(195.51,249.59) --
	(194.83,249.16) --
	(194.15,248.72) --
	(193.48,248.27) --
	(192.81,247.81) --
	(192.16,247.34) --
	(191.51,246.85) --
	(190.87,246.36) --
	(190.24,245.85) --
	(189.62,245.33) --
	(189.01,244.80) --
	(188.41,244.26) --
	(187.81,243.71) --
	(187.23,243.15) --
	(186.66,242.58) --
	(186.10,242.00) --
	(185.54,241.41) --
	(185.00,240.81) --
	(184.47,240.20) --
	(183.95,239.58) --
	(183.44,238.96) --
	(182.94,238.32) --
	(182.45,237.68) --
	(181.98,237.02) --
	(181.51,236.36) --
	(181.06,235.69) --
	(180.62,235.01) --
	(180.19,234.33) --
	(179.77,233.64) --
	(179.36,232.94) --
	(178.97,232.23) --
	(178.59,231.52) --
	(178.22,230.80) --
	(177.87,230.07) --
	(177.52,229.34) --
	(177.19,228.60) --
	(176.88,227.86) --
	(176.57,227.11) --
	(176.28,226.35) --
	(176.01,225.59) --
	(175.74,224.83) --
	(175.49,224.06) --
	(175.26,223.29) --
	(175.03,222.51) --
	(174.82,221.73) --
	(174.63,220.95) --
	(174.44,220.16) --
	(174.28,219.37) --
	(174.12,218.57) --
	(173.98,217.78) --
	(173.85,216.98) --
	(173.74,216.18) --
	(173.64,215.38) --
	(173.56,214.57) --
	(173.49,213.77) --
	(173.43,212.96) --
	(173.39,212.15) --
	(173.36,211.35) --
	(173.34,210.54) --
	(173.34,209.73) --
	(173.36,208.92) --
	(173.39,208.11) --
	(173.43,207.30) --
	(173.49,206.50) --
	(173.56,205.69) --
	(173.64,204.89) --
	(173.74,204.09) --
	(173.85,203.29) --
	(173.98,202.49) --
	(174.12,201.69) --
	(174.28,200.90) --
	(174.44,200.11) --
	(174.63,199.32) --
	(174.82,198.53) --
	(175.03,197.75) --
	(175.26,196.98) --
	(175.49,196.20) --
	(175.74,195.44) --
	(176.01,194.67) --
	(176.28,193.91) --
	(176.57,193.16) --
	(176.88,192.41) --
	(177.19,191.66) --
	(177.52,190.93) --
	(177.87,190.19) --
	(178.22,189.47) --
	(178.59,188.75) --
	(178.97,188.03) --
	(179.36,187.33) --
	(179.77,186.63) --
	(180.19,185.94) --
	(180.62,185.25) --
	(181.06,184.57) --
	(181.51,183.90) --
	(181.98,183.24) --
	(182.45,182.59) --
	(182.94,181.94) --
	(183.44,181.31) --
	(183.95,180.68) --
	(184.47,180.06) --
	(185.00,179.45) --
	(185.54,178.85) --
	(186.10,178.26) --
	(186.66,177.68) --
	(187.23,177.11) --
	(187.81,176.55) --
	(188.41,176.00) --
	(189.01,175.46) --
	(189.62,174.93) --
	(190.24,174.42) --
	(190.87,173.91) --
	(191.51,173.41) --
	(192.16,172.93) --
	(192.81,172.45) --
	(193.48,171.99) --
	(194.15,171.54) --
	(194.83,171.10) --
	(195.51,170.68) --
	(196.21,170.26) --
	(196.91,169.86) --
	(197.62,169.47) --
	(198.33,169.09) --
	(199.05,168.73) --
	(199.78,168.38) --
	(200.51,168.04) --
	(201.25,167.71) --
	(202.00,167.40) --
	(202.75,167.10) --
	(203.51,166.81) --
	(204.27,166.53) --
	(205.03,166.27) --
	(205.80,166.03) --
	(206.58,165.79) --
	(207.35,165.57) --
	(208.14,165.37) --
	(208.92,165.17) --
	(209.71,165.00) --
	(210.50,164.83) --
	(211.29,164.68) --
	(212.09,164.54) --
	(212.89,164.42) --
	(213.69,164.31) --
	(214.49,164.21) --
	(215.30,164.13) --
	(216.10,164.07) --
	(216.91,164.01) --
	(217.72,163.97) --
	(218.53,163.95) --
	(219.34,163.94) --
	(220.14,163.94) --
	(220.95,163.96) --
	(221.76,163.99) --
	(222.57,164.04) --
	(223.37,164.10) --
	(224.18,164.17) --
	(224.98,164.26) --
	(225.78,164.36) --
	(226.58,164.48) --
	(227.38,164.61) --
	(228.18,164.75) --
	(228.97,164.91) --
	(229.76,165.08) --
	(230.55,165.27) --
	(231.33,165.47) --
	(232.11,165.68) --
	(232.89,165.91) --
	(233.66,166.15) --
	(234.43,166.40) --
	(235.19,166.67) --
	(235.95,166.95) --
	(236.70,167.24) --
	(237.45,167.55) --
	(238.19,167.87) --
	(238.93,168.20) --
	(239.66,168.55) --
	(240.38,168.91) --
	(241.10,169.28) --
	(241.81,169.66) --
	(242.52,170.06) --
	(243.22,170.47) --
	(243.91,170.89) --
	(244.59,171.32) --
	(245.26,171.76) --
	(245.93,172.22) --
	(246.59,172.69) --
	(247.24,173.17) --
	(247.89,173.66) --
	(248.52,174.16) --
	(249.14,174.67) --
	(249.76,175.20) --
	(250.37,175.73) --
	(250.97,176.28) --
	(251.55,176.83) --
	(252.13,177.40) --
	(252.70,177.97) --
	(252.94,178.23);
\definecolor[named]{drawColor}{rgb}{0.00,0.00,0.00}

\path[draw=drawColor,line width= 0.4pt,dash pattern=on 1pt off 3pt ,line join=round,line cap=round] (252.94,248.42) --
	(252.78,248.56) --
	(252.11,249.14) --
	(251.42,249.70) --
	(250.72,250.25) --
	(250.01,250.79) --
	(249.30,251.32) --
	(248.57,251.84) --
	(247.84,252.34) --
	(247.09,252.83) --
	(246.91,252.94);

\path[draw=drawColor,line width= 0.4pt,dash pattern=on 1pt off 3pt ,line join=round,line cap=round] (192.17,252.94) --
	(191.61,252.58) --
	(190.87,252.09) --
	(190.14,251.58) --
	(189.42,251.06) --
	(188.71,250.53) --
	(188.00,249.98) --
	(187.31,249.42) --
	(186.63,248.85) --
	(185.96,248.27) --
	(185.29,247.68) --
	(184.64,247.07) --
	(184.00,246.45) --
	(183.37,245.83) --
	(182.75,245.19) --
	(182.14,244.54) --
	(181.55,243.88) --
	(180.96,243.21) --
	(180.39,242.53) --
	(179.83,241.84) --
	(179.28,241.14) --
	(178.74,240.43) --
	(178.22,239.71) --
	(177.71,238.98) --
	(177.21,238.25) --
	(176.72,237.50) --
	(176.25,236.75) --
	(175.79,235.99) --
	(175.35,235.22) --
	(174.91,234.44) --
	(174.50,233.66) --
	(174.09,232.86) --
	(173.70,232.07) --
	(173.32,231.26) --
	(172.96,230.45) --
	(172.61,229.63) --
	(172.28,228.81) --
	(171.96,227.98) --
	(171.65,227.14) --
	(171.36,226.30) --
	(171.09,225.45) --
	(170.83,224.60) --
	(170.58,223.75) --
	(170.35,222.89) --
	(170.13,222.03) --
	(169.93,221.16) --
	(169.75,220.29) --
	(169.58,219.42) --
	(169.42,218.54) --
	(169.28,217.66) --
	(169.16,216.78) --
	(169.05,215.90) --
	(168.96,215.02) --
	(168.88,214.13) --
	(168.82,213.24) --
	(168.77,212.36) --
	(168.74,211.47) --
	(168.72,210.58) --
	(168.72,209.69) --
	(168.74,208.80) --
	(168.77,207.91) --
	(168.82,207.02) --
	(168.88,206.13) --
	(168.96,205.25) --
	(169.05,204.36) --
	(169.16,203.48) --
	(169.28,202.60) --
	(169.42,201.72) --
	(169.58,200.85) --
	(169.75,199.97) --
	(169.93,199.10) --
	(170.13,198.24) --
	(170.35,197.37) --
	(170.58,196.52) --
	(170.83,195.66) --
	(171.09,194.81) --
	(171.36,193.97) --
	(171.65,193.12) --
	(171.96,192.29) --
	(172.28,191.46) --
	(172.61,190.64) --
	(172.96,189.82) --
	(173.32,189.01) --
	(173.70,188.20) --
	(174.09,187.40) --
	(174.50,186.61) --
	(174.91,185.82) --
	(175.35,185.05) --
	(175.79,184.28) --
	(176.25,183.52) --
	(176.72,182.76) --
	(177.21,182.02) --
	(177.71,181.28) --
	(178.22,180.55) --
	(178.74,179.83) --
	(179.28,179.13) --
	(179.83,178.43) --
	(180.39,177.74) --
	(180.96,177.06) --
	(181.55,176.39) --
	(182.14,175.73) --
	(182.75,175.08) --
	(183.37,174.44) --
	(184.00,173.81) --
	(184.64,173.19) --
	(185.29,172.59) --
	(185.96,172.00) --
	(186.63,171.41) --
	(187.31,170.84) --
	(188.00,170.29) --
	(188.71,169.74) --
	(189.42,169.21) --
	(190.14,168.69) --
	(190.87,168.18) --
	(191.61,167.68) --
	(192.35,167.20) --
	(193.11,166.73) --
	(193.87,166.27) --
	(194.65,165.83) --
	(195.42,165.40) --
	(196.21,164.99) --
	(197.00,164.59) --
	(197.81,164.20) --
	(198.61,163.83) --
	(199.43,163.47) --
	(200.25,163.12) --
	(201.07,162.79) --
	(201.90,162.48) --
	(202.74,162.17) --
	(203.58,161.89) --
	(204.43,161.62) --
	(205.28,161.36) --
	(206.14,161.12) --
	(207.00,160.89) --
	(207.86,160.68) --
	(208.73,160.48) --
	(209.60,160.30) --
	(210.47,160.13) --
	(211.35,159.98) --
	(212.23,159.85) --
	(213.11,159.73) --
	(213.99,159.62) --
	(214.87,159.53) --
	(215.76,159.46) --
	(216.65,159.40) --
	(217.54,159.36) --
	(218.43,159.33) --
	(219.32,159.32) --
	(220.20,159.32) --
	(221.09,159.34) --
	(221.98,159.38) --
	(222.87,159.43) --
	(223.76,159.49) --
	(224.64,159.58) --
	(225.53,159.67) --
	(226.41,159.78) --
	(227.29,159.91) --
	(228.17,160.06) --
	(229.04,160.21) --
	(229.91,160.39) --
	(230.78,160.58) --
	(231.65,160.78) --
	(232.51,161.00) --
	(233.37,161.24) --
	(234.22,161.49) --
	(235.07,161.75) --
	(235.92,162.03) --
	(236.75,162.32) --
	(237.59,162.63) --
	(238.42,162.96) --
	(239.24,163.29) --
	(240.06,163.65) --
	(240.87,164.01) --
	(241.67,164.39) --
	(242.47,164.79) --
	(243.26,165.19) --
	(244.04,165.62) --
	(244.82,166.05) --
	(245.58,166.50) --
	(246.34,166.96) --
	(247.09,167.44) --
	(247.84,167.93) --
	(248.57,168.43) --
	(249.30,168.94) --
	(250.01,169.47) --
	(250.72,170.01) --
	(251.42,170.56) --
	(252.11,171.13) --
	(252.78,171.70) --
	(252.94,171.85);
\definecolor[named]{drawColor}{rgb}{0.00,0.00,1.00}

\path[draw=drawColor,line width= 1.2pt,line join=round,line cap=round] (175.55,252.94) --
	(175.91,252.75) --
	(176.64,252.39) --
	(177.36,252.04) --
	(178.10,251.70) --
	(178.84,251.37) --
	(179.58,251.06) --
	(180.33,250.76) --
	(181.09,250.47) --
	(181.85,250.19) --
	(182.62,249.93) --
	(183.39,249.69) --
	(184.16,249.45) --
	(184.94,249.23) --
	(185.72,249.03) --
	(186.50,248.83) --
	(187.29,248.66) --
	(188.08,248.49) --
	(188.88,248.34) --
	(189.67,248.20) --
	(190.47,248.08) --
	(191.27,247.97) --
	(192.08,247.87) --
	(192.88,247.79) --
	(193.69,247.73) --
	(194.49,247.67) --
	(195.30,247.63) --
	(196.11,247.61) --
	(196.92,247.60) --
	(197.73,247.60) --
	(198.54,247.62) --
	(199.34,247.65) --
	(200.15,247.70) --
	(200.96,247.76) --
	(201.76,247.83) --
	(202.57,247.92) --
	(203.37,248.02) --
	(204.17,248.14) --
	(204.97,248.27) --
	(205.76,248.41) --
	(206.55,248.57) --
	(207.34,248.74) --
	(208.13,248.93) --
	(208.91,249.13) --
	(209.69,249.34) --
	(210.47,249.57) --
	(211.24,249.81) --
	(212.01,250.06) --
	(212.77,250.33) --
	(213.53,250.61) --
	(214.28,250.90) --
	(215.03,251.21) --
	(215.77,251.53) --
	(216.51,251.86) --
	(217.24,252.21) --
	(217.97,252.57) --
	(218.69,252.94) --
	(218.70,252.94);
\definecolor[named]{drawColor}{rgb}{0.00,0.00,0.00}

\path[draw=drawColor,line width= 0.4pt,dash pattern=on 1pt off 3pt ,line join=round,line cap=round] (166.90,252.94) --
	(167.00,252.87) --
	(167.72,252.35) --
	(168.45,251.84) --
	(169.19,251.34) --
	(169.94,250.86) --
	(170.69,250.39) --
	(171.46,249.93) --
	(172.23,249.49) --
	(173.01,249.06) --
	(173.79,248.65) --
	(174.59,248.25) --
	(175.39,247.86) --
	(176.20,247.49) --
	(177.01,247.13) --
	(177.83,246.78) --
	(178.65,246.45) --
	(179.49,246.14) --
	(180.32,245.83) --
	(181.16,245.55) --
	(182.01,245.28) --
	(182.86,245.02) --
	(183.72,244.78) --
	(184.58,244.55) --
	(185.44,244.34) --
	(186.31,244.14) --
	(187.18,243.96) --
	(188.05,243.79) --
	(188.93,243.64) --
	(189.81,243.51) --
	(190.69,243.39) --
	(191.57,243.28) --
	(192.46,243.19) --
	(193.34,243.12) --
	(194.23,243.06) --
	(195.12,243.02) --
	(196.01,242.99) --
	(196.90,242.98) --
	(197.79,242.98) --
	(198.68,243.00) --
	(199.57,243.04) --
	(200.45,243.09) --
	(201.34,243.15) --
	(202.23,243.24) --
	(203.11,243.33) --
	(203.99,243.44) --
	(204.87,243.57) --
	(205.75,243.72) --
	(206.63,243.87) --
	(207.50,244.05) --
	(208.37,244.24) --
	(209.23,244.44) --
	(210.09,244.66) --
	(210.95,244.90) --
	(211.80,245.15) --
	(212.65,245.41) --
	(213.50,245.69) --
	(214.34,245.98) --
	(215.17,246.29) --
	(216.00,246.62) --
	(216.82,246.95) --
	(217.64,247.31) --
	(218.45,247.67) --
	(219.25,248.05) --
	(220.05,248.45) --
	(220.84,248.85) --
	(221.62,249.28) --
	(222.40,249.71) --
	(223.17,250.16) --
	(223.93,250.62) --
	(224.68,251.10) --
	(225.42,251.59) --
	(226.16,252.09) --
	(226.88,252.60) --
	(227.34,252.94);

\path[draw=drawColor,line width= 0.4pt,dash pattern=on 1pt off 3pt ,line join=round,line cap=round] (252.94,146.38) --
	(252.77,145.97) --
	(252.44,145.15) --
	(252.12,144.32) --
	(251.81,143.48) --
	(251.52,142.64) --
	(251.25,141.79) --
	(250.98,140.94) --
	(250.74,140.09) --
	(250.51,139.23) --
	(250.29,138.37) --
	(250.09,137.50) --
	(249.91,136.63) --
	(249.74,135.76) --
	(249.58,134.88) --
	(249.44,134.00) --
	(249.32,133.12) --
	(249.21,132.24) --
	(249.12,131.36) --
	(249.04,130.47) --
	(248.98,129.58) --
	(248.93,128.70) --
	(248.90,127.81) --
	(248.88,126.92) --
	(248.88,126.03) --
	(248.90,125.14) --
	(248.93,124.25) --
	(248.98,123.36) --
	(249.04,122.47) --
	(249.12,121.59) --
	(249.21,120.70) --
	(249.32,119.82) --
	(249.44,118.94) --
	(249.58,118.06) --
	(249.74,117.19) --
	(249.91,116.31) --
	(250.09,115.44) --
	(250.29,114.58) --
	(250.51,113.71) --
	(250.74,112.86) --
	(250.98,112.00) --
	(251.25,111.15) --
	(251.52,110.31) --
	(251.81,109.46) --
	(252.12,108.63) --
	(252.44,107.80) --
	(252.77,106.98) --
	(252.94,106.56);
\definecolor[named]{drawColor}{rgb}{0.00,0.00,1.00}

\path[draw=drawColor,line width= 1.2pt,line join=round,line cap=round] (252.94,249.40) --
	(252.57,249.16) --
	(251.89,248.72) --
	(251.22,248.27) --
	(250.55,247.81) --
	(249.90,247.34) --
	(249.25,246.85) --
	(248.61,246.36) --
	(247.98,245.85) --
	(247.36,245.33) --
	(246.75,244.80) --
	(246.15,244.26) --
	(245.56,243.71) --
	(244.97,243.15) --
	(244.40,242.58) --
	(243.84,242.00) --
	(243.28,241.41) --
	(242.74,240.81) --
	(242.21,240.20) --
	(241.69,239.58) --
	(241.18,238.96) --
	(240.68,238.32) --
	(240.19,237.68) --
	(239.72,237.02) --
	(239.25,236.36) --
	(238.80,235.69) --
	(238.36,235.01) --
	(237.93,234.33) --
	(237.51,233.64) --
	(237.10,232.94) --
	(236.71,232.23) --
	(236.33,231.52) --
	(235.96,230.80) --
	(235.61,230.07) --
	(235.27,229.34) --
	(234.94,228.60) --
	(234.62,227.86) --
	(234.32,227.11) --
	(234.02,226.35) --
	(233.75,225.59) --
	(233.48,224.83) --
	(233.23,224.06) --
	(233.00,223.29) --
	(232.77,222.51) --
	(232.56,221.73) --
	(232.37,220.95) --
	(232.18,220.16) --
	(232.02,219.37) --
	(231.86,218.57) --
	(231.72,217.78) --
	(231.59,216.98) --
	(231.48,216.18) --
	(231.38,215.38) --
	(231.30,214.57) --
	(231.23,213.77) --
	(231.17,212.96) --
	(231.13,212.15) --
	(231.10,211.35) --
	(231.09,210.54) --
	(231.09,209.73) --
	(231.10,208.92) --
	(231.13,208.11) --
	(231.17,207.30) --
	(231.23,206.50) --
	(231.30,205.69) --
	(231.38,204.89) --
	(231.48,204.09) --
	(231.59,203.29) --
	(231.72,202.49) --
	(231.86,201.69) --
	(232.02,200.90) --
	(232.18,200.11) --
	(232.37,199.32) --
	(232.56,198.53) --
	(232.77,197.75) --
	(233.00,196.98) --
	(233.23,196.20) --
	(233.48,195.44) --
	(233.75,194.67) --
	(234.02,193.91) --
	(234.32,193.16) --
	(234.62,192.41) --
	(234.94,191.66) --
	(235.27,190.93) --
	(235.61,190.19) --
	(235.96,189.47) --
	(236.33,188.75) --
	(236.71,188.03) --
	(237.10,187.33) --
	(237.51,186.63) --
	(237.93,185.94) --
	(238.36,185.25) --
	(238.80,184.57) --
	(239.25,183.90) --
	(239.72,183.24) --
	(240.19,182.59) --
	(240.68,181.94) --
	(241.18,181.31) --
	(241.69,180.68) --
	(242.21,180.06) --
	(242.74,179.45) --
	(243.28,178.85) --
	(243.84,178.26) --
	(244.40,177.68) --
	(244.97,177.11) --
	(245.56,176.55) --
	(246.15,176.00) --
	(246.75,175.46) --
	(247.36,174.93) --
	(247.98,174.42) --
	(248.61,173.91) --
	(249.25,173.41) --
	(249.90,172.93) --
	(250.55,172.45) --
	(251.22,171.99) --
	(251.89,171.54) --
	(252.57,171.10) --
	(252.94,170.87);
\definecolor[named]{drawColor}{rgb}{0.00,0.00,0.00}

\path[draw=drawColor,line width= 0.4pt,dash pattern=on 1pt off 3pt ,line join=round,line cap=round] (249.91,252.94) --
	(249.35,252.58) --
	(248.61,252.09) --
	(247.88,251.58) --
	(247.16,251.06) --
	(246.45,250.53) --
	(245.75,249.98) --
	(245.05,249.42) --
	(244.37,248.85) --
	(243.70,248.27) --
	(243.04,247.68) --
	(242.38,247.07) --
	(241.74,246.45) --
	(241.11,245.83) --
	(240.49,245.19) --
	(239.88,244.54) --
	(239.29,243.88) --
	(238.70,243.21) --
	(238.13,242.53) --
	(237.57,241.84) --
	(237.02,241.14) --
	(236.48,240.43) --
	(235.96,239.71) --
	(235.45,238.98) --
	(234.95,238.25) --
	(234.46,237.50) --
	(233.99,236.75) --
	(233.53,235.99) --
	(233.09,235.22) --
	(232.66,234.44) --
	(232.24,233.66) --
	(231.83,232.86) --
	(231.44,232.07) --
	(231.06,231.26) --
	(230.70,230.45) --
	(230.35,229.63) --
	(230.02,228.81) --
	(229.70,227.98) --
	(229.39,227.14) --
	(229.10,226.30) --
	(228.83,225.45) --
	(228.57,224.60) --
	(228.32,223.75) --
	(228.09,222.89) --
	(227.88,222.03) --
	(227.68,221.16) --
	(227.49,220.29) --
	(227.32,219.42) --
	(227.16,218.54) --
	(227.03,217.66) --
	(226.90,216.78) --
	(226.79,215.90) --
	(226.70,215.02) --
	(226.62,214.13) --
	(226.56,213.24) --
	(226.51,212.36) --
	(226.48,211.47) --
	(226.47,210.58) --
	(226.47,209.69) --
	(226.48,208.80) --
	(226.51,207.91) --
	(226.56,207.02) --
	(226.62,206.13) --
	(226.70,205.25) --
	(226.79,204.36) --
	(226.90,203.48) --
	(227.03,202.60) --
	(227.16,201.72) --
	(227.32,200.85) --
	(227.49,199.97) --
	(227.68,199.10) --
	(227.88,198.24) --
	(228.09,197.37) --
	(228.32,196.52) --
	(228.57,195.66) --
	(228.83,194.81) --
	(229.10,193.97) --
	(229.39,193.12) --
	(229.70,192.29) --
	(230.02,191.46) --
	(230.35,190.64) --
	(230.70,189.82) --
	(231.06,189.01) --
	(231.44,188.20) --
	(231.83,187.40) --
	(232.24,186.61) --
	(232.66,185.82) --
	(233.09,185.05) --
	(233.53,184.28) --
	(233.99,183.52) --
	(234.46,182.76) --
	(234.95,182.02) --
	(235.45,181.28) --
	(235.96,180.55) --
	(236.48,179.83) --
	(237.02,179.13) --
	(237.57,178.43) --
	(238.13,177.74) --
	(238.70,177.06) --
	(239.29,176.39) --
	(239.88,175.73) --
	(240.49,175.08) --
	(241.11,174.44) --
	(241.74,173.81) --
	(242.38,173.19) --
	(243.04,172.59) --
	(243.70,172.00) --
	(244.37,171.41) --
	(245.05,170.84) --
	(245.75,170.29) --
	(246.45,169.74) --
	(247.16,169.21) --
	(247.88,168.69) --
	(248.61,168.18) --
	(249.35,167.68) --
	(250.10,167.20) --
	(250.85,166.73) --
	(251.61,166.27) --
	(252.39,165.83) --
	(252.94,165.52);
\definecolor[named]{drawColor}{rgb}{0.00,0.00,1.00}

\path[draw=drawColor,line width= 1.2pt,line join=round,line cap=round] (233.29,252.94) --
	(233.66,252.75) --
	(234.38,252.39) --
	(235.11,252.04) --
	(235.84,251.70) --
	(236.58,251.37) --
	(237.32,251.06) --
	(238.07,250.76) --
	(238.83,250.47) --
	(239.59,250.19) --
	(240.36,249.93) --
	(241.13,249.69) --
	(241.90,249.45) --
	(242.68,249.23) --
	(243.46,249.03) --
	(244.24,248.83) --
	(245.03,248.66) --
	(245.82,248.49) --
	(246.62,248.34) --
	(247.42,248.20) --
	(248.21,248.08) --
	(249.02,247.97) --
	(249.82,247.87) --
	(250.62,247.79) --
	(251.43,247.73) --
	(252.24,247.67) --
	(252.94,247.64);
\definecolor[named]{drawColor}{rgb}{0.00,0.00,0.00}

\path[draw=drawColor,line width= 0.4pt,dash pattern=on 1pt off 3pt ,line join=round,line cap=round] (224.64,252.94) --
	(224.74,252.87) --
	(225.46,252.35) --
	(226.19,251.84) --
	(226.93,251.34) --
	(227.68,250.86) --
	(228.43,250.39) --
	(229.20,249.93) --
	(229.97,249.49) --
	(230.75,249.06) --
	(231.54,248.65) --
	(232.33,248.25) --
	(233.13,247.86) --
	(233.94,247.49) --
	(234.75,247.13) --
	(235.57,246.78) --
	(236.40,246.45) --
	(237.23,246.14) --
	(238.06,245.83) --
	(238.91,245.55) --
	(239.75,245.28) --
	(240.60,245.02) --
	(241.46,244.78) --
	(242.32,244.55) --
	(243.18,244.34) --
	(244.05,244.14) --
	(244.92,243.96) --
	(245.79,243.79) --
	(246.67,243.64) --
	(247.55,243.51) --
	(248.43,243.39) --
	(249.31,243.28) --
	(250.20,243.19) --
	(251.09,243.12) --
	(251.97,243.06) --
	(252.86,243.02) --
	(252.94,243.01);

\path[draw=drawColor,line width= 1.2pt,line join=round,line cap=round] (172.67,126.47) --
	(172.66,127.28) --
	(172.64,128.09) --
	(172.60,128.90) --
	(172.55,129.70) --
	(172.49,130.51) --
	(172.41,131.31) --
	(172.32,132.12) --
	(172.22,132.92) --
	(172.10,133.72) --
	(171.96,134.52) --
	(171.81,135.31) --
	(171.65,136.10) --
	(171.48,136.89) --
	(171.29,137.68) --
	(171.08,138.46) --
	(170.87,139.24) --
	(170.64,140.02) --
	(170.39,140.79) --
	(170.14,141.55) --
	(169.87,142.31) --
	(169.58,143.07) --
	(169.29,143.82) --
	(168.97,144.57) --
	(168.65,145.31) --
	(168.32,146.05) --
	(167.97,146.78) --
	(167.60,147.50) --
	(167.23,148.22) --
	(166.84,148.93) --
	(166.44,149.63) --
	(166.03,150.32) --
	(165.61,151.01) --
	(165.17,151.69) --
	(164.73,152.37) --
	(164.27,153.03) --
	(163.80,153.69) --
	(163.32,154.34) --
	(162.82,154.98) --
	(162.32,155.61) --
	(161.80,156.23) --
	(161.28,156.85) --
	(160.74,157.45) --
	(160.19,158.05) --
	(159.63,158.63) --
	(159.07,159.21) --
	(158.49,159.77) --
	(157.90,160.33) --
	(157.30,160.87) --
	(156.70,161.41) --
	(156.08,161.93) --
	(155.45,162.44) --
	(154.82,162.95) --
	(154.18,163.44) --
	(153.53,163.92) --
	(152.87,164.38) --
	(152.20,164.84) --
	(151.52,165.28) --
	(150.84,165.72) --
	(150.15,166.14) --
	(149.45,166.55) --
	(148.75,166.94) --
	(148.04,167.33) --
	(147.32,167.70) --
	(146.59,168.06) --
	(145.86,168.40) --
	(145.13,168.73) --
	(144.38,169.05) --
	(143.64,169.36) --
	(142.88,169.65) --
	(142.12,169.94) --
	(141.36,170.20) --
	(140.59,170.46) --
	(139.82,170.70) --
	(139.05,170.92) --
	(138.27,171.14) --
	(137.48,171.34) --
	(136.70,171.52) --
	(135.91,171.69) --
	(135.11,171.85) --
	(134.32,172.00) --
	(133.52,172.13) --
	(132.72,172.24) --
	(131.92,172.35) --
	(131.11,172.43) --
	(130.31,172.51) --
	(129.50,172.57) --
	(128.70,172.61) --
	(127.89,172.65) --
	(127.08,172.66) --
	(126.27,172.67) --
	(125.46,172.66) --
	(124.65,172.63) --
	(123.85,172.59) --
	(123.04,172.54) --
	(122.23,172.47) --
	(121.43,172.39) --
	(120.63,172.30) --
	(119.83,172.19) --
	(119.03,172.06) --
	(118.23,171.93) --
	(117.44,171.77) --
	(116.64,171.61) --
	(115.86,171.43) --
	(115.07,171.24) --
	(114.29,171.03) --
	(113.51,170.81) --
	(112.74,170.58) --
	(111.97,170.33) --
	(111.20,170.07) --
	(110.44,169.80) --
	(109.69,169.51) --
	(108.93,169.21) --
	(108.19,168.90) --
	(107.45,168.57) --
	(106.72,168.23) --
	(105.99,167.88) --
	(105.27,167.51) --
	(104.55,167.14) --
	(103.84,166.75) --
	(103.14,166.34) --
	(102.45,165.93) --
	(101.76,165.50) --
	(101.08,165.06) --
	(100.41,164.61) --
	( 99.75,164.15) --
	( 99.09,163.68) --
	( 98.44,163.19) --
	( 97.81,162.70) --
	( 97.18,162.19) --
	( 96.56,161.67) --
	( 95.94,161.14) --
	( 95.34,160.60) --
	( 94.75,160.05) --
	( 94.17,159.49) --
	( 93.59,158.92) --
	( 93.03,158.34) --
	( 92.48,157.75) --
	( 91.94,157.15) --
	( 91.40,156.54) --
	( 90.88,155.92) --
	( 90.37,155.30) --
	( 89.87,154.66) --
	( 89.39,154.02) --
	( 88.91,153.36) --
	( 88.45,152.70) --
	( 87.99,152.03) --
	( 87.55,151.35) --
	( 87.12,150.67) --
	( 86.70,149.98) --
	( 86.30,149.28) --
	( 85.91,148.57) --
	( 85.53,147.86) --
	( 85.16,147.14) --
	( 84.80,146.41) --
	( 84.46,145.68) --
	( 84.13,144.94) --
	( 83.81,144.20) --
	( 83.51,143.45) --
	( 83.22,142.69) --
	( 82.94,141.93) --
	( 82.68,141.17) --
	( 82.43,140.40) --
	( 82.19,139.63) --
	( 81.97,138.85) --
	( 81.76,138.07) --
	( 81.56,137.29) --
	( 81.38,136.50) --
	( 81.21,135.71) --
	( 81.06,134.91) --
	( 80.91,134.12) --
	( 80.79,133.32) --
	( 80.67,132.52) --
	( 80.58,131.72) --
	( 80.49,130.91) --
	( 80.42,130.11) --
	( 80.36,129.30) --
	( 80.32,128.49) --
	( 80.29,127.69) --
	( 80.28,126.88) --
	( 80.28,126.07) --
	( 80.29,125.26) --
	( 80.32,124.45) --
	( 80.36,123.64) --
	( 80.42,122.84) --
	( 80.49,122.03) --
	( 80.58,121.23) --
	( 80.67,120.43) --
	( 80.79,119.63) --
	( 80.91,118.83) --
	( 81.06,118.03) --
	( 81.21,117.24) --
	( 81.38,116.45) --
	( 81.56,115.66) --
	( 81.76,114.87) --
	( 81.97,114.09) --
	( 82.19,113.32) --
	( 82.43,112.54) --
	( 82.68,111.78) --
	( 82.94,111.01) --
	( 83.22,110.25) --
	( 83.51,109.50) --
	( 83.81,108.75) --
	( 84.13,108.00) --
	( 84.46,107.27) --
	( 84.80,106.53) --
	( 85.16,105.81) --
	( 85.53,105.09) --
	( 85.91,104.37) --
	( 86.30,103.67) --
	( 86.70,102.97) --
	( 87.12,102.28) --
	( 87.55,101.59) --
	( 87.99,100.91) --
	( 88.45,100.24) --
	( 88.91, 99.58) --
	( 89.39, 98.93) --
	( 89.87, 98.28) --
	( 90.37, 97.65) --
	( 90.88, 97.02) --
	( 91.40, 96.40) --
	( 91.94, 95.79) --
	( 92.48, 95.19) --
	( 93.03, 94.60) --
	( 93.59, 94.02) --
	( 94.17, 93.45) --
	( 94.75, 92.89) --
	( 95.34, 92.34) --
	( 95.94, 91.80) --
	( 96.56, 91.27) --
	( 97.18, 90.76) --
	( 97.81, 90.25) --
	( 98.44, 89.75) --
	( 99.09, 89.27) --
	( 99.75, 88.79) --
	(100.41, 88.33) --
	(101.08, 87.88) --
	(101.76, 87.44) --
	(102.45, 87.02) --
	(103.14, 86.60) --
	(103.84, 86.20) --
	(104.55, 85.81) --
	(105.27, 85.43) --
	(105.99, 85.07) --
	(106.72, 84.72) --
	(107.45, 84.38) --
	(108.19, 84.05) --
	(108.93, 83.74) --
	(109.69, 83.44) --
	(110.44, 83.15) --
	(111.20, 82.87) --
	(111.97, 82.61) --
	(112.74, 82.37) --
	(113.51, 82.13) --
	(114.29, 81.91) --
	(115.07, 81.71) --
	(115.86, 81.51) --
	(116.64, 81.34) --
	(117.44, 81.17) --
	(118.23, 81.02) --
	(119.03, 80.88) --
	(119.83, 80.76) --
	(120.63, 80.65) --
	(121.43, 80.55) --
	(122.23, 80.47) --
	(123.04, 80.41) --
	(123.85, 80.35) --
	(124.65, 80.31) --
	(125.46, 80.29) --
	(126.27, 80.28) --
	(127.08, 80.28) --
	(127.89, 80.30) --
	(128.70, 80.33) --
	(129.50, 80.38) --
	(130.31, 80.44) --
	(131.11, 80.51) --
	(131.92, 80.60) --
	(132.72, 80.70) --
	(133.52, 80.82) --
	(134.32, 80.95) --
	(135.11, 81.09) --
	(135.91, 81.25) --
	(136.70, 81.42) --
	(137.48, 81.61) --
	(138.27, 81.81) --
	(139.05, 82.02) --
	(139.82, 82.25) --
	(140.59, 82.49) --
	(141.36, 82.74) --
	(142.12, 83.01) --
	(142.88, 83.29) --
	(143.64, 83.58) --
	(144.38, 83.89) --
	(145.13, 84.21) --
	(145.86, 84.54) --
	(146.59, 84.89) --
	(147.32, 85.25) --
	(148.04, 85.62) --
	(148.75, 86.00) --
	(149.45, 86.40) --
	(150.15, 86.81) --
	(150.84, 87.23) --
	(151.52, 87.66) --
	(152.20, 88.10) --
	(152.87, 88.56) --
	(153.53, 89.03) --
	(154.18, 89.51) --
	(154.82, 90.00) --
	(155.45, 90.50) --
	(156.08, 91.01) --
	(156.70, 91.54) --
	(157.30, 92.07) --
	(157.90, 92.62) --
	(158.49, 93.17) --
	(159.07, 93.74) --
	(159.63, 94.31) --
	(160.19, 94.90) --
	(160.74, 95.49) --
	(161.28, 96.10) --
	(161.80, 96.71) --
	(162.32, 97.33) --
	(162.82, 97.96) --
	(163.32, 98.61) --
	(163.80, 99.25) --
	(164.27, 99.91) --
	(164.73,100.58) --
	(165.17,101.25) --
	(165.61,101.93) --
	(166.03,102.62) --
	(166.44,103.32) --
	(166.84,104.02) --
	(167.23,104.73) --
	(167.60,105.45) --
	(167.97,106.17) --
	(168.32,106.90) --
	(168.65,107.63) --
	(168.97,108.38) --
	(169.29,109.12) --
	(169.58,109.87) --
	(169.87,110.63) --
	(170.14,111.39) --
	(170.39,112.16) --
	(170.64,112.93) --
	(170.87,113.70) --
	(171.08,114.48) --
	(171.29,115.27) --
	(171.48,116.05) --
	(171.65,116.84) --
	(171.81,117.63) --
	(171.96,118.43) --
	(172.10,119.23) --
	(172.22,120.03) --
	(172.32,120.83) --
	(172.41,121.63) --
	(172.49,122.44) --
	(172.55,123.24) --
	(172.60,124.05) --
	(172.64,124.86) --
	(172.66,125.66) --
	(172.67,126.47);
\definecolor[named]{drawColor}{rgb}{1.00,0.00,0.00}

\node[text=drawColor,rotate=-75.00,anchor=base,inner sep=0pt, outer sep=0pt, scale=  1.00] at (102.00,180.96) {$\mathbf{k}_{gv}$};

\node[text=drawColor,anchor=base,inner sep=0pt, outer sep=0pt, scale=  1.00] at (154.04,129.49) {$\mathbf{k}_{gx}$};
\end{scope}
\end{tikzpicture}
% Created by tikzDevice version 0.6.2-92-0ad2792 on 2012-10-24 14:24:45
% !TEX encoding = UTF-8 Unicode
\begin{tikzpicture}[x=1pt,y=1pt]
\definecolor[named]{fillColor}{rgb}{1.00,1.00,1.00}
\path[use as bounding box,fill=fillColor,fill opacity=0.00] (0,0) rectangle (252.94,252.94);
\begin{scope}
\path[clip] (  0.00,  0.00) rectangle (252.94,252.94);
\definecolor[named]{drawColor}{rgb}{0.00,0.00,0.00}

\path[draw=drawColor,line width= 0.4pt,line join=round,line cap=round] ( 24.00, 24.00) --
	(228.94, 24.00) --
	(228.94,228.94) --
	( 24.00,228.94) --
	( 24.00, 24.00);
\end{scope}
\begin{scope}
\path[clip] ( 24.00, 24.00) rectangle (228.94,228.94);
\definecolor[named]{drawColor}{rgb}{0.00,0.00,0.00}

\path[draw=drawColor,line width= 0.4pt,line join=round,line cap=round] (126.47,126.47) --
	(126.47,126.47);
\definecolor[named]{drawColor}{rgb}{1.00,0.00,0.00}
\definecolor[named]{fillColor}{rgb}{1.00,1.00,1.00}

\path[draw=drawColor,line width= 0.8pt,line join=round,line cap=round,fill=fillColor] (126.47,126.47) -- (184.21,126.47);

\path[draw=drawColor,line width= 0.8pt,line join=round,line cap=round] (168.57,117.44) --
	(184.21,126.47) --
	(168.57,135.51);

\path[draw=drawColor,line width= 0.8pt,line join=round,line cap=round,fill=fillColor] (126.47,126.47) -- ( 83.17,201.48);

\path[draw=drawColor,line width= 0.8pt,line join=round,line cap=round] ( 98.81,192.45) --
	( 83.17,201.48) --
	( 83.17,183.41);
\definecolor[named]{drawColor}{rgb}{0.00,0.00,1.00}

\path[draw=drawColor,line width= 1.2pt,line join=round,line cap=round] ( 21.86,  0.00) --
	( 21.68,  0.31) --
	( 21.26,  1.00) --
	( 20.82,  1.68) --
	( 20.38,  2.35) --
	( 19.92,  3.02) --
	( 19.45,  3.68) --
	( 18.96,  4.32) --
	( 18.47,  4.97) --
	( 17.97,  5.60) --
	( 17.45,  6.22) --
	( 16.92,  6.83) --
	( 16.39,  7.44) --
	( 15.84,  8.03) --
	( 15.28,  8.62) --
	( 14.71,  9.19) --
	( 14.14,  9.76) --
	( 13.55, 10.31) --
	( 12.95, 10.86) --
	( 12.34, 11.39) --
	( 11.73, 11.92) --
	( 11.10, 12.43) --
	( 10.47, 12.93) --
	(  9.83, 13.42) --
	(  9.18, 13.90) --
	(  8.52, 14.37) --
	(  7.85, 14.83) --
	(  7.17, 15.27) --
	(  6.49, 15.70) --
	(  5.80, 16.12) --
	(  5.10, 16.53) --
	(  4.40, 16.93) --
	(  3.68, 17.31) --
	(  2.97, 17.68) --
	(  2.24, 18.04) --
	(  1.51, 18.39) --
	(  0.77, 18.72) --
	(  0.03, 19.04) --
	(  0.00, 19.05);
\definecolor[named]{drawColor}{rgb}{0.00,0.00,0.00}

\path[draw=drawColor,line width= 0.4pt,dash pattern=on 1pt off 3pt ,line join=round,line cap=round] ( 27.15,  0.00) --
	( 26.95,  0.37) --
	( 26.53,  1.16) --
	( 26.09,  1.93) --
	( 25.64,  2.69) --
	( 25.17,  3.45) --
	( 24.69,  4.20) --
	( 24.20,  4.94) --
	( 23.70,  5.67) --
	( 23.18,  6.40) --
	( 22.65,  7.11) --
	( 22.10,  7.82) --
	( 21.55,  8.51) --
	( 20.98,  9.20) --
	( 20.40,  9.87) --
	( 19.81, 10.54) --
	( 19.21, 11.19) --
	( 18.60, 11.83) --
	( 17.97, 12.47) --
	( 17.34, 13.09) --
	( 16.69, 13.70) --
	( 16.03, 14.30) --
	( 15.37, 14.89) --
	( 14.69, 15.46) --
	( 14.00, 16.03) --
	( 13.30, 16.58) --
	( 12.60, 17.12) --
	( 11.88, 17.65) --
	( 11.16, 18.16) --
	( 10.42, 18.66) --
	(  9.68, 19.15) --
	(  8.93, 19.63) --
	(  8.17, 20.09) --
	(  7.40, 20.54) --
	(  6.62, 20.97) --
	(  5.84, 21.40) --
	(  5.05, 21.80) --
	(  4.25, 22.20) --
	(  3.45, 22.58) --
	(  2.64, 22.94) --
	(  1.82, 23.30) --
	(  1.00, 23.63) --
	(  0.17, 23.96) --
	(  0.00, 24.02);
\definecolor[named]{drawColor}{rgb}{0.00,0.00,1.00}

\path[draw=drawColor,line width= 1.2pt,line join=round,line cap=round] ( 79.61,  0.00) --
	( 79.42,  0.31) --
	( 79.00,  1.00) --
	( 78.56,  1.68) --
	( 78.12,  2.35) --
	( 77.66,  3.02) --
	( 77.19,  3.68) --
	( 76.70,  4.32) --
	( 76.21,  4.97) --
	( 75.71,  5.60) --
	( 75.19,  6.22) --
	( 74.66,  6.83) --
	( 74.13,  7.44) --
	( 73.58,  8.03) --
	( 73.02,  8.62) --
	( 72.45,  9.19) --
	( 71.88,  9.76) --
	( 71.29, 10.31) --
	( 70.69, 10.86) --
	( 70.08, 11.39) --
	( 69.47, 11.92) --
	( 68.84, 12.43) --
	( 68.21, 12.93) --
	( 67.57, 13.42) --
	( 66.92, 13.90) --
	( 66.26, 14.37) --
	( 65.59, 14.83) --
	( 64.91, 15.27) --
	( 64.23, 15.70) --
	( 63.54, 16.12) --
	( 62.84, 16.53) --
	( 62.14, 16.93) --
	( 61.43, 17.31) --
	( 60.71, 17.68) --
	( 59.98, 18.04) --
	( 59.25, 18.39) --
	( 58.51, 18.72) --
	( 57.77, 19.04) --
	( 57.02, 19.35) --
	( 56.27, 19.64) --
	( 55.51, 19.92) --
	( 54.75, 20.19) --
	( 53.98, 20.44) --
	( 53.21, 20.68) --
	( 52.43, 20.91) --
	( 51.65, 21.12) --
	( 50.87, 21.32) --
	( 50.08, 21.51) --
	( 49.29, 21.68) --
	( 48.50, 21.84) --
	( 47.71, 21.98) --
	( 46.91, 22.11) --
	( 46.11, 22.23) --
	( 45.31, 22.33) --
	( 44.50, 22.42) --
	( 43.70, 22.49) --
	( 42.89, 22.55) --
	( 42.08, 22.60) --
	( 41.28, 22.63) --
	( 40.47, 22.65) --
	( 39.66, 22.65) --
	( 38.85, 22.64) --
	( 38.04, 22.62) --
	( 37.24, 22.58) --
	( 36.43, 22.52) --
	( 35.62, 22.46) --
	( 34.82, 22.38) --
	( 34.02, 22.28) --
	( 33.21, 22.17) --
	( 32.42, 22.05) --
	( 31.62, 21.91) --
	( 30.82, 21.76) --
	( 30.03, 21.59) --
	( 29.24, 21.42) --
	( 28.46, 21.22) --
	( 27.68, 21.02) --
	( 26.90, 20.80) --
	( 26.13, 20.56) --
	( 25.36, 20.32) --
	( 24.59, 20.06) --
	( 23.83, 19.78) --
	( 23.07, 19.49) --
	( 22.32, 19.19) --
	( 21.58, 18.88) --
	( 20.84, 18.55) --
	( 20.10, 18.21) --
	( 19.38, 17.86) --
	( 18.66, 17.50) --
	( 17.94, 17.12) --
	( 17.23, 16.73) --
	( 16.53, 16.33) --
	( 15.84, 15.91) --
	( 15.15, 15.49) --
	( 14.47, 15.05) --
	( 13.80, 14.60) --
	( 13.14, 14.14) --
	( 12.48, 13.66) --
	( 11.83, 13.18) --
	( 11.19, 12.68) --
	( 10.57, 12.17) --
	(  9.94, 11.66) --
	(  9.33, 11.13) --
	(  8.73, 10.59) --
	(  8.14, 10.04) --
	(  7.56,  9.48) --
	(  6.98,  8.91) --
	(  6.42,  8.33) --
	(  5.87,  7.74) --
	(  5.33,  7.14) --
	(  4.79,  6.53) --
	(  4.27,  5.91) --
	(  3.76,  5.28) --
	(  3.26,  4.65) --
	(  2.78,  4.00) --
	(  2.30,  3.35) --
	(  1.83,  2.69) --
	(  1.38,  2.02) --
	(  0.94,  1.34) --
	(  0.51,  0.65) --
	(  0.12,  0.00);
\definecolor[named]{drawColor}{rgb}{0.00,0.00,0.00}

\path[draw=drawColor,line width= 0.4pt,dash pattern=on 1pt off 3pt ,line join=round,line cap=round] ( 84.89,  0.00) --
	( 84.70,  0.37) --
	( 84.27,  1.16) --
	( 83.83,  1.93) --
	( 83.38,  2.69) --
	( 82.91,  3.45) --
	( 82.43,  4.20) --
	( 81.94,  4.94) --
	( 81.44,  5.67) --
	( 80.92,  6.40) --
	( 80.39,  7.11) --
	( 79.85,  7.82) --
	( 79.29,  8.51) --
	( 78.72,  9.20) --
	( 78.14,  9.87) --
	( 77.55, 10.54) --
	( 76.95, 11.19) --
	( 76.34, 11.83) --
	( 75.71, 12.47) --
	( 75.08, 13.09) --
	( 74.43, 13.70) --
	( 73.77, 14.30) --
	( 73.11, 14.89) --
	( 72.43, 15.46) --
	( 71.74, 16.03) --
	( 71.04, 16.58) --
	( 70.34, 17.12) --
	( 69.62, 17.65) --
	( 68.90, 18.16) --
	( 68.16, 18.66) --
	( 67.42, 19.15) --
	( 66.67, 19.63) --
	( 65.91, 20.09) --
	( 65.14, 20.54) --
	( 64.36, 20.97) --
	( 63.58, 21.40) --
	( 62.79, 21.80) --
	( 61.99, 22.20) --
	( 61.19, 22.58) --
	( 60.38, 22.94) --
	( 59.56, 23.30) --
	( 58.74, 23.63) --
	( 57.91, 23.96) --
	( 57.08, 24.27) --
	( 56.24, 24.56) --
	( 55.39, 24.84) --
	( 54.55, 25.10) --
	( 53.69, 25.35) --
	( 52.83, 25.59) --
	( 51.97, 25.81) --
	( 51.11, 26.01) --
	( 50.24, 26.20) --
	( 49.37, 26.38) --
	( 48.49, 26.53) --
	( 47.61, 26.68) --
	( 46.73, 26.81) --
	( 45.85, 26.92) --
	( 44.97, 27.01) --
	( 44.08, 27.10) --
	( 43.19, 27.16) --
	( 42.31, 27.21) --
	( 41.42, 27.25) --
	( 40.53, 27.27) --
	( 39.64, 27.27) --
	( 38.75, 27.26) --
	( 37.86, 27.23) --
	( 36.97, 27.19) --
	( 36.09, 27.13) --
	( 35.20, 27.06) --
	( 34.31, 26.97) --
	( 33.43, 26.86) --
	( 32.55, 26.74) --
	( 31.67, 26.61) --
	( 30.79, 26.46) --
	( 29.92, 26.29) --
	( 29.05, 26.11) --
	( 28.18, 25.91) --
	( 27.32, 25.70) --
	( 26.46, 25.47) --
	( 25.60, 25.23) --
	( 24.75, 24.97) --
	( 23.91, 24.70) --
	( 23.06, 24.42) --
	( 22.23, 24.11) --
	( 21.40, 23.80) --
	( 20.57, 23.47) --
	( 19.75, 23.12) --
	( 18.94, 22.76) --
	( 18.13, 22.39) --
	( 17.33, 22.00) --
	( 16.53, 21.60) --
	( 15.75, 21.19) --
	( 14.97, 20.76) --
	( 14.20, 20.32) --
	( 13.43, 19.86) --
	( 12.68, 19.39) --
	( 11.93, 18.91) --
	( 11.19, 18.41) --
	( 10.46, 17.90) --
	(  9.74, 17.38) --
	(  9.03, 16.85) --
	(  8.33, 16.30) --
	(  7.64, 15.75) --
	(  6.95, 15.18) --
	(  6.28, 14.59) --
	(  5.62, 14.00) --
	(  4.97, 13.40) --
	(  4.33, 12.78) --
	(  3.70, 12.15) --
	(  3.08, 11.51) --
	(  2.47, 10.86) --
	(  1.87, 10.20) --
	(  1.29,  9.53) --
	(  0.71,  8.85) --
	(  0.15,  8.16) --
	(  0.00,  7.97);
\definecolor[named]{drawColor}{rgb}{0.00,0.00,1.00}

\path[draw=drawColor,line width= 1.2pt,line join=round,line cap=round] ( 42.75, 51.47) --
	( 42.74, 52.27) --
	( 42.72, 53.08) --
	( 42.69, 53.89) --
	( 42.64, 54.70) --
	( 42.57, 55.50) --
	( 42.50, 56.31) --
	( 42.40, 57.11) --
	( 42.30, 57.91) --
	( 42.18, 58.71) --
	( 42.04, 59.51) --
	( 41.90, 60.30) --
	( 41.74, 61.10) --
	( 41.56, 61.89) --
	( 41.37, 62.67) --
	( 41.17, 63.45) --
	( 40.95, 64.23) --
	( 40.72, 65.01) --
	( 40.48, 65.78) --
	( 40.22, 66.54) --
	( 39.95, 67.31) --
	( 39.67, 68.06) --
	( 39.37, 68.82) --
	( 39.06, 69.56) --
	( 38.73, 70.30) --
	( 38.40, 71.04) --
	( 38.05, 71.77) --
	( 37.69, 72.49) --
	( 37.31, 73.21) --
	( 36.93, 73.92) --
	( 36.53, 74.62) --
	( 36.12, 75.32) --
	( 35.69, 76.01) --
	( 35.26, 76.69) --
	( 34.81, 77.36) --
	( 34.35, 78.03) --
	( 33.88, 78.68) --
	( 33.40, 79.33) --
	( 32.91, 79.97) --
	( 32.40, 80.60) --
	( 31.89, 81.23) --
	( 31.36, 81.84) --
	( 30.82, 82.45) --
	( 30.27, 83.04) --
	( 29.72, 83.63) --
	( 29.15, 84.20) --
	( 28.57, 84.77) --
	( 27.98, 85.32) --
	( 27.39, 85.87) --
	( 26.78, 86.40) --
	( 26.16, 86.92) --
	( 25.54, 87.44) --
	( 24.90, 87.94) --
	( 24.26, 88.43) --
	( 23.61, 88.91) --
	( 22.95, 89.38) --
	( 22.28, 89.83) --
	( 21.61, 90.28) --
	( 20.92, 90.71) --
	( 20.23, 91.13) --
	( 19.54, 91.54) --
	( 18.83, 91.93) --
	( 18.12, 92.32) --
	( 17.40, 92.69) --
	( 16.68, 93.05) --
	( 15.95, 93.39) --
	( 15.21, 93.73) --
	( 14.47, 94.05) --
	( 13.72, 94.35) --
	( 12.97, 94.65) --
	( 12.21, 94.93) --
	( 11.44, 95.19) --
	( 10.68, 95.45) --
	(  9.90, 95.69) --
	(  9.13, 95.92) --
	(  8.35, 96.13) --
	(  7.57, 96.33) --
	(  6.78, 96.51) --
	(  5.99, 96.69) --
	(  5.20, 96.84) --
	(  4.40, 96.99) --
	(  3.60, 97.12) --
	(  2.80, 97.24) --
	(  2.00, 97.34) --
	(  1.20, 97.43) --
	(  0.39, 97.50) --
	(  0.00, 97.53);

\path[draw=drawColor,line width= 1.2pt,line join=round,line cap=round] (  0.00,  5.40) --
	(  0.39,  5.43) --
	(  1.20,  5.50) --
	(  2.00,  5.59) --
	(  2.80,  5.69) --
	(  3.60,  5.81) --
	(  4.40,  5.94) --
	(  5.20,  6.09) --
	(  5.99,  6.24) --
	(  6.78,  6.42) --
	(  7.57,  6.60) --
	(  8.35,  6.80) --
	(  9.13,  7.01) --
	(  9.90,  7.24) --
	( 10.68,  7.48) --
	( 11.44,  7.74) --
	( 12.21,  8.00) --
	( 12.97,  8.28) --
	( 13.72,  8.58) --
	( 14.47,  8.88) --
	( 15.21,  9.20) --
	( 15.95,  9.54) --
	( 16.68,  9.88) --
	( 17.40, 10.24) --
	( 18.12, 10.61) --
	( 18.83, 11.00) --
	( 19.54, 11.39) --
	( 20.23, 11.80) --
	( 20.92, 12.22) --
	( 21.61, 12.65) --
	( 22.28, 13.10) --
	( 22.95, 13.55) --
	( 23.61, 14.02) --
	( 24.26, 14.50) --
	( 24.90, 14.99) --
	( 25.54, 15.49) --
	( 26.16, 16.01) --
	( 26.78, 16.53) --
	( 27.39, 17.06) --
	( 27.98, 17.61) --
	( 28.57, 18.16) --
	( 29.15, 18.73) --
	( 29.72, 19.30) --
	( 30.27, 19.89) --
	( 30.82, 20.48) --
	( 31.36, 21.09) --
	( 31.89, 21.70) --
	( 32.40, 22.33) --
	( 32.91, 22.96) --
	( 33.40, 23.60) --
	( 33.88, 24.25) --
	( 34.35, 24.90) --
	( 34.81, 25.57) --
	( 35.26, 26.24) --
	( 35.69, 26.92) --
	( 36.12, 27.61) --
	( 36.53, 28.31) --
	( 36.93, 29.01) --
	( 37.31, 29.72) --
	( 37.69, 30.44) --
	( 38.05, 31.16) --
	( 38.40, 31.89) --
	( 38.73, 32.63) --
	( 39.06, 33.37) --
	( 39.37, 34.11) --
	( 39.67, 34.87) --
	( 39.95, 35.62) --
	( 40.22, 36.39) --
	( 40.48, 37.15) --
	( 40.72, 37.92) --
	( 40.95, 38.70) --
	( 41.17, 39.48) --
	( 41.37, 40.26) --
	( 41.56, 41.04) --
	( 41.74, 41.83) --
	( 41.90, 42.63) --
	( 42.04, 43.42) --
	( 42.18, 44.22) --
	( 42.30, 45.02) --
	( 42.40, 45.82) --
	( 42.50, 46.62) --
	( 42.57, 47.43) --
	( 42.64, 48.23) --
	( 42.69, 49.04) --
	( 42.72, 49.85) --
	( 42.74, 50.66) --
	( 42.75, 51.47);
\definecolor[named]{drawColor}{rgb}{0.00,0.00,0.00}

\path[draw=drawColor,line width= 0.4pt,dash pattern=on 1pt off 3pt ,line join=round,line cap=round] ( 47.37, 51.47) --
	( 47.36, 52.35) --
	( 47.34, 53.24) --
	( 47.30, 54.13) --
	( 47.25, 55.02) --
	( 47.18, 55.91) --
	( 47.09, 56.79) --
	( 46.99, 57.67) --
	( 46.87, 58.56) --
	( 46.74, 59.44) --
	( 46.59, 60.31) --
	( 46.43, 61.19) --
	( 46.25, 62.06) --
	( 46.06, 62.93) --
	( 45.85, 63.79) --
	( 45.63, 64.65) --
	( 45.39, 65.51) --
	( 45.14, 66.36) --
	( 44.87, 67.21) --
	( 44.59, 68.05) --
	( 44.29, 68.89) --
	( 43.98, 69.72) --
	( 43.65, 70.55) --
	( 43.31, 71.37) --
	( 42.95, 72.19) --
	( 42.58, 73.00) --
	( 42.20, 73.80) --
	( 41.80, 74.59) --
	( 41.39, 75.38) --
	( 40.96, 76.16) --
	( 40.53, 76.94) --
	( 40.07, 77.70) --
	( 39.61, 78.46) --
	( 39.13, 79.21) --
	( 38.64, 79.95) --
	( 38.13, 80.68) --
	( 37.61, 81.40) --
	( 37.08, 82.12) --
	( 36.54, 82.82) --
	( 35.99, 83.52) --
	( 35.42, 84.20) --
	( 34.84, 84.88) --
	( 34.25, 85.54) --
	( 33.65, 86.20) --
	( 33.03, 86.84) --
	( 32.41, 87.47) --
	( 31.77, 88.10) --
	( 31.13, 88.71) --
	( 30.47, 89.31) --
	( 29.80, 89.89) --
	( 29.12, 90.47) --
	( 28.44, 91.03) --
	( 27.74, 91.59) --
	( 27.03, 92.13) --
	( 26.32, 92.65) --
	( 25.59, 93.17) --
	( 24.86, 93.67) --
	( 24.11, 94.16) --
	( 23.36, 94.63) --
	( 22.60, 95.10) --
	( 21.83, 95.55) --
	( 21.06, 95.98) --
	( 20.28, 96.40) --
	( 19.49, 96.81) --
	( 18.69, 97.21) --
	( 17.89, 97.59) --
	( 17.07, 97.95) --
	( 16.26, 98.30) --
	( 15.44, 98.64) --
	( 14.61, 98.97) --
	( 13.77, 99.27) --
	( 12.93, 99.57) --
	( 12.09, 99.85) --
	( 11.24,100.11) --
	( 10.39,100.36) --
	(  9.53,100.60) --
	(  8.67,100.82) --
	(  7.80,101.02) --
	(  6.93,101.21) --
	(  6.06,101.38) --
	(  5.18,101.54) --
	(  4.31,101.68) --
	(  3.43,101.81) --
	(  2.54,101.93) --
	(  1.66,102.02) --
	(  0.78,102.10) --
	(  0.00,102.16);

\path[draw=drawColor,line width= 0.4pt,dash pattern=on 1pt off 3pt ,line join=round,line cap=round] (  0.00,  0.77) --
	(  0.78,  0.83) --
	(  1.66,  0.91) --
	(  2.54,  1.00) --
	(  3.43,  1.12) --
	(  4.31,  1.25) --
	(  5.18,  1.39) --
	(  6.06,  1.55) --
	(  6.93,  1.72) --
	(  7.80,  1.91) --
	(  8.67,  2.11) --
	(  9.53,  2.33) --
	( 10.39,  2.57) --
	( 11.24,  2.82) --
	( 12.09,  3.08) --
	( 12.93,  3.36) --
	( 13.77,  3.66) --
	( 14.61,  3.96) --
	( 15.44,  4.29) --
	( 16.26,  4.63) --
	( 17.07,  4.98) --
	( 17.89,  5.34) --
	( 18.69,  5.72) --
	( 19.49,  6.12) --
	( 20.28,  6.53) --
	( 21.06,  6.95) --
	( 21.83,  7.38) --
	( 22.60,  7.83) --
	( 23.36,  8.30) --
	( 24.11,  8.77) --
	( 24.86,  9.26) --
	( 25.59,  9.76) --
	( 26.32, 10.28) --
	( 27.03, 10.80) --
	( 27.74, 11.34) --
	( 28.44, 11.90) --
	( 29.12, 12.46) --
	( 29.80, 13.04) --
	( 30.47, 13.62) --
	( 31.13, 14.22) --
	( 31.77, 14.83) --
	( 32.41, 15.46) --
	( 33.03, 16.09) --
	( 33.65, 16.73) --
	( 34.25, 17.39) --
	( 34.84, 18.05) --
	( 35.42, 18.73) --
	( 35.99, 19.41) --
	( 36.54, 20.11) --
	( 37.08, 20.81) --
	( 37.61, 21.53) --
	( 38.13, 22.25) --
	( 38.64, 22.98) --
	( 39.13, 23.72) --
	( 39.61, 24.47) --
	( 40.07, 25.23) --
	( 40.53, 25.99) --
	( 40.96, 26.77) --
	( 41.39, 27.55) --
	( 41.80, 28.34) --
	( 42.20, 29.13) --
	( 42.58, 29.93) --
	( 42.95, 30.74) --
	( 43.31, 31.56) --
	( 43.65, 32.38) --
	( 43.98, 33.21) --
	( 44.29, 34.04) --
	( 44.59, 34.88) --
	( 44.87, 35.72) --
	( 45.14, 36.57) --
	( 45.39, 37.42) --
	( 45.63, 38.28) --
	( 45.85, 39.14) --
	( 46.06, 40.00) --
	( 46.25, 40.87) --
	( 46.43, 41.74) --
	( 46.59, 42.62) --
	( 46.74, 43.49) --
	( 46.87, 44.37) --
	( 46.99, 45.26) --
	( 47.09, 46.14) --
	( 47.18, 47.02) --
	( 47.25, 47.91) --
	( 47.30, 48.80) --
	( 47.34, 49.69) --
	( 47.36, 50.58) --
	( 47.37, 51.47);

\path[draw=drawColor,line width= 0.4pt,dash pattern=on 1pt off 3pt ,line join=round,line cap=round] (  4.06,126.47) --
	(  4.06,127.36) --
	(  4.03,128.25) --
	(  3.99,129.14) --
	(  3.94,130.03) --
	(  3.87,130.91) --
	(  3.78,131.80) --
	(  3.68,132.68) --
	(  3.57,133.56) --
	(  3.44,134.44) --
	(  3.29,135.32) --
	(  3.13,136.20) --
	(  2.95,137.07) --
	(  2.75,137.93) --
	(  2.55,138.80) --
	(  2.32,139.66) --
	(  2.09,140.52) --
	(  1.83,141.37) --
	(  1.56,142.22) --
	(  1.28,143.06) --
	(  0.98,143.90) --
	(  0.67,144.73) --
	(  0.34,145.56) --
	(  0.00,146.38) --
	(  0.00,146.39);

\path[draw=drawColor,line width= 0.4pt,dash pattern=on 1pt off 3pt ,line join=round,line cap=round] (  0.00,106.56) --
	(  0.00,106.57) --
	(  0.34,107.39) --
	(  0.67,108.21) --
	(  0.98,109.05) --
	(  1.28,109.88) --
	(  1.56,110.73) --
	(  1.83,111.58) --
	(  2.09,112.43) --
	(  2.32,113.28) --
	(  2.55,114.15) --
	(  2.75,115.01) --
	(  2.95,115.88) --
	(  3.13,116.75) --
	(  3.29,117.62) --
	(  3.44,118.50) --
	(  3.57,119.38) --
	(  3.68,120.26) --
	(  3.78,121.15) --
	(  3.87,122.03) --
	(  3.94,122.92) --
	(  3.99,123.81) --
	(  4.03,124.69) --
	(  4.06,125.58) --
	(  4.06,126.47);
\definecolor[named]{drawColor}{rgb}{0.00,0.00,1.00}

\path[draw=drawColor,line width= 1.2pt,line join=round,line cap=round] (137.35,  0.00) --
	(137.16,  0.31) --
	(136.74,  1.00) --
	(136.30,  1.68) --
	(135.86,  2.35) --
	(135.40,  3.02) --
	(134.93,  3.68) --
	(134.44,  4.32) --
	(133.95,  4.97) --
	(133.45,  5.60) --
	(132.93,  6.22) --
	(132.41,  6.83) --
	(131.87,  7.44) --
	(131.32,  8.03) --
	(130.76,  8.62) --
	(130.20,  9.19) --
	(129.62,  9.76) --
	(129.03, 10.31) --
	(128.43, 10.86) --
	(127.83, 11.39) --
	(127.21, 11.92) --
	(126.58, 12.43) --
	(125.95, 12.93) --
	(125.31, 13.42) --
	(124.66, 13.90) --
	(124.00, 14.37) --
	(123.33, 14.83) --
	(122.65, 15.27) --
	(121.97, 15.70) --
	(121.28, 16.12) --
	(120.58, 16.53) --
	(119.88, 16.93) --
	(119.17, 17.31) --
	(118.45, 17.68) --
	(117.72, 18.04) --
	(116.99, 18.39) --
	(116.26, 18.72) --
	(115.51, 19.04) --
	(114.77, 19.35) --
	(114.01, 19.64) --
	(113.25, 19.92) --
	(112.49, 20.19) --
	(111.72, 20.44) --
	(110.95, 20.68) --
	(110.18, 20.91) --
	(109.40, 21.12) --
	(108.61, 21.32) --
	(107.83, 21.51) --
	(107.04, 21.68) --
	(106.24, 21.84) --
	(105.45, 21.98) --
	(104.65, 22.11) --
	(103.85, 22.23) --
	(103.05, 22.33) --
	(102.24, 22.42) --
	(101.44, 22.49) --
	(100.63, 22.55) --
	( 99.82, 22.60) --
	( 99.02, 22.63) --
	( 98.21, 22.65) --
	( 97.40, 22.65) --
	( 96.59, 22.64) --
	( 95.78, 22.62) --
	( 94.98, 22.58) --
	( 94.17, 22.52) --
	( 93.36, 22.46) --
	( 92.56, 22.38) --
	( 91.76, 22.28) --
	( 90.96, 22.17) --
	( 90.16, 22.05) --
	( 89.36, 21.91) --
	( 88.57, 21.76) --
	( 87.77, 21.59) --
	( 86.99, 21.42) --
	( 86.20, 21.22) --
	( 85.42, 21.02) --
	( 84.64, 20.80) --
	( 83.87, 20.56) --
	( 83.10, 20.32) --
	( 82.33, 20.06) --
	( 81.57, 19.78) --
	( 80.81, 19.49) --
	( 80.06, 19.19) --
	( 79.32, 18.88) --
	( 78.58, 18.55) --
	( 77.85, 18.21) --
	( 77.12, 17.86) --
	( 76.40, 17.50) --
	( 75.68, 17.12) --
	( 74.97, 16.73) --
	( 74.27, 16.33) --
	( 73.58, 15.91) --
	( 72.89, 15.49) --
	( 72.21, 15.05) --
	( 71.54, 14.60) --
	( 70.88, 14.14) --
	( 70.22, 13.66) --
	( 69.57, 13.18) --
	( 68.94, 12.68) --
	( 68.31, 12.17) --
	( 67.69, 11.66) --
	( 67.07, 11.13) --
	( 66.47, 10.59) --
	( 65.88, 10.04) --
	( 65.30,  9.48) --
	( 64.72,  8.91) --
	( 64.16,  8.33) --
	( 63.61,  7.74) --
	( 63.07,  7.14) --
	( 62.53,  6.53) --
	( 62.01,  5.91) --
	( 61.50,  5.28) --
	( 61.00,  4.65) --
	( 60.52,  4.00) --
	( 60.04,  3.35) --
	( 59.58,  2.69) --
	( 59.12,  2.02) --
	( 58.68,  1.34) --
	( 58.25,  0.65) --
	( 57.86,  0.00);
\definecolor[named]{drawColor}{rgb}{0.00,0.00,0.00}

\path[draw=drawColor,line width= 0.4pt,dash pattern=on 1pt off 3pt ,line join=round,line cap=round] (142.63,  0.00) --
	(142.44,  0.37) --
	(142.01,  1.16) --
	(141.57,  1.93) --
	(141.12,  2.69) --
	(140.65,  3.45) --
	(140.17,  4.20) --
	(139.68,  4.94) --
	(139.18,  5.67) --
	(138.66,  6.40) --
	(138.13,  7.11) --
	(137.59,  7.82) --
	(137.03,  8.51) --
	(136.46,  9.20) --
	(135.89,  9.87) --
	(135.30, 10.54) --
	(134.69, 11.19) --
	(134.08, 11.83) --
	(133.45, 12.47) --
	(132.82, 13.09) --
	(132.17, 13.70) --
	(131.52, 14.30) --
	(130.85, 14.89) --
	(130.17, 15.46) --
	(129.48, 16.03) --
	(128.79, 16.58) --
	(128.08, 17.12) --
	(127.36, 17.65) --
	(126.64, 18.16) --
	(125.90, 18.66) --
	(125.16, 19.15) --
	(124.41, 19.63) --
	(123.65, 20.09) --
	(122.88, 20.54) --
	(122.11, 20.97) --
	(121.32, 21.40) --
	(120.53, 21.80) --
	(119.74, 22.20) --
	(118.93, 22.58) --
	(118.12, 22.94) --
	(117.30, 23.30) --
	(116.48, 23.63) --
	(115.65, 23.96) --
	(114.82, 24.27) --
	(113.98, 24.56) --
	(113.14, 24.84) --
	(112.29, 25.10) --
	(111.43, 25.35) --
	(110.57, 25.59) --
	(109.71, 25.81) --
	(108.85, 26.01) --
	(107.98, 26.20) --
	(107.11, 26.38) --
	(106.23, 26.53) --
	(105.35, 26.68) --
	(104.47, 26.81) --
	(103.59, 26.92) --
	(102.71, 27.01) --
	(101.82, 27.10) --
	(100.93, 27.16) --
	(100.05, 27.21) --
	( 99.16, 27.25) --
	( 98.27, 27.27) --
	( 97.38, 27.27) --
	( 96.49, 27.26) --
	( 95.60, 27.23) --
	( 94.71, 27.19) --
	( 93.83, 27.13) --
	( 92.94, 27.06) --
	( 92.05, 26.97) --
	( 91.17, 26.86) --
	( 90.29, 26.74) --
	( 89.41, 26.61) --
	( 88.54, 26.46) --
	( 87.66, 26.29) --
	( 86.79, 26.11) --
	( 85.92, 25.91) --
	( 85.06, 25.70) --
	( 84.20, 25.47) --
	( 83.34, 25.23) --
	( 82.49, 24.97) --
	( 81.65, 24.70) --
	( 80.80, 24.42) --
	( 79.97, 24.11) --
	( 79.14, 23.80) --
	( 78.31, 23.47) --
	( 77.49, 23.12) --
	( 76.68, 22.76) --
	( 75.87, 22.39) --
	( 75.07, 22.00) --
	( 74.28, 21.60) --
	( 73.49, 21.19) --
	( 72.71, 20.76) --
	( 71.94, 20.32) --
	( 71.17, 19.86) --
	( 70.42, 19.39) --
	( 69.67, 18.91) --
	( 68.93, 18.41) --
	( 68.20, 17.90) --
	( 67.48, 17.38) --
	( 66.77, 16.85) --
	( 66.07, 16.30) --
	( 65.38, 15.75) --
	( 64.69, 15.18) --
	( 64.02, 14.59) --
	( 63.36, 14.00) --
	( 62.71, 13.40) --
	( 62.07, 12.78) --
	( 61.44, 12.15) --
	( 60.82, 11.51) --
	( 60.21, 10.86) --
	( 59.61, 10.20) --
	( 59.03,  9.53) --
	( 58.45,  8.85) --
	( 57.89,  8.16) --
	( 57.34,  7.46) --
	( 56.81,  6.76) --
	( 56.28,  6.04) --
	( 55.77,  5.31) --
	( 55.27,  4.57) --
	( 54.79,  3.83) --
	( 54.32,  3.07) --
	( 53.86,  2.31) --
	( 53.41,  1.54) --
	( 52.98,  0.77) --
	( 52.57,  0.00);
\definecolor[named]{fillColor}{rgb}{0.00,0.00,0.00}

\path[fill=fillColor] ( 54.30, 51.47) circle (  2.25);
\definecolor[named]{drawColor}{rgb}{0.00,0.00,1.00}

\path[draw=drawColor,line width= 1.2pt,line join=round,line cap=round] (100.49, 51.47) --
	(100.48, 52.27) --
	(100.46, 53.08) --
	(100.43, 53.89) --
	(100.38, 54.70) --
	(100.31, 55.50) --
	(100.24, 56.31) --
	(100.15, 57.11) --
	(100.04, 57.91) --
	( 99.92, 58.71) --
	( 99.79, 59.51) --
	( 99.64, 60.30) --
	( 99.48, 61.10) --
	( 99.30, 61.89) --
	( 99.11, 62.67) --
	( 98.91, 63.45) --
	( 98.69, 64.23) --
	( 98.46, 65.01) --
	( 98.22, 65.78) --
	( 97.96, 66.54) --
	( 97.69, 67.31) --
	( 97.41, 68.06) --
	( 97.11, 68.82) --
	( 96.80, 69.56) --
	( 96.48, 70.30) --
	( 96.14, 71.04) --
	( 95.79, 71.77) --
	( 95.43, 72.49) --
	( 95.05, 73.21) --
	( 94.67, 73.92) --
	( 94.27, 74.62) --
	( 93.86, 75.32) --
	( 93.43, 76.01) --
	( 93.00, 76.69) --
	( 92.55, 77.36) --
	( 92.09, 78.03) --
	( 91.62, 78.68) --
	( 91.14, 79.33) --
	( 90.65, 79.97) --
	( 90.14, 80.60) --
	( 89.63, 81.23) --
	( 89.10, 81.84) --
	( 88.56, 82.45) --
	( 88.02, 83.04) --
	( 87.46, 83.63) --
	( 86.89, 84.20) --
	( 86.31, 84.77) --
	( 85.72, 85.32) --
	( 85.13, 85.87) --
	( 84.52, 86.40) --
	( 83.90, 86.92) --
	( 83.28, 87.44) --
	( 82.64, 87.94) --
	( 82.00, 88.43) --
	( 81.35, 88.91) --
	( 80.69, 89.38) --
	( 80.02, 89.83) --
	( 79.35, 90.28) --
	( 78.67, 90.71) --
	( 77.97, 91.13) --
	( 77.28, 91.54) --
	( 76.57, 91.93) --
	( 75.86, 92.32) --
	( 75.14, 92.69) --
	( 74.42, 93.05) --
	( 73.69, 93.39) --
	( 72.95, 93.73) --
	( 72.21, 94.05) --
	( 71.46, 94.35) --
	( 70.71, 94.65) --
	( 69.95, 94.93) --
	( 69.19, 95.19) --
	( 68.42, 95.45) --
	( 67.65, 95.69) --
	( 66.87, 95.92) --
	( 66.09, 96.13) --
	( 65.31, 96.33) --
	( 64.52, 96.51) --
	( 63.73, 96.69) --
	( 62.94, 96.84) --
	( 62.14, 96.99) --
	( 61.34, 97.12) --
	( 60.54, 97.24) --
	( 59.74, 97.34) --
	( 58.94, 97.43) --
	( 58.13, 97.50) --
	( 57.33, 97.56) --
	( 56.52, 97.61) --
	( 55.71, 97.64) --
	( 54.90, 97.66) --
	( 54.09, 97.66) --
	( 53.29, 97.65) --
	( 52.48, 97.62) --
	( 51.67, 97.59) --
	( 50.86, 97.53) --
	( 50.06, 97.47) --
	( 49.25, 97.38) --
	( 48.45, 97.29) --
	( 47.65, 97.18) --
	( 46.85, 97.06) --
	( 46.05, 96.92) --
	( 45.26, 96.77) --
	( 44.47, 96.60) --
	( 43.68, 96.42) --
	( 42.89, 96.23) --
	( 42.11, 96.02) --
	( 41.33, 95.80) --
	( 40.56, 95.57) --
	( 39.79, 95.32) --
	( 39.03, 95.06) --
	( 38.26, 94.79) --
	( 37.51, 94.50) --
	( 36.76, 94.20) --
	( 36.01, 93.89) --
	( 35.27, 93.56) --
	( 34.54, 93.22) --
	( 33.81, 92.87) --
	( 33.09, 92.51) --
	( 32.38, 92.13) --
	( 31.67, 91.74) --
	( 30.97, 91.34) --
	( 30.27, 90.92) --
	( 29.58, 90.49) --
	( 28.91, 90.06) --
	( 28.23, 89.61) --
	( 27.57, 89.14) --
	( 26.92, 88.67) --
	( 26.27, 88.19) --
	( 25.63, 87.69) --
	( 25.00, 87.18) --
	( 24.38, 86.66) --
	( 23.77, 86.14) --
	( 23.17, 85.60) --
	( 22.57, 85.05) --
	( 21.99, 84.49) --
	( 21.42, 83.91) --
	( 20.86, 83.33) --
	( 20.30, 82.74) --
	( 19.76, 82.14) --
	( 19.23, 81.54) --
	( 18.71, 80.92) --
	( 18.20, 80.29) --
	( 17.70, 79.65) --
	( 17.21, 79.01) --
	( 16.73, 78.36) --
	( 16.27, 77.69) --
	( 15.82, 77.02) --
	( 15.38, 76.35) --
	( 14.95, 75.66) --
	( 14.53, 74.97) --
	( 14.12, 74.27) --
	( 13.73, 73.56) --
	( 13.35, 72.85) --
	( 12.98, 72.13) --
	( 12.63, 71.40) --
	( 12.28, 70.67) --
	( 11.95, 69.93) --
	( 11.64, 69.19) --
	( 11.33, 68.44) --
	( 11.04, 67.69) --
	( 10.77, 66.93) --
	( 10.50, 66.16) --
	( 10.25, 65.39) --
	( 10.01, 64.62) --
	(  9.79, 63.84) --
	(  9.58, 63.06) --
	(  9.39, 62.28) --
	(  9.20, 61.49) --
	(  9.03, 60.70) --
	(  8.88, 59.91) --
	(  8.74, 59.11) --
	(  8.61, 58.31) --
	(  8.50, 57.51) --
	(  8.40, 56.71) --
	(  8.32, 55.90) --
	(  8.24, 55.10) --
	(  8.19, 54.29) --
	(  8.15, 53.49) --
	(  8.12, 52.68) --
	(  8.10, 51.87) --
	(  8.10, 51.06) --
	(  8.12, 50.25) --
	(  8.15, 49.44) --
	(  8.19, 48.64) --
	(  8.24, 47.83) --
	(  8.32, 47.03) --
	(  8.40, 46.22) --
	(  8.50, 45.42) --
	(  8.61, 44.62) --
	(  8.74, 43.82) --
	(  8.88, 43.02) --
	(  9.03, 42.23) --
	(  9.20, 41.44) --
	(  9.39, 40.65) --
	(  9.58, 39.87) --
	(  9.79, 39.09) --
	( 10.01, 38.31) --
	( 10.25, 37.54) --
	( 10.50, 36.77) --
	( 10.77, 36.00) --
	( 11.04, 35.24) --
	( 11.33, 34.49) --
	( 11.64, 33.74) --
	( 11.95, 33.00) --
	( 12.28, 32.26) --
	( 12.63, 31.53) --
	( 12.98, 30.80) --
	( 13.35, 30.08) --
	( 13.73, 29.37) --
	( 14.12, 28.66) --
	( 14.53, 27.96) --
	( 14.95, 27.27) --
	( 15.38, 26.58) --
	( 15.82, 25.91) --
	( 16.27, 25.24) --
	( 16.73, 24.57) --
	( 17.21, 23.92) --
	( 17.70, 23.28) --
	( 18.20, 22.64) --
	( 18.71, 22.01) --
	( 19.23, 21.39) --
	( 19.76, 20.79) --
	( 20.30, 20.19) --
	( 20.86, 19.60) --
	( 21.42, 19.02) --
	( 21.99, 18.44) --
	( 22.57, 17.88) --
	( 23.17, 17.33) --
	( 23.77, 16.79) --
	( 24.38, 16.27) --
	( 25.00, 15.75) --
	( 25.63, 15.24) --
	( 26.27, 14.74) --
	( 26.92, 14.26) --
	( 27.57, 13.79) --
	( 28.23, 13.32) --
	( 28.91, 12.87) --
	( 29.58, 12.44) --
	( 30.27, 12.01) --
	( 30.97, 11.59) --
	( 31.67, 11.19) --
	( 32.38, 10.80) --
	( 33.09, 10.42) --
	( 33.81, 10.06) --
	( 34.54,  9.71) --
	( 35.27,  9.37) --
	( 36.01,  9.04) --
	( 36.76,  8.73) --
	( 37.51,  8.43) --
	( 38.26,  8.14) --
	( 39.03,  7.87) --
	( 39.79,  7.61) --
	( 40.56,  7.36) --
	( 41.33,  7.13) --
	( 42.11,  6.91) --
	( 42.89,  6.70) --
	( 43.68,  6.51) --
	( 44.47,  6.33) --
	( 45.26,  6.16) --
	( 46.05,  6.01) --
	( 46.85,  5.87) --
	( 47.65,  5.75) --
	( 48.45,  5.64) --
	( 49.25,  5.55) --
	( 50.06,  5.46) --
	( 50.86,  5.40) --
	( 51.67,  5.34) --
	( 52.48,  5.31) --
	( 53.29,  5.28) --
	( 54.09,  5.27) --
	( 54.90,  5.27) --
	( 55.71,  5.29) --
	( 56.52,  5.32) --
	( 57.33,  5.37) --
	( 58.13,  5.43) --
	( 58.94,  5.50) --
	( 59.74,  5.59) --
	( 60.54,  5.69) --
	( 61.34,  5.81) --
	( 62.14,  5.94) --
	( 62.94,  6.09) --
	( 63.73,  6.24) --
	( 64.52,  6.42) --
	( 65.31,  6.60) --
	( 66.09,  6.80) --
	( 66.87,  7.01) --
	( 67.65,  7.24) --
	( 68.42,  7.48) --
	( 69.19,  7.74) --
	( 69.95,  8.00) --
	( 70.71,  8.28) --
	( 71.46,  8.58) --
	( 72.21,  8.88) --
	( 72.95,  9.20) --
	( 73.69,  9.54) --
	( 74.42,  9.88) --
	( 75.14, 10.24) --
	( 75.86, 10.61) --
	( 76.57, 11.00) --
	( 77.28, 11.39) --
	( 77.97, 11.80) --
	( 78.67, 12.22) --
	( 79.35, 12.65) --
	( 80.02, 13.10) --
	( 80.69, 13.55) --
	( 81.35, 14.02) --
	( 82.00, 14.50) --
	( 82.64, 14.99) --
	( 83.28, 15.49) --
	( 83.90, 16.01) --
	( 84.52, 16.53) --
	( 85.13, 17.06) --
	( 85.72, 17.61) --
	( 86.31, 18.16) --
	( 86.89, 18.73) --
	( 87.46, 19.30) --
	( 88.02, 19.89) --
	( 88.56, 20.48) --
	( 89.10, 21.09) --
	( 89.63, 21.70) --
	( 90.14, 22.33) --
	( 90.65, 22.96) --
	( 91.14, 23.60) --
	( 91.62, 24.25) --
	( 92.09, 24.90) --
	( 92.55, 25.57) --
	( 93.00, 26.24) --
	( 93.43, 26.92) --
	( 93.86, 27.61) --
	( 94.27, 28.31) --
	( 94.67, 29.01) --
	( 95.05, 29.72) --
	( 95.43, 30.44) --
	( 95.79, 31.16) --
	( 96.14, 31.89) --
	( 96.48, 32.63) --
	( 96.80, 33.37) --
	( 97.11, 34.11) --
	( 97.41, 34.87) --
	( 97.69, 35.62) --
	( 97.96, 36.39) --
	( 98.22, 37.15) --
	( 98.46, 37.92) --
	( 98.69, 38.70) --
	( 98.91, 39.48) --
	( 99.11, 40.26) --
	( 99.30, 41.04) --
	( 99.48, 41.83) --
	( 99.64, 42.63) --
	( 99.79, 43.42) --
	( 99.92, 44.22) --
	(100.04, 45.02) --
	(100.15, 45.82) --
	(100.24, 46.62) --
	(100.31, 47.43) --
	(100.38, 48.23) --
	(100.43, 49.04) --
	(100.46, 49.85) --
	(100.48, 50.66) --
	(100.49, 51.47);
\definecolor[named]{drawColor}{rgb}{0.00,0.00,0.00}

\path[draw=drawColor,line width= 0.4pt,dash pattern=on 1pt off 3pt ,line join=round,line cap=round] (105.11, 51.47) --
	(105.10, 52.35) --
	(105.08, 53.24) --
	(105.04, 54.13) --
	(104.99, 55.02) --
	(104.92, 55.91) --
	(104.83, 56.79) --
	(104.73, 57.67) --
	(104.61, 58.56) --
	(104.48, 59.44) --
	(104.33, 60.31) --
	(104.17, 61.19) --
	(103.99, 62.06) --
	(103.80, 62.93) --
	(103.59, 63.79) --
	(103.37, 64.65) --
	(103.13, 65.51) --
	(102.88, 66.36) --
	(102.61, 67.21) --
	(102.33, 68.05) --
	(102.03, 68.89) --
	(101.72, 69.72) --
	(101.39, 70.55) --
	(101.05, 71.37) --
	(100.69, 72.19) --
	(100.32, 73.00) --
	( 99.94, 73.80) --
	( 99.54, 74.59) --
	( 99.13, 75.38) --
	( 98.71, 76.16) --
	( 98.27, 76.94) --
	( 97.81, 77.70) --
	( 97.35, 78.46) --
	( 96.87, 79.21) --
	( 96.38, 79.95) --
	( 95.87, 80.68) --
	( 95.35, 81.40) --
	( 94.82, 82.12) --
	( 94.28, 82.82) --
	( 93.73, 83.52) --
	( 93.16, 84.20) --
	( 92.58, 84.88) --
	( 91.99, 85.54) --
	( 91.39, 86.20) --
	( 90.77, 86.84) --
	( 90.15, 87.47) --
	( 89.51, 88.10) --
	( 88.87, 88.71) --
	( 88.21, 89.31) --
	( 87.54, 89.89) --
	( 86.86, 90.47) --
	( 86.18, 91.03) --
	( 85.48, 91.59) --
	( 84.77, 92.13) --
	( 84.06, 92.65) --
	( 83.33, 93.17) --
	( 82.60, 93.67) --
	( 81.85, 94.16) --
	( 81.10, 94.63) --
	( 80.34, 95.10) --
	( 79.58, 95.55) --
	( 78.80, 95.98) --
	( 78.02, 96.40) --
	( 77.23, 96.81) --
	( 76.43, 97.21) --
	( 75.63, 97.59) --
	( 74.82, 97.95) --
	( 74.00, 98.30) --
	( 73.18, 98.64) --
	( 72.35, 98.97) --
	( 71.51, 99.27) --
	( 70.67, 99.57) --
	( 69.83, 99.85) --
	( 68.98,100.11) --
	( 68.13,100.36) --
	( 67.27,100.60) --
	( 66.41,100.82) --
	( 65.54,101.02) --
	( 64.67,101.21) --
	( 63.80,101.38) --
	( 62.93,101.54) --
	( 62.05,101.68) --
	( 61.17,101.81) --
	( 60.29,101.93) --
	( 59.40,102.02) --
	( 58.52,102.10) --
	( 57.63,102.17) --
	( 56.74,102.22) --
	( 55.85,102.26) --
	( 54.96,102.28) --
	( 54.07,102.28) --
	( 53.18,102.27) --
	( 52.30,102.24) --
	( 51.41,102.20) --
	( 50.52,102.14) --
	( 49.63,102.07) --
	( 48.75,101.98) --
	( 47.87,101.87) --
	( 46.98,101.75) --
	( 46.11,101.62) --
	( 45.23,101.46) --
	( 44.36,101.30) --
	( 43.49,101.12) --
	( 42.62,100.92) --
	( 41.75,100.71) --
	( 40.89,100.48) --
	( 40.04,100.24) --
	( 39.19, 99.98) --
	( 38.34, 99.71) --
	( 37.50, 99.42) --
	( 36.66, 99.12) --
	( 35.83, 98.81) --
	( 35.00, 98.47) --
	( 34.19, 98.13) --
	( 33.37, 97.77) --
	( 32.56, 97.40) --
	( 31.76, 97.01) --
	( 30.97, 96.61) --
	( 30.18, 96.19) --
	( 29.40, 95.77) --
	( 28.63, 95.32) --
	( 27.87, 94.87) --
	( 27.11, 94.40) --
	( 26.37, 93.92) --
	( 25.63, 93.42) --
	( 24.90, 92.91) --
	( 24.18, 92.39) --
	( 23.47, 91.86) --
	( 22.76, 91.31) --
	( 22.07, 90.75) --
	( 21.39, 90.18) --
	( 20.72, 89.60) --
	( 20.05, 89.01) --
	( 19.40, 88.40) --
	( 18.76, 87.79) --
	( 18.13, 87.16) --
	( 17.51, 86.52) --
	( 16.90, 85.87) --
	( 16.31, 85.21) --
	( 15.72, 84.54) --
	( 15.15, 83.86) --
	( 14.59, 83.17) --
	( 14.04, 82.47) --
	( 13.50, 81.76) --
	( 12.98, 81.04) --
	( 12.47, 80.32) --
	( 11.97, 79.58) --
	( 11.48, 78.84) --
	( 11.01, 78.08) --
	( 10.55, 77.32) --
	( 10.11, 76.55) --
	(  9.67, 75.77) --
	(  9.25, 74.99) --
	(  8.85, 74.20) --
	(  8.46, 73.40) --
	(  8.08, 72.59) --
	(  7.72, 71.78) --
	(  7.37, 70.96) --
	(  7.04, 70.14) --
	(  6.72, 69.31) --
	(  6.41, 68.47) --
	(  6.12, 67.63) --
	(  5.85, 66.79) --
	(  5.59, 65.94) --
	(  5.34, 65.08) --
	(  5.11, 64.22) --
	(  4.89, 63.36) --
	(  4.69, 62.49) --
	(  4.51, 61.62) --
	(  4.34, 60.75) --
	(  4.18, 59.87) --
	(  4.04, 59.00) --
	(  3.92, 58.12) --
	(  3.81, 57.23) --
	(  3.72, 56.35) --
	(  3.64, 55.46) --
	(  3.58, 54.58) --
	(  3.53, 53.69) --
	(  3.50, 52.80) --
	(  3.48, 51.91) --
	(  3.48, 51.02) --
	(  3.50, 50.13) --
	(  3.53, 49.24) --
	(  3.58, 48.35) --
	(  3.64, 47.47) --
	(  3.72, 46.58) --
	(  3.81, 45.70) --
	(  3.92, 44.81) --
	(  4.04, 43.93) --
	(  4.18, 43.06) --
	(  4.34, 42.18) --
	(  4.51, 41.31) --
	(  4.69, 40.44) --
	(  4.89, 39.57) --
	(  5.11, 38.71) --
	(  5.34, 37.85) --
	(  5.59, 36.99) --
	(  5.85, 36.14) --
	(  6.12, 35.30) --
	(  6.41, 34.46) --
	(  6.72, 33.62) --
	(  7.04, 32.79) --
	(  7.37, 31.97) --
	(  7.72, 31.15) --
	(  8.08, 30.34) --
	(  8.46, 29.53) --
	(  8.85, 28.73) --
	(  9.25, 27.94) --
	(  9.67, 27.16) --
	( 10.11, 26.38) --
	( 10.55, 25.61) --
	( 11.01, 24.85) --
	( 11.48, 24.09) --
	( 11.97, 23.35) --
	( 12.47, 22.61) --
	( 12.98, 21.89) --
	( 13.50, 21.17) --
	( 14.04, 20.46) --
	( 14.59, 19.76) --
	( 15.15, 19.07) --
	( 15.72, 18.39) --
	( 16.31, 17.72) --
	( 16.90, 17.06) --
	( 17.51, 16.41) --
	( 18.13, 15.77) --
	( 18.76, 15.14) --
	( 19.40, 14.53) --
	( 20.05, 13.92) --
	( 20.72, 13.33) --
	( 21.39, 12.75) --
	( 22.07, 12.18) --
	( 22.76, 11.62) --
	( 23.47, 11.07) --
	( 24.18, 10.54) --
	( 24.90, 10.02) --
	( 25.63,  9.51) --
	( 26.37,  9.01) --
	( 27.11,  8.53) --
	( 27.87,  8.06) --
	( 28.63,  7.61) --
	( 29.40,  7.16) --
	( 30.18,  6.74) --
	( 30.97,  6.32) --
	( 31.76,  5.92) --
	( 32.56,  5.53) --
	( 33.37,  5.16) --
	( 34.19,  4.80) --
	( 35.00,  4.46) --
	( 35.83,  4.12) --
	( 36.66,  3.81) --
	( 37.50,  3.51) --
	( 38.34,  3.22) --
	( 39.19,  2.95) --
	( 40.04,  2.69) --
	( 40.89,  2.45) --
	( 41.75,  2.22) --
	( 42.62,  2.01) --
	( 43.49,  1.81) --
	( 44.36,  1.63) --
	( 45.23,  1.47) --
	( 46.11,  1.32) --
	( 46.98,  1.18) --
	( 47.87,  1.06) --
	( 48.75,  0.95) --
	( 49.63,  0.86) --
	( 50.52,  0.79) --
	( 51.41,  0.73) --
	( 52.30,  0.69) --
	( 53.18,  0.66) --
	( 54.07,  0.65) --
	( 54.96,  0.65) --
	( 55.85,  0.67) --
	( 56.74,  0.71) --
	( 57.63,  0.76) --
	( 58.52,  0.83) --
	( 59.40,  0.91) --
	( 60.29,  1.00) --
	( 61.17,  1.12) --
	( 62.05,  1.25) --
	( 62.93,  1.39) --
	( 63.80,  1.55) --
	( 64.67,  1.72) --
	( 65.54,  1.91) --
	( 66.41,  2.11) --
	( 67.27,  2.33) --
	( 68.13,  2.57) --
	( 68.98,  2.82) --
	( 69.83,  3.08) --
	( 70.67,  3.36) --
	( 71.51,  3.66) --
	( 72.35,  3.96) --
	( 73.18,  4.29) --
	( 74.00,  4.63) --
	( 74.82,  4.98) --
	( 75.63,  5.34) --
	( 76.43,  5.72) --
	( 77.23,  6.12) --
	( 78.02,  6.53) --
	( 78.80,  6.95) --
	( 79.58,  7.38) --
	( 80.34,  7.83) --
	( 81.10,  8.30) --
	( 81.85,  8.77) --
	( 82.60,  9.26) --
	( 83.33,  9.76) --
	( 84.06, 10.28) --
	( 84.77, 10.80) --
	( 85.48, 11.34) --
	( 86.18, 11.90) --
	( 86.86, 12.46) --
	( 87.54, 13.04) --
	( 88.21, 13.62) --
	( 88.87, 14.22) --
	( 89.51, 14.83) --
	( 90.15, 15.46) --
	( 90.77, 16.09) --
	( 91.39, 16.73) --
	( 91.99, 17.39) --
	( 92.58, 18.05) --
	( 93.16, 18.73) --
	( 93.73, 19.41) --
	( 94.28, 20.11) --
	( 94.82, 20.81) --
	( 95.35, 21.53) --
	( 95.87, 22.25) --
	( 96.38, 22.98) --
	( 96.87, 23.72) --
	( 97.35, 24.47) --
	( 97.81, 25.23) --
	( 98.27, 25.99) --
	( 98.71, 26.77) --
	( 99.13, 27.55) --
	( 99.54, 28.34) --
	( 99.94, 29.13) --
	(100.32, 29.93) --
	(100.69, 30.74) --
	(101.05, 31.56) --
	(101.39, 32.38) --
	(101.72, 33.21) --
	(102.03, 34.04) --
	(102.33, 34.88) --
	(102.61, 35.72) --
	(102.88, 36.57) --
	(103.13, 37.42) --
	(103.37, 38.28) --
	(103.59, 39.14) --
	(103.80, 40.00) --
	(103.99, 40.87) --
	(104.17, 41.74) --
	(104.33, 42.62) --
	(104.48, 43.49) --
	(104.61, 44.37) --
	(104.73, 45.26) --
	(104.83, 46.14) --
	(104.92, 47.02) --
	(104.99, 47.91) --
	(105.04, 48.80) --
	(105.08, 49.69) --
	(105.10, 50.58) --
	(105.11, 51.47);

\path[fill=fillColor] ( 10.99,126.47) circle (  2.25);
\definecolor[named]{drawColor}{rgb}{0.00,0.00,1.00}

\path[draw=drawColor,line width= 1.2pt,line join=round,line cap=round] ( 57.19,126.47) --
	( 57.18,127.28) --
	( 57.16,128.09) --
	( 57.12,128.90) --
	( 57.07,129.70) --
	( 57.01,130.51) --
	( 56.93,131.31) --
	( 56.84,132.12) --
	( 56.73,132.92) --
	( 56.61,133.72) --
	( 56.48,134.52) --
	( 56.33,135.31) --
	( 56.17,136.10) --
	( 56.00,136.89) --
	( 55.81,137.68) --
	( 55.60,138.46) --
	( 55.39,139.24) --
	( 55.16,140.02) --
	( 54.91,140.79) --
	( 54.66,141.55) --
	( 54.38,142.31) --
	( 54.10,143.07) --
	( 53.80,143.82) --
	( 53.49,144.57) --
	( 53.17,145.31) --
	( 52.83,146.05) --
	( 52.49,146.78) --
	( 52.12,147.50) --
	( 51.75,148.22) --
	( 51.36,148.93) --
	( 50.96,149.63) --
	( 50.55,150.32) --
	( 50.13,151.01) --
	( 49.69,151.69) --
	( 49.25,152.37) --
	( 48.79,153.03) --
	( 48.32,153.69) --
	( 47.83,154.34) --
	( 47.34,154.98) --
	( 46.84,155.61) --
	( 46.32,156.23) --
	( 45.79,156.85) --
	( 45.26,157.45) --
	( 44.71,158.05) --
	( 44.15,158.63) --
	( 43.58,159.21) --
	( 43.01,159.77) --
	( 42.42,160.33) --
	( 41.82,160.87) --
	( 41.21,161.41) --
	( 40.60,161.93) --
	( 39.97,162.44) --
	( 39.34,162.95) --
	( 38.70,163.44) --
	( 38.05,163.92) --
	( 37.39,164.38) --
	( 36.72,164.84) --
	( 36.04,165.28) --
	( 35.36,165.72) --
	( 34.67,166.14) --
	( 33.97,166.55) --
	( 33.27,166.94) --
	( 32.56,167.33) --
	( 31.84,167.70) --
	( 31.11,168.06) --
	( 30.38,168.40) --
	( 29.64,168.73) --
	( 28.90,169.05) --
	( 28.15,169.36) --
	( 27.40,169.65) --
	( 26.64,169.94) --
	( 25.88,170.20) --
	( 25.11,170.46) --
	( 24.34,170.70) --
	( 23.56,170.92) --
	( 22.78,171.14) --
	( 22.00,171.34) --
	( 21.21,171.52) --
	( 20.42,171.69) --
	( 19.63,171.85) --
	( 18.84,172.00) --
	( 18.04,172.13) --
	( 17.24,172.24) --
	( 16.44,172.35) --
	( 15.63,172.43) --
	( 14.83,172.51) --
	( 14.02,172.57) --
	( 13.21,172.61) --
	( 12.41,172.65) --
	( 11.60,172.66) --
	( 10.79,172.67) --
	(  9.98,172.66) --
	(  9.17,172.63) --
	(  8.36,172.59) --
	(  7.56,172.54) --
	(  6.75,172.47) --
	(  5.95,172.39) --
	(  5.15,172.30) --
	(  4.34,172.19) --
	(  3.54,172.06) --
	(  2.75,171.93) --
	(  1.95,171.77) --
	(  1.16,171.61) --
	(  0.37,171.43) --
	(  0.00,171.34);

\path[draw=drawColor,line width= 1.2pt,line join=round,line cap=round] (  0.00, 81.61) --
	(  0.37, 81.51) --
	(  1.16, 81.34) --
	(  1.95, 81.17) --
	(  2.75, 81.02) --
	(  3.54, 80.88) --
	(  4.34, 80.76) --
	(  5.15, 80.65) --
	(  5.95, 80.55) --
	(  6.75, 80.47) --
	(  7.56, 80.41) --
	(  8.36, 80.35) --
	(  9.17, 80.31) --
	(  9.98, 80.29) --
	( 10.79, 80.28) --
	( 11.60, 80.28) --
	( 12.41, 80.30) --
	( 13.21, 80.33) --
	( 14.02, 80.38) --
	( 14.83, 80.44) --
	( 15.63, 80.51) --
	( 16.44, 80.60) --
	( 17.24, 80.70) --
	( 18.04, 80.82) --
	( 18.84, 80.95) --
	( 19.63, 81.09) --
	( 20.42, 81.25) --
	( 21.21, 81.42) --
	( 22.00, 81.61) --
	( 22.78, 81.81) --
	( 23.56, 82.02) --
	( 24.34, 82.25) --
	( 25.11, 82.49) --
	( 25.88, 82.74) --
	( 26.64, 83.01) --
	( 27.40, 83.29) --
	( 28.15, 83.58) --
	( 28.90, 83.89) --
	( 29.64, 84.21) --
	( 30.38, 84.54) --
	( 31.11, 84.89) --
	( 31.84, 85.25) --
	( 32.56, 85.62) --
	( 33.27, 86.00) --
	( 33.97, 86.40) --
	( 34.67, 86.81) --
	( 35.36, 87.23) --
	( 36.04, 87.66) --
	( 36.72, 88.10) --
	( 37.39, 88.56) --
	( 38.05, 89.03) --
	( 38.70, 89.51) --
	( 39.34, 90.00) --
	( 39.97, 90.50) --
	( 40.60, 91.01) --
	( 41.21, 91.54) --
	( 41.82, 92.07) --
	( 42.42, 92.62) --
	( 43.01, 93.17) --
	( 43.58, 93.74) --
	( 44.15, 94.31) --
	( 44.71, 94.90) --
	( 45.26, 95.49) --
	( 45.79, 96.10) --
	( 46.32, 96.71) --
	( 46.84, 97.33) --
	( 47.34, 97.96) --
	( 47.83, 98.61) --
	( 48.32, 99.25) --
	( 48.79, 99.91) --
	( 49.25,100.58) --
	( 49.69,101.25) --
	( 50.13,101.93) --
	( 50.55,102.62) --
	( 50.96,103.32) --
	( 51.36,104.02) --
	( 51.75,104.73) --
	( 52.12,105.45) --
	( 52.49,106.17) --
	( 52.83,106.90) --
	( 53.17,107.63) --
	( 53.49,108.38) --
	( 53.80,109.12) --
	( 54.10,109.87) --
	( 54.38,110.63) --
	( 54.66,111.39) --
	( 54.91,112.16) --
	( 55.16,112.93) --
	( 55.39,113.70) --
	( 55.60,114.48) --
	( 55.81,115.27) --
	( 56.00,116.05) --
	( 56.17,116.84) --
	( 56.33,117.63) --
	( 56.48,118.43) --
	( 56.61,119.23) --
	( 56.73,120.03) --
	( 56.84,120.83) --
	( 56.93,121.63) --
	( 57.01,122.44) --
	( 57.07,123.24) --
	( 57.12,124.05) --
	( 57.16,124.86) --
	( 57.18,125.66) --
	( 57.19,126.47);
\definecolor[named]{drawColor}{rgb}{0.00,0.00,0.00}

\path[draw=drawColor,line width= 0.4pt,dash pattern=on 1pt off 3pt ,line join=round,line cap=round] ( 61.81,126.47) --
	( 61.80,127.36) --
	( 61.77,128.25) --
	( 61.74,129.14) --
	( 61.68,130.03) --
	( 61.61,130.91) --
	( 61.53,131.80) --
	( 61.42,132.68) --
	( 61.31,133.56) --
	( 61.18,134.44) --
	( 61.03,135.32) --
	( 60.87,136.20) --
	( 60.69,137.07) --
	( 60.50,137.93) --
	( 60.29,138.80) --
	( 60.06,139.66) --
	( 59.83,140.52) --
	( 59.57,141.37) --
	( 59.30,142.22) --
	( 59.02,143.06) --
	( 58.72,143.90) --
	( 58.41,144.73) --
	( 58.08,145.56) --
	( 57.74,146.38) --
	( 57.39,147.19) --
	( 57.02,148.00) --
	( 56.63,148.81) --
	( 56.24,149.60) --
	( 55.82,150.39) --
	( 55.40,151.17) --
	( 54.96,151.94) --
	( 54.51,152.71) --
	( 54.04,153.47) --
	( 53.56,154.22) --
	( 53.07,154.96) --
	( 52.57,155.69) --
	( 52.05,156.41) --
	( 51.52,157.13) --
	( 50.98,157.83) --
	( 50.42,158.53) --
	( 49.85,159.21) --
	( 49.27,159.89) --
	( 48.68,160.55) --
	( 48.08,161.21) --
	( 47.47,161.85) --
	( 46.84,162.48) --
	( 46.21,163.10) --
	( 45.56,163.71) --
	( 44.90,164.31) --
	( 44.24,164.90) --
	( 43.56,165.48) --
	( 42.87,166.04) --
	( 42.17,166.59) --
	( 41.47,167.13) --
	( 40.75,167.66) --
	( 40.03,168.18) --
	( 39.29,168.68) --
	( 38.55,169.17) --
	( 37.80,169.64) --
	( 37.04,170.10) --
	( 36.27,170.55) --
	( 35.49,170.99) --
	( 34.71,171.41) --
	( 33.92,171.82) --
	( 33.12,172.21) --
	( 32.32,172.59) --
	( 31.51,172.96) --
	( 30.69,173.31) --
	( 29.87,173.65) --
	( 29.04,173.97) --
	( 28.21,174.28) --
	( 27.37,174.58) --
	( 26.52,174.85) --
	( 25.68,175.12) --
	( 24.82,175.37) --
	( 23.96,175.60) --
	( 23.10,175.82) --
	( 22.24,176.03) --
	( 21.37,176.22) --
	( 20.50,176.39) --
	( 19.62,176.55) --
	( 18.74,176.69) --
	( 17.86,176.82) --
	( 16.98,176.93) --
	( 16.10,177.03) --
	( 15.21,177.11) --
	( 14.32,177.18) --
	( 13.44,177.23) --
	( 12.55,177.26) --
	( 11.66,177.28) --
	( 10.77,177.29) --
	(  9.88,177.27) --
	(  8.99,177.25) --
	(  8.10,177.20) --
	(  7.21,177.15) --
	(  6.33,177.07) --
	(  5.44,176.98) --
	(  4.56,176.88) --
	(  3.68,176.76) --
	(  2.80,176.62) --
	(  1.92,176.47) --
	(  1.05,176.31) --
	(  0.18,176.12) --
	(  0.00,176.08);

\path[draw=drawColor,line width= 0.4pt,dash pattern=on 1pt off 3pt ,line join=round,line cap=round] (  0.00, 76.86) --
	(  0.18, 76.82) --
	(  1.05, 76.64) --
	(  1.92, 76.47) --
	(  2.80, 76.32) --
	(  3.68, 76.19) --
	(  4.56, 76.07) --
	(  5.44, 75.96) --
	(  6.33, 75.87) --
	(  7.21, 75.80) --
	(  8.10, 75.74) --
	(  8.99, 75.70) --
	(  9.88, 75.67) --
	( 10.77, 75.66) --
	( 11.66, 75.66) --
	( 12.55, 75.68) --
	( 13.44, 75.72) --
	( 14.32, 75.77) --
	( 15.21, 75.83) --
	( 16.10, 75.92) --
	( 16.98, 76.01) --
	( 17.86, 76.12) --
	( 18.74, 76.25) --
	( 19.62, 76.40) --
	( 20.50, 76.55) --
	( 21.37, 76.73) --
	( 22.24, 76.92) --
	( 23.10, 77.12) --
	( 23.96, 77.34) --
	( 24.82, 77.58) --
	( 25.68, 77.83) --
	( 26.52, 78.09) --
	( 27.37, 78.37) --
	( 28.21, 78.66) --
	( 29.04, 78.97) --
	( 29.87, 79.30) --
	( 30.69, 79.63) --
	( 31.51, 79.99) --
	( 32.32, 80.35) --
	( 33.12, 80.73) --
	( 33.92, 81.13) --
	( 34.71, 81.53) --
	( 35.49, 81.96) --
	( 36.27, 82.39) --
	( 37.04, 82.84) --
	( 37.80, 83.30) --
	( 38.55, 83.78) --
	( 39.29, 84.27) --
	( 40.03, 84.77) --
	( 40.75, 85.28) --
	( 41.47, 85.81) --
	( 42.17, 86.35) --
	( 42.87, 86.90) --
	( 43.56, 87.47) --
	( 44.24, 88.04) --
	( 44.90, 88.63) --
	( 45.56, 89.23) --
	( 46.21, 89.84) --
	( 46.84, 90.46) --
	( 47.47, 91.10) --
	( 48.08, 91.74) --
	( 48.68, 92.39) --
	( 49.27, 93.06) --
	( 49.85, 93.73) --
	( 50.42, 94.42) --
	( 50.98, 95.11) --
	( 51.52, 95.82) --
	( 52.05, 96.53) --
	( 52.57, 97.26) --
	( 53.07, 97.99) --
	( 53.56, 98.73) --
	( 54.04, 99.48) --
	( 54.51,100.24) --
	( 54.96,101.00) --
	( 55.40,101.77) --
	( 55.82,102.56) --
	( 56.24,103.34) --
	( 56.63,104.14) --
	( 57.02,104.94) --
	( 57.39,105.75) --
	( 57.74,106.57) --
	( 58.08,107.39) --
	( 58.41,108.21) --
	( 58.72,109.05) --
	( 59.02,109.88) --
	( 59.30,110.73) --
	( 59.57,111.58) --
	( 59.83,112.43) --
	( 60.06,113.28) --
	( 60.29,114.15) --
	( 60.50,115.01) --
	( 60.69,115.88) --
	( 60.87,116.75) --
	( 61.03,117.62) --
	( 61.18,118.50) --
	( 61.31,119.38) --
	( 61.42,120.26) --
	( 61.53,121.15) --
	( 61.61,122.03) --
	( 61.68,122.92) --
	( 61.74,123.81) --
	( 61.77,124.69) --
	( 61.80,125.58) --
	( 61.81,126.47);
\definecolor[named]{drawColor}{rgb}{0.00,0.00,1.00}

\path[draw=drawColor,line width= 1.2pt,line join=round,line cap=round] ( 13.88,201.48) --
	( 13.87,202.29) --
	( 13.85,203.10) --
	( 13.82,203.90) --
	( 13.77,204.71) --
	( 13.70,205.52) --
	( 13.63,206.32) --
	( 13.53,207.13) --
	( 13.43,207.93) --
	( 13.31,208.73) --
	( 13.17,209.52) --
	( 13.03,210.32) --
	( 12.87,211.11) --
	( 12.69,211.90) --
	( 12.50,212.69) --
	( 12.30,213.47) --
	( 12.08,214.25) --
	( 11.85,215.02) --
	( 11.61,215.79) --
	( 11.35,216.56) --
	( 11.08,217.32) --
	( 10.80,218.08) --
	( 10.50,218.83) --
	( 10.19,219.58) --
	(  9.86,220.32) --
	(  9.53,221.05) --
	(  9.18,221.78) --
	(  8.82,222.51) --
	(  8.44,223.22) --
	(  8.06,223.93) --
	(  7.66,224.64) --
	(  7.25,225.33) --
	(  6.82,226.02) --
	(  6.39,226.70) --
	(  5.94,227.38) --
	(  5.48,228.04) --
	(  5.01,228.70) --
	(  4.53,229.35) --
	(  4.03,229.99) --
	(  3.53,230.62) --
	(  3.01,231.24) --
	(  2.49,231.86) --
	(  1.95,232.46) --
	(  1.40,233.06) --
	(  0.85,233.64) --
	(  0.28,234.22) --
	(  0.00,234.49);

\path[draw=drawColor,line width= 1.2pt,line join=round,line cap=round] (  0.00,168.47) --
	(  0.28,168.74) --
	(  0.85,169.32) --
	(  1.40,169.90) --
	(  1.95,170.50) --
	(  2.49,171.10) --
	(  3.01,171.72) --
	(  3.53,172.34) --
	(  4.03,172.97) --
	(  4.53,173.61) --
	(  5.01,174.26) --
	(  5.48,174.92) --
	(  5.94,175.58) --
	(  6.39,176.26) --
	(  6.82,176.94) --
	(  7.25,177.63) --
	(  7.66,178.32) --
	(  8.06,179.03) --
	(  8.44,179.74) --
	(  8.82,180.45) --
	(  9.18,181.18) --
	(  9.53,181.91) --
	(  9.86,182.64) --
	( 10.19,183.38) --
	( 10.50,184.13) --
	( 10.80,184.88) --
	( 11.08,185.64) --
	( 11.35,186.40) --
	( 11.61,187.17) --
	( 11.85,187.94) --
	( 12.08,188.71) --
	( 12.30,189.49) --
	( 12.50,190.27) --
	( 12.69,191.06) --
	( 12.87,191.85) --
	( 13.03,192.64) --
	( 13.17,193.44) --
	( 13.31,194.23) --
	( 13.43,195.03) --
	( 13.53,195.83) --
	( 13.63,196.64) --
	( 13.70,197.44) --
	( 13.77,198.25) --
	( 13.82,199.06) --
	( 13.85,199.86) --
	( 13.87,200.67) --
	( 13.88,201.48);
\definecolor[named]{drawColor}{rgb}{0.00,0.00,0.00}

\path[draw=drawColor,line width= 0.4pt,dash pattern=on 1pt off 3pt ,line join=round,line cap=round] ( 18.50,201.48) --
	( 18.49,202.37) --
	( 18.47,203.26) --
	( 18.43,204.15) --
	( 18.38,205.03) --
	( 18.31,205.92) --
	( 18.22,206.81) --
	( 18.12,207.69) --
	( 18.00,208.57) --
	( 17.87,209.45) --
	( 17.72,210.33) --
	( 17.56,211.20) --
	( 17.38,212.07) --
	( 17.19,212.94) --
	( 16.98,213.81) --
	( 16.76,214.67) --
	( 16.52,215.52) --
	( 16.27,216.38) --
	( 16.00,217.22) --
	( 15.72,218.07) --
	( 15.42,218.91) --
	( 15.11,219.74) --
	( 14.78,220.57) --
	( 14.44,221.39) --
	( 14.08,222.20) --
	( 13.71,223.01) --
	( 13.33,223.81) --
	( 12.93,224.61) --
	( 12.52,225.40) --
	( 12.09,226.18) --
	( 11.65,226.95) --
	( 11.20,227.72) --
	( 10.74,228.47) --
	( 10.26,229.22) --
	(  9.77,229.96) --
	(  9.26,230.70) --
	(  8.74,231.42) --
	(  8.21,232.13) --
	(  7.67,232.84) --
	(  7.11,233.53) --
	(  6.55,234.22) --
	(  5.97,234.89) --
	(  5.38,235.56) --
	(  4.78,236.21) --
	(  4.16,236.86) --
	(  3.54,237.49) --
	(  2.90,238.11) --
	(  2.26,238.72) --
	(  1.60,239.32) --
	(  0.93,239.91) --
	(  0.25,240.49) --
	(  0.00,240.69);

\path[draw=drawColor,line width= 0.4pt,dash pattern=on 1pt off 3pt ,line join=round,line cap=round] (  0.00,162.27) --
	(  0.25,162.47) --
	(  0.93,163.05) --
	(  1.60,163.64) --
	(  2.26,164.24) --
	(  2.90,164.85) --
	(  3.54,165.47) --
	(  4.16,166.10) --
	(  4.78,166.75) --
	(  5.38,167.40) --
	(  5.97,168.07) --
	(  6.55,168.74) --
	(  7.11,169.43) --
	(  7.67,170.12) --
	(  8.21,170.83) --
	(  8.74,171.54) --
	(  9.26,172.26) --
	(  9.77,173.00) --
	( 10.26,173.74) --
	( 10.74,174.49) --
	( 11.20,175.24) --
	( 11.65,176.01) --
	( 12.09,176.78) --
	( 12.52,177.56) --
	( 12.93,178.35) --
	( 13.33,179.15) --
	( 13.71,179.95) --
	( 14.08,180.76) --
	( 14.44,181.57) --
	( 14.78,182.39) --
	( 15.11,183.22) --
	( 15.42,184.05) --
	( 15.72,184.89) --
	( 16.00,185.74) --
	( 16.27,186.58) --
	( 16.52,187.44) --
	( 16.76,188.29) --
	( 16.98,189.15) --
	( 17.19,190.02) --
	( 17.38,190.89) --
	( 17.56,191.76) --
	( 17.72,192.63) --
	( 17.87,193.51) --
	( 18.00,194.39) --
	( 18.12,195.27) --
	( 18.22,196.15) --
	( 18.31,197.04) --
	( 18.38,197.93) --
	( 18.43,198.81) --
	( 18.47,199.70) --
	( 18.49,200.59) --
	( 18.50,201.48);
\definecolor[named]{drawColor}{rgb}{0.00,0.00,1.00}

\path[draw=drawColor,line width= 1.2pt,line join=round,line cap=round] (195.09,  0.00) --
	(194.90,  0.31) --
	(194.48,  1.00) --
	(194.04,  1.68) --
	(193.60,  2.35) --
	(193.14,  3.02) --
	(192.67,  3.68) --
	(192.19,  4.32) --
	(191.69,  4.97) --
	(191.19,  5.60) --
	(190.67,  6.22) --
	(190.15,  6.83) --
	(189.61,  7.44) --
	(189.06,  8.03) --
	(188.50,  8.62) --
	(187.94,  9.19) --
	(187.36,  9.76) --
	(186.77, 10.31) --
	(186.17, 10.86) --
	(185.57, 11.39) --
	(184.95, 11.92) --
	(184.33, 12.43) --
	(183.69, 12.93) --
	(183.05, 13.42) --
	(182.40, 13.90) --
	(181.74, 14.37) --
	(181.07, 14.83) --
	(180.40, 15.27) --
	(179.71, 15.70) --
	(179.02, 16.12) --
	(178.32, 16.53) --
	(177.62, 16.93) --
	(176.91, 17.31) --
	(176.19, 17.68) --
	(175.46, 18.04) --
	(174.73, 18.39) --
	(174.00, 18.72) --
	(173.25, 19.04) --
	(172.51, 19.35) --
	(171.75, 19.64) --
	(170.99, 19.92) --
	(170.23, 20.19) --
	(169.46, 20.44) --
	(168.69, 20.68) --
	(167.92, 20.91) --
	(167.14, 21.12) --
	(166.35, 21.32) --
	(165.57, 21.51) --
	(164.78, 21.68) --
	(163.98, 21.84) --
	(163.19, 21.98) --
	(162.39, 22.11) --
	(161.59, 22.23) --
	(160.79, 22.33) --
	(159.98, 22.42) --
	(159.18, 22.49) --
	(158.37, 22.55) --
	(157.57, 22.60) --
	(156.76, 22.63) --
	(155.95, 22.65) --
	(155.14, 22.65) --
	(154.33, 22.64) --
	(153.52, 22.62) --
	(152.72, 22.58) --
	(151.91, 22.52) --
	(151.10, 22.46) --
	(150.30, 22.38) --
	(149.50, 22.28) --
	(148.70, 22.17) --
	(147.90, 22.05) --
	(147.10, 21.91) --
	(146.31, 21.76) --
	(145.51, 21.59) --
	(144.73, 21.42) --
	(143.94, 21.22) --
	(143.16, 21.02) --
	(142.38, 20.80) --
	(141.61, 20.56) --
	(140.84, 20.32) --
	(140.07, 20.06) --
	(139.31, 19.78) --
	(138.56, 19.49) --
	(137.81, 19.19) --
	(137.06, 18.88) --
	(136.32, 18.55) --
	(135.59, 18.21) --
	(134.86, 17.86) --
	(134.14, 17.50) --
	(133.42, 17.12) --
	(132.71, 16.73) --
	(132.01, 16.33) --
	(131.32, 15.91) --
	(130.63, 15.49) --
	(129.95, 15.05) --
	(129.28, 14.60) --
	(128.62, 14.14) --
	(127.96, 13.66) --
	(127.31, 13.18) --
	(126.68, 12.68) --
	(126.05, 12.17) --
	(125.43, 11.66) --
	(124.81, 11.13) --
	(124.21, 10.59) --
	(123.62, 10.04) --
	(123.04,  9.48) --
	(122.46,  8.91) --
	(121.90,  8.33) --
	(121.35,  7.74) --
	(120.81,  7.14) --
	(120.28,  6.53) --
	(119.75,  5.91) --
	(119.24,  5.28) --
	(118.75,  4.65) --
	(118.26,  4.00) --
	(117.78,  3.35) --
	(117.32,  2.69) --
	(116.86,  2.02) --
	(116.42,  1.34) --
	(115.99,  0.65) --
	(115.60,  0.00);
\definecolor[named]{drawColor}{rgb}{0.00,0.00,0.00}

\path[draw=drawColor,line width= 0.4pt,dash pattern=on 1pt off 3pt ,line join=round,line cap=round] (200.37,  0.00) --
	(200.18,  0.37) --
	(199.75,  1.16) --
	(199.31,  1.93) --
	(198.86,  2.69) --
	(198.39,  3.45) --
	(197.92,  4.20) --
	(197.42,  4.94) --
	(196.92,  5.67) --
	(196.40,  6.40) --
	(195.87,  7.11) --
	(195.33,  7.82) --
	(194.77,  8.51) --
	(194.21,  9.20) --
	(193.63,  9.87) --
	(193.04, 10.54) --
	(192.43, 11.19) --
	(191.82, 11.83) --
	(191.20, 12.47) --
	(190.56, 13.09) --
	(189.91, 13.70) --
	(189.26, 14.30) --
	(188.59, 14.89) --
	(187.91, 15.46) --
	(187.22, 16.03) --
	(186.53, 16.58) --
	(185.82, 17.12) --
	(185.10, 17.65) --
	(184.38, 18.16) --
	(183.64, 18.66) --
	(182.90, 19.15) --
	(182.15, 19.63) --
	(181.39, 20.09) --
	(180.62, 20.54) --
	(179.85, 20.97) --
	(179.06, 21.40) --
	(178.27, 21.80) --
	(177.48, 22.20) --
	(176.67, 22.58) --
	(175.86, 22.94) --
	(175.05, 23.30) --
	(174.22, 23.63) --
	(173.39, 23.96) --
	(172.56, 24.27) --
	(171.72, 24.56) --
	(170.88, 24.84) --
	(170.03, 25.10) --
	(169.17, 25.35) --
	(168.32, 25.59) --
	(167.45, 25.81) --
	(166.59, 26.01) --
	(165.72, 26.20) --
	(164.85, 26.38) --
	(163.97, 26.53) --
	(163.09, 26.68) --
	(162.21, 26.81) --
	(161.33, 26.92) --
	(160.45, 27.01) --
	(159.56, 27.10) --
	(158.68, 27.16) --
	(157.79, 27.21) --
	(156.90, 27.25) --
	(156.01, 27.27) --
	(155.12, 27.27) --
	(154.23, 27.26) --
	(153.34, 27.23) --
	(152.45, 27.19) --
	(151.57, 27.13) --
	(150.68, 27.06) --
	(149.80, 26.97) --
	(148.91, 26.86) --
	(148.03, 26.74) --
	(147.15, 26.61) --
	(146.28, 26.46) --
	(145.40, 26.29) --
	(144.53, 26.11) --
	(143.66, 25.91) --
	(142.80, 25.70) --
	(141.94, 25.47) --
	(141.09, 25.23) --
	(140.23, 24.97) --
	(139.39, 24.70) --
	(138.54, 24.42) --
	(137.71, 24.11) --
	(136.88, 23.80) --
	(136.05, 23.47) --
	(135.23, 23.12) --
	(134.42, 22.76) --
	(133.61, 22.39) --
	(132.81, 22.00) --
	(132.02, 21.60) --
	(131.23, 21.19) --
	(130.45, 20.76) --
	(129.68, 20.32) --
	(128.92, 19.86) --
	(128.16, 19.39) --
	(127.41, 18.91) --
	(126.67, 18.41) --
	(125.94, 17.90) --
	(125.22, 17.38) --
	(124.51, 16.85) --
	(123.81, 16.30) --
	(123.12, 15.75) --
	(122.43, 15.18) --
	(121.76, 14.59) --
	(121.10, 14.00) --
	(120.45, 13.40) --
	(119.81, 12.78) --
	(119.18, 12.15) --
	(118.56, 11.51) --
	(117.95, 10.86) --
	(117.35, 10.20) --
	(116.77,  9.53) --
	(116.20,  8.85) --
	(115.63,  8.16) --
	(115.09,  7.46) --
	(114.55,  6.76) --
	(114.03,  6.04) --
	(113.51,  5.31) --
	(113.02,  4.57) --
	(112.53,  3.83) --
	(112.06,  3.07) --
	(111.60,  2.31) --
	(111.15,  1.54) --
	(110.72,  0.77) --
	(110.31,  0.00);

\path[fill=fillColor] (112.04, 51.47) circle (  2.25);
\definecolor[named]{drawColor}{rgb}{0.00,0.00,1.00}

\path[draw=drawColor,line width= 1.2pt,line join=round,line cap=round] (158.23, 51.47) --
	(158.23, 52.27) --
	(158.20, 53.08) --
	(158.17, 53.89) --
	(158.12, 54.70) --
	(158.06, 55.50) --
	(157.98, 56.31) --
	(157.89, 57.11) --
	(157.78, 57.91) --
	(157.66, 58.71) --
	(157.53, 59.51) --
	(157.38, 60.30) --
	(157.22, 61.10) --
	(157.04, 61.89) --
	(156.85, 62.67) --
	(156.65, 63.45) --
	(156.43, 64.23) --
	(156.20, 65.01) --
	(155.96, 65.78) --
	(155.70, 66.54) --
	(155.43, 67.31) --
	(155.15, 68.06) --
	(154.85, 68.82) --
	(154.54, 69.56) --
	(154.22, 70.30) --
	(153.88, 71.04) --
	(153.53, 71.77) --
	(153.17, 72.49) --
	(152.80, 73.21) --
	(152.41, 73.92) --
	(152.01, 74.62) --
	(151.60, 75.32) --
	(151.17, 76.01) --
	(150.74, 76.69) --
	(150.29, 77.36) --
	(149.83, 78.03) --
	(149.36, 78.68) --
	(148.88, 79.33) --
	(148.39, 79.97) --
	(147.88, 80.60) --
	(147.37, 81.23) --
	(146.84, 81.84) --
	(146.30, 82.45) --
	(145.76, 83.04) --
	(145.20, 83.63) --
	(144.63, 84.20) --
	(144.05, 84.77) --
	(143.46, 85.32) --
	(142.87, 85.87) --
	(142.26, 86.40) --
	(141.64, 86.92) --
	(141.02, 87.44) --
	(140.39, 87.94) --
	(139.74, 88.43) --
	(139.09, 88.91) --
	(138.43, 89.38) --
	(137.76, 89.83) --
	(137.09, 90.28) --
	(136.41, 90.71) --
	(135.72, 91.13) --
	(135.02, 91.54) --
	(134.31, 91.93) --
	(133.60, 92.32) --
	(132.88, 92.69) --
	(132.16, 93.05) --
	(131.43, 93.39) --
	(130.69, 93.73) --
	(129.95, 94.05) --
	(129.20, 94.35) --
	(128.45, 94.65) --
	(127.69, 94.93) --
	(126.93, 95.19) --
	(126.16, 95.45) --
	(125.39, 95.69) --
	(124.61, 95.92) --
	(123.83, 96.13) --
	(123.05, 96.33) --
	(122.26, 96.51) --
	(121.47, 96.69) --
	(120.68, 96.84) --
	(119.88, 96.99) --
	(119.08, 97.12) --
	(118.28, 97.24) --
	(117.48, 97.34) --
	(116.68, 97.43) --
	(115.87, 97.50) --
	(115.07, 97.56) --
	(114.26, 97.61) --
	(113.45, 97.64) --
	(112.64, 97.66) --
	(111.84, 97.66) --
	(111.03, 97.65) --
	(110.22, 97.62) --
	(109.41, 97.59) --
	(108.60, 97.53) --
	(107.80, 97.47) --
	(106.99, 97.38) --
	(106.19, 97.29) --
	(105.39, 97.18) --
	(104.59, 97.06) --
	(103.79, 96.92) --
	(103.00, 96.77) --
	(102.21, 96.60) --
	(101.42, 96.42) --
	(100.64, 96.23) --
	( 99.85, 96.02) --
	( 99.08, 95.80) --
	( 98.30, 95.57) --
	( 97.53, 95.32) --
	( 96.77, 95.06) --
	( 96.01, 94.79) --
	( 95.25, 94.50) --
	( 94.50, 94.20) --
	( 93.75, 93.89) --
	( 93.01, 93.56) --
	( 92.28, 93.22) --
	( 91.55, 92.87) --
	( 90.83, 92.51) --
	( 90.12, 92.13) --
	( 89.41, 91.74) --
	( 88.71, 91.34) --
	( 88.01, 90.92) --
	( 87.33, 90.49) --
	( 86.65, 90.06) --
	( 85.98, 89.61) --
	( 85.31, 89.14) --
	( 84.66, 88.67) --
	( 84.01, 88.19) --
	( 83.37, 87.69) --
	( 82.74, 87.18) --
	( 82.12, 86.66) --
	( 81.51, 86.14) --
	( 80.91, 85.60) --
	( 80.31, 85.05) --
	( 79.73, 84.49) --
	( 79.16, 83.91) --
	( 78.60, 83.33) --
	( 78.04, 82.74) --
	( 77.50, 82.14) --
	( 76.97, 81.54) --
	( 76.45, 80.92) --
	( 75.94, 80.29) --
	( 75.44, 79.65) --
	( 74.95, 79.01) --
	( 74.48, 78.36) --
	( 74.01, 77.69) --
	( 73.56, 77.02) --
	( 73.12, 76.35) --
	( 72.69, 75.66) --
	( 72.27, 74.97) --
	( 71.86, 74.27) --
	( 71.47, 73.56) --
	( 71.09, 72.85) --
	( 70.72, 72.13) --
	( 70.37, 71.40) --
	( 70.02, 70.67) --
	( 69.69, 69.93) --
	( 69.38, 69.19) --
	( 69.07, 68.44) --
	( 68.78, 67.69) --
	( 68.51, 66.93) --
	( 68.24, 66.16) --
	( 67.99, 65.39) --
	( 67.76, 64.62) --
	( 67.53, 63.84) --
	( 67.32, 63.06) --
	( 67.13, 62.28) --
	( 66.94, 61.49) --
	( 66.77, 60.70) --
	( 66.62, 59.91) --
	( 66.48, 59.11) --
	( 66.35, 58.31) --
	( 66.24, 57.51) --
	( 66.14, 56.71) --
	( 66.06, 55.90) --
	( 65.99, 55.10) --
	( 65.93, 54.29) --
	( 65.89, 53.49) --
	( 65.86, 52.68) --
	( 65.84, 51.87) --
	( 65.84, 51.06) --
	( 65.86, 50.25) --
	( 65.89, 49.44) --
	( 65.93, 48.64) --
	( 65.99, 47.83) --
	( 66.06, 47.03) --
	( 66.14, 46.22) --
	( 66.24, 45.42) --
	( 66.35, 44.62) --
	( 66.48, 43.82) --
	( 66.62, 43.02) --
	( 66.77, 42.23) --
	( 66.94, 41.44) --
	( 67.13, 40.65) --
	( 67.32, 39.87) --
	( 67.53, 39.09) --
	( 67.76, 38.31) --
	( 67.99, 37.54) --
	( 68.24, 36.77) --
	( 68.51, 36.00) --
	( 68.78, 35.24) --
	( 69.07, 34.49) --
	( 69.38, 33.74) --
	( 69.69, 33.00) --
	( 70.02, 32.26) --
	( 70.37, 31.53) --
	( 70.72, 30.80) --
	( 71.09, 30.08) --
	( 71.47, 29.37) --
	( 71.86, 28.66) --
	( 72.27, 27.96) --
	( 72.69, 27.27) --
	( 73.12, 26.58) --
	( 73.56, 25.91) --
	( 74.01, 25.24) --
	( 74.48, 24.57) --
	( 74.95, 23.92) --
	( 75.44, 23.28) --
	( 75.94, 22.64) --
	( 76.45, 22.01) --
	( 76.97, 21.39) --
	( 77.50, 20.79) --
	( 78.04, 20.19) --
	( 78.60, 19.60) --
	( 79.16, 19.02) --
	( 79.73, 18.44) --
	( 80.31, 17.88) --
	( 80.91, 17.33) --
	( 81.51, 16.79) --
	( 82.12, 16.27) --
	( 82.74, 15.75) --
	( 83.37, 15.24) --
	( 84.01, 14.74) --
	( 84.66, 14.26) --
	( 85.31, 13.79) --
	( 85.98, 13.32) --
	( 86.65, 12.87) --
	( 87.33, 12.44) --
	( 88.01, 12.01) --
	( 88.71, 11.59) --
	( 89.41, 11.19) --
	( 90.12, 10.80) --
	( 90.83, 10.42) --
	( 91.55, 10.06) --
	( 92.28,  9.71) --
	( 93.01,  9.37) --
	( 93.75,  9.04) --
	( 94.50,  8.73) --
	( 95.25,  8.43) --
	( 96.01,  8.14) --
	( 96.77,  7.87) --
	( 97.53,  7.61) --
	( 98.30,  7.36) --
	( 99.08,  7.13) --
	( 99.85,  6.91) --
	(100.64,  6.70) --
	(101.42,  6.51) --
	(102.21,  6.33) --
	(103.00,  6.16) --
	(103.79,  6.01) --
	(104.59,  5.87) --
	(105.39,  5.75) --
	(106.19,  5.64) --
	(106.99,  5.55) --
	(107.80,  5.46) --
	(108.60,  5.40) --
	(109.41,  5.34) --
	(110.22,  5.31) --
	(111.03,  5.28) --
	(111.84,  5.27) --
	(112.64,  5.27) --
	(113.45,  5.29) --
	(114.26,  5.32) --
	(115.07,  5.37) --
	(115.87,  5.43) --
	(116.68,  5.50) --
	(117.48,  5.59) --
	(118.28,  5.69) --
	(119.08,  5.81) --
	(119.88,  5.94) --
	(120.68,  6.09) --
	(121.47,  6.24) --
	(122.26,  6.42) --
	(123.05,  6.60) --
	(123.83,  6.80) --
	(124.61,  7.01) --
	(125.39,  7.24) --
	(126.16,  7.48) --
	(126.93,  7.74) --
	(127.69,  8.00) --
	(128.45,  8.28) --
	(129.20,  8.58) --
	(129.95,  8.88) --
	(130.69,  9.20) --
	(131.43,  9.54) --
	(132.16,  9.88) --
	(132.88, 10.24) --
	(133.60, 10.61) --
	(134.31, 11.00) --
	(135.02, 11.39) --
	(135.72, 11.80) --
	(136.41, 12.22) --
	(137.09, 12.65) --
	(137.76, 13.10) --
	(138.43, 13.55) --
	(139.09, 14.02) --
	(139.74, 14.50) --
	(140.39, 14.99) --
	(141.02, 15.49) --
	(141.64, 16.01) --
	(142.26, 16.53) --
	(142.87, 17.06) --
	(143.46, 17.61) --
	(144.05, 18.16) --
	(144.63, 18.73) --
	(145.20, 19.30) --
	(145.76, 19.89) --
	(146.30, 20.48) --
	(146.84, 21.09) --
	(147.37, 21.70) --
	(147.88, 22.33) --
	(148.39, 22.96) --
	(148.88, 23.60) --
	(149.36, 24.25) --
	(149.83, 24.90) --
	(150.29, 25.57) --
	(150.74, 26.24) --
	(151.17, 26.92) --
	(151.60, 27.61) --
	(152.01, 28.31) --
	(152.41, 29.01) --
	(152.80, 29.72) --
	(153.17, 30.44) --
	(153.53, 31.16) --
	(153.88, 31.89) --
	(154.22, 32.63) --
	(154.54, 33.37) --
	(154.85, 34.11) --
	(155.15, 34.87) --
	(155.43, 35.62) --
	(155.70, 36.39) --
	(155.96, 37.15) --
	(156.20, 37.92) --
	(156.43, 38.70) --
	(156.65, 39.48) --
	(156.85, 40.26) --
	(157.04, 41.04) --
	(157.22, 41.83) --
	(157.38, 42.63) --
	(157.53, 43.42) --
	(157.66, 44.22) --
	(157.78, 45.02) --
	(157.89, 45.82) --
	(157.98, 46.62) --
	(158.06, 47.43) --
	(158.12, 48.23) --
	(158.17, 49.04) --
	(158.20, 49.85) --
	(158.23, 50.66) --
	(158.23, 51.47);
\definecolor[named]{drawColor}{rgb}{0.00,0.00,0.00}

\path[draw=drawColor,line width= 0.4pt,dash pattern=on 1pt off 3pt ,line join=round,line cap=round] (162.85, 51.47) --
	(162.84, 52.35) --
	(162.82, 53.24) --
	(162.78, 54.13) --
	(162.73, 55.02) --
	(162.66, 55.91) --
	(162.57, 56.79) --
	(162.47, 57.67) --
	(162.35, 58.56) --
	(162.22, 59.44) --
	(162.08, 60.31) --
	(161.91, 61.19) --
	(161.74, 62.06) --
	(161.54, 62.93) --
	(161.33, 63.79) --
	(161.11, 64.65) --
	(160.87, 65.51) --
	(160.62, 66.36) --
	(160.35, 67.21) --
	(160.07, 68.05) --
	(159.77, 68.89) --
	(159.46, 69.72) --
	(159.13, 70.55) --
	(158.79, 71.37) --
	(158.43, 72.19) --
	(158.06, 73.00) --
	(157.68, 73.80) --
	(157.28, 74.59) --
	(156.87, 75.38) --
	(156.45, 76.16) --
	(156.01, 76.94) --
	(155.55, 77.70) --
	(155.09, 78.46) --
	(154.61, 79.21) --
	(154.12, 79.95) --
	(153.61, 80.68) --
	(153.09, 81.40) --
	(152.56, 82.12) --
	(152.02, 82.82) --
	(151.47, 83.52) --
	(150.90, 84.20) --
	(150.32, 84.88) --
	(149.73, 85.54) --
	(149.13, 86.20) --
	(148.51, 86.84) --
	(147.89, 87.47) --
	(147.25, 88.10) --
	(146.61, 88.71) --
	(145.95, 89.31) --
	(145.28, 89.89) --
	(144.61, 90.47) --
	(143.92, 91.03) --
	(143.22, 91.59) --
	(142.51, 92.13) --
	(141.80, 92.65) --
	(141.07, 93.17) --
	(140.34, 93.67) --
	(139.59, 94.16) --
	(138.84, 94.63) --
	(138.08, 95.10) --
	(137.32, 95.55) --
	(136.54, 95.98) --
	(135.76, 96.40) --
	(134.97, 96.81) --
	(134.17, 97.21) --
	(133.37, 97.59) --
	(132.56, 97.95) --
	(131.74, 98.30) --
	(130.92, 98.64) --
	(130.09, 98.97) --
	(129.25, 99.27) --
	(128.41, 99.57) --
	(127.57, 99.85) --
	(126.72,100.11) --
	(125.87,100.36) --
	(125.01,100.60) --
	(124.15,100.82) --
	(123.28,101.02) --
	(122.41,101.21) --
	(121.54,101.38) --
	(120.67,101.54) --
	(119.79,101.68) --
	(118.91,101.81) --
	(118.03,101.93) --
	(117.14,102.02) --
	(116.26,102.10) --
	(115.37,102.17) --
	(114.48,102.22) --
	(113.59,102.26) --
	(112.70,102.28) --
	(111.81,102.28) --
	(110.93,102.27) --
	(110.04,102.24) --
	(109.15,102.20) --
	(108.26,102.14) --
	(107.37,102.07) --
	(106.49,101.98) --
	(105.61,101.87) --
	(104.73,101.75) --
	(103.85,101.62) --
	(102.97,101.46) --
	(102.10,101.30) --
	(101.23,101.12) --
	(100.36,100.92) --
	( 99.50,100.71) --
	( 98.64,100.48) --
	( 97.78,100.24) --
	( 96.93, 99.98) --
	( 96.08, 99.71) --
	( 95.24, 99.42) --
	( 94.40, 99.12) --
	( 93.57, 98.81) --
	( 92.75, 98.47) --
	( 91.93, 98.13) --
	( 91.11, 97.77) --
	( 90.31, 97.40) --
	( 89.50, 97.01) --
	( 88.71, 96.61) --
	( 87.92, 96.19) --
	( 87.15, 95.77) --
	( 86.37, 95.32) --
	( 85.61, 94.87) --
	( 84.85, 94.40) --
	( 84.11, 93.92) --
	( 83.37, 93.42) --
	( 82.64, 92.91) --
	( 81.92, 92.39) --
	( 81.21, 91.86) --
	( 80.50, 91.31) --
	( 79.81, 90.75) --
	( 79.13, 90.18) --
	( 78.46, 89.60) --
	( 77.79, 89.01) --
	( 77.14, 88.40) --
	( 76.50, 87.79) --
	( 75.87, 87.16) --
	( 75.25, 86.52) --
	( 74.64, 85.87) --
	( 74.05, 85.21) --
	( 73.46, 84.54) --
	( 72.89, 83.86) --
	( 72.33, 83.17) --
	( 71.78, 82.47) --
	( 71.24, 81.76) --
	( 70.72, 81.04) --
	( 70.21, 80.32) --
	( 69.71, 79.58) --
	( 69.22, 78.84) --
	( 68.75, 78.08) --
	( 68.29, 77.32) --
	( 67.85, 76.55) --
	( 67.41, 75.77) --
	( 67.00, 74.99) --
	( 66.59, 74.20) --
	( 66.20, 73.40) --
	( 65.82, 72.59) --
	( 65.46, 71.78) --
	( 65.11, 70.96) --
	( 64.78, 70.14) --
	( 64.46, 69.31) --
	( 64.15, 68.47) --
	( 63.86, 67.63) --
	( 63.59, 66.79) --
	( 63.33, 65.94) --
	( 63.08, 65.08) --
	( 62.85, 64.22) --
	( 62.63, 63.36) --
	( 62.43, 62.49) --
	( 62.25, 61.62) --
	( 62.08, 60.75) --
	( 61.92, 59.87) --
	( 61.78, 59.00) --
	( 61.66, 58.12) --
	( 61.55, 57.23) --
	( 61.46, 56.35) --
	( 61.38, 55.46) --
	( 61.32, 54.58) --
	( 61.27, 53.69) --
	( 61.24, 52.80) --
	( 61.22, 51.91) --
	( 61.22, 51.02) --
	( 61.24, 50.13) --
	( 61.27, 49.24) --
	( 61.32, 48.35) --
	( 61.38, 47.47) --
	( 61.46, 46.58) --
	( 61.55, 45.70) --
	( 61.66, 44.81) --
	( 61.78, 43.93) --
	( 61.92, 43.06) --
	( 62.08, 42.18) --
	( 62.25, 41.31) --
	( 62.43, 40.44) --
	( 62.63, 39.57) --
	( 62.85, 38.71) --
	( 63.08, 37.85) --
	( 63.33, 36.99) --
	( 63.59, 36.14) --
	( 63.86, 35.30) --
	( 64.15, 34.46) --
	( 64.46, 33.62) --
	( 64.78, 32.79) --
	( 65.11, 31.97) --
	( 65.46, 31.15) --
	( 65.82, 30.34) --
	( 66.20, 29.53) --
	( 66.59, 28.73) --
	( 67.00, 27.94) --
	( 67.41, 27.16) --
	( 67.85, 26.38) --
	( 68.29, 25.61) --
	( 68.75, 24.85) --
	( 69.22, 24.09) --
	( 69.71, 23.35) --
	( 70.21, 22.61) --
	( 70.72, 21.89) --
	( 71.24, 21.17) --
	( 71.78, 20.46) --
	( 72.33, 19.76) --
	( 72.89, 19.07) --
	( 73.46, 18.39) --
	( 74.05, 17.72) --
	( 74.64, 17.06) --
	( 75.25, 16.41) --
	( 75.87, 15.77) --
	( 76.50, 15.14) --
	( 77.14, 14.53) --
	( 77.79, 13.92) --
	( 78.46, 13.33) --
	( 79.13, 12.75) --
	( 79.81, 12.18) --
	( 80.50, 11.62) --
	( 81.21, 11.07) --
	( 81.92, 10.54) --
	( 82.64, 10.02) --
	( 83.37,  9.51) --
	( 84.11,  9.01) --
	( 84.85,  8.53) --
	( 85.61,  8.06) --
	( 86.37,  7.61) --
	( 87.15,  7.16) --
	( 87.92,  6.74) --
	( 88.71,  6.32) --
	( 89.50,  5.92) --
	( 90.31,  5.53) --
	( 91.11,  5.16) --
	( 91.93,  4.80) --
	( 92.75,  4.46) --
	( 93.57,  4.12) --
	( 94.40,  3.81) --
	( 95.24,  3.51) --
	( 96.08,  3.22) --
	( 96.93,  2.95) --
	( 97.78,  2.69) --
	( 98.64,  2.45) --
	( 99.50,  2.22) --
	(100.36,  2.01) --
	(101.23,  1.81) --
	(102.10,  1.63) --
	(102.97,  1.47) --
	(103.85,  1.32) --
	(104.73,  1.18) --
	(105.61,  1.06) --
	(106.49,  0.95) --
	(107.37,  0.86) --
	(108.26,  0.79) --
	(109.15,  0.73) --
	(110.04,  0.69) --
	(110.93,  0.66) --
	(111.81,  0.65) --
	(112.70,  0.65) --
	(113.59,  0.67) --
	(114.48,  0.71) --
	(115.37,  0.76) --
	(116.26,  0.83) --
	(117.14,  0.91) --
	(118.03,  1.00) --
	(118.91,  1.12) --
	(119.79,  1.25) --
	(120.67,  1.39) --
	(121.54,  1.55) --
	(122.41,  1.72) --
	(123.28,  1.91) --
	(124.15,  2.11) --
	(125.01,  2.33) --
	(125.87,  2.57) --
	(126.72,  2.82) --
	(127.57,  3.08) --
	(128.41,  3.36) --
	(129.25,  3.66) --
	(130.09,  3.96) --
	(130.92,  4.29) --
	(131.74,  4.63) --
	(132.56,  4.98) --
	(133.37,  5.34) --
	(134.17,  5.72) --
	(134.97,  6.12) --
	(135.76,  6.53) --
	(136.54,  6.95) --
	(137.32,  7.38) --
	(138.08,  7.83) --
	(138.84,  8.30) --
	(139.59,  8.77) --
	(140.34,  9.26) --
	(141.07,  9.76) --
	(141.80, 10.28) --
	(142.51, 10.80) --
	(143.22, 11.34) --
	(143.92, 11.90) --
	(144.61, 12.46) --
	(145.28, 13.04) --
	(145.95, 13.62) --
	(146.61, 14.22) --
	(147.25, 14.83) --
	(147.89, 15.46) --
	(148.51, 16.09) --
	(149.13, 16.73) --
	(149.73, 17.39) --
	(150.32, 18.05) --
	(150.90, 18.73) --
	(151.47, 19.41) --
	(152.02, 20.11) --
	(152.56, 20.81) --
	(153.09, 21.53) --
	(153.61, 22.25) --
	(154.12, 22.98) --
	(154.61, 23.72) --
	(155.09, 24.47) --
	(155.55, 25.23) --
	(156.01, 25.99) --
	(156.45, 26.77) --
	(156.87, 27.55) --
	(157.28, 28.34) --
	(157.68, 29.13) --
	(158.06, 29.93) --
	(158.43, 30.74) --
	(158.79, 31.56) --
	(159.13, 32.38) --
	(159.46, 33.21) --
	(159.77, 34.04) --
	(160.07, 34.88) --
	(160.35, 35.72) --
	(160.62, 36.57) --
	(160.87, 37.42) --
	(161.11, 38.28) --
	(161.33, 39.14) --
	(161.54, 40.00) --
	(161.74, 40.87) --
	(161.91, 41.74) --
	(162.08, 42.62) --
	(162.22, 43.49) --
	(162.35, 44.37) --
	(162.47, 45.26) --
	(162.57, 46.14) --
	(162.66, 47.02) --
	(162.73, 47.91) --
	(162.78, 48.80) --
	(162.82, 49.69) --
	(162.84, 50.58) --
	(162.85, 51.47);

\path[fill=fillColor] ( 68.73,126.47) circle (  2.25);
\definecolor[named]{drawColor}{rgb}{0.00,0.00,1.00}

\path[draw=drawColor,line width= 1.2pt,line join=round,line cap=round] (114.93,126.47) --
	(114.92,127.28) --
	(114.90,128.09) --
	(114.86,128.90) --
	(114.81,129.70) --
	(114.75,130.51) --
	(114.67,131.31) --
	(114.58,132.12) --
	(114.47,132.92) --
	(114.35,133.72) --
	(114.22,134.52) --
	(114.07,135.31) --
	(113.91,136.10) --
	(113.74,136.89) --
	(113.55,137.68) --
	(113.34,138.46) --
	(113.13,139.24) --
	(112.90,140.02) --
	(112.65,140.79) --
	(112.40,141.55) --
	(112.13,142.31) --
	(111.84,143.07) --
	(111.54,143.82) --
	(111.23,144.57) --
	(110.91,145.31) --
	(110.57,146.05) --
	(110.23,146.78) --
	(109.86,147.50) --
	(109.49,148.22) --
	(109.10,148.93) --
	(108.70,149.63) --
	(108.29,150.32) --
	(107.87,151.01) --
	(107.43,151.69) --
	(106.99,152.37) --
	(106.53,153.03) --
	(106.06,153.69) --
	(105.57,154.34) --
	(105.08,154.98) --
	(104.58,155.61) --
	(104.06,156.23) --
	(103.53,156.85) --
	(103.00,157.45) --
	(102.45,158.05) --
	(101.89,158.63) --
	(101.32,159.21) --
	(100.75,159.77) --
	(100.16,160.33) --
	( 99.56,160.87) --
	( 98.96,161.41) --
	( 98.34,161.93) --
	( 97.71,162.44) --
	( 97.08,162.95) --
	( 96.44,163.44) --
	( 95.79,163.92) --
	( 95.13,164.38) --
	( 94.46,164.84) --
	( 93.78,165.28) --
	( 93.10,165.72) --
	( 92.41,166.14) --
	( 91.71,166.55) --
	( 91.01,166.94) --
	( 90.30,167.33) --
	( 89.58,167.70) --
	( 88.85,168.06) --
	( 88.12,168.40) --
	( 87.39,168.73) --
	( 86.64,169.05) --
	( 85.89,169.36) --
	( 85.14,169.65) --
	( 84.38,169.94) --
	( 83.62,170.20) --
	( 82.85,170.46) --
	( 82.08,170.70) --
	( 81.30,170.92) --
	( 80.53,171.14) --
	( 79.74,171.34) --
	( 78.95,171.52) --
	( 78.16,171.69) --
	( 77.37,171.85) --
	( 76.58,172.00) --
	( 75.78,172.13) --
	( 74.98,172.24) --
	( 74.18,172.35) --
	( 73.37,172.43) --
	( 72.57,172.51) --
	( 71.76,172.57) --
	( 70.95,172.61) --
	( 70.15,172.65) --
	( 69.34,172.66) --
	( 68.53,172.67) --
	( 67.72,172.66) --
	( 66.91,172.63) --
	( 66.11,172.59) --
	( 65.30,172.54) --
	( 64.49,172.47) --
	( 63.69,172.39) --
	( 62.89,172.30) --
	( 62.08,172.19) --
	( 61.29,172.06) --
	( 60.49,171.93) --
	( 59.69,171.77) --
	( 58.90,171.61) --
	( 58.11,171.43) --
	( 57.33,171.24) --
	( 56.55,171.03) --
	( 55.77,170.81) --
	( 55.00,170.58) --
	( 54.23,170.33) --
	( 53.46,170.07) --
	( 52.70,169.80) --
	( 51.94,169.51) --
	( 51.19,169.21) --
	( 50.45,168.90) --
	( 49.71,168.57) --
	( 48.98,168.23) --
	( 48.25,167.88) --
	( 47.53,167.51) --
	( 46.81,167.14) --
	( 46.10,166.75) --
	( 45.40,166.34) --
	( 44.71,165.93) --
	( 44.02,165.50) --
	( 43.34,165.06) --
	( 42.67,164.61) --
	( 42.01,164.15) --
	( 41.35,163.68) --
	( 40.70,163.19) --
	( 40.07,162.70) --
	( 39.44,162.19) --
	( 38.82,161.67) --
	( 38.20,161.14) --
	( 37.60,160.60) --
	( 37.01,160.05) --
	( 36.43,159.49) --
	( 35.85,158.92) --
	( 35.29,158.34) --
	( 34.74,157.75) --
	( 34.20,157.15) --
	( 33.66,156.54) --
	( 33.14,155.92) --
	( 32.63,155.30) --
	( 32.13,154.66) --
	( 31.65,154.02) --
	( 31.17,153.36) --
	( 30.71,152.70) --
	( 30.25,152.03) --
	( 29.81,151.35) --
	( 29.38,150.67) --
	( 28.96,149.98) --
	( 28.56,149.28) --
	( 28.17,148.57) --
	( 27.78,147.86) --
	( 27.42,147.14) --
	( 27.06,146.41) --
	( 26.72,145.68) --
	( 26.39,144.94) --
	( 26.07,144.20) --
	( 25.77,143.45) --
	( 25.48,142.69) --
	( 25.20,141.93) --
	( 24.94,141.17) --
	( 24.69,140.40) --
	( 24.45,139.63) --
	( 24.23,138.85) --
	( 24.02,138.07) --
	( 23.82,137.29) --
	( 23.64,136.50) --
	( 23.47,135.71) --
	( 23.31,134.91) --
	( 23.17,134.12) --
	( 23.05,133.32) --
	( 22.93,132.52) --
	( 22.84,131.72) --
	( 22.75,130.91) --
	( 22.68,130.11) --
	( 22.62,129.30) --
	( 22.58,128.49) --
	( 22.55,127.69) --
	( 22.54,126.88) --
	( 22.54,126.07) --
	( 22.55,125.26) --
	( 22.58,124.45) --
	( 22.62,123.64) --
	( 22.68,122.84) --
	( 22.75,122.03) --
	( 22.84,121.23) --
	( 22.93,120.43) --
	( 23.05,119.63) --
	( 23.17,118.83) --
	( 23.31,118.03) --
	( 23.47,117.24) --
	( 23.64,116.45) --
	( 23.82,115.66) --
	( 24.02,114.87) --
	( 24.23,114.09) --
	( 24.45,113.32) --
	( 24.69,112.54) --
	( 24.94,111.78) --
	( 25.20,111.01) --
	( 25.48,110.25) --
	( 25.77,109.50) --
	( 26.07,108.75) --
	( 26.39,108.00) --
	( 26.72,107.27) --
	( 27.06,106.53) --
	( 27.42,105.81) --
	( 27.78,105.09) --
	( 28.17,104.37) --
	( 28.56,103.67) --
	( 28.96,102.97) --
	( 29.38,102.28) --
	( 29.81,101.59) --
	( 30.25,100.91) --
	( 30.71,100.24) --
	( 31.17, 99.58) --
	( 31.65, 98.93) --
	( 32.13, 98.28) --
	( 32.63, 97.65) --
	( 33.14, 97.02) --
	( 33.66, 96.40) --
	( 34.20, 95.79) --
	( 34.74, 95.19) --
	( 35.29, 94.60) --
	( 35.85, 94.02) --
	( 36.43, 93.45) --
	( 37.01, 92.89) --
	( 37.60, 92.34) --
	( 38.20, 91.80) --
	( 38.82, 91.27) --
	( 39.44, 90.76) --
	( 40.07, 90.25) --
	( 40.70, 89.75) --
	( 41.35, 89.27) --
	( 42.01, 88.79) --
	( 42.67, 88.33) --
	( 43.34, 87.88) --
	( 44.02, 87.44) --
	( 44.71, 87.02) --
	( 45.40, 86.60) --
	( 46.10, 86.20) --
	( 46.81, 85.81) --
	( 47.53, 85.43) --
	( 48.25, 85.07) --
	( 48.98, 84.72) --
	( 49.71, 84.38) --
	( 50.45, 84.05) --
	( 51.19, 83.74) --
	( 51.94, 83.44) --
	( 52.70, 83.15) --
	( 53.46, 82.87) --
	( 54.23, 82.61) --
	( 55.00, 82.37) --
	( 55.77, 82.13) --
	( 56.55, 81.91) --
	( 57.33, 81.71) --
	( 58.11, 81.51) --
	( 58.90, 81.34) --
	( 59.69, 81.17) --
	( 60.49, 81.02) --
	( 61.29, 80.88) --
	( 62.08, 80.76) --
	( 62.89, 80.65) --
	( 63.69, 80.55) --
	( 64.49, 80.47) --
	( 65.30, 80.41) --
	( 66.11, 80.35) --
	( 66.91, 80.31) --
	( 67.72, 80.29) --
	( 68.53, 80.28) --
	( 69.34, 80.28) --
	( 70.15, 80.30) --
	( 70.95, 80.33) --
	( 71.76, 80.38) --
	( 72.57, 80.44) --
	( 73.37, 80.51) --
	( 74.18, 80.60) --
	( 74.98, 80.70) --
	( 75.78, 80.82) --
	( 76.58, 80.95) --
	( 77.37, 81.09) --
	( 78.16, 81.25) --
	( 78.95, 81.42) --
	( 79.74, 81.61) --
	( 80.53, 81.81) --
	( 81.30, 82.02) --
	( 82.08, 82.25) --
	( 82.85, 82.49) --
	( 83.62, 82.74) --
	( 84.38, 83.01) --
	( 85.14, 83.29) --
	( 85.89, 83.58) --
	( 86.64, 83.89) --
	( 87.39, 84.21) --
	( 88.12, 84.54) --
	( 88.85, 84.89) --
	( 89.58, 85.25) --
	( 90.30, 85.62) --
	( 91.01, 86.00) --
	( 91.71, 86.40) --
	( 92.41, 86.81) --
	( 93.10, 87.23) --
	( 93.78, 87.66) --
	( 94.46, 88.10) --
	( 95.13, 88.56) --
	( 95.79, 89.03) --
	( 96.44, 89.51) --
	( 97.08, 90.00) --
	( 97.71, 90.50) --
	( 98.34, 91.01) --
	( 98.96, 91.54) --
	( 99.56, 92.07) --
	(100.16, 92.62) --
	(100.75, 93.17) --
	(101.32, 93.74) --
	(101.89, 94.31) --
	(102.45, 94.90) --
	(103.00, 95.49) --
	(103.53, 96.10) --
	(104.06, 96.71) --
	(104.58, 97.33) --
	(105.08, 97.96) --
	(105.57, 98.61) --
	(106.06, 99.25) --
	(106.53, 99.91) --
	(106.99,100.58) --
	(107.43,101.25) --
	(107.87,101.93) --
	(108.29,102.62) --
	(108.70,103.32) --
	(109.10,104.02) --
	(109.49,104.73) --
	(109.86,105.45) --
	(110.23,106.17) --
	(110.57,106.90) --
	(110.91,107.63) --
	(111.23,108.38) --
	(111.54,109.12) --
	(111.84,109.87) --
	(112.13,110.63) --
	(112.40,111.39) --
	(112.65,112.16) --
	(112.90,112.93) --
	(113.13,113.70) --
	(113.34,114.48) --
	(113.55,115.27) --
	(113.74,116.05) --
	(113.91,116.84) --
	(114.07,117.63) --
	(114.22,118.43) --
	(114.35,119.23) --
	(114.47,120.03) --
	(114.58,120.83) --
	(114.67,121.63) --
	(114.75,122.44) --
	(114.81,123.24) --
	(114.86,124.05) --
	(114.90,124.86) --
	(114.92,125.66) --
	(114.93,126.47);
\definecolor[named]{drawColor}{rgb}{0.00,0.00,0.00}

\path[draw=drawColor,line width= 0.4pt,dash pattern=on 1pt off 3pt ,line join=round,line cap=round] (119.55,126.47) --
	(119.54,127.36) --
	(119.52,128.25) --
	(119.48,129.14) --
	(119.42,130.03) --
	(119.35,130.91) --
	(119.27,131.80) --
	(119.17,132.68) --
	(119.05,133.56) --
	(118.92,134.44) --
	(118.77,135.32) --
	(118.61,136.20) --
	(118.43,137.07) --
	(118.24,137.93) --
	(118.03,138.80) --
	(117.81,139.66) --
	(117.57,140.52) --
	(117.31,141.37) --
	(117.05,142.22) --
	(116.76,143.06) --
	(116.46,143.90) --
	(116.15,144.73) --
	(115.83,145.56) --
	(115.48,146.38) --
	(115.13,147.19) --
	(114.76,148.00) --
	(114.38,148.81) --
	(113.98,149.60) --
	(113.57,150.39) --
	(113.14,151.17) --
	(112.70,151.94) --
	(112.25,152.71) --
	(111.78,153.47) --
	(111.30,154.22) --
	(110.81,154.96) --
	(110.31,155.69) --
	(109.79,156.41) --
	(109.26,157.13) --
	(108.72,157.83) --
	(108.16,158.53) --
	(107.59,159.21) --
	(107.02,159.89) --
	(106.42,160.55) --
	(105.82,161.21) --
	(105.21,161.85) --
	(104.58,162.48) --
	(103.95,163.10) --
	(103.30,163.71) --
	(102.64,164.31) --
	(101.98,164.90) --
	(101.30,165.48) --
	(100.61,166.04) --
	( 99.91,166.59) --
	( 99.21,167.13) --
	( 98.49,167.66) --
	( 97.77,168.18) --
	( 97.03,168.68) --
	( 96.29,169.17) --
	( 95.54,169.64) --
	( 94.78,170.10) --
	( 94.01,170.55) --
	( 93.24,170.99) --
	( 92.45,171.41) --
	( 91.66,171.82) --
	( 90.87,172.21) --
	( 90.06,172.59) --
	( 89.25,172.96) --
	( 88.43,173.31) --
	( 87.61,173.65) --
	( 86.78,173.97) --
	( 85.95,174.28) --
	( 85.11,174.58) --
	( 84.26,174.85) --
	( 83.42,175.12) --
	( 82.56,175.37) --
	( 81.70,175.60) --
	( 80.84,175.82) --
	( 79.98,176.03) --
	( 79.11,176.22) --
	( 78.24,176.39) --
	( 77.36,176.55) --
	( 76.48,176.69) --
	( 75.60,176.82) --
	( 74.72,176.93) --
	( 73.84,177.03) --
	( 72.95,177.11) --
	( 72.06,177.18) --
	( 71.18,177.23) --
	( 70.29,177.26) --
	( 69.40,177.28) --
	( 68.51,177.29) --
	( 67.62,177.27) --
	( 66.73,177.25) --
	( 65.84,177.20) --
	( 64.96,177.15) --
	( 64.07,177.07) --
	( 63.18,176.98) --
	( 62.30,176.88) --
	( 61.42,176.76) --
	( 60.54,176.62) --
	( 59.66,176.47) --
	( 58.79,176.31) --
	( 57.92,176.12) --
	( 57.05,175.93) --
	( 56.19,175.71) --
	( 55.33,175.49) --
	( 54.47,175.25) --
	( 53.62,174.99) --
	( 52.78,174.72) --
	( 51.93,174.43) --
	( 51.10,174.13) --
	( 50.27,173.81) --
	( 49.44,173.48) --
	( 48.62,173.14) --
	( 47.81,172.78) --
	( 47.00,172.41) --
	( 46.20,172.02) --
	( 45.41,171.62) --
	( 44.62,171.20) --
	( 43.84,170.77) --
	( 43.07,170.33) --
	( 42.30,169.87) --
	( 41.55,169.41) --
	( 40.80,168.92) --
	( 40.06,168.43) --
	( 39.33,167.92) --
	( 38.61,167.40) --
	( 37.90,166.87) --
	( 37.20,166.32) --
	( 36.51,165.76) --
	( 35.82,165.19) --
	( 35.15,164.61) --
	( 34.49,164.02) --
	( 33.84,163.41) --
	( 33.20,162.79) --
	( 32.57,162.17) --
	( 31.95,161.53) --
	( 31.34,160.88) --
	( 30.74,160.22) --
	( 30.16,159.55) --
	( 29.58,158.87) --
	( 29.02,158.18) --
	( 28.47,157.48) --
	( 27.94,156.77) --
	( 27.41,156.05) --
	( 26.90,155.32) --
	( 26.40,154.59) --
	( 25.92,153.84) --
	( 25.45,153.09) --
	( 24.99,152.33) --
	( 24.54,151.56) --
	( 24.11,150.78) --
	( 23.69,150.00) --
	( 23.29,149.20) --
	( 22.89,148.41) --
	( 22.52,147.60) --
	( 22.15,146.79) --
	( 21.81,145.97) --
	( 21.47,145.15) --
	( 21.15,144.32) --
	( 20.85,143.48) --
	( 20.56,142.64) --
	( 20.28,141.79) --
	( 20.02,140.94) --
	( 19.78,140.09) --
	( 19.54,139.23) --
	( 19.33,138.37) --
	( 19.13,137.50) --
	( 18.94,136.63) --
	( 18.77,135.76) --
	( 18.62,134.88) --
	( 18.48,134.00) --
	( 18.35,133.12) --
	( 18.25,132.24) --
	( 18.15,131.36) --
	( 18.07,130.47) --
	( 18.01,129.58) --
	( 17.97,128.70) --
	( 17.93,127.81) --
	( 17.92,126.92) --
	( 17.92,126.03) --
	( 17.93,125.14) --
	( 17.97,124.25) --
	( 18.01,123.36) --
	( 18.07,122.47) --
	( 18.15,121.59) --
	( 18.25,120.70) --
	( 18.35,119.82) --
	( 18.48,118.94) --
	( 18.62,118.06) --
	( 18.77,117.19) --
	( 18.94,116.31) --
	( 19.13,115.44) --
	( 19.33,114.58) --
	( 19.54,113.71) --
	( 19.78,112.86) --
	( 20.02,112.00) --
	( 20.28,111.15) --
	( 20.56,110.31) --
	( 20.85,109.46) --
	( 21.15,108.63) --
	( 21.47,107.80) --
	( 21.81,106.98) --
	( 22.15,106.16) --
	( 22.52,105.35) --
	( 22.89,104.54) --
	( 23.29,103.74) --
	( 23.69,102.95) --
	( 24.11,102.16) --
	( 24.54,101.39) --
	( 24.99,100.62) --
	( 25.45, 99.86) --
	( 25.92, 99.10) --
	( 26.40, 98.36) --
	( 26.90, 97.62) --
	( 27.41, 96.89) --
	( 27.94, 96.17) --
	( 28.47, 95.47) --
	( 29.02, 94.77) --
	( 29.58, 94.08) --
	( 30.16, 93.40) --
	( 30.74, 92.73) --
	( 31.34, 92.07) --
	( 31.95, 91.42) --
	( 32.57, 90.78) --
	( 33.20, 90.15) --
	( 33.84, 89.53) --
	( 34.49, 88.93) --
	( 35.15, 88.34) --
	( 35.82, 87.75) --
	( 36.51, 87.18) --
	( 37.20, 86.63) --
	( 37.90, 86.08) --
	( 38.61, 85.55) --
	( 39.33, 85.03) --
	( 40.06, 84.52) --
	( 40.80, 84.02) --
	( 41.55, 83.54) --
	( 42.30, 83.07) --
	( 43.07, 82.61) --
	( 43.84, 82.17) --
	( 44.62, 81.74) --
	( 45.41, 81.33) --
	( 46.20, 80.93) --
	( 47.00, 80.54) --
	( 47.81, 80.17) --
	( 48.62, 79.81) --
	( 49.44, 79.46) --
	( 50.27, 79.13) --
	( 51.10, 78.82) --
	( 51.93, 78.51) --
	( 52.78, 78.23) --
	( 53.62, 77.96) --
	( 54.47, 77.70) --
	( 55.33, 77.46) --
	( 56.19, 77.23) --
	( 57.05, 77.02) --
	( 57.92, 76.82) --
	( 58.79, 76.64) --
	( 59.66, 76.47) --
	( 60.54, 76.32) --
	( 61.42, 76.19) --
	( 62.30, 76.07) --
	( 63.18, 75.96) --
	( 64.07, 75.87) --
	( 64.96, 75.80) --
	( 65.84, 75.74) --
	( 66.73, 75.70) --
	( 67.62, 75.67) --
	( 68.51, 75.66) --
	( 69.40, 75.66) --
	( 70.29, 75.68) --
	( 71.18, 75.72) --
	( 72.06, 75.77) --
	( 72.95, 75.83) --
	( 73.84, 75.92) --
	( 74.72, 76.01) --
	( 75.60, 76.12) --
	( 76.48, 76.25) --
	( 77.36, 76.40) --
	( 78.24, 76.55) --
	( 79.11, 76.73) --
	( 79.98, 76.92) --
	( 80.84, 77.12) --
	( 81.70, 77.34) --
	( 82.56, 77.58) --
	( 83.42, 77.83) --
	( 84.26, 78.09) --
	( 85.11, 78.37) --
	( 85.95, 78.66) --
	( 86.78, 78.97) --
	( 87.61, 79.30) --
	( 88.43, 79.63) --
	( 89.25, 79.99) --
	( 90.06, 80.35) --
	( 90.87, 80.73) --
	( 91.66, 81.13) --
	( 92.45, 81.53) --
	( 93.24, 81.96) --
	( 94.01, 82.39) --
	( 94.78, 82.84) --
	( 95.54, 83.30) --
	( 96.29, 83.78) --
	( 97.03, 84.27) --
	( 97.77, 84.77) --
	( 98.49, 85.28) --
	( 99.21, 85.81) --
	( 99.91, 86.35) --
	(100.61, 86.90) --
	(101.30, 87.47) --
	(101.98, 88.04) --
	(102.64, 88.63) --
	(103.30, 89.23) --
	(103.95, 89.84) --
	(104.58, 90.46) --
	(105.21, 91.10) --
	(105.82, 91.74) --
	(106.42, 92.39) --
	(107.02, 93.06) --
	(107.59, 93.73) --
	(108.16, 94.42) --
	(108.72, 95.11) --
	(109.26, 95.82) --
	(109.79, 96.53) --
	(110.31, 97.26) --
	(110.81, 97.99) --
	(111.30, 98.73) --
	(111.78, 99.48) --
	(112.25,100.24) --
	(112.70,101.00) --
	(113.14,101.77) --
	(113.57,102.56) --
	(113.98,103.34) --
	(114.38,104.14) --
	(114.76,104.94) --
	(115.13,105.75) --
	(115.48,106.57) --
	(115.83,107.39) --
	(116.15,108.21) --
	(116.46,109.05) --
	(116.76,109.88) --
	(117.05,110.73) --
	(117.31,111.58) --
	(117.57,112.43) --
	(117.81,113.28) --
	(118.03,114.15) --
	(118.24,115.01) --
	(118.43,115.88) --
	(118.61,116.75) --
	(118.77,117.62) --
	(118.92,118.50) --
	(119.05,119.38) --
	(119.17,120.26) --
	(119.27,121.15) --
	(119.35,122.03) --
	(119.42,122.92) --
	(119.48,123.81) --
	(119.52,124.69) --
	(119.54,125.58) --
	(119.55,126.47);

\path[fill=fillColor] ( 25.43,201.48) circle (  2.25);
\definecolor[named]{drawColor}{rgb}{0.00,0.00,1.00}

\path[draw=drawColor,line width= 1.2pt,line join=round,line cap=round] ( 71.62,201.48) --
	( 71.61,202.29) --
	( 71.59,203.10) --
	( 71.56,203.90) --
	( 71.51,204.71) --
	( 71.44,205.52) --
	( 71.37,206.32) --
	( 71.27,207.13) --
	( 71.17,207.93) --
	( 71.05,208.73) --
	( 70.92,209.52) --
	( 70.77,210.32) --
	( 70.61,211.11) --
	( 70.43,211.90) --
	( 70.24,212.69) --
	( 70.04,213.47) --
	( 69.82,214.25) --
	( 69.59,215.02) --
	( 69.35,215.79) --
	( 69.09,216.56) --
	( 68.82,217.32) --
	( 68.54,218.08) --
	( 68.24,218.83) --
	( 67.93,219.58) --
	( 67.61,220.32) --
	( 67.27,221.05) --
	( 66.92,221.78) --
	( 66.56,222.51) --
	( 66.18,223.22) --
	( 65.80,223.93) --
	( 65.40,224.64) --
	( 64.99,225.33) --
	( 64.56,226.02) --
	( 64.13,226.70) --
	( 63.68,227.38) --
	( 63.22,228.04) --
	( 62.75,228.70) --
	( 62.27,229.35) --
	( 61.78,229.99) --
	( 61.27,230.62) --
	( 60.76,231.24) --
	( 60.23,231.86) --
	( 59.69,232.46) --
	( 59.15,233.06) --
	( 58.59,233.64) --
	( 58.02,234.22) --
	( 57.44,234.78) --
	( 56.85,235.34) --
	( 56.26,235.88) --
	( 55.65,236.42) --
	( 55.03,236.94) --
	( 54.41,237.45) --
	( 53.77,237.95) --
	( 53.13,238.44) --
	( 52.48,238.92) --
	( 51.82,239.39) --
	( 51.15,239.85) --
	( 50.48,240.29) --
	( 49.80,240.72) --
	( 49.10,241.14) --
	( 48.41,241.55) --
	( 47.70,241.95) --
	( 46.99,242.33) --
	( 46.27,242.70) --
	( 45.55,243.06) --
	( 44.82,243.41) --
	( 44.08,243.74) --
	( 43.34,244.06) --
	( 42.59,244.37) --
	( 41.84,244.66) --
	( 41.08,244.94) --
	( 40.31,245.21) --
	( 39.55,245.46) --
	( 38.78,245.70) --
	( 38.00,245.93) --
	( 37.22,246.14) --
	( 36.44,246.34) --
	( 35.65,246.53) --
	( 34.86,246.70) --
	( 34.07,246.86) --
	( 33.27,247.00) --
	( 32.47,247.13) --
	( 31.67,247.25) --
	( 30.87,247.35) --
	( 30.07,247.44) --
	( 29.26,247.52) --
	( 28.46,247.58) --
	( 27.65,247.62) --
	( 26.84,247.65) --
	( 26.03,247.67) --
	( 25.22,247.67) --
	( 24.42,247.66) --
	( 23.61,247.64) --
	( 22.80,247.60) --
	( 21.99,247.55) --
	( 21.19,247.48) --
	( 20.38,247.40) --
	( 19.58,247.30) --
	( 18.78,247.19) --
	( 17.98,247.07) --
	( 17.18,246.93) --
	( 16.39,246.78) --
	( 15.60,246.62) --
	( 14.81,246.44) --
	( 14.02,246.25) --
	( 13.24,246.04) --
	( 12.46,245.82) --
	( 11.69,245.59) --
	( 10.92,245.34) --
	( 10.16,245.08) --
	(  9.39,244.80) --
	(  8.64,244.52) --
	(  7.89,244.22) --
	(  7.14,243.90) --
	(  6.40,243.58) --
	(  5.67,243.24) --
	(  4.94,242.88) --
	(  4.22,242.52) --
	(  3.51,242.14) --
	(  2.80,241.75) --
	(  2.10,241.35) --
	(  1.40,240.94) --
	(  0.71,240.51) --
	(  0.04,240.07) --
	(  0.00,240.05);

\path[draw=drawColor,line width= 1.2pt,line join=round,line cap=round] (  0.00,162.91) --
	(  0.04,162.89) --
	(  0.71,162.45) --
	(  1.40,162.02) --
	(  2.10,161.61) --
	(  2.80,161.21) --
	(  3.51,160.82) --
	(  4.22,160.44) --
	(  4.94,160.08) --
	(  5.67,159.72) --
	(  6.40,159.38) --
	(  7.14,159.06) --
	(  7.89,158.74) --
	(  8.64,158.44) --
	(  9.39,158.16) --
	( 10.16,157.88) --
	( 10.92,157.62) --
	( 11.69,157.37) --
	( 12.46,157.14) --
	( 13.24,156.92) --
	( 14.02,156.71) --
	( 14.81,156.52) --
	( 15.60,156.34) --
	( 16.39,156.18) --
	( 17.18,156.03) --
	( 17.98,155.89) --
	( 18.78,155.77) --
	( 19.58,155.66) --
	( 20.38,155.56) --
	( 21.19,155.48) --
	( 21.99,155.41) --
	( 22.80,155.36) --
	( 23.61,155.32) --
	( 24.42,155.30) --
	( 25.22,155.29) --
	( 26.03,155.29) --
	( 26.84,155.31) --
	( 27.65,155.34) --
	( 28.46,155.38) --
	( 29.26,155.44) --
	( 30.07,155.52) --
	( 30.87,155.61) --
	( 31.67,155.71) --
	( 32.47,155.83) --
	( 33.27,155.96) --
	( 34.07,156.10) --
	( 34.86,156.26) --
	( 35.65,156.43) --
	( 36.44,156.62) --
	( 37.22,156.82) --
	( 38.00,157.03) --
	( 38.78,157.26) --
	( 39.55,157.50) --
	( 40.31,157.75) --
	( 41.08,158.02) --
	( 41.84,158.30) --
	( 42.59,158.59) --
	( 43.34,158.90) --
	( 44.08,159.22) --
	( 44.82,159.55) --
	( 45.55,159.90) --
	( 46.27,160.26) --
	( 46.99,160.63) --
	( 47.70,161.01) --
	( 48.41,161.41) --
	( 49.10,161.82) --
	( 49.80,162.24) --
	( 50.48,162.67) --
	( 51.15,163.11) --
	( 51.82,163.57) --
	( 52.48,164.04) --
	( 53.13,164.52) --
	( 53.77,165.01) --
	( 54.41,165.51) --
	( 55.03,166.02) --
	( 55.65,166.54) --
	( 56.26,167.08) --
	( 56.85,167.62) --
	( 57.44,168.18) --
	( 58.02,168.74) --
	( 58.59,169.32) --
	( 59.15,169.90) --
	( 59.69,170.50) --
	( 60.23,171.10) --
	( 60.76,171.72) --
	( 61.27,172.34) --
	( 61.78,172.97) --
	( 62.27,173.61) --
	( 62.75,174.26) --
	( 63.22,174.92) --
	( 63.68,175.58) --
	( 64.13,176.26) --
	( 64.56,176.94) --
	( 64.99,177.63) --
	( 65.40,178.32) --
	( 65.80,179.03) --
	( 66.18,179.74) --
	( 66.56,180.45) --
	( 66.92,181.18) --
	( 67.27,181.91) --
	( 67.61,182.64) --
	( 67.93,183.38) --
	( 68.24,184.13) --
	( 68.54,184.88) --
	( 68.82,185.64) --
	( 69.09,186.40) --
	( 69.35,187.17) --
	( 69.59,187.94) --
	( 69.82,188.71) --
	( 70.04,189.49) --
	( 70.24,190.27) --
	( 70.43,191.06) --
	( 70.61,191.85) --
	( 70.77,192.64) --
	( 70.92,193.44) --
	( 71.05,194.23) --
	( 71.17,195.03) --
	( 71.27,195.83) --
	( 71.37,196.64) --
	( 71.44,197.44) --
	( 71.51,198.25) --
	( 71.56,199.06) --
	( 71.59,199.86) --
	( 71.61,200.67) --
	( 71.62,201.48);
\definecolor[named]{drawColor}{rgb}{0.00,0.00,0.00}

\path[draw=drawColor,line width= 0.4pt,dash pattern=on 1pt off 3pt ,line join=round,line cap=round] ( 76.24,201.48) --
	( 76.23,202.37) --
	( 76.21,203.26) --
	( 76.17,204.15) --
	( 76.12,205.03) --
	( 76.05,205.92) --
	( 75.96,206.81) --
	( 75.86,207.69) --
	( 75.74,208.57) --
	( 75.61,209.45) --
	( 75.46,210.33) --
	( 75.30,211.20) --
	( 75.12,212.07) --
	( 74.93,212.94) --
	( 74.72,213.81) --
	( 74.50,214.67) --
	( 74.26,215.52) --
	( 74.01,216.38) --
	( 73.74,217.22) --
	( 73.46,218.07) --
	( 73.16,218.91) --
	( 72.85,219.74) --
	( 72.52,220.57) --
	( 72.18,221.39) --
	( 71.82,222.20) --
	( 71.45,223.01) --
	( 71.07,223.81) --
	( 70.67,224.61) --
	( 70.26,225.40) --
	( 69.83,226.18) --
	( 69.40,226.95) --
	( 68.94,227.72) --
	( 68.48,228.47) --
	( 68.00,229.22) --
	( 67.51,229.96) --
	( 67.00,230.70) --
	( 66.48,231.42) --
	( 65.95,232.13) --
	( 65.41,232.84) --
	( 64.86,233.53) --
	( 64.29,234.22) --
	( 63.71,234.89) --
	( 63.12,235.56) --
	( 62.52,236.21) --
	( 61.90,236.86) --
	( 61.28,237.49) --
	( 60.64,238.11) --
	( 60.00,238.72) --
	( 59.34,239.32) --
	( 58.67,239.91) --
	( 57.99,240.49) --
	( 57.31,241.05) --
	( 56.61,241.60) --
	( 55.90,242.14) --
	( 55.19,242.67) --
	( 54.46,243.18) --
	( 53.73,243.68) --
	( 52.98,244.17) --
	( 52.23,244.65) --
	( 51.47,245.11) --
	( 50.70,245.56) --
	( 49.93,246.00) --
	( 49.15,246.42) --
	( 48.36,246.83) --
	( 47.56,247.22) --
	( 46.76,247.60) --
	( 45.95,247.97) --
	( 45.13,248.32) --
	( 44.31,248.66) --
	( 43.48,248.98) --
	( 42.64,249.29) --
	( 41.80,249.58) --
	( 40.96,249.86) --
	( 40.11,250.13) --
	( 39.26,250.38) --
	( 38.40,250.61) --
	( 37.54,250.83) --
	( 36.67,251.03) --
	( 35.80,251.22) --
	( 34.93,251.40) --
	( 34.06,251.56) --
	( 33.18,251.70) --
	( 32.30,251.83) --
	( 31.42,251.94) --
	( 30.53,252.04) --
	( 29.65,252.12) --
	( 28.76,252.19) --
	( 27.87,252.24) --
	( 26.98,252.27) --
	( 26.09,252.29) --
	( 25.20,252.29) --
	( 24.31,252.28) --
	( 23.43,252.26) --
	( 22.54,252.21) --
	( 21.65,252.15) --
	( 20.76,252.08) --
	( 19.88,251.99) --
	( 19.00,251.89) --
	( 18.11,251.77) --
	( 17.24,251.63) --
	( 16.36,251.48) --
	( 15.49,251.31) --
	( 14.61,251.13) --
	( 13.75,250.93) --
	( 12.88,250.72) --
	( 12.02,250.50) --
	( 11.17,250.25) --
	( 10.32,250.00) --
	(  9.47,249.72) --
	(  8.63,249.44) --
	(  7.79,249.14) --
	(  6.96,248.82) --
	(  6.13,248.49) --
	(  5.31,248.15) --
	(  4.50,247.79) --
	(  3.69,247.41) --
	(  2.89,247.03) --
	(  2.10,246.62) --
	(  1.31,246.21) --
	(  0.53,245.78) --
	(  0.00,245.47);

\path[draw=drawColor,line width= 0.4pt,dash pattern=on 1pt off 3pt ,line join=round,line cap=round] (  0.00,157.49) --
	(  0.53,157.18) --
	(  1.31,156.75) --
	(  2.10,156.34) --
	(  2.89,155.93) --
	(  3.69,155.55) --
	(  4.50,155.17) --
	(  5.31,154.81) --
	(  6.13,154.47) --
	(  6.96,154.14) --
	(  7.79,153.82) --
	(  8.63,153.52) --
	(  9.47,153.24) --
	( 10.32,152.96) --
	( 11.17,152.71) --
	( 12.02,152.46) --
	( 12.88,152.24) --
	( 13.75,152.03) --
	( 14.61,151.83) --
	( 15.49,151.65) --
	( 16.36,151.48) --
	( 17.24,151.33) --
	( 18.11,151.19) --
	( 19.00,151.07) --
	( 19.88,150.97) --
	( 20.76,150.88) --
	( 21.65,150.81) --
	( 22.54,150.75) --
	( 23.43,150.70) --
	( 24.31,150.68) --
	( 25.20,150.67) --
	( 26.09,150.67) --
	( 26.98,150.69) --
	( 27.87,150.72) --
	( 28.76,150.77) --
	( 29.65,150.84) --
	( 30.53,150.92) --
	( 31.42,151.02) --
	( 32.30,151.13) --
	( 33.18,151.26) --
	( 34.06,151.40) --
	( 34.93,151.56) --
	( 35.80,151.74) --
	( 36.67,151.93) --
	( 37.54,152.13) --
	( 38.40,152.35) --
	( 39.26,152.58) --
	( 40.11,152.83) --
	( 40.96,153.10) --
	( 41.80,153.38) --
	( 42.64,153.67) --
	( 43.48,153.98) --
	( 44.31,154.30) --
	( 45.13,154.64) --
	( 45.95,154.99) --
	( 46.76,155.36) --
	( 47.56,155.74) --
	( 48.36,156.13) --
	( 49.15,156.54) --
	( 49.93,156.96) --
	( 50.70,157.40) --
	( 51.47,157.85) --
	( 52.23,158.31) --
	( 52.98,158.79) --
	( 53.73,159.28) --
	( 54.46,159.78) --
	( 55.19,160.29) --
	( 55.90,160.82) --
	( 56.61,161.36) --
	( 57.31,161.91) --
	( 57.99,162.47) --
	( 58.67,163.05) --
	( 59.34,163.64) --
	( 60.00,164.24) --
	( 60.64,164.85) --
	( 61.28,165.47) --
	( 61.90,166.10) --
	( 62.52,166.75) --
	( 63.12,167.40) --
	( 63.71,168.07) --
	( 64.29,168.74) --
	( 64.86,169.43) --
	( 65.41,170.12) --
	( 65.95,170.83) --
	( 66.48,171.54) --
	( 67.00,172.26) --
	( 67.51,173.00) --
	( 68.00,173.74) --
	( 68.48,174.49) --
	( 68.94,175.24) --
	( 69.40,176.01) --
	( 69.83,176.78) --
	( 70.26,177.56) --
	( 70.67,178.35) --
	( 71.07,179.15) --
	( 71.45,179.95) --
	( 71.82,180.76) --
	( 72.18,181.57) --
	( 72.52,182.39) --
	( 72.85,183.22) --
	( 73.16,184.05) --
	( 73.46,184.89) --
	( 73.74,185.74) --
	( 74.01,186.58) --
	( 74.26,187.44) --
	( 74.50,188.29) --
	( 74.72,189.15) --
	( 74.93,190.02) --
	( 75.12,190.89) --
	( 75.30,191.76) --
	( 75.46,192.63) --
	( 75.61,193.51) --
	( 75.74,194.39) --
	( 75.86,195.27) --
	( 75.96,196.15) --
	( 76.05,197.04) --
	( 76.12,197.93) --
	( 76.17,198.81) --
	( 76.21,199.70) --
	( 76.23,200.59) --
	( 76.24,201.48);
\definecolor[named]{drawColor}{rgb}{0.00,0.00,1.00}

\path[draw=drawColor,line width= 1.2pt,line join=round,line cap=round] (  0.00,233.89) --
	(  0.03,233.91) --
	(  0.77,234.23) --
	(  1.51,234.56) --
	(  2.24,234.90) --
	(  2.97,235.26) --
	(  3.68,235.63) --
	(  4.40,236.02) --
	(  5.10,236.41) --
	(  5.80,236.82) --
	(  6.49,237.24) --
	(  7.17,237.68) --
	(  7.85,238.12) --
	(  8.52,238.58) --
	(  9.18,239.04) --
	(  9.83,239.52) --
	( 10.47,240.01) --
	( 11.10,240.52) --
	( 11.73,241.03) --
	( 12.34,241.55) --
	( 12.95,242.09) --
	( 13.55,242.63) --
	( 14.14,243.19) --
	( 14.71,243.75) --
	( 15.28,244.33) --
	( 15.84,244.91) --
	( 16.39,245.51) --
	( 16.92,246.11) --
	( 17.45,246.73) --
	( 17.97,247.35) --
	( 18.47,247.98) --
	( 18.96,248.62) --
	( 19.45,249.27) --
	( 19.92,249.93) --
	( 20.38,250.59) --
	( 20.82,251.27) --
	( 21.26,251.95) --
	( 21.68,252.64) --
	( 21.86,252.94);
\definecolor[named]{drawColor}{rgb}{0.00,0.00,0.00}

\path[draw=drawColor,line width= 0.4pt,dash pattern=on 1pt off 3pt ,line join=round,line cap=round] (  0.00,228.92) --
	(  0.17,228.99) --
	(  1.00,229.31) --
	(  1.82,229.65) --
	(  2.64,230.00) --
	(  3.45,230.37) --
	(  4.25,230.75) --
	(  5.05,231.14) --
	(  5.84,231.55) --
	(  6.62,231.97) --
	(  7.40,232.41) --
	(  8.17,232.86) --
	(  8.93,233.32) --
	(  9.68,233.79) --
	( 10.42,234.28) --
	( 11.16,234.78) --
	( 11.88,235.30) --
	( 12.60,235.83) --
	( 13.30,236.37) --
	( 14.00,236.92) --
	( 14.69,237.48) --
	( 15.37,238.06) --
	( 16.03,238.65) --
	( 16.69,239.25) --
	( 17.34,239.86) --
	( 17.97,240.48) --
	( 18.60,241.11) --
	( 19.21,241.75) --
	( 19.81,242.41) --
	( 20.40,243.07) --
	( 20.98,243.75) --
	( 21.55,244.43) --
	( 22.10,245.13) --
	( 22.65,245.83) --
	( 23.18,246.55) --
	( 23.70,247.27) --
	( 24.20,248.00) --
	( 24.69,248.74) --
	( 25.17,249.49) --
	( 25.64,250.25) --
	( 26.09,251.02) --
	( 26.53,251.79) --
	( 26.95,252.57) --
	( 27.15,252.94);
\definecolor[named]{drawColor}{rgb}{0.00,0.00,1.00}

\path[draw=drawColor,line width= 1.2pt,line join=round,line cap=round] (252.83,  0.00) --
	(252.64,  0.31) --
	(252.22,  1.00) --
	(251.79,  1.68) --
	(251.34,  2.35) --
	(250.88,  3.02) --
	(250.41,  3.68) --
	(249.93,  4.32) --
	(249.43,  4.97) --
	(248.93,  5.60) --
	(248.41,  6.22) --
	(247.89,  6.83) --
	(247.35,  7.44) --
	(246.80,  8.03) --
	(246.24,  8.62) --
	(245.68,  9.19) --
	(245.10,  9.76) --
	(244.51, 10.31) --
	(243.91, 10.86) --
	(243.31, 11.39) --
	(242.69, 11.92) --
	(242.07, 12.43) --
	(241.43, 12.93) --
	(240.79, 13.42) --
	(240.14, 13.90) --
	(239.48, 14.37) --
	(238.81, 14.83) --
	(238.14, 15.27) --
	(237.45, 15.70) --
	(236.76, 16.12) --
	(236.06, 16.53) --
	(235.36, 16.93) --
	(234.65, 17.31) --
	(233.93, 17.68) --
	(233.20, 18.04) --
	(232.47, 18.39) --
	(231.74, 18.72) --
	(230.99, 19.04) --
	(230.25, 19.35) --
	(229.49, 19.64) --
	(228.74, 19.92) --
	(227.97, 20.19) --
	(227.20, 20.44) --
	(226.43, 20.68) --
	(225.66, 20.91) --
	(224.88, 21.12) --
	(224.09, 21.32) --
	(223.31, 21.51) --
	(222.52, 21.68) --
	(221.72, 21.84) --
	(220.93, 21.98) --
	(220.13, 22.11) --
	(219.33, 22.23) --
	(218.53, 22.33) --
	(217.72, 22.42) --
	(216.92, 22.49) --
	(216.11, 22.55) --
	(215.31, 22.60) --
	(214.50, 22.63) --
	(213.69, 22.65) --
	(212.88, 22.65) --
	(212.07, 22.64) --
	(211.27, 22.62) --
	(210.46, 22.58) --
	(209.65, 22.52) --
	(208.85, 22.46) --
	(208.04, 22.38) --
	(207.24, 22.28) --
	(206.44, 22.17) --
	(205.64, 22.05) --
	(204.84, 21.91) --
	(204.05, 21.76) --
	(203.26, 21.59) --
	(202.47, 21.42) --
	(201.68, 21.22) --
	(200.90, 21.02) --
	(200.12, 20.80) --
	(199.35, 20.56) --
	(198.58, 20.32) --
	(197.81, 20.06) --
	(197.05, 19.78) --
	(196.30, 19.49) --
	(195.55, 19.19) --
	(194.80, 18.88) --
	(194.06, 18.55) --
	(193.33, 18.21) --
	(192.60, 17.86) --
	(191.88, 17.50) --
	(191.16, 17.12) --
	(190.45, 16.73) --
	(189.75, 16.33) --
	(189.06, 15.91) --
	(188.37, 15.49) --
	(187.69, 15.05) --
	(187.02, 14.60) --
	(186.36, 14.14) --
	(185.70, 13.66) --
	(185.06, 13.18) --
	(184.42, 12.68) --
	(183.79, 12.17) --
	(183.17, 11.66) --
	(182.56, 11.13) --
	(181.95, 10.59) --
	(181.36, 10.04) --
	(180.78,  9.48) --
	(180.21,  8.91) --
	(179.64,  8.33) --
	(179.09,  7.74) --
	(178.55,  7.14) --
	(178.02,  6.53) --
	(177.50,  5.91) --
	(176.99,  5.28) --
	(176.49,  4.65) --
	(176.00,  4.00) --
	(175.52,  3.35) --
	(175.06,  2.69) --
	(174.60,  2.02) --
	(174.16,  1.34) --
	(173.73,  0.65) --
	(173.34,  0.00);
\definecolor[named]{drawColor}{rgb}{0.00,0.00,0.00}

\path[draw=drawColor,line width= 0.4pt,dash pattern=on 1pt off 3pt ,line join=round,line cap=round] (252.94,  7.97) --
	(252.51,  8.51) --
	(251.95,  9.20) --
	(251.37,  9.87) --
	(250.78, 10.54) --
	(250.17, 11.19) --
	(249.56, 11.83) --
	(248.94, 12.47) --
	(248.30, 13.09) --
	(247.65, 13.70) --
	(247.00, 14.30) --
	(246.33, 14.89) --
	(245.65, 15.46) --
	(244.96, 16.03) --
	(244.27, 16.58) --
	(243.56, 17.12) --
	(242.84, 17.65) --
	(242.12, 18.16) --
	(241.38, 18.66) --
	(240.64, 19.15) --
	(239.89, 19.63) --
	(239.13, 20.09) --
	(238.36, 20.54) --
	(237.59, 20.97) --
	(236.80, 21.40) --
	(236.01, 21.80) --
	(235.22, 22.20) --
	(234.41, 22.58) --
	(233.60, 22.94) --
	(232.79, 23.30) --
	(231.96, 23.63) --
	(231.13, 23.96) --
	(230.30, 24.27) --
	(229.46, 24.56) --
	(228.62, 24.84) --
	(227.77, 25.10) --
	(226.91, 25.35) --
	(226.06, 25.59) --
	(225.19, 25.81) --
	(224.33, 26.01) --
	(223.46, 26.20) --
	(222.59, 26.38) --
	(221.71, 26.53) --
	(220.84, 26.68) --
	(219.96, 26.81) --
	(219.07, 26.92) --
	(218.19, 27.01) --
	(217.30, 27.10) --
	(216.42, 27.16) --
	(215.53, 27.21) --
	(214.64, 27.25) --
	(213.75, 27.27) --
	(212.86, 27.27) --
	(211.97, 27.26) --
	(211.08, 27.23) --
	(210.19, 27.19) --
	(209.31, 27.13) --
	(208.42, 27.06) --
	(207.54, 26.97) --
	(206.65, 26.86) --
	(205.77, 26.74) --
	(204.89, 26.61) --
	(204.02, 26.46) --
	(203.14, 26.29) --
	(202.27, 26.11) --
	(201.41, 25.91) --
	(200.54, 25.70) --
	(199.68, 25.47) --
	(198.83, 25.23) --
	(197.97, 24.97) --
	(197.13, 24.70) --
	(196.29, 24.42) --
	(195.45, 24.11) --
	(194.62, 23.80) --
	(193.79, 23.47) --
	(192.97, 23.12) --
	(192.16, 22.76) --
	(191.35, 22.39) --
	(190.55, 22.00) --
	(189.76, 21.60) --
	(188.97, 21.19) --
	(188.19, 20.76) --
	(187.42, 20.32) --
	(186.66, 19.86) --
	(185.90, 19.39) --
	(185.15, 18.91) --
	(184.42, 18.41) --
	(183.69, 17.90) --
	(182.96, 17.38) --
	(182.25, 16.85) --
	(181.55, 16.30) --
	(180.86, 15.75) --
	(180.18, 15.18) --
	(179.50, 14.59) --
	(178.84, 14.00) --
	(178.19, 13.40) --
	(177.55, 12.78) --
	(176.92, 12.15) --
	(176.30, 11.51) --
	(175.69, 10.86) --
	(175.09, 10.20) --
	(174.51,  9.53) --
	(173.94,  8.85) --
	(173.38,  8.16) --
	(172.83,  7.46) --
	(172.29,  6.76) --
	(171.77,  6.04) --
	(171.25,  5.31) --
	(170.76,  4.57) --
	(170.27,  3.83) --
	(169.80,  3.07) --
	(169.34,  2.31) --
	(168.89,  1.54) --
	(168.46,  0.77) --
	(168.05,  0.00);

\path[fill=fillColor] (169.78, 51.47) circle (  2.25);
\definecolor[named]{drawColor}{rgb}{0.00,0.00,1.00}

\path[draw=drawColor,line width= 1.2pt,line join=round,line cap=round] (215.97, 51.47) --
	(215.97, 52.27) --
	(215.94, 53.08) --
	(215.91, 53.89) --
	(215.86, 54.70) --
	(215.80, 55.50) --
	(215.72, 56.31) --
	(215.63, 57.11) --
	(215.52, 57.91) --
	(215.40, 58.71) --
	(215.27, 59.51) --
	(215.12, 60.30) --
	(214.96, 61.10) --
	(214.78, 61.89) --
	(214.59, 62.67) --
	(214.39, 63.45) --
	(214.17, 64.23) --
	(213.94, 65.01) --
	(213.70, 65.78) --
	(213.44, 66.54) --
	(213.17, 67.31) --
	(212.89, 68.06) --
	(212.59, 68.82) --
	(212.28, 69.56) --
	(211.96, 70.30) --
	(211.62, 71.04) --
	(211.27, 71.77) --
	(210.91, 72.49) --
	(210.54, 73.21) --
	(210.15, 73.92) --
	(209.75, 74.62) --
	(209.34, 75.32) --
	(208.92, 76.01) --
	(208.48, 76.69) --
	(208.03, 77.36) --
	(207.57, 78.03) --
	(207.10, 78.68) --
	(206.62, 79.33) --
	(206.13, 79.97) --
	(205.62, 80.60) --
	(205.11, 81.23) --
	(204.58, 81.84) --
	(204.04, 82.45) --
	(203.50, 83.04) --
	(202.94, 83.63) --
	(202.37, 84.20) --
	(201.79, 84.77) --
	(201.21, 85.32) --
	(200.61, 85.87) --
	(200.00, 86.40) --
	(199.39, 86.92) --
	(198.76, 87.44) --
	(198.13, 87.94) --
	(197.48, 88.43) --
	(196.83, 88.91) --
	(196.17, 89.38) --
	(195.51, 89.83) --
	(194.83, 90.28) --
	(194.15, 90.71) --
	(193.46, 91.13) --
	(192.76, 91.54) --
	(192.05, 91.93) --
	(191.34, 92.32) --
	(190.62, 92.69) --
	(189.90, 93.05) --
	(189.17, 93.39) --
	(188.43, 93.73) --
	(187.69, 94.05) --
	(186.94, 94.35) --
	(186.19, 94.65) --
	(185.43, 94.93) --
	(184.67, 95.19) --
	(183.90, 95.45) --
	(183.13, 95.69) --
	(182.35, 95.92) --
	(181.57, 96.13) --
	(180.79, 96.33) --
	(180.00, 96.51) --
	(179.21, 96.69) --
	(178.42, 96.84) --
	(177.62, 96.99) --
	(176.82, 97.12) --
	(176.02, 97.24) --
	(175.22, 97.34) --
	(174.42, 97.43) --
	(173.61, 97.50) --
	(172.81, 97.56) --
	(172.00, 97.61) --
	(171.19, 97.64) --
	(170.38, 97.66) --
	(169.58, 97.66) --
	(168.77, 97.65) --
	(167.96, 97.62) --
	(167.15, 97.59) --
	(166.35, 97.53) --
	(165.54, 97.47) --
	(164.74, 97.38) --
	(163.93, 97.29) --
	(163.13, 97.18) --
	(162.33, 97.06) --
	(161.54, 96.92) --
	(160.74, 96.77) --
	(159.95, 96.60) --
	(159.16, 96.42) --
	(158.38, 96.23) --
	(157.59, 96.02) --
	(156.82, 95.80) --
	(156.04, 95.57) --
	(155.27, 95.32) --
	(154.51, 95.06) --
	(153.75, 94.79) --
	(152.99, 94.50) --
	(152.24, 94.20) --
	(151.49, 93.89) --
	(150.76, 93.56) --
	(150.02, 93.22) --
	(149.29, 92.87) --
	(148.57, 92.51) --
	(147.86, 92.13) --
	(147.15, 91.74) --
	(146.45, 91.34) --
	(145.75, 90.92) --
	(145.07, 90.49) --
	(144.39, 90.06) --
	(143.72, 89.61) --
	(143.05, 89.14) --
	(142.40, 88.67) --
	(141.75, 88.19) --
	(141.11, 87.69) --
	(140.48, 87.18) --
	(139.86, 86.66) --
	(139.25, 86.14) --
	(138.65, 85.60) --
	(138.06, 85.05) --
	(137.47, 84.49) --
	(136.90, 83.91) --
	(136.34, 83.33) --
	(135.78, 82.74) --
	(135.24, 82.14) --
	(134.71, 81.54) --
	(134.19, 80.92) --
	(133.68, 80.29) --
	(133.18, 79.65) --
	(132.69, 79.01) --
	(132.22, 78.36) --
	(131.75, 77.69) --
	(131.30, 77.02) --
	(130.86, 76.35) --
	(130.43, 75.66) --
	(130.01, 74.97) --
	(129.60, 74.27) --
	(129.21, 73.56) --
	(128.83, 72.85) --
	(128.46, 72.13) --
	(128.11, 71.40) --
	(127.77, 70.67) --
	(127.44, 69.93) --
	(127.12, 69.19) --
	(126.82, 68.44) --
	(126.52, 67.69) --
	(126.25, 66.93) --
	(125.98, 66.16) --
	(125.73, 65.39) --
	(125.50, 64.62) --
	(125.27, 63.84) --
	(125.06, 63.06) --
	(124.87, 62.28) --
	(124.68, 61.49) --
	(124.52, 60.70) --
	(124.36, 59.91) --
	(124.22, 59.11) --
	(124.09, 58.31) --
	(123.98, 57.51) --
	(123.88, 56.71) --
	(123.80, 55.90) --
	(123.73, 55.10) --
	(123.67, 54.29) --
	(123.63, 53.49) --
	(123.60, 52.68) --
	(123.58, 51.87) --
	(123.58, 51.06) --
	(123.60, 50.25) --
	(123.63, 49.44) --
	(123.67, 48.64) --
	(123.73, 47.83) --
	(123.80, 47.03) --
	(123.88, 46.22) --
	(123.98, 45.42) --
	(124.09, 44.62) --
	(124.22, 43.82) --
	(124.36, 43.02) --
	(124.52, 42.23) --
	(124.68, 41.44) --
	(124.87, 40.65) --
	(125.06, 39.87) --
	(125.27, 39.09) --
	(125.50, 38.31) --
	(125.73, 37.54) --
	(125.98, 36.77) --
	(126.25, 36.00) --
	(126.52, 35.24) --
	(126.82, 34.49) --
	(127.12, 33.74) --
	(127.44, 33.00) --
	(127.77, 32.26) --
	(128.11, 31.53) --
	(128.46, 30.80) --
	(128.83, 30.08) --
	(129.21, 29.37) --
	(129.60, 28.66) --
	(130.01, 27.96) --
	(130.43, 27.27) --
	(130.86, 26.58) --
	(131.30, 25.91) --
	(131.75, 25.24) --
	(132.22, 24.57) --
	(132.69, 23.92) --
	(133.18, 23.28) --
	(133.68, 22.64) --
	(134.19, 22.01) --
	(134.71, 21.39) --
	(135.24, 20.79) --
	(135.78, 20.19) --
	(136.34, 19.60) --
	(136.90, 19.02) --
	(137.47, 18.44) --
	(138.06, 17.88) --
	(138.65, 17.33) --
	(139.25, 16.79) --
	(139.86, 16.27) --
	(140.48, 15.75) --
	(141.11, 15.24) --
	(141.75, 14.74) --
	(142.40, 14.26) --
	(143.05, 13.79) --
	(143.72, 13.32) --
	(144.39, 12.87) --
	(145.07, 12.44) --
	(145.75, 12.01) --
	(146.45, 11.59) --
	(147.15, 11.19) --
	(147.86, 10.80) --
	(148.57, 10.42) --
	(149.29, 10.06) --
	(150.02,  9.71) --
	(150.76,  9.37) --
	(151.49,  9.04) --
	(152.24,  8.73) --
	(152.99,  8.43) --
	(153.75,  8.14) --
	(154.51,  7.87) --
	(155.27,  7.61) --
	(156.04,  7.36) --
	(156.82,  7.13) --
	(157.59,  6.91) --
	(158.38,  6.70) --
	(159.16,  6.51) --
	(159.95,  6.33) --
	(160.74,  6.16) --
	(161.54,  6.01) --
	(162.33,  5.87) --
	(163.13,  5.75) --
	(163.93,  5.64) --
	(164.74,  5.55) --
	(165.54,  5.46) --
	(166.35,  5.40) --
	(167.15,  5.34) --
	(167.96,  5.31) --
	(168.77,  5.28) --
	(169.58,  5.27) --
	(170.38,  5.27) --
	(171.19,  5.29) --
	(172.00,  5.32) --
	(172.81,  5.37) --
	(173.61,  5.43) --
	(174.42,  5.50) --
	(175.22,  5.59) --
	(176.02,  5.69) --
	(176.82,  5.81) --
	(177.62,  5.94) --
	(178.42,  6.09) --
	(179.21,  6.24) --
	(180.00,  6.42) --
	(180.79,  6.60) --
	(181.57,  6.80) --
	(182.35,  7.01) --
	(183.13,  7.24) --
	(183.90,  7.48) --
	(184.67,  7.74) --
	(185.43,  8.00) --
	(186.19,  8.28) --
	(186.94,  8.58) --
	(187.69,  8.88) --
	(188.43,  9.20) --
	(189.17,  9.54) --
	(189.90,  9.88) --
	(190.62, 10.24) --
	(191.34, 10.61) --
	(192.05, 11.00) --
	(192.76, 11.39) --
	(193.46, 11.80) --
	(194.15, 12.22) --
	(194.83, 12.65) --
	(195.51, 13.10) --
	(196.17, 13.55) --
	(196.83, 14.02) --
	(197.48, 14.50) --
	(198.13, 14.99) --
	(198.76, 15.49) --
	(199.39, 16.01) --
	(200.00, 16.53) --
	(200.61, 17.06) --
	(201.21, 17.61) --
	(201.79, 18.16) --
	(202.37, 18.73) --
	(202.94, 19.30) --
	(203.50, 19.89) --
	(204.04, 20.48) --
	(204.58, 21.09) --
	(205.11, 21.70) --
	(205.62, 22.33) --
	(206.13, 22.96) --
	(206.62, 23.60) --
	(207.10, 24.25) --
	(207.57, 24.90) --
	(208.03, 25.57) --
	(208.48, 26.24) --
	(208.92, 26.92) --
	(209.34, 27.61) --
	(209.75, 28.31) --
	(210.15, 29.01) --
	(210.54, 29.72) --
	(210.91, 30.44) --
	(211.27, 31.16) --
	(211.62, 31.89) --
	(211.96, 32.63) --
	(212.28, 33.37) --
	(212.59, 34.11) --
	(212.89, 34.87) --
	(213.17, 35.62) --
	(213.44, 36.39) --
	(213.70, 37.15) --
	(213.94, 37.92) --
	(214.17, 38.70) --
	(214.39, 39.48) --
	(214.59, 40.26) --
	(214.78, 41.04) --
	(214.96, 41.83) --
	(215.12, 42.63) --
	(215.27, 43.42) --
	(215.40, 44.22) --
	(215.52, 45.02) --
	(215.63, 45.82) --
	(215.72, 46.62) --
	(215.80, 47.43) --
	(215.86, 48.23) --
	(215.91, 49.04) --
	(215.94, 49.85) --
	(215.97, 50.66) --
	(215.97, 51.47);
\definecolor[named]{drawColor}{rgb}{0.00,0.00,0.00}

\path[draw=drawColor,line width= 0.4pt,dash pattern=on 1pt off 3pt ,line join=round,line cap=round] (220.59, 51.47) --
	(220.58, 52.35) --
	(220.56, 53.24) --
	(220.52, 54.13) --
	(220.47, 55.02) --
	(220.40, 55.91) --
	(220.31, 56.79) --
	(220.21, 57.67) --
	(220.10, 58.56) --
	(219.96, 59.44) --
	(219.82, 60.31) --
	(219.65, 61.19) --
	(219.48, 62.06) --
	(219.28, 62.93) --
	(219.07, 63.79) --
	(218.85, 64.65) --
	(218.61, 65.51) --
	(218.36, 66.36) --
	(218.09, 67.21) --
	(217.81, 68.05) --
	(217.51, 68.89) --
	(217.20, 69.72) --
	(216.87, 70.55) --
	(216.53, 71.37) --
	(216.18, 72.19) --
	(215.81, 73.00) --
	(215.42, 73.80) --
	(215.02, 74.59) --
	(214.61, 75.38) --
	(214.19, 76.16) --
	(213.75, 76.94) --
	(213.30, 77.70) --
	(212.83, 78.46) --
	(212.35, 79.21) --
	(211.86, 79.95) --
	(211.35, 80.68) --
	(210.84, 81.40) --
	(210.31, 82.12) --
	(209.76, 82.82) --
	(209.21, 83.52) --
	(208.64, 84.20) --
	(208.06, 84.88) --
	(207.47, 85.54) --
	(206.87, 86.20) --
	(206.26, 86.84) --
	(205.63, 87.47) --
	(204.99, 88.10) --
	(204.35, 88.71) --
	(203.69, 89.31) --
	(203.02, 89.89) --
	(202.35, 90.47) --
	(201.66, 91.03) --
	(200.96, 91.59) --
	(200.25, 92.13) --
	(199.54, 92.65) --
	(198.81, 93.17) --
	(198.08, 93.67) --
	(197.34, 94.16) --
	(196.58, 94.63) --
	(195.82, 95.10) --
	(195.06, 95.55) --
	(194.28, 95.98) --
	(193.50, 96.40) --
	(192.71, 96.81) --
	(191.91, 97.21) --
	(191.11, 97.59) --
	(190.30, 97.95) --
	(189.48, 98.30) --
	(188.66, 98.64) --
	(187.83, 98.97) --
	(187.00, 99.27) --
	(186.16, 99.57) --
	(185.31, 99.85) --
	(184.46,100.11) --
	(183.61,100.36) --
	(182.75,100.60) --
	(181.89,100.82) --
	(181.02,101.02) --
	(180.15,101.21) --
	(179.28,101.38) --
	(178.41,101.54) --
	(177.53,101.68) --
	(176.65,101.81) --
	(175.77,101.93) --
	(174.88,102.02) --
	(174.00,102.10) --
	(173.11,102.17) --
	(172.22,102.22) --
	(171.33,102.26) --
	(170.45,102.28) --
	(169.56,102.28) --
	(168.67,102.27) --
	(167.78,102.24) --
	(166.89,102.20) --
	(166.00,102.14) --
	(165.12,102.07) --
	(164.23,101.98) --
	(163.35,101.87) --
	(162.47,101.75) --
	(161.59,101.62) --
	(160.71,101.46) --
	(159.84,101.30) --
	(158.97,101.12) --
	(158.10,100.92) --
	(157.24,100.71) --
	(156.38,100.48) --
	(155.52,100.24) --
	(154.67, 99.98) --
	(153.82, 99.71) --
	(152.98, 99.42) --
	(152.14, 99.12) --
	(151.31, 98.81) --
	(150.49, 98.47) --
	(149.67, 98.13) --
	(148.85, 97.77) --
	(148.05, 97.40) --
	(147.25, 97.01) --
	(146.45, 96.61) --
	(145.67, 96.19) --
	(144.89, 95.77) --
	(144.11, 95.32) --
	(143.35, 94.87) --
	(142.60, 94.40) --
	(141.85, 93.92) --
	(141.11, 93.42) --
	(140.38, 92.91) --
	(139.66, 92.39) --
	(138.95, 91.86) --
	(138.25, 91.31) --
	(137.55, 90.75) --
	(136.87, 90.18) --
	(136.20, 89.60) --
	(135.53, 89.01) --
	(134.88, 88.40) --
	(134.24, 87.79) --
	(133.61, 87.16) --
	(132.99, 86.52) --
	(132.38, 85.87) --
	(131.79, 85.21) --
	(131.20, 84.54) --
	(130.63, 83.86) --
	(130.07, 83.17) --
	(129.52, 82.47) --
	(128.98, 81.76) --
	(128.46, 81.04) --
	(127.95, 80.32) --
	(127.45, 79.58) --
	(126.96, 78.84) --
	(126.49, 78.08) --
	(126.03, 77.32) --
	(125.59, 76.55) --
	(125.16, 75.77) --
	(124.74, 74.99) --
	(124.33, 74.20) --
	(123.94, 73.40) --
	(123.56, 72.59) --
	(123.20, 71.78) --
	(122.85, 70.96) --
	(122.52, 70.14) --
	(122.20, 69.31) --
	(121.89, 68.47) --
	(121.60, 67.63) --
	(121.33, 66.79) --
	(121.07, 65.94) --
	(120.82, 65.08) --
	(120.59, 64.22) --
	(120.38, 63.36) --
	(120.17, 62.49) --
	(119.99, 61.62) --
	(119.82, 60.75) --
	(119.66, 59.87) --
	(119.52, 59.00) --
	(119.40, 58.12) --
	(119.29, 57.23) --
	(119.20, 56.35) --
	(119.12, 55.46) --
	(119.06, 54.58) --
	(119.01, 53.69) --
	(118.98, 52.80) --
	(118.97, 51.91) --
	(118.97, 51.02) --
	(118.98, 50.13) --
	(119.01, 49.24) --
	(119.06, 48.35) --
	(119.12, 47.47) --
	(119.20, 46.58) --
	(119.29, 45.70) --
	(119.40, 44.81) --
	(119.52, 43.93) --
	(119.66, 43.06) --
	(119.82, 42.18) --
	(119.99, 41.31) --
	(120.17, 40.44) --
	(120.38, 39.57) --
	(120.59, 38.71) --
	(120.82, 37.85) --
	(121.07, 36.99) --
	(121.33, 36.14) --
	(121.60, 35.30) --
	(121.89, 34.46) --
	(122.20, 33.62) --
	(122.52, 32.79) --
	(122.85, 31.97) --
	(123.20, 31.15) --
	(123.56, 30.34) --
	(123.94, 29.53) --
	(124.33, 28.73) --
	(124.74, 27.94) --
	(125.16, 27.16) --
	(125.59, 26.38) --
	(126.03, 25.61) --
	(126.49, 24.85) --
	(126.96, 24.09) --
	(127.45, 23.35) --
	(127.95, 22.61) --
	(128.46, 21.89) --
	(128.98, 21.17) --
	(129.52, 20.46) --
	(130.07, 19.76) --
	(130.63, 19.07) --
	(131.20, 18.39) --
	(131.79, 17.72) --
	(132.38, 17.06) --
	(132.99, 16.41) --
	(133.61, 15.77) --
	(134.24, 15.14) --
	(134.88, 14.53) --
	(135.53, 13.92) --
	(136.20, 13.33) --
	(136.87, 12.75) --
	(137.55, 12.18) --
	(138.25, 11.62) --
	(138.95, 11.07) --
	(139.66, 10.54) --
	(140.38, 10.02) --
	(141.11,  9.51) --
	(141.85,  9.01) --
	(142.60,  8.53) --
	(143.35,  8.06) --
	(144.11,  7.61) --
	(144.89,  7.16) --
	(145.67,  6.74) --
	(146.45,  6.32) --
	(147.25,  5.92) --
	(148.05,  5.53) --
	(148.85,  5.16) --
	(149.67,  4.80) --
	(150.49,  4.46) --
	(151.31,  4.12) --
	(152.14,  3.81) --
	(152.98,  3.51) --
	(153.82,  3.22) --
	(154.67,  2.95) --
	(155.52,  2.69) --
	(156.38,  2.45) --
	(157.24,  2.22) --
	(158.10,  2.01) --
	(158.97,  1.81) --
	(159.84,  1.63) --
	(160.71,  1.47) --
	(161.59,  1.32) --
	(162.47,  1.18) --
	(163.35,  1.06) --
	(164.23,  0.95) --
	(165.12,  0.86) --
	(166.00,  0.79) --
	(166.89,  0.73) --
	(167.78,  0.69) --
	(168.67,  0.66) --
	(169.56,  0.65) --
	(170.45,  0.65) --
	(171.33,  0.67) --
	(172.22,  0.71) --
	(173.11,  0.76) --
	(174.00,  0.83) --
	(174.88,  0.91) --
	(175.77,  1.00) --
	(176.65,  1.12) --
	(177.53,  1.25) --
	(178.41,  1.39) --
	(179.28,  1.55) --
	(180.15,  1.72) --
	(181.02,  1.91) --
	(181.89,  2.11) --
	(182.75,  2.33) --
	(183.61,  2.57) --
	(184.46,  2.82) --
	(185.31,  3.08) --
	(186.16,  3.36) --
	(187.00,  3.66) --
	(187.83,  3.96) --
	(188.66,  4.29) --
	(189.48,  4.63) --
	(190.30,  4.98) --
	(191.11,  5.34) --
	(191.91,  5.72) --
	(192.71,  6.12) --
	(193.50,  6.53) --
	(194.28,  6.95) --
	(195.06,  7.38) --
	(195.82,  7.83) --
	(196.58,  8.30) --
	(197.34,  8.77) --
	(198.08,  9.26) --
	(198.81,  9.76) --
	(199.54, 10.28) --
	(200.25, 10.80) --
	(200.96, 11.34) --
	(201.66, 11.90) --
	(202.35, 12.46) --
	(203.02, 13.04) --
	(203.69, 13.62) --
	(204.35, 14.22) --
	(204.99, 14.83) --
	(205.63, 15.46) --
	(206.26, 16.09) --
	(206.87, 16.73) --
	(207.47, 17.39) --
	(208.06, 18.05) --
	(208.64, 18.73) --
	(209.21, 19.41) --
	(209.76, 20.11) --
	(210.31, 20.81) --
	(210.84, 21.53) --
	(211.35, 22.25) --
	(211.86, 22.98) --
	(212.35, 23.72) --
	(212.83, 24.47) --
	(213.30, 25.23) --
	(213.75, 25.99) --
	(214.19, 26.77) --
	(214.61, 27.55) --
	(215.02, 28.34) --
	(215.42, 29.13) --
	(215.81, 29.93) --
	(216.18, 30.74) --
	(216.53, 31.56) --
	(216.87, 32.38) --
	(217.20, 33.21) --
	(217.51, 34.04) --
	(217.81, 34.88) --
	(218.09, 35.72) --
	(218.36, 36.57) --
	(218.61, 37.42) --
	(218.85, 38.28) --
	(219.07, 39.14) --
	(219.28, 40.00) --
	(219.48, 40.87) --
	(219.65, 41.74) --
	(219.82, 42.62) --
	(219.96, 43.49) --
	(220.10, 44.37) --
	(220.21, 45.26) --
	(220.31, 46.14) --
	(220.40, 47.02) --
	(220.47, 47.91) --
	(220.52, 48.80) --
	(220.56, 49.69) --
	(220.58, 50.58) --
	(220.59, 51.47);

\path[fill=fillColor] (126.47,126.47) circle (  2.25);
\definecolor[named]{drawColor}{rgb}{0.00,0.00,1.00}

\path[draw=drawColor,line width= 1.2pt,line join=round,line cap=round] (172.67,126.47) --
	(172.66,127.28) --
	(172.64,128.09) --
	(172.60,128.90) --
	(172.55,129.70) --
	(172.49,130.51) --
	(172.41,131.31) --
	(172.32,132.12) --
	(172.22,132.92) --
	(172.10,133.72) --
	(171.96,134.52) --
	(171.81,135.31) --
	(171.65,136.10) --
	(171.48,136.89) --
	(171.29,137.68) --
	(171.08,138.46) --
	(170.87,139.24) --
	(170.64,140.02) --
	(170.39,140.79) --
	(170.14,141.55) --
	(169.87,142.31) --
	(169.58,143.07) --
	(169.29,143.82) --
	(168.97,144.57) --
	(168.65,145.31) --
	(168.32,146.05) --
	(167.97,146.78) --
	(167.60,147.50) --
	(167.23,148.22) --
	(166.84,148.93) --
	(166.44,149.63) --
	(166.03,150.32) --
	(165.61,151.01) --
	(165.17,151.69) --
	(164.73,152.37) --
	(164.27,153.03) --
	(163.80,153.69) --
	(163.32,154.34) --
	(162.82,154.98) --
	(162.32,155.61) --
	(161.80,156.23) --
	(161.28,156.85) --
	(160.74,157.45) --
	(160.19,158.05) --
	(159.63,158.63) --
	(159.07,159.21) --
	(158.49,159.77) --
	(157.90,160.33) --
	(157.30,160.87) --
	(156.70,161.41) --
	(156.08,161.93) --
	(155.45,162.44) --
	(154.82,162.95) --
	(154.18,163.44) --
	(153.53,163.92) --
	(152.87,164.38) --
	(152.20,164.84) --
	(151.52,165.28) --
	(150.84,165.72) --
	(150.15,166.14) --
	(149.45,166.55) --
	(148.75,166.94) --
	(148.04,167.33) --
	(147.32,167.70) --
	(146.59,168.06) --
	(145.86,168.40) --
	(145.13,168.73) --
	(144.38,169.05) --
	(143.64,169.36) --
	(142.88,169.65) --
	(142.12,169.94) --
	(141.36,170.20) --
	(140.59,170.46) --
	(139.82,170.70) --
	(139.05,170.92) --
	(138.27,171.14) --
	(137.48,171.34) --
	(136.70,171.52) --
	(135.91,171.69) --
	(135.11,171.85) --
	(134.32,172.00) --
	(133.52,172.13) --
	(132.72,172.24) --
	(131.92,172.35) --
	(131.11,172.43) --
	(130.31,172.51) --
	(129.50,172.57) --
	(128.70,172.61) --
	(127.89,172.65) --
	(127.08,172.66) --
	(126.27,172.67) --
	(125.46,172.66) --
	(124.65,172.63) --
	(123.85,172.59) --
	(123.04,172.54) --
	(122.23,172.47) --
	(121.43,172.39) --
	(120.63,172.30) --
	(119.83,172.19) --
	(119.03,172.06) --
	(118.23,171.93) --
	(117.44,171.77) --
	(116.64,171.61) --
	(115.86,171.43) --
	(115.07,171.24) --
	(114.29,171.03) --
	(113.51,170.81) --
	(112.74,170.58) --
	(111.97,170.33) --
	(111.20,170.07) --
	(110.44,169.80) --
	(109.69,169.51) --
	(108.93,169.21) --
	(108.19,168.90) --
	(107.45,168.57) --
	(106.72,168.23) --
	(105.99,167.88) --
	(105.27,167.51) --
	(104.55,167.14) --
	(103.84,166.75) --
	(103.14,166.34) --
	(102.45,165.93) --
	(101.76,165.50) --
	(101.08,165.06) --
	(100.41,164.61) --
	( 99.75,164.15) --
	( 99.09,163.68) --
	( 98.44,163.19) --
	( 97.81,162.70) --
	( 97.18,162.19) --
	( 96.56,161.67) --
	( 95.94,161.14) --
	( 95.34,160.60) --
	( 94.75,160.05) --
	( 94.17,159.49) --
	( 93.59,158.92) --
	( 93.03,158.34) --
	( 92.48,157.75) --
	( 91.94,157.15) --
	( 91.40,156.54) --
	( 90.88,155.92) --
	( 90.37,155.30) --
	( 89.87,154.66) --
	( 89.39,154.02) --
	( 88.91,153.36) --
	( 88.45,152.70) --
	( 87.99,152.03) --
	( 87.55,151.35) --
	( 87.12,150.67) --
	( 86.70,149.98) --
	( 86.30,149.28) --
	( 85.91,148.57) --
	( 85.53,147.86) --
	( 85.16,147.14) --
	( 84.80,146.41) --
	( 84.46,145.68) --
	( 84.13,144.94) --
	( 83.81,144.20) --
	( 83.51,143.45) --
	( 83.22,142.69) --
	( 82.94,141.93) --
	( 82.68,141.17) --
	( 82.43,140.40) --
	( 82.19,139.63) --
	( 81.97,138.85) --
	( 81.76,138.07) --
	( 81.56,137.29) --
	( 81.38,136.50) --
	( 81.21,135.71) --
	( 81.06,134.91) --
	( 80.91,134.12) --
	( 80.79,133.32) --
	( 80.67,132.52) --
	( 80.58,131.72) --
	( 80.49,130.91) --
	( 80.42,130.11) --
	( 80.36,129.30) --
	( 80.32,128.49) --
	( 80.29,127.69) --
	( 80.28,126.88) --
	( 80.28,126.07) --
	( 80.29,125.26) --
	( 80.32,124.45) --
	( 80.36,123.64) --
	( 80.42,122.84) --
	( 80.49,122.03) --
	( 80.58,121.23) --
	( 80.67,120.43) --
	( 80.79,119.63) --
	( 80.91,118.83) --
	( 81.06,118.03) --
	( 81.21,117.24) --
	( 81.38,116.45) --
	( 81.56,115.66) --
	( 81.76,114.87) --
	( 81.97,114.09) --
	( 82.19,113.32) --
	( 82.43,112.54) --
	( 82.68,111.78) --
	( 82.94,111.01) --
	( 83.22,110.25) --
	( 83.51,109.50) --
	( 83.81,108.75) --
	( 84.13,108.00) --
	( 84.46,107.27) --
	( 84.80,106.53) --
	( 85.16,105.81) --
	( 85.53,105.09) --
	( 85.91,104.37) --
	( 86.30,103.67) --
	( 86.70,102.97) --
	( 87.12,102.28) --
	( 87.55,101.59) --
	( 87.99,100.91) --
	( 88.45,100.24) --
	( 88.91, 99.58) --
	( 89.39, 98.93) --
	( 89.87, 98.28) --
	( 90.37, 97.65) --
	( 90.88, 97.02) --
	( 91.40, 96.40) --
	( 91.94, 95.79) --
	( 92.48, 95.19) --
	( 93.03, 94.60) --
	( 93.59, 94.02) --
	( 94.17, 93.45) --
	( 94.75, 92.89) --
	( 95.34, 92.34) --
	( 95.94, 91.80) --
	( 96.56, 91.27) --
	( 97.18, 90.76) --
	( 97.81, 90.25) --
	( 98.44, 89.75) --
	( 99.09, 89.27) --
	( 99.75, 88.79) --
	(100.41, 88.33) --
	(101.08, 87.88) --
	(101.76, 87.44) --
	(102.45, 87.02) --
	(103.14, 86.60) --
	(103.84, 86.20) --
	(104.55, 85.81) --
	(105.27, 85.43) --
	(105.99, 85.07) --
	(106.72, 84.72) --
	(107.45, 84.38) --
	(108.19, 84.05) --
	(108.93, 83.74) --
	(109.69, 83.44) --
	(110.44, 83.15) --
	(111.20, 82.87) --
	(111.97, 82.61) --
	(112.74, 82.37) --
	(113.51, 82.13) --
	(114.29, 81.91) --
	(115.07, 81.71) --
	(115.86, 81.51) --
	(116.64, 81.34) --
	(117.44, 81.17) --
	(118.23, 81.02) --
	(119.03, 80.88) --
	(119.83, 80.76) --
	(120.63, 80.65) --
	(121.43, 80.55) --
	(122.23, 80.47) --
	(123.04, 80.41) --
	(123.85, 80.35) --
	(124.65, 80.31) --
	(125.46, 80.29) --
	(126.27, 80.28) --
	(127.08, 80.28) --
	(127.89, 80.30) --
	(128.70, 80.33) --
	(129.50, 80.38) --
	(130.31, 80.44) --
	(131.11, 80.51) --
	(131.92, 80.60) --
	(132.72, 80.70) --
	(133.52, 80.82) --
	(134.32, 80.95) --
	(135.11, 81.09) --
	(135.91, 81.25) --
	(136.70, 81.42) --
	(137.48, 81.61) --
	(138.27, 81.81) --
	(139.05, 82.02) --
	(139.82, 82.25) --
	(140.59, 82.49) --
	(141.36, 82.74) --
	(142.12, 83.01) --
	(142.88, 83.29) --
	(143.64, 83.58) --
	(144.38, 83.89) --
	(145.13, 84.21) --
	(145.86, 84.54) --
	(146.59, 84.89) --
	(147.32, 85.25) --
	(148.04, 85.62) --
	(148.75, 86.00) --
	(149.45, 86.40) --
	(150.15, 86.81) --
	(150.84, 87.23) --
	(151.52, 87.66) --
	(152.20, 88.10) --
	(152.87, 88.56) --
	(153.53, 89.03) --
	(154.18, 89.51) --
	(154.82, 90.00) --
	(155.45, 90.50) --
	(156.08, 91.01) --
	(156.70, 91.54) --
	(157.30, 92.07) --
	(157.90, 92.62) --
	(158.49, 93.17) --
	(159.07, 93.74) --
	(159.63, 94.31) --
	(160.19, 94.90) --
	(160.74, 95.49) --
	(161.28, 96.10) --
	(161.80, 96.71) --
	(162.32, 97.33) --
	(162.82, 97.96) --
	(163.32, 98.61) --
	(163.80, 99.25) --
	(164.27, 99.91) --
	(164.73,100.58) --
	(165.17,101.25) --
	(165.61,101.93) --
	(166.03,102.62) --
	(166.44,103.32) --
	(166.84,104.02) --
	(167.23,104.73) --
	(167.60,105.45) --
	(167.97,106.17) --
	(168.32,106.90) --
	(168.65,107.63) --
	(168.97,108.38) --
	(169.29,109.12) --
	(169.58,109.87) --
	(169.87,110.63) --
	(170.14,111.39) --
	(170.39,112.16) --
	(170.64,112.93) --
	(170.87,113.70) --
	(171.08,114.48) --
	(171.29,115.27) --
	(171.48,116.05) --
	(171.65,116.84) --
	(171.81,117.63) --
	(171.96,118.43) --
	(172.10,119.23) --
	(172.22,120.03) --
	(172.32,120.83) --
	(172.41,121.63) --
	(172.49,122.44) --
	(172.55,123.24) --
	(172.60,124.05) --
	(172.64,124.86) --
	(172.66,125.66) --
	(172.67,126.47);
\definecolor[named]{drawColor}{rgb}{0.00,0.00,0.00}

\path[draw=drawColor,line width= 0.4pt,dash pattern=on 1pt off 3pt ,line join=round,line cap=round] (177.29,126.47) --
	(177.28,127.36) --
	(177.26,128.25) --
	(177.22,129.14) --
	(177.16,130.03) --
	(177.09,130.91) --
	(177.01,131.80) --
	(176.91,132.68) --
	(176.79,133.56) --
	(176.66,134.44) --
	(176.51,135.32) --
	(176.35,136.20) --
	(176.17,137.07) --
	(175.98,137.93) --
	(175.77,138.80) --
	(175.55,139.66) --
	(175.31,140.52) --
	(175.05,141.37) --
	(174.79,142.22) --
	(174.50,143.06) --
	(174.21,143.90) --
	(173.89,144.73) --
	(173.57,145.56) --
	(173.23,146.38) --
	(172.87,147.19) --
	(172.50,148.00) --
	(172.12,148.81) --
	(171.72,149.60) --
	(171.31,150.39) --
	(170.88,151.17) --
	(170.44,151.94) --
	(169.99,152.71) --
	(169.52,153.47) --
	(169.04,154.22) --
	(168.55,154.96) --
	(168.05,155.69) --
	(167.53,156.41) --
	(167.00,157.13) --
	(166.46,157.83) --
	(165.90,158.53) --
	(165.34,159.21) --
	(164.76,159.89) --
	(164.17,160.55) --
	(163.56,161.21) --
	(162.95,161.85) --
	(162.33,162.48) --
	(161.69,163.10) --
	(161.04,163.71) --
	(160.39,164.31) --
	(159.72,164.90) --
	(159.04,165.48) --
	(158.35,166.04) --
	(157.66,166.59) --
	(156.95,167.13) --
	(156.23,167.66) --
	(155.51,168.18) --
	(154.77,168.68) --
	(154.03,169.17) --
	(153.28,169.64) --
	(152.52,170.10) --
	(151.75,170.55) --
	(150.98,170.99) --
	(150.19,171.41) --
	(149.40,171.82) --
	(148.61,172.21) --
	(147.80,172.59) --
	(146.99,172.96) --
	(146.17,173.31) --
	(145.35,173.65) --
	(144.52,173.97) --
	(143.69,174.28) --
	(142.85,174.58) --
	(142.01,174.85) --
	(141.16,175.12) --
	(140.30,175.37) --
	(139.45,175.60) --
	(138.58,175.82) --
	(137.72,176.03) --
	(136.85,176.22) --
	(135.98,176.39) --
	(135.10,176.55) --
	(134.22,176.69) --
	(133.34,176.82) --
	(132.46,176.93) --
	(131.58,177.03) --
	(130.69,177.11) --
	(129.81,177.18) --
	(128.92,177.23) --
	(128.03,177.26) --
	(127.14,177.28) --
	(126.25,177.29) --
	(125.36,177.27) --
	(124.47,177.25) --
	(123.58,177.20) --
	(122.70,177.15) --
	(121.81,177.07) --
	(120.93,176.98) --
	(120.04,176.88) --
	(119.16,176.76) --
	(118.28,176.62) --
	(117.41,176.47) --
	(116.53,176.31) --
	(115.66,176.12) --
	(114.79,175.93) --
	(113.93,175.71) --
	(113.07,175.49) --
	(112.21,175.25) --
	(111.36,174.99) --
	(110.52,174.72) --
	(109.67,174.43) --
	(108.84,174.13) --
	(108.01,173.81) --
	(107.18,173.48) --
	(106.36,173.14) --
	(105.55,172.78) --
	(104.74,172.41) --
	(103.94,172.02) --
	(103.15,171.62) --
	(102.36,171.20) --
	(101.58,170.77) --
	(100.81,170.33) --
	(100.05,169.87) --
	( 99.29,169.41) --
	( 98.54,168.92) --
	( 97.80,168.43) --
	( 97.07,167.92) --
	( 96.35,167.40) --
	( 95.64,166.87) --
	( 94.94,166.32) --
	( 94.25,165.76) --
	( 93.56,165.19) --
	( 92.89,164.61) --
	( 92.23,164.02) --
	( 91.58,163.41) --
	( 90.94,162.79) --
	( 90.31,162.17) --
	( 89.69,161.53) --
	( 89.08,160.88) --
	( 88.48,160.22) --
	( 87.90,159.55) --
	( 87.32,158.87) --
	( 86.76,158.18) --
	( 86.22,157.48) --
	( 85.68,156.77) --
	( 85.15,156.05) --
	( 84.64,155.32) --
	( 84.14,154.59) --
	( 83.66,153.84) --
	( 83.19,153.09) --
	( 82.73,152.33) --
	( 82.28,151.56) --
	( 81.85,150.78) --
	( 81.43,150.00) --
	( 81.03,149.20) --
	( 80.64,148.41) --
	( 80.26,147.60) --
	( 79.90,146.79) --
	( 79.55,145.97) --
	( 79.21,145.15) --
	( 78.89,144.32) --
	( 78.59,143.48) --
	( 78.30,142.64) --
	( 78.02,141.79) --
	( 77.76,140.94) --
	( 77.52,140.09) --
	( 77.29,139.23) --
	( 77.07,138.37) --
	( 76.87,137.50) --
	( 76.68,136.63) --
	( 76.51,135.76) --
	( 76.36,134.88) --
	( 76.22,134.00) --
	( 76.10,133.12) --
	( 75.99,132.24) --
	( 75.89,131.36) --
	( 75.82,130.47) --
	( 75.75,129.58) --
	( 75.71,128.70) --
	( 75.68,127.81) --
	( 75.66,126.92) --
	( 75.66,126.03) --
	( 75.68,125.14) --
	( 75.71,124.25) --
	( 75.75,123.36) --
	( 75.82,122.47) --
	( 75.89,121.59) --
	( 75.99,120.70) --
	( 76.10,119.82) --
	( 76.22,118.94) --
	( 76.36,118.06) --
	( 76.51,117.19) --
	( 76.68,116.31) --
	( 76.87,115.44) --
	( 77.07,114.58) --
	( 77.29,113.71) --
	( 77.52,112.86) --
	( 77.76,112.00) --
	( 78.02,111.15) --
	( 78.30,110.31) --
	( 78.59,109.46) --
	( 78.89,108.63) --
	( 79.21,107.80) --
	( 79.55,106.98) --
	( 79.90,106.16) --
	( 80.26,105.35) --
	( 80.64,104.54) --
	( 81.03,103.74) --
	( 81.43,102.95) --
	( 81.85,102.16) --
	( 82.28,101.39) --
	( 82.73,100.62) --
	( 83.19, 99.86) --
	( 83.66, 99.10) --
	( 84.14, 98.36) --
	( 84.64, 97.62) --
	( 85.15, 96.89) --
	( 85.68, 96.17) --
	( 86.22, 95.47) --
	( 86.76, 94.77) --
	( 87.32, 94.08) --
	( 87.90, 93.40) --
	( 88.48, 92.73) --
	( 89.08, 92.07) --
	( 89.69, 91.42) --
	( 90.31, 90.78) --
	( 90.94, 90.15) --
	( 91.58, 89.53) --
	( 92.23, 88.93) --
	( 92.89, 88.34) --
	( 93.56, 87.75) --
	( 94.25, 87.18) --
	( 94.94, 86.63) --
	( 95.64, 86.08) --
	( 96.35, 85.55) --
	( 97.07, 85.03) --
	( 97.80, 84.52) --
	( 98.54, 84.02) --
	( 99.29, 83.54) --
	(100.05, 83.07) --
	(100.81, 82.61) --
	(101.58, 82.17) --
	(102.36, 81.74) --
	(103.15, 81.33) --
	(103.94, 80.93) --
	(104.74, 80.54) --
	(105.55, 80.17) --
	(106.36, 79.81) --
	(107.18, 79.46) --
	(108.01, 79.13) --
	(108.84, 78.82) --
	(109.67, 78.51) --
	(110.52, 78.23) --
	(111.36, 77.96) --
	(112.21, 77.70) --
	(113.07, 77.46) --
	(113.93, 77.23) --
	(114.79, 77.02) --
	(115.66, 76.82) --
	(116.53, 76.64) --
	(117.41, 76.47) --
	(118.28, 76.32) --
	(119.16, 76.19) --
	(120.04, 76.07) --
	(120.93, 75.96) --
	(121.81, 75.87) --
	(122.70, 75.80) --
	(123.58, 75.74) --
	(124.47, 75.70) --
	(125.36, 75.67) --
	(126.25, 75.66) --
	(127.14, 75.66) --
	(128.03, 75.68) --
	(128.92, 75.72) --
	(129.81, 75.77) --
	(130.69, 75.83) --
	(131.58, 75.92) --
	(132.46, 76.01) --
	(133.34, 76.12) --
	(134.22, 76.25) --
	(135.10, 76.40) --
	(135.98, 76.55) --
	(136.85, 76.73) --
	(137.72, 76.92) --
	(138.58, 77.12) --
	(139.45, 77.34) --
	(140.30, 77.58) --
	(141.16, 77.83) --
	(142.01, 78.09) --
	(142.85, 78.37) --
	(143.69, 78.66) --
	(144.52, 78.97) --
	(145.35, 79.30) --
	(146.17, 79.63) --
	(146.99, 79.99) --
	(147.80, 80.35) --
	(148.61, 80.73) --
	(149.40, 81.13) --
	(150.19, 81.53) --
	(150.98, 81.96) --
	(151.75, 82.39) --
	(152.52, 82.84) --
	(153.28, 83.30) --
	(154.03, 83.78) --
	(154.77, 84.27) --
	(155.51, 84.77) --
	(156.23, 85.28) --
	(156.95, 85.81) --
	(157.66, 86.35) --
	(158.35, 86.90) --
	(159.04, 87.47) --
	(159.72, 88.04) --
	(160.39, 88.63) --
	(161.04, 89.23) --
	(161.69, 89.84) --
	(162.33, 90.46) --
	(162.95, 91.10) --
	(163.56, 91.74) --
	(164.17, 92.39) --
	(164.76, 93.06) --
	(165.34, 93.73) --
	(165.90, 94.42) --
	(166.46, 95.11) --
	(167.00, 95.82) --
	(167.53, 96.53) --
	(168.05, 97.26) --
	(168.55, 97.99) --
	(169.04, 98.73) --
	(169.52, 99.48) --
	(169.99,100.24) --
	(170.44,101.00) --
	(170.88,101.77) --
	(171.31,102.56) --
	(171.72,103.34) --
	(172.12,104.14) --
	(172.50,104.94) --
	(172.87,105.75) --
	(173.23,106.57) --
	(173.57,107.39) --
	(173.89,108.21) --
	(174.21,109.05) --
	(174.50,109.88) --
	(174.79,110.73) --
	(175.05,111.58) --
	(175.31,112.43) --
	(175.55,113.28) --
	(175.77,114.15) --
	(175.98,115.01) --
	(176.17,115.88) --
	(176.35,116.75) --
	(176.51,117.62) --
	(176.66,118.50) --
	(176.79,119.38) --
	(176.91,120.26) --
	(177.01,121.15) --
	(177.09,122.03) --
	(177.16,122.92) --
	(177.22,123.81) --
	(177.26,124.69) --
	(177.28,125.58) --
	(177.29,126.47);

\path[fill=fillColor] ( 83.17,201.48) circle (  2.25);
\definecolor[named]{drawColor}{rgb}{0.00,0.00,1.00}

\path[draw=drawColor,line width= 1.2pt,line join=round,line cap=round] (129.36,201.48) --
	(129.35,202.29) --
	(129.33,203.10) --
	(129.30,203.90) --
	(129.25,204.71) --
	(129.19,205.52) --
	(129.11,206.32) --
	(129.02,207.13) --
	(128.91,207.93) --
	(128.79,208.73) --
	(128.66,209.52) --
	(128.51,210.32) --
	(128.35,211.11) --
	(128.17,211.90) --
	(127.98,212.69) --
	(127.78,213.47) --
	(127.56,214.25) --
	(127.33,215.02) --
	(127.09,215.79) --
	(126.83,216.56) --
	(126.56,217.32) --
	(126.28,218.08) --
	(125.98,218.83) --
	(125.67,219.58) --
	(125.35,220.32) --
	(125.01,221.05) --
	(124.66,221.78) --
	(124.30,222.51) --
	(123.93,223.22) --
	(123.54,223.93) --
	(123.14,224.64) --
	(122.73,225.33) --
	(122.30,226.02) --
	(121.87,226.70) --
	(121.42,227.38) --
	(120.96,228.04) --
	(120.49,228.70) --
	(120.01,229.35) --
	(119.52,229.99) --
	(119.01,230.62) --
	(118.50,231.24) --
	(117.97,231.86) --
	(117.43,232.46) --
	(116.89,233.06) --
	(116.33,233.64) --
	(115.76,234.22) --
	(115.18,234.78) --
	(114.59,235.34) --
	(114.00,235.88) --
	(113.39,236.42) --
	(112.77,236.94) --
	(112.15,237.45) --
	(111.52,237.95) --
	(110.87,238.44) --
	(110.22,238.92) --
	(109.56,239.39) --
	(108.89,239.85) --
	(108.22,240.29) --
	(107.54,240.72) --
	(106.85,241.14) --
	(106.15,241.55) --
	(105.44,241.95) --
	(104.73,242.33) --
	(104.01,242.70) --
	(103.29,243.06) --
	(102.56,243.41) --
	(101.82,243.74) --
	(101.08,244.06) --
	(100.33,244.37) --
	( 99.58,244.66) --
	( 98.82,244.94) --
	( 98.06,245.21) --
	( 97.29,245.46) --
	( 96.52,245.70) --
	( 95.74,245.93) --
	( 94.96,246.14) --
	( 94.18,246.34) --
	( 93.39,246.53) --
	( 92.60,246.70) --
	( 91.81,246.86) --
	( 91.01,247.00) --
	( 90.21,247.13) --
	( 89.41,247.25) --
	( 88.61,247.35) --
	( 87.81,247.44) --
	( 87.00,247.52) --
	( 86.20,247.58) --
	( 85.39,247.62) --
	( 84.58,247.65) --
	( 83.77,247.67) --
	( 82.96,247.67) --
	( 82.16,247.66) --
	( 81.35,247.64) --
	( 80.54,247.60) --
	( 79.73,247.55) --
	( 78.93,247.48) --
	( 78.12,247.40) --
	( 77.32,247.30) --
	( 76.52,247.19) --
	( 75.72,247.07) --
	( 74.92,246.93) --
	( 74.13,246.78) --
	( 73.34,246.62) --
	( 72.55,246.44) --
	( 71.76,246.25) --
	( 70.98,246.04) --
	( 70.21,245.82) --
	( 69.43,245.59) --
	( 68.66,245.34) --
	( 67.90,245.08) --
	( 67.14,244.80) --
	( 66.38,244.52) --
	( 65.63,244.22) --
	( 64.88,243.90) --
	( 64.14,243.58) --
	( 63.41,243.24) --
	( 62.68,242.88) --
	( 61.96,242.52) --
	( 61.25,242.14) --
	( 60.54,241.75) --
	( 59.84,241.35) --
	( 59.14,240.94) --
	( 58.46,240.51) --
	( 57.78,240.07) --
	( 57.10,239.62) --
	( 56.44,239.16) --
	( 55.79,238.69) --
	( 55.14,238.20) --
	( 54.50,237.70) --
	( 53.87,237.20) --
	( 53.25,236.68) --
	( 52.64,236.15) --
	( 52.04,235.61) --
	( 51.44,235.06) --
	( 50.86,234.50) --
	( 50.29,233.93) --
	( 49.73,233.35) --
	( 49.17,232.76) --
	( 48.63,232.16) --
	( 48.10,231.55) --
	( 47.58,230.93) --
	( 47.07,230.30) --
	( 46.57,229.67) --
	( 46.08,229.02) --
	( 45.61,228.37) --
	( 45.14,227.71) --
	( 44.69,227.04) --
	( 44.25,226.36) --
	( 43.82,225.68) --
	( 43.40,224.98) --
	( 42.99,224.29) --
	( 42.60,223.58) --
	( 42.22,222.87) --
	( 41.85,222.15) --
	( 41.50,221.42) --
	( 41.15,220.69) --
	( 40.82,219.95) --
	( 40.51,219.20) --
	( 40.20,218.46) --
	( 39.91,217.70) --
	( 39.64,216.94) --
	( 39.37,216.18) --
	( 39.12,215.41) --
	( 38.88,214.64) --
	( 38.66,213.86) --
	( 38.45,213.08) --
	( 38.26,212.29) --
	( 38.07,211.51) --
	( 37.90,210.72) --
	( 37.75,209.92) --
	( 37.61,209.13) --
	( 37.48,208.33) --
	( 37.37,207.53) --
	( 37.27,206.72) --
	( 37.19,205.92) --
	( 37.12,205.11) --
	( 37.06,204.31) --
	( 37.02,203.50) --
	( 36.99,202.69) --
	( 36.97,201.88) --
	( 36.97,201.08) --
	( 36.99,200.27) --
	( 37.02,199.46) --
	( 37.06,198.65) --
	( 37.12,197.85) --
	( 37.19,197.04) --
	( 37.27,196.24) --
	( 37.37,195.43) --
	( 37.48,194.63) --
	( 37.61,193.83) --
	( 37.75,193.04) --
	( 37.90,192.24) --
	( 38.07,191.45) --
	( 38.26,190.67) --
	( 38.45,189.88) --
	( 38.66,189.10) --
	( 38.88,188.32) --
	( 39.12,187.55) --
	( 39.37,186.78) --
	( 39.64,186.02) --
	( 39.91,185.26) --
	( 40.20,184.50) --
	( 40.51,183.76) --
	( 40.82,183.01) --
	( 41.15,182.27) --
	( 41.50,181.54) --
	( 41.85,180.81) --
	( 42.22,180.09) --
	( 42.60,179.38) --
	( 42.99,178.67) --
	( 43.40,177.98) --
	( 43.82,177.28) --
	( 44.25,176.60) --
	( 44.69,175.92) --
	( 45.14,175.25) --
	( 45.61,174.59) --
	( 46.08,173.94) --
	( 46.57,173.29) --
	( 47.07,172.66) --
	( 47.58,172.03) --
	( 48.10,171.41) --
	( 48.63,170.80) --
	( 49.17,170.20) --
	( 49.73,169.61) --
	( 50.29,169.03) --
	( 50.86,168.46) --
	( 51.44,167.90) --
	( 52.04,167.35) --
	( 52.64,166.81) --
	( 53.25,166.28) --
	( 53.87,165.76) --
	( 54.50,165.26) --
	( 55.14,164.76) --
	( 55.79,164.27) --
	( 56.44,163.80) --
	( 57.10,163.34) --
	( 57.78,162.89) --
	( 58.46,162.45) --
	( 59.14,162.02) --
	( 59.84,161.61) --
	( 60.54,161.21) --
	( 61.25,160.82) --
	( 61.96,160.44) --
	( 62.68,160.08) --
	( 63.41,159.72) --
	( 64.14,159.38) --
	( 64.88,159.06) --
	( 65.63,158.74) --
	( 66.38,158.44) --
	( 67.14,158.16) --
	( 67.90,157.88) --
	( 68.66,157.62) --
	( 69.43,157.37) --
	( 70.21,157.14) --
	( 70.98,156.92) --
	( 71.76,156.71) --
	( 72.55,156.52) --
	( 73.34,156.34) --
	( 74.13,156.18) --
	( 74.92,156.03) --
	( 75.72,155.89) --
	( 76.52,155.77) --
	( 77.32,155.66) --
	( 78.12,155.56) --
	( 78.93,155.48) --
	( 79.73,155.41) --
	( 80.54,155.36) --
	( 81.35,155.32) --
	( 82.16,155.30) --
	( 82.96,155.29) --
	( 83.77,155.29) --
	( 84.58,155.31) --
	( 85.39,155.34) --
	( 86.20,155.38) --
	( 87.00,155.44) --
	( 87.81,155.52) --
	( 88.61,155.61) --
	( 89.41,155.71) --
	( 90.21,155.83) --
	( 91.01,155.96) --
	( 91.81,156.10) --
	( 92.60,156.26) --
	( 93.39,156.43) --
	( 94.18,156.62) --
	( 94.96,156.82) --
	( 95.74,157.03) --
	( 96.52,157.26) --
	( 97.29,157.50) --
	( 98.06,157.75) --
	( 98.82,158.02) --
	( 99.58,158.30) --
	(100.33,158.59) --
	(101.08,158.90) --
	(101.82,159.22) --
	(102.56,159.55) --
	(103.29,159.90) --
	(104.01,160.26) --
	(104.73,160.63) --
	(105.44,161.01) --
	(106.15,161.41) --
	(106.85,161.82) --
	(107.54,162.24) --
	(108.22,162.67) --
	(108.89,163.11) --
	(109.56,163.57) --
	(110.22,164.04) --
	(110.87,164.52) --
	(111.52,165.01) --
	(112.15,165.51) --
	(112.77,166.02) --
	(113.39,166.54) --
	(114.00,167.08) --
	(114.59,167.62) --
	(115.18,168.18) --
	(115.76,168.74) --
	(116.33,169.32) --
	(116.89,169.90) --
	(117.43,170.50) --
	(117.97,171.10) --
	(118.50,171.72) --
	(119.01,172.34) --
	(119.52,172.97) --
	(120.01,173.61) --
	(120.49,174.26) --
	(120.96,174.92) --
	(121.42,175.58) --
	(121.87,176.26) --
	(122.30,176.94) --
	(122.73,177.63) --
	(123.14,178.32) --
	(123.54,179.03) --
	(123.93,179.74) --
	(124.30,180.45) --
	(124.66,181.18) --
	(125.01,181.91) --
	(125.35,182.64) --
	(125.67,183.38) --
	(125.98,184.13) --
	(126.28,184.88) --
	(126.56,185.64) --
	(126.83,186.40) --
	(127.09,187.17) --
	(127.33,187.94) --
	(127.56,188.71) --
	(127.78,189.49) --
	(127.98,190.27) --
	(128.17,191.06) --
	(128.35,191.85) --
	(128.51,192.64) --
	(128.66,193.44) --
	(128.79,194.23) --
	(128.91,195.03) --
	(129.02,195.83) --
	(129.11,196.64) --
	(129.19,197.44) --
	(129.25,198.25) --
	(129.30,199.06) --
	(129.33,199.86) --
	(129.35,200.67) --
	(129.36,201.48);
\definecolor[named]{drawColor}{rgb}{0.00,0.00,0.00}

\path[draw=drawColor,line width= 0.4pt,dash pattern=on 1pt off 3pt ,line join=round,line cap=round] (133.98,201.48) --
	(133.97,202.37) --
	(133.95,203.26) --
	(133.91,204.15) --
	(133.86,205.03) --
	(133.79,205.92) --
	(133.70,206.81) --
	(133.60,207.69) --
	(133.48,208.57) --
	(133.35,209.45) --
	(133.21,210.33) --
	(133.04,211.20) --
	(132.86,212.07) --
	(132.67,212.94) --
	(132.46,213.81) --
	(132.24,214.67) --
	(132.00,215.52) --
	(131.75,216.38) --
	(131.48,217.22) --
	(131.20,218.07) --
	(130.90,218.91) --
	(130.59,219.74) --
	(130.26,220.57) --
	(129.92,221.39) --
	(129.56,222.20) --
	(129.19,223.01) --
	(128.81,223.81) --
	(128.41,224.61) --
	(128.00,225.40) --
	(127.58,226.18) --
	(127.14,226.95) --
	(126.68,227.72) --
	(126.22,228.47) --
	(125.74,229.22) --
	(125.25,229.96) --
	(124.74,230.70) --
	(124.22,231.42) --
	(123.69,232.13) --
	(123.15,232.84) --
	(122.60,233.53) --
	(122.03,234.22) --
	(121.45,234.89) --
	(120.86,235.56) --
	(120.26,236.21) --
	(119.64,236.86) --
	(119.02,237.49) --
	(118.38,238.11) --
	(117.74,238.72) --
	(117.08,239.32) --
	(116.41,239.91) --
	(115.74,240.49) --
	(115.05,241.05) --
	(114.35,241.60) --
	(113.64,242.14) --
	(112.93,242.67) --
	(112.20,243.18) --
	(111.47,243.68) --
	(110.72,244.17) --
	(109.97,244.65) --
	(109.21,245.11) --
	(108.45,245.56) --
	(107.67,246.00) --
	(106.89,246.42) --
	(106.10,246.83) --
	(105.30,247.22) --
	(104.50,247.60) --
	(103.69,247.97) --
	(102.87,248.32) --
	(102.05,248.66) --
	(101.22,248.98) --
	(100.38,249.29) --
	( 99.54,249.58) --
	( 98.70,249.86) --
	( 97.85,250.13) --
	( 97.00,250.38) --
	( 96.14,250.61) --
	( 95.28,250.83) --
	( 94.41,251.03) --
	( 93.54,251.22) --
	( 92.67,251.40) --
	( 91.80,251.56) --
	( 90.92,251.70) --
	( 90.04,251.83) --
	( 89.16,251.94) --
	( 88.27,252.04) --
	( 87.39,252.12) --
	( 86.50,252.19) --
	( 85.61,252.24) --
	( 84.72,252.27) --
	( 83.83,252.29) --
	( 82.94,252.29) --
	( 82.06,252.28) --
	( 81.17,252.26) --
	( 80.28,252.21) --
	( 79.39,252.15) --
	( 78.50,252.08) --
	( 77.62,251.99) --
	( 76.74,251.89) --
	( 75.86,251.77) --
	( 74.98,251.63) --
	( 74.10,251.48) --
	( 73.23,251.31) --
	( 72.36,251.13) --
	( 71.49,250.93) --
	( 70.62,250.72) --
	( 69.76,250.50) --
	( 68.91,250.25) --
	( 68.06,250.00) --
	( 67.21,249.72) --
	( 66.37,249.44) --
	( 65.53,249.14) --
	( 64.70,248.82) --
	( 63.88,248.49) --
	( 63.06,248.15) --
	( 62.24,247.79) --
	( 61.43,247.41) --
	( 60.63,247.03) --
	( 59.84,246.62) --
	( 59.05,246.21) --
	( 58.27,245.78) --
	( 57.50,245.34) --
	( 56.74,244.88) --
	( 55.98,244.41) --
	( 55.24,243.93) --
	( 54.50,243.44) --
	( 53.77,242.93) --
	( 53.05,242.41) --
	( 52.34,241.87) --
	( 51.63,241.33) --
	( 50.94,240.77) --
	( 50.26,240.20) --
	( 49.59,239.62) --
	( 48.92,239.02) --
	( 48.27,238.42) --
	( 47.63,237.80) --
	( 47.00,237.17) --
	( 46.38,236.54) --
	( 45.77,235.89) --
	( 45.18,235.23) --
	( 44.59,234.56) --
	( 44.02,233.88) --
	( 43.46,233.19) --
	( 42.91,232.49) --
	( 42.37,231.78) --
	( 41.85,231.06) --
	( 41.34,230.33) --
	( 40.84,229.60) --
	( 40.35,228.85) --
	( 39.88,228.10) --
	( 39.42,227.34) --
	( 38.98,226.57) --
	( 38.54,225.79) --
	( 38.13,225.00) --
	( 37.72,224.21) --
	( 37.33,223.41) --
	( 36.95,222.61) --
	( 36.59,221.80) --
	( 36.24,220.98) --
	( 35.91,220.15) --
	( 35.59,219.32) --
	( 35.28,218.49) --
	( 34.99,217.65) --
	( 34.72,216.80) --
	( 34.46,215.95) --
	( 34.21,215.10) --
	( 33.98,214.24) --
	( 33.76,213.37) --
	( 33.56,212.51) --
	( 33.38,211.64) --
	( 33.21,210.77) --
	( 33.05,209.89) --
	( 32.91,209.01) --
	( 32.79,208.13) --
	( 32.68,207.25) --
	( 32.59,206.36) --
	( 32.51,205.48) --
	( 32.45,204.59) --
	( 32.40,203.70) --
	( 32.37,202.81) --
	( 32.35,201.92) --
	( 32.35,201.04) --
	( 32.37,200.15) --
	( 32.40,199.26) --
	( 32.45,198.37) --
	( 32.51,197.48) --
	( 32.59,196.60) --
	( 32.68,195.71) --
	( 32.79,194.83) --
	( 32.91,193.95) --
	( 33.05,193.07) --
	( 33.21,192.19) --
	( 33.38,191.32) --
	( 33.56,190.45) --
	( 33.76,189.59) --
	( 33.98,188.72) --
	( 34.21,187.86) --
	( 34.46,187.01) --
	( 34.72,186.16) --
	( 34.99,185.31) --
	( 35.28,184.47) --
	( 35.59,183.64) --
	( 35.91,182.81) --
	( 36.24,181.98) --
	( 36.59,181.16) --
	( 36.95,180.35) --
	( 37.33,179.55) --
	( 37.72,178.75) --
	( 38.13,177.96) --
	( 38.54,177.17) --
	( 38.98,176.39) --
	( 39.42,175.62) --
	( 39.88,174.86) --
	( 40.35,174.11) --
	( 40.84,173.36) --
	( 41.34,172.63) --
	( 41.85,171.90) --
	( 42.37,171.18) --
	( 42.91,170.47) --
	( 43.46,169.77) --
	( 44.02,169.08) --
	( 44.59,168.40) --
	( 45.18,167.73) --
	( 45.77,167.07) --
	( 46.38,166.42) --
	( 47.00,165.79) --
	( 47.63,165.16) --
	( 48.27,164.54) --
	( 48.92,163.94) --
	( 49.59,163.34) --
	( 50.26,162.76) --
	( 50.94,162.19) --
	( 51.63,161.63) --
	( 52.34,161.09) --
	( 53.05,160.55) --
	( 53.77,160.03) --
	( 54.50,159.52) --
	( 55.24,159.03) --
	( 55.98,158.55) --
	( 56.74,158.08) --
	( 57.50,157.62) --
	( 58.27,157.18) --
	( 59.05,156.75) --
	( 59.84,156.34) --
	( 60.63,155.93) --
	( 61.43,155.55) --
	( 62.24,155.17) --
	( 63.06,154.81) --
	( 63.88,154.47) --
	( 64.70,154.14) --
	( 65.53,153.82) --
	( 66.37,153.52) --
	( 67.21,153.24) --
	( 68.06,152.96) --
	( 68.91,152.71) --
	( 69.76,152.46) --
	( 70.62,152.24) --
	( 71.49,152.03) --
	( 72.36,151.83) --
	( 73.23,151.65) --
	( 74.10,151.48) --
	( 74.98,151.33) --
	( 75.86,151.19) --
	( 76.74,151.07) --
	( 77.62,150.97) --
	( 78.50,150.88) --
	( 79.39,150.81) --
	( 80.28,150.75) --
	( 81.17,150.70) --
	( 82.06,150.68) --
	( 82.94,150.67) --
	( 83.83,150.67) --
	( 84.72,150.69) --
	( 85.61,150.72) --
	( 86.50,150.77) --
	( 87.39,150.84) --
	( 88.27,150.92) --
	( 89.16,151.02) --
	( 90.04,151.13) --
	( 90.92,151.26) --
	( 91.80,151.40) --
	( 92.67,151.56) --
	( 93.54,151.74) --
	( 94.41,151.93) --
	( 95.28,152.13) --
	( 96.14,152.35) --
	( 97.00,152.58) --
	( 97.85,152.83) --
	( 98.70,153.10) --
	( 99.54,153.38) --
	(100.38,153.67) --
	(101.22,153.98) --
	(102.05,154.30) --
	(102.87,154.64) --
	(103.69,154.99) --
	(104.50,155.36) --
	(105.30,155.74) --
	(106.10,156.13) --
	(106.89,156.54) --
	(107.67,156.96) --
	(108.45,157.40) --
	(109.21,157.85) --
	(109.97,158.31) --
	(110.72,158.79) --
	(111.47,159.28) --
	(112.20,159.78) --
	(112.93,160.29) --
	(113.64,160.82) --
	(114.35,161.36) --
	(115.05,161.91) --
	(115.74,162.47) --
	(116.41,163.05) --
	(117.08,163.64) --
	(117.74,164.24) --
	(118.38,164.85) --
	(119.02,165.47) --
	(119.64,166.10) --
	(120.26,166.75) --
	(120.86,167.40) --
	(121.45,168.07) --
	(122.03,168.74) --
	(122.60,169.43) --
	(123.15,170.12) --
	(123.69,170.83) --
	(124.22,171.54) --
	(124.74,172.26) --
	(125.25,173.00) --
	(125.74,173.74) --
	(126.22,174.49) --
	(126.68,175.24) --
	(127.14,176.01) --
	(127.58,176.78) --
	(128.00,177.56) --
	(128.41,178.35) --
	(128.81,179.15) --
	(129.19,179.95) --
	(129.56,180.76) --
	(129.92,181.57) --
	(130.26,182.39) --
	(130.59,183.22) --
	(130.90,184.05) --
	(131.20,184.89) --
	(131.48,185.74) --
	(131.75,186.58) --
	(132.00,187.44) --
	(132.24,188.29) --
	(132.46,189.15) --
	(132.67,190.02) --
	(132.86,190.89) --
	(133.04,191.76) --
	(133.21,192.63) --
	(133.35,193.51) --
	(133.48,194.39) --
	(133.60,195.27) --
	(133.70,196.15) --
	(133.79,197.04) --
	(133.86,197.93) --
	(133.91,198.81) --
	(133.95,199.70) --
	(133.97,200.59) --
	(133.98,201.48);
\definecolor[named]{drawColor}{rgb}{0.00,0.00,1.00}

\path[draw=drawColor,line width= 1.2pt,line join=round,line cap=round] (  0.12,252.94) --
	(  0.51,252.29) --
	(  0.94,251.61) --
	(  1.38,250.93) --
	(  1.83,250.26) --
	(  2.30,249.60) --
	(  2.78,248.94) --
	(  3.26,248.30) --
	(  3.76,247.66) --
	(  4.27,247.04) --
	(  4.79,246.42) --
	(  5.33,245.81) --
	(  5.87,245.21) --
	(  6.42,244.62) --
	(  6.98,244.04) --
	(  7.56,243.47) --
	(  8.14,242.91) --
	(  8.73,242.36) --
	(  9.33,241.82) --
	(  9.94,241.29) --
	( 10.57,240.77) --
	( 11.19,240.26) --
	( 11.83,239.77) --
	( 12.48,239.28) --
	( 13.14,238.81) --
	( 13.80,238.35) --
	( 14.47,237.90) --
	( 15.15,237.46) --
	( 15.84,237.03) --
	( 16.53,236.62) --
	( 17.23,236.21) --
	( 17.94,235.82) --
	( 18.66,235.45) --
	( 19.38,235.08) --
	( 20.10,234.73) --
	( 20.84,234.39) --
	( 21.58,234.06) --
	( 22.32,233.75) --
	( 23.07,233.45) --
	( 23.83,233.16) --
	( 24.59,232.89) --
	( 25.36,232.63) --
	( 26.13,232.38) --
	( 26.90,232.15) --
	( 27.68,231.93) --
	( 28.46,231.72) --
	( 29.24,231.53) --
	( 30.03,231.35) --
	( 30.82,231.19) --
	( 31.62,231.03) --
	( 32.42,230.90) --
	( 33.21,230.77) --
	( 34.02,230.66) --
	( 34.82,230.57) --
	( 35.62,230.49) --
	( 36.43,230.42) --
	( 37.24,230.37) --
	( 38.04,230.33) --
	( 38.85,230.30) --
	( 39.66,230.29) --
	( 40.47,230.30) --
	( 41.28,230.31) --
	( 42.08,230.35) --
	( 42.89,230.39) --
	( 43.70,230.45) --
	( 44.50,230.53) --
	( 45.31,230.61) --
	( 46.11,230.72) --
	( 46.91,230.83) --
	( 47.71,230.96) --
	( 48.50,231.11) --
	( 49.29,231.27) --
	( 50.08,231.44) --
	( 50.87,231.62) --
	( 51.65,231.82) --
	( 52.43,232.04) --
	( 53.21,232.26) --
	( 53.98,232.50) --
	( 54.75,232.76) --
	( 55.51,233.02) --
	( 56.27,233.31) --
	( 57.02,233.60) --
	( 57.77,233.91) --
	( 58.51,234.23) --
	( 59.25,234.56) --
	( 59.98,234.90) --
	( 60.71,235.26) --
	( 61.43,235.63) --
	( 62.14,236.02) --
	( 62.84,236.41) --
	( 63.54,236.82) --
	( 64.23,237.24) --
	( 64.91,237.68) --
	( 65.59,238.12) --
	( 66.26,238.58) --
	( 66.92,239.04) --
	( 67.57,239.52) --
	( 68.21,240.01) --
	( 68.84,240.52) --
	( 69.47,241.03) --
	( 70.08,241.55) --
	( 70.69,242.09) --
	( 71.29,242.63) --
	( 71.88,243.19) --
	( 72.45,243.75) --
	( 73.02,244.33) --
	( 73.58,244.91) --
	( 74.13,245.51) --
	( 74.66,246.11) --
	( 75.19,246.73) --
	( 75.71,247.35) --
	( 76.21,247.98) --
	( 76.70,248.62) --
	( 77.19,249.27) --
	( 77.66,249.93) --
	( 78.12,250.59) --
	( 78.56,251.27) --
	( 79.00,251.95) --
	( 79.42,252.64) --
	( 79.61,252.94);
\definecolor[named]{drawColor}{rgb}{0.00,0.00,0.00}

\path[draw=drawColor,line width= 0.4pt,dash pattern=on 1pt off 3pt ,line join=round,line cap=round] (  0.00,244.98) --
	(  0.15,244.78) --
	(  0.71,244.09) --
	(  1.29,243.41) --
	(  1.87,242.74) --
	(  2.47,242.08) --
	(  3.08,241.43) --
	(  3.70,240.79) --
	(  4.33,240.17) --
	(  4.97,239.55) --
	(  5.62,238.94) --
	(  6.28,238.35) --
	(  6.95,237.77) --
	(  7.64,237.20) --
	(  8.33,236.64) --
	(  9.03,236.09) --
	(  9.74,235.56) --
	( 10.46,235.04) --
	( 11.19,234.53) --
	( 11.93,234.04) --
	( 12.68,233.55) --
	( 13.43,233.09) --
	( 14.20,232.63) --
	( 14.97,232.19) --
	( 15.75,231.76) --
	( 16.53,231.34) --
	( 17.33,230.94) --
	( 18.13,230.55) --
	( 18.94,230.18) --
	( 19.75,229.82) --
	( 20.57,229.48) --
	( 21.40,229.15) --
	( 22.23,228.83) --
	( 23.06,228.53) --
	( 23.91,228.24) --
	( 24.75,227.97) --
	( 25.60,227.71) --
	( 26.46,227.47) --
	( 27.32,227.25) --
	( 28.18,227.03) --
	( 29.05,226.84) --
	( 29.92,226.65) --
	( 30.79,226.49) --
	( 31.67,226.34) --
	( 32.55,226.20) --
	( 33.43,226.08) --
	( 34.31,225.98) --
	( 35.20,225.89) --
	( 36.09,225.81) --
	( 36.97,225.76) --
	( 37.86,225.71) --
	( 38.75,225.69) --
	( 39.64,225.67) --
	( 40.53,225.68) --
	( 41.42,225.70) --
	( 42.31,225.73) --
	( 43.19,225.78) --
	( 44.08,225.85) --
	( 44.97,225.93) --
	( 45.85,226.03) --
	( 46.73,226.14) --
	( 47.61,226.27) --
	( 48.49,226.41) --
	( 49.37,226.57) --
	( 50.24,226.74) --
	( 51.11,226.93) --
	( 51.97,227.14) --
	( 52.83,227.36) --
	( 53.69,227.59) --
	( 54.55,227.84) --
	( 55.39,228.11) --
	( 56.24,228.38) --
	( 57.08,228.68) --
	( 57.91,228.99) --
	( 58.74,229.31) --
	( 59.56,229.65) --
	( 60.38,230.00) --
	( 61.19,230.37) --
	( 61.99,230.75) --
	( 62.79,231.14) --
	( 63.58,231.55) --
	( 64.36,231.97) --
	( 65.14,232.41) --
	( 65.91,232.86) --
	( 66.67,233.32) --
	( 67.42,233.79) --
	( 68.16,234.28) --
	( 68.90,234.78) --
	( 69.62,235.30) --
	( 70.34,235.83) --
	( 71.04,236.37) --
	( 71.74,236.92) --
	( 72.43,237.48) --
	( 73.11,238.06) --
	( 73.77,238.65) --
	( 74.43,239.25) --
	( 75.08,239.86) --
	( 75.71,240.48) --
	( 76.34,241.11) --
	( 76.95,241.75) --
	( 77.55,242.41) --
	( 78.14,243.07) --
	( 78.72,243.75) --
	( 79.29,244.43) --
	( 79.85,245.13) --
	( 80.39,245.83) --
	( 80.92,246.55) --
	( 81.44,247.27) --
	( 81.94,248.00) --
	( 82.43,248.74) --
	( 82.91,249.49) --
	( 83.38,250.25) --
	( 83.83,251.02) --
	( 84.27,251.79) --
	( 84.70,252.57) --
	( 84.89,252.94);
\definecolor[named]{drawColor}{rgb}{0.00,0.00,1.00}

\path[draw=drawColor,line width= 1.2pt,line join=round,line cap=round] (252.94, 19.05) --
	(252.54, 18.88) --
	(251.80, 18.55) --
	(251.07, 18.21) --
	(250.34, 17.86) --
	(249.62, 17.50) --
	(248.90, 17.12) --
	(248.20, 16.73) --
	(247.49, 16.33) --
	(246.80, 15.91) --
	(246.11, 15.49) --
	(245.43, 15.05) --
	(244.76, 14.60) --
	(244.10, 14.14) --
	(243.44, 13.66) --
	(242.80, 13.18) --
	(242.16, 12.68) --
	(241.53, 12.17) --
	(240.91, 11.66) --
	(240.30, 11.13) --
	(239.69, 10.59) --
	(239.10, 10.04) --
	(238.52,  9.48) --
	(237.95,  8.91) --
	(237.38,  8.33) --
	(236.83,  7.74) --
	(236.29,  7.14) --
	(235.76,  6.53) --
	(235.24,  5.91) --
	(234.73,  5.28) --
	(234.23,  4.65) --
	(233.74,  4.00) --
	(233.26,  3.35) --
	(232.80,  2.69) --
	(232.34,  2.02) --
	(231.90,  1.34) --
	(231.47,  0.65) --
	(231.08,  0.00);
\definecolor[named]{drawColor}{rgb}{0.00,0.00,0.00}

\path[draw=drawColor,line width= 0.4pt,dash pattern=on 1pt off 3pt ,line join=round,line cap=round] (252.94, 24.02) --
	(252.36, 23.80) --
	(251.53, 23.47) --
	(250.71, 23.12) --
	(249.90, 22.76) --
	(249.09, 22.39) --
	(248.29, 22.00) --
	(247.50, 21.60) --
	(246.71, 21.19) --
	(245.93, 20.76) --
	(245.16, 20.32) --
	(244.40, 19.86) --
	(243.64, 19.39) --
	(242.89, 18.91) --
	(242.16, 18.41) --
	(241.43, 17.90) --
	(240.71, 17.38) --
	(239.99, 16.85) --
	(239.29, 16.30) --
	(238.60, 15.75) --
	(237.92, 15.18) --
	(237.24, 14.59) --
	(236.58, 14.00) --
	(235.93, 13.40) --
	(235.29, 12.78) --
	(234.66, 12.15) --
	(234.04, 11.51) --
	(233.43, 10.86) --
	(232.83, 10.20) --
	(232.25,  9.53) --
	(231.68,  8.85) --
	(231.12,  8.16) --
	(230.57,  7.46) --
	(230.03,  6.76) --
	(229.51,  6.04) --
	(229.00,  5.31) --
	(228.50,  4.57) --
	(228.01,  3.83) --
	(227.54,  3.07) --
	(227.08,  2.31) --
	(226.63,  1.54) --
	(226.20,  0.77) --
	(225.79,  0.00);

\path[fill=fillColor] (227.52, 51.47) circle (  2.25);
\definecolor[named]{drawColor}{rgb}{0.00,0.00,1.00}

\path[draw=drawColor,line width= 1.2pt,line join=round,line cap=round] (252.94, 90.03) --
	(252.57, 90.28) --
	(251.89, 90.71) --
	(251.20, 91.13) --
	(250.50, 91.54) --
	(249.79, 91.93) --
	(249.08, 92.32) --
	(248.36, 92.69) --
	(247.64, 93.05) --
	(246.91, 93.39) --
	(246.17, 93.73) --
	(245.43, 94.05) --
	(244.68, 94.35) --
	(243.93, 94.65) --
	(243.17, 94.93) --
	(242.41, 95.19) --
	(241.64, 95.45) --
	(240.87, 95.69) --
	(240.09, 95.92) --
	(239.31, 96.13) --
	(238.53, 96.33) --
	(237.74, 96.51) --
	(236.95, 96.69) --
	(236.16, 96.84) --
	(235.36, 96.99) --
	(234.57, 97.12) --
	(233.77, 97.24) --
	(232.96, 97.34) --
	(232.16, 97.43) --
	(231.35, 97.50) --
	(230.55, 97.56) --
	(229.74, 97.61) --
	(228.93, 97.64) --
	(228.13, 97.66) --
	(227.32, 97.66) --
	(226.51, 97.65) --
	(225.70, 97.62) --
	(224.89, 97.59) --
	(224.09, 97.53) --
	(223.28, 97.47) --
	(222.48, 97.38) --
	(221.67, 97.29) --
	(220.87, 97.18) --
	(220.07, 97.06) --
	(219.28, 96.92) --
	(218.48, 96.77) --
	(217.69, 96.60) --
	(216.90, 96.42) --
	(216.12, 96.23) --
	(215.34, 96.02) --
	(214.56, 95.80) --
	(213.78, 95.57) --
	(213.01, 95.32) --
	(212.25, 95.06) --
	(211.49, 94.79) --
	(210.73, 94.50) --
	(209.98, 94.20) --
	(209.24, 93.89) --
	(208.50, 93.56) --
	(207.76, 93.22) --
	(207.03, 92.87) --
	(206.31, 92.51) --
	(205.60, 92.13) --
	(204.89, 91.74) --
	(204.19, 91.34) --
	(203.49, 90.92) --
	(202.81, 90.49) --
	(202.13, 90.06) --
	(201.46, 89.61) --
	(200.79, 89.14) --
	(200.14, 88.67) --
	(199.49, 88.19) --
	(198.85, 87.69) --
	(198.22, 87.18) --
	(197.60, 86.66) --
	(196.99, 86.14) --
	(196.39, 85.60) --
	(195.80, 85.05) --
	(195.21, 84.49) --
	(194.64, 83.91) --
	(194.08, 83.33) --
	(193.52, 82.74) --
	(192.98, 82.14) --
	(192.45, 81.54) --
	(191.93, 80.92) --
	(191.42, 80.29) --
	(190.92, 79.65) --
	(190.43, 79.01) --
	(189.96, 78.36) --
	(189.49, 77.69) --
	(189.04, 77.02) --
	(188.60, 76.35) --
	(188.17, 75.66) --
	(187.75, 74.97) --
	(187.35, 74.27) --
	(186.95, 73.56) --
	(186.57, 72.85) --
	(186.20, 72.13) --
	(185.85, 71.40) --
	(185.51, 70.67) --
	(185.18, 69.93) --
	(184.86, 69.19) --
	(184.56, 68.44) --
	(184.27, 67.69) --
	(183.99, 66.93) --
	(183.72, 66.16) --
	(183.47, 65.39) --
	(183.24, 64.62) --
	(183.01, 63.84) --
	(182.80, 63.06) --
	(182.61, 62.28) --
	(182.43, 61.49) --
	(182.26, 60.70) --
	(182.10, 59.91) --
	(181.96, 59.11) --
	(181.83, 58.31) --
	(181.72, 57.51) --
	(181.62, 56.71) --
	(181.54, 55.90) --
	(181.47, 55.10) --
	(181.41, 54.29) --
	(181.37, 53.49) --
	(181.34, 52.68) --
	(181.33, 51.87) --
	(181.33, 51.06) --
	(181.34, 50.25) --
	(181.37, 49.44) --
	(181.41, 48.64) --
	(181.47, 47.83) --
	(181.54, 47.03) --
	(181.62, 46.22) --
	(181.72, 45.42) --
	(181.83, 44.62) --
	(181.96, 43.82) --
	(182.10, 43.02) --
	(182.26, 42.23) --
	(182.43, 41.44) --
	(182.61, 40.65) --
	(182.80, 39.87) --
	(183.01, 39.09) --
	(183.24, 38.31) --
	(183.47, 37.54) --
	(183.72, 36.77) --
	(183.99, 36.00) --
	(184.27, 35.24) --
	(184.56, 34.49) --
	(184.86, 33.74) --
	(185.18, 33.00) --
	(185.51, 32.26) --
	(185.85, 31.53) --
	(186.20, 30.80) --
	(186.57, 30.08) --
	(186.95, 29.37) --
	(187.35, 28.66) --
	(187.75, 27.96) --
	(188.17, 27.27) --
	(188.60, 26.58) --
	(189.04, 25.91) --
	(189.49, 25.24) --
	(189.96, 24.57) --
	(190.43, 23.92) --
	(190.92, 23.28) --
	(191.42, 22.64) --
	(191.93, 22.01) --
	(192.45, 21.39) --
	(192.98, 20.79) --
	(193.52, 20.19) --
	(194.08, 19.60) --
	(194.64, 19.02) --
	(195.21, 18.44) --
	(195.80, 17.88) --
	(196.39, 17.33) --
	(196.99, 16.79) --
	(197.60, 16.27) --
	(198.22, 15.75) --
	(198.85, 15.24) --
	(199.49, 14.74) --
	(200.14, 14.26) --
	(200.79, 13.79) --
	(201.46, 13.32) --
	(202.13, 12.87) --
	(202.81, 12.44) --
	(203.49, 12.01) --
	(204.19, 11.59) --
	(204.89, 11.19) --
	(205.60, 10.80) --
	(206.31, 10.42) --
	(207.03, 10.06) --
	(207.76,  9.71) --
	(208.50,  9.37) --
	(209.24,  9.04) --
	(209.98,  8.73) --
	(210.73,  8.43) --
	(211.49,  8.14) --
	(212.25,  7.87) --
	(213.01,  7.61) --
	(213.78,  7.36) --
	(214.56,  7.13) --
	(215.34,  6.91) --
	(216.12,  6.70) --
	(216.90,  6.51) --
	(217.69,  6.33) --
	(218.48,  6.16) --
	(219.28,  6.01) --
	(220.07,  5.87) --
	(220.87,  5.75) --
	(221.67,  5.64) --
	(222.48,  5.55) --
	(223.28,  5.46) --
	(224.09,  5.40) --
	(224.89,  5.34) --
	(225.70,  5.31) --
	(226.51,  5.28) --
	(227.32,  5.27) --
	(228.13,  5.27) --
	(228.93,  5.29) --
	(229.74,  5.32) --
	(230.55,  5.37) --
	(231.35,  5.43) --
	(232.16,  5.50) --
	(232.96,  5.59) --
	(233.77,  5.69) --
	(234.57,  5.81) --
	(235.36,  5.94) --
	(236.16,  6.09) --
	(236.95,  6.24) --
	(237.74,  6.42) --
	(238.53,  6.60) --
	(239.31,  6.80) --
	(240.09,  7.01) --
	(240.87,  7.24) --
	(241.64,  7.48) --
	(242.41,  7.74) --
	(243.17,  8.00) --
	(243.93,  8.28) --
	(244.68,  8.58) --
	(245.43,  8.88) --
	(246.17,  9.20) --
	(246.91,  9.54) --
	(247.64,  9.88) --
	(248.36, 10.24) --
	(249.08, 10.61) --
	(249.79, 11.00) --
	(250.50, 11.39) --
	(251.20, 11.80) --
	(251.89, 12.22) --
	(252.57, 12.65) --
	(252.94, 12.90);
\definecolor[named]{drawColor}{rgb}{0.00,0.00,0.00}

\path[draw=drawColor,line width= 0.4pt,dash pattern=on 1pt off 3pt ,line join=round,line cap=round] (252.94, 95.46) --
	(252.80, 95.55) --
	(252.02, 95.98) --
	(251.24, 96.40) --
	(250.45, 96.81) --
	(249.65, 97.21) --
	(248.85, 97.59) --
	(248.04, 97.95) --
	(247.22, 98.30) --
	(246.40, 98.64) --
	(245.57, 98.97) --
	(244.74, 99.27) --
	(243.90, 99.57) --
	(243.05, 99.85) --
	(242.20,100.11) --
	(241.35,100.36) --
	(240.49,100.60) --
	(239.63,100.82) --
	(238.76,101.02) --
	(237.90,101.21) --
	(237.02,101.38) --
	(236.15,101.54) --
	(235.27,101.68) --
	(234.39,101.81) --
	(233.51,101.93) --
	(232.62,102.02) --
	(231.74,102.10) --
	(230.85,102.17) --
	(229.96,102.22) --
	(229.08,102.26) --
	(228.19,102.28) --
	(227.30,102.28) --
	(226.41,102.27) --
	(225.52,102.24) --
	(224.63,102.20) --
	(223.74,102.14) --
	(222.86,102.07) --
	(221.97,101.98) --
	(221.09,101.87) --
	(220.21,101.75) --
	(219.33,101.62) --
	(218.45,101.46) --
	(217.58,101.30) --
	(216.71,101.12) --
	(215.84,100.92) --
	(214.98,100.71) --
	(214.12,100.48) --
	(213.26,100.24) --
	(212.41, 99.98) --
	(211.56, 99.71) --
	(210.72, 99.42) --
	(209.88, 99.12) --
	(209.05, 98.81) --
	(208.23, 98.47) --
	(207.41, 98.13) --
	(206.59, 97.77) --
	(205.79, 97.40) --
	(204.99, 97.01) --
	(204.19, 96.61) --
	(203.41, 96.19) --
	(202.63, 95.77) --
	(201.86, 95.32) --
	(201.09, 94.87) --
	(200.34, 94.40) --
	(199.59, 93.92) --
	(198.85, 93.42) --
	(198.12, 92.91) --
	(197.40, 92.39) --
	(196.69, 91.86) --
	(195.99, 91.31) --
	(195.29, 90.75) --
	(194.61, 90.18) --
	(193.94, 89.60) --
	(193.28, 89.01) --
	(192.62, 88.40) --
	(191.98, 87.79) --
	(191.35, 87.16) --
	(190.73, 86.52) --
	(190.13, 85.87) --
	(189.53, 85.21) --
	(188.94, 84.54) --
	(188.37, 83.86) --
	(187.81, 83.17) --
	(187.26, 82.47) --
	(186.73, 81.76) --
	(186.20, 81.04) --
	(185.69, 80.32) --
	(185.19, 79.58) --
	(184.71, 78.84) --
	(184.23, 78.08) --
	(183.77, 77.32) --
	(183.33, 76.55) --
	(182.90, 75.77) --
	(182.48, 74.99) --
	(182.07, 74.20) --
	(181.68, 73.40) --
	(181.30, 72.59) --
	(180.94, 71.78) --
	(180.59, 70.96) --
	(180.26, 70.14) --
	(179.94, 69.31) --
	(179.64, 68.47) --
	(179.34, 67.63) --
	(179.07, 66.79) --
	(178.81, 65.94) --
	(178.56, 65.08) --
	(178.33, 64.22) --
	(178.12, 63.36) --
	(177.92, 62.49) --
	(177.73, 61.62) --
	(177.56, 60.75) --
	(177.41, 59.87) --
	(177.27, 59.00) --
	(177.14, 58.12) --
	(177.03, 57.23) --
	(176.94, 56.35) --
	(176.86, 55.46) --
	(176.80, 54.58) --
	(176.75, 53.69) --
	(176.72, 52.80) --
	(176.71, 51.91) --
	(176.71, 51.02) --
	(176.72, 50.13) --
	(176.75, 49.24) --
	(176.80, 48.35) --
	(176.86, 47.47) --
	(176.94, 46.58) --
	(177.03, 45.70) --
	(177.14, 44.81) --
	(177.27, 43.93) --
	(177.41, 43.06) --
	(177.56, 42.18) --
	(177.73, 41.31) --
	(177.92, 40.44) --
	(178.12, 39.57) --
	(178.33, 38.71) --
	(178.56, 37.85) --
	(178.81, 36.99) --
	(179.07, 36.14) --
	(179.34, 35.30) --
	(179.64, 34.46) --
	(179.94, 33.62) --
	(180.26, 32.79) --
	(180.59, 31.97) --
	(180.94, 31.15) --
	(181.30, 30.34) --
	(181.68, 29.53) --
	(182.07, 28.73) --
	(182.48, 27.94) --
	(182.90, 27.16) --
	(183.33, 26.38) --
	(183.77, 25.61) --
	(184.23, 24.85) --
	(184.71, 24.09) --
	(185.19, 23.35) --
	(185.69, 22.61) --
	(186.20, 21.89) --
	(186.73, 21.17) --
	(187.26, 20.46) --
	(187.81, 19.76) --
	(188.37, 19.07) --
	(188.94, 18.39) --
	(189.53, 17.72) --
	(190.13, 17.06) --
	(190.73, 16.41) --
	(191.35, 15.77) --
	(191.98, 15.14) --
	(192.62, 14.53) --
	(193.28, 13.92) --
	(193.94, 13.33) --
	(194.61, 12.75) --
	(195.29, 12.18) --
	(195.99, 11.62) --
	(196.69, 11.07) --
	(197.40, 10.54) --
	(198.12, 10.02) --
	(198.85,  9.51) --
	(199.59,  9.01) --
	(200.34,  8.53) --
	(201.09,  8.06) --
	(201.86,  7.61) --
	(202.63,  7.16) --
	(203.41,  6.74) --
	(204.19,  6.32) --
	(204.99,  5.92) --
	(205.79,  5.53) --
	(206.59,  5.16) --
	(207.41,  4.80) --
	(208.23,  4.46) --
	(209.05,  4.12) --
	(209.88,  3.81) --
	(210.72,  3.51) --
	(211.56,  3.22) --
	(212.41,  2.95) --
	(213.26,  2.69) --
	(214.12,  2.45) --
	(214.98,  2.22) --
	(215.84,  2.01) --
	(216.71,  1.81) --
	(217.58,  1.63) --
	(218.45,  1.47) --
	(219.33,  1.32) --
	(220.21,  1.18) --
	(221.09,  1.06) --
	(221.97,  0.95) --
	(222.86,  0.86) --
	(223.74,  0.79) --
	(224.63,  0.73) --
	(225.52,  0.69) --
	(226.41,  0.66) --
	(227.30,  0.65) --
	(228.19,  0.65) --
	(229.08,  0.67) --
	(229.96,  0.71) --
	(230.85,  0.76) --
	(231.74,  0.83) --
	(232.62,  0.91) --
	(233.51,  1.00) --
	(234.39,  1.12) --
	(235.27,  1.25) --
	(236.15,  1.39) --
	(237.02,  1.55) --
	(237.90,  1.72) --
	(238.76,  1.91) --
	(239.63,  2.11) --
	(240.49,  2.33) --
	(241.35,  2.57) --
	(242.20,  2.82) --
	(243.05,  3.08) --
	(243.90,  3.36) --
	(244.74,  3.66) --
	(245.57,  3.96) --
	(246.40,  4.29) --
	(247.22,  4.63) --
	(248.04,  4.98) --
	(248.85,  5.34) --
	(249.65,  5.72) --
	(250.45,  6.12) --
	(251.24,  6.53) --
	(252.02,  6.95) --
	(252.80,  7.38) --
	(252.94,  7.47);

\path[fill=fillColor] (184.21,126.47) circle (  2.25);
\definecolor[named]{drawColor}{rgb}{0.00,0.00,1.00}

\path[draw=drawColor,line width= 1.2pt,line join=round,line cap=round] (230.41,126.47) --
	(230.40,127.28) --
	(230.38,128.09) --
	(230.34,128.90) --
	(230.30,129.70) --
	(230.23,130.51) --
	(230.15,131.31) --
	(230.06,132.12) --
	(229.96,132.92) --
	(229.84,133.72) --
	(229.70,134.52) --
	(229.55,135.31) --
	(229.39,136.10) --
	(229.22,136.89) --
	(229.03,137.68) --
	(228.83,138.46) --
	(228.61,139.24) --
	(228.38,140.02) --
	(228.13,140.79) --
	(227.88,141.55) --
	(227.61,142.31) --
	(227.32,143.07) --
	(227.03,143.82) --
	(226.72,144.57) --
	(226.39,145.31) --
	(226.06,146.05) --
	(225.71,146.78) --
	(225.35,147.50) --
	(224.97,148.22) --
	(224.58,148.93) --
	(224.19,149.63) --
	(223.77,150.32) --
	(223.35,151.01) --
	(222.92,151.69) --
	(222.47,152.37) --
	(222.01,153.03) --
	(221.54,153.69) --
	(221.06,154.34) --
	(220.56,154.98) --
	(220.06,155.61) --
	(219.54,156.23) --
	(219.02,156.85) --
	(218.48,157.45) --
	(217.93,158.05) --
	(217.37,158.63) --
	(216.81,159.21) --
	(216.23,159.77) --
	(215.64,160.33) --
	(215.04,160.87) --
	(214.44,161.41) --
	(213.82,161.93) --
	(213.20,162.44) --
	(212.56,162.95) --
	(211.92,163.44) --
	(211.27,163.92) --
	(210.61,164.38) --
	(209.94,164.84) --
	(209.27,165.28) --
	(208.58,165.72) --
	(207.89,166.14) --
	(207.19,166.55) --
	(206.49,166.94) --
	(205.78,167.33) --
	(205.06,167.70) --
	(204.33,168.06) --
	(203.60,168.40) --
	(202.87,168.73) --
	(202.12,169.05) --
	(201.38,169.36) --
	(200.62,169.65) --
	(199.87,169.94) --
	(199.10,170.20) --
	(198.33,170.46) --
	(197.56,170.70) --
	(196.79,170.92) --
	(196.01,171.14) --
	(195.22,171.34) --
	(194.44,171.52) --
	(193.65,171.69) --
	(192.85,171.85) --
	(192.06,172.00) --
	(191.26,172.13) --
	(190.46,172.24) --
	(189.66,172.35) --
	(188.85,172.43) --
	(188.05,172.51) --
	(187.24,172.57) --
	(186.44,172.61) --
	(185.63,172.65) --
	(184.82,172.66) --
	(184.01,172.67) --
	(183.20,172.66) --
	(182.39,172.63) --
	(181.59,172.59) --
	(180.78,172.54) --
	(179.97,172.47) --
	(179.17,172.39) --
	(178.37,172.30) --
	(177.57,172.19) --
	(176.77,172.06) --
	(175.97,171.93) --
	(175.18,171.77) --
	(174.38,171.61) --
	(173.60,171.43) --
	(172.81,171.24) --
	(172.03,171.03) --
	(171.25,170.81) --
	(170.48,170.58) --
	(169.71,170.33) --
	(168.94,170.07) --
	(168.18,169.80) --
	(167.43,169.51) --
	(166.68,169.21) --
	(165.93,168.90) --
	(165.19,168.57) --
	(164.46,168.23) --
	(163.73,167.88) --
	(163.01,167.51) --
	(162.29,167.14) --
	(161.58,166.75) --
	(160.88,166.34) --
	(160.19,165.93) --
	(159.50,165.50) --
	(158.82,165.06) --
	(158.15,164.61) --
	(157.49,164.15) --
	(156.83,163.68) --
	(156.19,163.19) --
	(155.55,162.70) --
	(154.92,162.19) --
	(154.30,161.67) --
	(153.69,161.14) --
	(153.08,160.60) --
	(152.49,160.05) --
	(151.91,159.49) --
	(151.33,158.92) --
	(150.77,158.34) --
	(150.22,157.75) --
	(149.68,157.15) --
	(149.15,156.54) --
	(148.62,155.92) --
	(148.11,155.30) --
	(147.62,154.66) --
	(147.13,154.02) --
	(146.65,153.36) --
	(146.19,152.70) --
	(145.73,152.03) --
	(145.29,151.35) --
	(144.86,150.67) --
	(144.45,149.98) --
	(144.04,149.28) --
	(143.65,148.57) --
	(143.27,147.86) --
	(142.90,147.14) --
	(142.54,146.41) --
	(142.20,145.68) --
	(141.87,144.94) --
	(141.55,144.20) --
	(141.25,143.45) --
	(140.96,142.69) --
	(140.68,141.93) --
	(140.42,141.17) --
	(140.17,140.40) --
	(139.93,139.63) --
	(139.71,138.85) --
	(139.50,138.07) --
	(139.30,137.29) --
	(139.12,136.50) --
	(138.95,135.71) --
	(138.80,134.91) --
	(138.66,134.12) --
	(138.53,133.32) --
	(138.42,132.52) --
	(138.32,131.72) --
	(138.23,130.91) --
	(138.16,130.11) --
	(138.10,129.30) --
	(138.06,128.49) --
	(138.03,127.69) --
	(138.02,126.88) --
	(138.02,126.07) --
	(138.03,125.26) --
	(138.06,124.45) --
	(138.10,123.64) --
	(138.16,122.84) --
	(138.23,122.03) --
	(138.32,121.23) --
	(138.42,120.43) --
	(138.53,119.63) --
	(138.66,118.83) --
	(138.80,118.03) --
	(138.95,117.24) --
	(139.12,116.45) --
	(139.30,115.66) --
	(139.50,114.87) --
	(139.71,114.09) --
	(139.93,113.32) --
	(140.17,112.54) --
	(140.42,111.78) --
	(140.68,111.01) --
	(140.96,110.25) --
	(141.25,109.50) --
	(141.55,108.75) --
	(141.87,108.00) --
	(142.20,107.27) --
	(142.54,106.53) --
	(142.90,105.81) --
	(143.27,105.09) --
	(143.65,104.37) --
	(144.04,103.67) --
	(144.45,102.97) --
	(144.86,102.28) --
	(145.29,101.59) --
	(145.73,100.91) --
	(146.19,100.24) --
	(146.65, 99.58) --
	(147.13, 98.93) --
	(147.62, 98.28) --
	(148.11, 97.65) --
	(148.62, 97.02) --
	(149.15, 96.40) --
	(149.68, 95.79) --
	(150.22, 95.19) --
	(150.77, 94.60) --
	(151.33, 94.02) --
	(151.91, 93.45) --
	(152.49, 92.89) --
	(153.08, 92.34) --
	(153.69, 91.80) --
	(154.30, 91.27) --
	(154.92, 90.76) --
	(155.55, 90.25) --
	(156.19, 89.75) --
	(156.83, 89.27) --
	(157.49, 88.79) --
	(158.15, 88.33) --
	(158.82, 87.88) --
	(159.50, 87.44) --
	(160.19, 87.02) --
	(160.88, 86.60) --
	(161.58, 86.20) --
	(162.29, 85.81) --
	(163.01, 85.43) --
	(163.73, 85.07) --
	(164.46, 84.72) --
	(165.19, 84.38) --
	(165.93, 84.05) --
	(166.68, 83.74) --
	(167.43, 83.44) --
	(168.18, 83.15) --
	(168.94, 82.87) --
	(169.71, 82.61) --
	(170.48, 82.37) --
	(171.25, 82.13) --
	(172.03, 81.91) --
	(172.81, 81.71) --
	(173.60, 81.51) --
	(174.38, 81.34) --
	(175.18, 81.17) --
	(175.97, 81.02) --
	(176.77, 80.88) --
	(177.57, 80.76) --
	(178.37, 80.65) --
	(179.17, 80.55) --
	(179.97, 80.47) --
	(180.78, 80.41) --
	(181.59, 80.35) --
	(182.39, 80.31) --
	(183.20, 80.29) --
	(184.01, 80.28) --
	(184.82, 80.28) --
	(185.63, 80.30) --
	(186.44, 80.33) --
	(187.24, 80.38) --
	(188.05, 80.44) --
	(188.85, 80.51) --
	(189.66, 80.60) --
	(190.46, 80.70) --
	(191.26, 80.82) --
	(192.06, 80.95) --
	(192.85, 81.09) --
	(193.65, 81.25) --
	(194.44, 81.42) --
	(195.22, 81.61) --
	(196.01, 81.81) --
	(196.79, 82.02) --
	(197.56, 82.25) --
	(198.33, 82.49) --
	(199.10, 82.74) --
	(199.87, 83.01) --
	(200.62, 83.29) --
	(201.38, 83.58) --
	(202.12, 83.89) --
	(202.87, 84.21) --
	(203.60, 84.54) --
	(204.33, 84.89) --
	(205.06, 85.25) --
	(205.78, 85.62) --
	(206.49, 86.00) --
	(207.19, 86.40) --
	(207.89, 86.81) --
	(208.58, 87.23) --
	(209.27, 87.66) --
	(209.94, 88.10) --
	(210.61, 88.56) --
	(211.27, 89.03) --
	(211.92, 89.51) --
	(212.56, 90.00) --
	(213.20, 90.50) --
	(213.82, 91.01) --
	(214.44, 91.54) --
	(215.04, 92.07) --
	(215.64, 92.62) --
	(216.23, 93.17) --
	(216.81, 93.74) --
	(217.37, 94.31) --
	(217.93, 94.90) --
	(218.48, 95.49) --
	(219.02, 96.10) --
	(219.54, 96.71) --
	(220.06, 97.33) --
	(220.56, 97.96) --
	(221.06, 98.61) --
	(221.54, 99.25) --
	(222.01, 99.91) --
	(222.47,100.58) --
	(222.92,101.25) --
	(223.35,101.93) --
	(223.77,102.62) --
	(224.19,103.32) --
	(224.58,104.02) --
	(224.97,104.73) --
	(225.35,105.45) --
	(225.71,106.17) --
	(226.06,106.90) --
	(226.39,107.63) --
	(226.72,108.38) --
	(227.03,109.12) --
	(227.32,109.87) --
	(227.61,110.63) --
	(227.88,111.39) --
	(228.13,112.16) --
	(228.38,112.93) --
	(228.61,113.70) --
	(228.83,114.48) --
	(229.03,115.27) --
	(229.22,116.05) --
	(229.39,116.84) --
	(229.55,117.63) --
	(229.70,118.43) --
	(229.84,119.23) --
	(229.96,120.03) --
	(230.06,120.83) --
	(230.15,121.63) --
	(230.23,122.44) --
	(230.30,123.24) --
	(230.34,124.05) --
	(230.38,124.86) --
	(230.40,125.66) --
	(230.41,126.47);
\definecolor[named]{drawColor}{rgb}{0.00,0.00,0.00}

\path[draw=drawColor,line width= 0.4pt,dash pattern=on 1pt off 3pt ,line join=round,line cap=round] (235.03,126.47) --
	(235.02,127.36) --
	(235.00,128.25) --
	(234.96,129.14) --
	(234.90,130.03) --
	(234.83,130.91) --
	(234.75,131.80) --
	(234.65,132.68) --
	(234.53,133.56) --
	(234.40,134.44) --
	(234.25,135.32) --
	(234.09,136.20) --
	(233.91,137.07) --
	(233.72,137.93) --
	(233.51,138.80) --
	(233.29,139.66) --
	(233.05,140.52) --
	(232.80,141.37) --
	(232.53,142.22) --
	(232.24,143.06) --
	(231.95,143.90) --
	(231.63,144.73) --
	(231.31,145.56) --
	(230.97,146.38) --
	(230.61,147.19) --
	(230.24,148.00) --
	(229.86,148.81) --
	(229.46,149.60) --
	(229.05,150.39) --
	(228.62,151.17) --
	(228.18,151.94) --
	(227.73,152.71) --
	(227.26,153.47) --
	(226.79,154.22) --
	(226.29,154.96) --
	(225.79,155.69) --
	(225.27,156.41) --
	(224.74,157.13) --
	(224.20,157.83) --
	(223.64,158.53) --
	(223.08,159.21) --
	(222.50,159.89) --
	(221.91,160.55) --
	(221.30,161.21) --
	(220.69,161.85) --
	(220.07,162.48) --
	(219.43,163.10) --
	(218.78,163.71) --
	(218.13,164.31) --
	(217.46,164.90) --
	(216.78,165.48) --
	(216.09,166.04) --
	(215.40,166.59) --
	(214.69,167.13) --
	(213.97,167.66) --
	(213.25,168.18) --
	(212.51,168.68) --
	(211.77,169.17) --
	(211.02,169.64) --
	(210.26,170.10) --
	(209.49,170.55) --
	(208.72,170.99) --
	(207.93,171.41) --
	(207.14,171.82) --
	(206.35,172.21) --
	(205.54,172.59) --
	(204.73,172.96) --
	(203.92,173.31) --
	(203.09,173.65) --
	(202.26,173.97) --
	(201.43,174.28) --
	(200.59,174.58) --
	(199.75,174.85) --
	(198.90,175.12) --
	(198.04,175.37) --
	(197.19,175.60) --
	(196.32,175.82) --
	(195.46,176.03) --
	(194.59,176.22) --
	(193.72,176.39) --
	(192.84,176.55) --
	(191.96,176.69) --
	(191.08,176.82) --
	(190.20,176.93) --
	(189.32,177.03) --
	(188.43,177.11) --
	(187.55,177.18) --
	(186.66,177.23) --
	(185.77,177.26) --
	(184.88,177.28) --
	(183.99,177.29) --
	(183.10,177.27) --
	(182.21,177.25) --
	(181.32,177.20) --
	(180.44,177.15) --
	(179.55,177.07) --
	(178.67,176.98) --
	(177.78,176.88) --
	(176.90,176.76) --
	(176.02,176.62) --
	(175.15,176.47) --
	(174.27,176.31) --
	(173.40,176.12) --
	(172.53,175.93) --
	(171.67,175.71) --
	(170.81,175.49) --
	(169.96,175.25) --
	(169.10,174.99) --
	(168.26,174.72) --
	(167.42,174.43) --
	(166.58,174.13) --
	(165.75,173.81) --
	(164.92,173.48) --
	(164.10,173.14) --
	(163.29,172.78) --
	(162.48,172.41) --
	(161.68,172.02) --
	(160.89,171.62) --
	(160.10,171.20) --
	(159.32,170.77) --
	(158.55,170.33) --
	(157.79,169.87) --
	(157.03,169.41) --
	(156.28,168.92) --
	(155.54,168.43) --
	(154.81,167.92) --
	(154.09,167.40) --
	(153.38,166.87) --
	(152.68,166.32) --
	(151.99,165.76) --
	(151.31,165.19) --
	(150.63,164.61) --
	(149.97,164.02) --
	(149.32,163.41) --
	(148.68,162.79) --
	(148.05,162.17) --
	(147.43,161.53) --
	(146.82,160.88) --
	(146.22,160.22) --
	(145.64,159.55) --
	(145.07,158.87) --
	(144.50,158.18) --
	(143.96,157.48) --
	(143.42,156.77) --
	(142.90,156.05) --
	(142.38,155.32) --
	(141.89,154.59) --
	(141.40,153.84) --
	(140.93,153.09) --
	(140.47,152.33) --
	(140.02,151.56) --
	(139.59,150.78) --
	(139.17,150.00) --
	(138.77,149.20) --
	(138.38,148.41) --
	(138.00,147.60) --
	(137.64,146.79) --
	(137.29,145.97) --
	(136.95,145.15) --
	(136.63,144.32) --
	(136.33,143.48) --
	(136.04,142.64) --
	(135.76,141.79) --
	(135.50,140.94) --
	(135.26,140.09) --
	(135.03,139.23) --
	(134.81,138.37) --
	(134.61,137.50) --
	(134.42,136.63) --
	(134.25,135.76) --
	(134.10,134.88) --
	(133.96,134.00) --
	(133.84,133.12) --
	(133.73,132.24) --
	(133.63,131.36) --
	(133.56,130.47) --
	(133.49,129.58) --
	(133.45,128.70) --
	(133.42,127.81) --
	(133.40,126.92) --
	(133.40,126.03) --
	(133.42,125.14) --
	(133.45,124.25) --
	(133.49,123.36) --
	(133.56,122.47) --
	(133.63,121.59) --
	(133.73,120.70) --
	(133.84,119.82) --
	(133.96,118.94) --
	(134.10,118.06) --
	(134.25,117.19) --
	(134.42,116.31) --
	(134.61,115.44) --
	(134.81,114.58) --
	(135.03,113.71) --
	(135.26,112.86) --
	(135.50,112.00) --
	(135.76,111.15) --
	(136.04,110.31) --
	(136.33,109.46) --
	(136.63,108.63) --
	(136.95,107.80) --
	(137.29,106.98) --
	(137.64,106.16) --
	(138.00,105.35) --
	(138.38,104.54) --
	(138.77,103.74) --
	(139.17,102.95) --
	(139.59,102.16) --
	(140.02,101.39) --
	(140.47,100.62) --
	(140.93, 99.86) --
	(141.40, 99.10) --
	(141.89, 98.36) --
	(142.38, 97.62) --
	(142.90, 96.89) --
	(143.42, 96.17) --
	(143.96, 95.47) --
	(144.50, 94.77) --
	(145.07, 94.08) --
	(145.64, 93.40) --
	(146.22, 92.73) --
	(146.82, 92.07) --
	(147.43, 91.42) --
	(148.05, 90.78) --
	(148.68, 90.15) --
	(149.32, 89.53) --
	(149.97, 88.93) --
	(150.63, 88.34) --
	(151.31, 87.75) --
	(151.99, 87.18) --
	(152.68, 86.63) --
	(153.38, 86.08) --
	(154.09, 85.55) --
	(154.81, 85.03) --
	(155.54, 84.52) --
	(156.28, 84.02) --
	(157.03, 83.54) --
	(157.79, 83.07) --
	(158.55, 82.61) --
	(159.32, 82.17) --
	(160.10, 81.74) --
	(160.89, 81.33) --
	(161.68, 80.93) --
	(162.48, 80.54) --
	(163.29, 80.17) --
	(164.10, 79.81) --
	(164.92, 79.46) --
	(165.75, 79.13) --
	(166.58, 78.82) --
	(167.42, 78.51) --
	(168.26, 78.23) --
	(169.10, 77.96) --
	(169.96, 77.70) --
	(170.81, 77.46) --
	(171.67, 77.23) --
	(172.53, 77.02) --
	(173.40, 76.82) --
	(174.27, 76.64) --
	(175.15, 76.47) --
	(176.02, 76.32) --
	(176.90, 76.19) --
	(177.78, 76.07) --
	(178.67, 75.96) --
	(179.55, 75.87) --
	(180.44, 75.80) --
	(181.32, 75.74) --
	(182.21, 75.70) --
	(183.10, 75.67) --
	(183.99, 75.66) --
	(184.88, 75.66) --
	(185.77, 75.68) --
	(186.66, 75.72) --
	(187.55, 75.77) --
	(188.43, 75.83) --
	(189.32, 75.92) --
	(190.20, 76.01) --
	(191.08, 76.12) --
	(191.96, 76.25) --
	(192.84, 76.40) --
	(193.72, 76.55) --
	(194.59, 76.73) --
	(195.46, 76.92) --
	(196.32, 77.12) --
	(197.19, 77.34) --
	(198.04, 77.58) --
	(198.90, 77.83) --
	(199.75, 78.09) --
	(200.59, 78.37) --
	(201.43, 78.66) --
	(202.26, 78.97) --
	(203.09, 79.30) --
	(203.92, 79.63) --
	(204.73, 79.99) --
	(205.54, 80.35) --
	(206.35, 80.73) --
	(207.14, 81.13) --
	(207.93, 81.53) --
	(208.72, 81.96) --
	(209.49, 82.39) --
	(210.26, 82.84) --
	(211.02, 83.30) --
	(211.77, 83.78) --
	(212.51, 84.27) --
	(213.25, 84.77) --
	(213.97, 85.28) --
	(214.69, 85.81) --
	(215.40, 86.35) --
	(216.09, 86.90) --
	(216.78, 87.47) --
	(217.46, 88.04) --
	(218.13, 88.63) --
	(218.78, 89.23) --
	(219.43, 89.84) --
	(220.07, 90.46) --
	(220.69, 91.10) --
	(221.30, 91.74) --
	(221.91, 92.39) --
	(222.50, 93.06) --
	(223.08, 93.73) --
	(223.64, 94.42) --
	(224.20, 95.11) --
	(224.74, 95.82) --
	(225.27, 96.53) --
	(225.79, 97.26) --
	(226.29, 97.99) --
	(226.79, 98.73) --
	(227.26, 99.48) --
	(227.73,100.24) --
	(228.18,101.00) --
	(228.62,101.77) --
	(229.05,102.56) --
	(229.46,103.34) --
	(229.86,104.14) --
	(230.24,104.94) --
	(230.61,105.75) --
	(230.97,106.57) --
	(231.31,107.39) --
	(231.63,108.21) --
	(231.95,109.05) --
	(232.24,109.88) --
	(232.53,110.73) --
	(232.80,111.58) --
	(233.05,112.43) --
	(233.29,113.28) --
	(233.51,114.15) --
	(233.72,115.01) --
	(233.91,115.88) --
	(234.09,116.75) --
	(234.25,117.62) --
	(234.40,118.50) --
	(234.53,119.38) --
	(234.65,120.26) --
	(234.75,121.15) --
	(234.83,122.03) --
	(234.90,122.92) --
	(234.96,123.81) --
	(235.00,124.69) --
	(235.02,125.58) --
	(235.03,126.47);

\path[fill=fillColor] (140.91,201.48) circle (  2.25);
\definecolor[named]{drawColor}{rgb}{0.00,0.00,1.00}

\path[draw=drawColor,line width= 1.2pt,line join=round,line cap=round] (187.10,201.48) --
	(187.10,202.29) --
	(187.07,203.10) --
	(187.04,203.90) --
	(186.99,204.71) --
	(186.93,205.52) --
	(186.85,206.32) --
	(186.76,207.13) --
	(186.65,207.93) --
	(186.53,208.73) --
	(186.40,209.52) --
	(186.25,210.32) --
	(186.09,211.11) --
	(185.91,211.90) --
	(185.72,212.69) --
	(185.52,213.47) --
	(185.30,214.25) --
	(185.07,215.02) --
	(184.83,215.79) --
	(184.57,216.56) --
	(184.30,217.32) --
	(184.02,218.08) --
	(183.72,218.83) --
	(183.41,219.58) --
	(183.09,220.32) --
	(182.75,221.05) --
	(182.40,221.78) --
	(182.04,222.51) --
	(181.67,223.22) --
	(181.28,223.93) --
	(180.88,224.64) --
	(180.47,225.33) --
	(180.05,226.02) --
	(179.61,226.70) --
	(179.16,227.38) --
	(178.70,228.04) --
	(178.23,228.70) --
	(177.75,229.35) --
	(177.26,229.99) --
	(176.75,230.62) --
	(176.24,231.24) --
	(175.71,231.86) --
	(175.17,232.46) --
	(174.63,233.06) --
	(174.07,233.64) --
	(173.50,234.22) --
	(172.92,234.78) --
	(172.34,235.34) --
	(171.74,235.88) --
	(171.13,236.42) --
	(170.52,236.94) --
	(169.89,237.45) --
	(169.26,237.95) --
	(168.61,238.44) --
	(167.96,238.92) --
	(167.30,239.39) --
	(166.64,239.85) --
	(165.96,240.29) --
	(165.28,240.72) --
	(164.59,241.14) --
	(163.89,241.55) --
	(163.18,241.95) --
	(162.47,242.33) --
	(161.75,242.70) --
	(161.03,243.06) --
	(160.30,243.41) --
	(159.56,243.74) --
	(158.82,244.06) --
	(158.07,244.37) --
	(157.32,244.66) --
	(156.56,244.94) --
	(155.80,245.21) --
	(155.03,245.46) --
	(154.26,245.70) --
	(153.48,245.93) --
	(152.70,246.14) --
	(151.92,246.34) --
	(151.13,246.53) --
	(150.34,246.70) --
	(149.55,246.86) --
	(148.75,247.00) --
	(147.95,247.13) --
	(147.15,247.25) --
	(146.35,247.35) --
	(145.55,247.44) --
	(144.74,247.52) --
	(143.94,247.58) --
	(143.13,247.62) --
	(142.32,247.65) --
	(141.51,247.67) --
	(140.71,247.67) --
	(139.90,247.66) --
	(139.09,247.64) --
	(138.28,247.60) --
	(137.47,247.55) --
	(136.67,247.48) --
	(135.86,247.40) --
	(135.06,247.30) --
	(134.26,247.19) --
	(133.46,247.07) --
	(132.66,246.93) --
	(131.87,246.78) --
	(131.08,246.62) --
	(130.29,246.44) --
	(129.51,246.25) --
	(128.72,246.04) --
	(127.95,245.82) --
	(127.17,245.59) --
	(126.40,245.34) --
	(125.64,245.08) --
	(124.88,244.80) --
	(124.12,244.52) --
	(123.37,244.22) --
	(122.62,243.90) --
	(121.88,243.58) --
	(121.15,243.24) --
	(120.42,242.88) --
	(119.70,242.52) --
	(118.99,242.14) --
	(118.28,241.75) --
	(117.58,241.35) --
	(116.88,240.94) --
	(116.20,240.51) --
	(115.52,240.07) --
	(114.85,239.62) --
	(114.18,239.16) --
	(113.53,238.69) --
	(112.88,238.20) --
	(112.24,237.70) --
	(111.61,237.20) --
	(110.99,236.68) --
	(110.38,236.15) --
	(109.78,235.61) --
	(109.19,235.06) --
	(108.60,234.50) --
	(108.03,233.93) --
	(107.47,233.35) --
	(106.91,232.76) --
	(106.37,232.16) --
	(105.84,231.55) --
	(105.32,230.93) --
	(104.81,230.30) --
	(104.31,229.67) --
	(103.82,229.02) --
	(103.35,228.37) --
	(102.88,227.71) --
	(102.43,227.04) --
	(101.99,226.36) --
	(101.56,225.68) --
	(101.14,224.98) --
	(100.73,224.29) --
	(100.34,223.58) --
	( 99.96,222.87) --
	( 99.59,222.15) --
	( 99.24,221.42) --
	( 98.89,220.69) --
	( 98.57,219.95) --
	( 98.25,219.20) --
	( 97.94,218.46) --
	( 97.65,217.70) --
	( 97.38,216.94) --
	( 97.11,216.18) --
	( 96.86,215.41) --
	( 96.63,214.64) --
	( 96.40,213.86) --
	( 96.19,213.08) --
	( 96.00,212.29) --
	( 95.81,211.51) --
	( 95.65,210.72) --
	( 95.49,209.92) --
	( 95.35,209.13) --
	( 95.22,208.33) --
	( 95.11,207.53) --
	( 95.01,206.72) --
	( 94.93,205.92) --
	( 94.86,205.11) --
	( 94.80,204.31) --
	( 94.76,203.50) --
	( 94.73,202.69) --
	( 94.71,201.88) --
	( 94.71,201.08) --
	( 94.73,200.27) --
	( 94.76,199.46) --
	( 94.80,198.65) --
	( 94.86,197.85) --
	( 94.93,197.04) --
	( 95.01,196.24) --
	( 95.11,195.43) --
	( 95.22,194.63) --
	( 95.35,193.83) --
	( 95.49,193.04) --
	( 95.65,192.24) --
	( 95.81,191.45) --
	( 96.00,190.67) --
	( 96.19,189.88) --
	( 96.40,189.10) --
	( 96.63,188.32) --
	( 96.86,187.55) --
	( 97.11,186.78) --
	( 97.38,186.02) --
	( 97.65,185.26) --
	( 97.94,184.50) --
	( 98.25,183.76) --
	( 98.57,183.01) --
	( 98.89,182.27) --
	( 99.24,181.54) --
	( 99.59,180.81) --
	( 99.96,180.09) --
	(100.34,179.38) --
	(100.73,178.67) --
	(101.14,177.98) --
	(101.56,177.28) --
	(101.99,176.60) --
	(102.43,175.92) --
	(102.88,175.25) --
	(103.35,174.59) --
	(103.82,173.94) --
	(104.31,173.29) --
	(104.81,172.66) --
	(105.32,172.03) --
	(105.84,171.41) --
	(106.37,170.80) --
	(106.91,170.20) --
	(107.47,169.61) --
	(108.03,169.03) --
	(108.60,168.46) --
	(109.19,167.90) --
	(109.78,167.35) --
	(110.38,166.81) --
	(110.99,166.28) --
	(111.61,165.76) --
	(112.24,165.26) --
	(112.88,164.76) --
	(113.53,164.27) --
	(114.18,163.80) --
	(114.85,163.34) --
	(115.52,162.89) --
	(116.20,162.45) --
	(116.88,162.02) --
	(117.58,161.61) --
	(118.28,161.21) --
	(118.99,160.82) --
	(119.70,160.44) --
	(120.42,160.08) --
	(121.15,159.72) --
	(121.88,159.38) --
	(122.62,159.06) --
	(123.37,158.74) --
	(124.12,158.44) --
	(124.88,158.16) --
	(125.64,157.88) --
	(126.40,157.62) --
	(127.17,157.37) --
	(127.95,157.14) --
	(128.72,156.92) --
	(129.51,156.71) --
	(130.29,156.52) --
	(131.08,156.34) --
	(131.87,156.18) --
	(132.66,156.03) --
	(133.46,155.89) --
	(134.26,155.77) --
	(135.06,155.66) --
	(135.86,155.56) --
	(136.67,155.48) --
	(137.47,155.41) --
	(138.28,155.36) --
	(139.09,155.32) --
	(139.90,155.30) --
	(140.71,155.29) --
	(141.51,155.29) --
	(142.32,155.31) --
	(143.13,155.34) --
	(143.94,155.38) --
	(144.74,155.44) --
	(145.55,155.52) --
	(146.35,155.61) --
	(147.15,155.71) --
	(147.95,155.83) --
	(148.75,155.96) --
	(149.55,156.10) --
	(150.34,156.26) --
	(151.13,156.43) --
	(151.92,156.62) --
	(152.70,156.82) --
	(153.48,157.03) --
	(154.26,157.26) --
	(155.03,157.50) --
	(155.80,157.75) --
	(156.56,158.02) --
	(157.32,158.30) --
	(158.07,158.59) --
	(158.82,158.90) --
	(159.56,159.22) --
	(160.30,159.55) --
	(161.03,159.90) --
	(161.75,160.26) --
	(162.47,160.63) --
	(163.18,161.01) --
	(163.89,161.41) --
	(164.59,161.82) --
	(165.28,162.24) --
	(165.96,162.67) --
	(166.64,163.11) --
	(167.30,163.57) --
	(167.96,164.04) --
	(168.61,164.52) --
	(169.26,165.01) --
	(169.89,165.51) --
	(170.52,166.02) --
	(171.13,166.54) --
	(171.74,167.08) --
	(172.34,167.62) --
	(172.92,168.18) --
	(173.50,168.74) --
	(174.07,169.32) --
	(174.63,169.90) --
	(175.17,170.50) --
	(175.71,171.10) --
	(176.24,171.72) --
	(176.75,172.34) --
	(177.26,172.97) --
	(177.75,173.61) --
	(178.23,174.26) --
	(178.70,174.92) --
	(179.16,175.58) --
	(179.61,176.26) --
	(180.05,176.94) --
	(180.47,177.63) --
	(180.88,178.32) --
	(181.28,179.03) --
	(181.67,179.74) --
	(182.04,180.45) --
	(182.40,181.18) --
	(182.75,181.91) --
	(183.09,182.64) --
	(183.41,183.38) --
	(183.72,184.13) --
	(184.02,184.88) --
	(184.30,185.64) --
	(184.57,186.40) --
	(184.83,187.17) --
	(185.07,187.94) --
	(185.30,188.71) --
	(185.52,189.49) --
	(185.72,190.27) --
	(185.91,191.06) --
	(186.09,191.85) --
	(186.25,192.64) --
	(186.40,193.44) --
	(186.53,194.23) --
	(186.65,195.03) --
	(186.76,195.83) --
	(186.85,196.64) --
	(186.93,197.44) --
	(186.99,198.25) --
	(187.04,199.06) --
	(187.07,199.86) --
	(187.10,200.67) --
	(187.10,201.48);
\definecolor[named]{drawColor}{rgb}{0.00,0.00,0.00}

\path[draw=drawColor,line width= 0.4pt,dash pattern=on 1pt off 3pt ,line join=round,line cap=round] (191.72,201.48) --
	(191.71,202.37) --
	(191.69,203.26) --
	(191.65,204.15) --
	(191.60,205.03) --
	(191.53,205.92) --
	(191.44,206.81) --
	(191.34,207.69) --
	(191.22,208.57) --
	(191.09,209.45) --
	(190.95,210.33) --
	(190.78,211.20) --
	(190.61,212.07) --
	(190.41,212.94) --
	(190.20,213.81) --
	(189.98,214.67) --
	(189.74,215.52) --
	(189.49,216.38) --
	(189.22,217.22) --
	(188.94,218.07) --
	(188.64,218.91) --
	(188.33,219.74) --
	(188.00,220.57) --
	(187.66,221.39) --
	(187.30,222.20) --
	(186.94,223.01) --
	(186.55,223.81) --
	(186.15,224.61) --
	(185.74,225.40) --
	(185.32,226.18) --
	(184.88,226.95) --
	(184.42,227.72) --
	(183.96,228.47) --
	(183.48,229.22) --
	(182.99,229.96) --
	(182.48,230.70) --
	(181.97,231.42) --
	(181.43,232.13) --
	(180.89,232.84) --
	(180.34,233.53) --
	(179.77,234.22) --
	(179.19,234.89) --
	(178.60,235.56) --
	(178.00,236.21) --
	(177.38,236.86) --
	(176.76,237.49) --
	(176.12,238.11) --
	(175.48,238.72) --
	(174.82,239.32) --
	(174.15,239.91) --
	(173.48,240.49) --
	(172.79,241.05) --
	(172.09,241.60) --
	(171.38,242.14) --
	(170.67,242.67) --
	(169.94,243.18) --
	(169.21,243.68) --
	(168.47,244.17) --
	(167.71,244.65) --
	(166.95,245.11) --
	(166.19,245.56) --
	(165.41,246.00) --
	(164.63,246.42) --
	(163.84,246.83) --
	(163.04,247.22) --
	(162.24,247.60) --
	(161.43,247.97) --
	(160.61,248.32) --
	(159.79,248.66) --
	(158.96,248.98) --
	(158.12,249.29) --
	(157.29,249.58) --
	(156.44,249.86) --
	(155.59,250.13) --
	(154.74,250.38) --
	(153.88,250.61) --
	(153.02,250.83) --
	(152.15,251.03) --
	(151.28,251.22) --
	(150.41,251.40) --
	(149.54,251.56) --
	(148.66,251.70) --
	(147.78,251.83) --
	(146.90,251.94) --
	(146.01,252.04) --
	(145.13,252.12) --
	(144.24,252.19) --
	(143.35,252.24) --
	(142.46,252.27) --
	(141.57,252.29) --
	(140.69,252.29) --
	(139.80,252.28) --
	(138.91,252.26) --
	(138.02,252.21) --
	(137.13,252.15) --
	(136.25,252.08) --
	(135.36,251.99) --
	(134.48,251.89) --
	(133.60,251.77) --
	(132.72,251.63) --
	(131.84,251.48) --
	(130.97,251.31) --
	(130.10,251.13) --
	(129.23,250.93) --
	(128.37,250.72) --
	(127.51,250.50) --
	(126.65,250.25) --
	(125.80,250.00) --
	(124.95,249.72) --
	(124.11,249.44) --
	(123.27,249.14) --
	(122.44,248.82) --
	(121.62,248.49) --
	(120.80,248.15) --
	(119.98,247.79) --
	(119.18,247.41) --
	(118.37,247.03) --
	(117.58,246.62) --
	(116.79,246.21) --
	(116.02,245.78) --
	(115.24,245.34) --
	(114.48,244.88) --
	(113.73,244.41) --
	(112.98,243.93) --
	(112.24,243.44) --
	(111.51,242.93) --
	(110.79,242.41) --
	(110.08,241.87) --
	(109.37,241.33) --
	(108.68,240.77) --
	(108.00,240.20) --
	(107.33,239.62) --
	(106.66,239.02) --
	(106.01,238.42) --
	(105.37,237.80) --
	(104.74,237.17) --
	(104.12,236.54) --
	(103.51,235.89) --
	(102.92,235.23) --
	(102.33,234.56) --
	(101.76,233.88) --
	(101.20,233.19) --
	(100.65,232.49) --
	(100.11,231.78) --
	( 99.59,231.06) --
	( 99.08,230.33) --
	( 98.58,229.60) --
	( 98.09,228.85) --
	( 97.62,228.10) --
	( 97.16,227.34) --
	( 96.72,226.57) --
	( 96.28,225.79) --
	( 95.87,225.00) --
	( 95.46,224.21) --
	( 95.07,223.41) --
	( 94.69,222.61) --
	( 94.33,221.80) --
	( 93.98,220.98) --
	( 93.65,220.15) --
	( 93.33,219.32) --
	( 93.02,218.49) --
	( 92.73,217.65) --
	( 92.46,216.80) --
	( 92.20,215.95) --
	( 91.95,215.10) --
	( 91.72,214.24) --
	( 91.51,213.37) --
	( 91.30,212.51) --
	( 91.12,211.64) --
	( 90.95,210.77) --
	( 90.79,209.89) --
	( 90.65,209.01) --
	( 90.53,208.13) --
	( 90.42,207.25) --
	( 90.33,206.36) --
	( 90.25,205.48) --
	( 90.19,204.59) --
	( 90.14,203.70) --
	( 90.11,202.81) --
	( 90.10,201.92) --
	( 90.10,201.04) --
	( 90.11,200.15) --
	( 90.14,199.26) --
	( 90.19,198.37) --
	( 90.25,197.48) --
	( 90.33,196.60) --
	( 90.42,195.71) --
	( 90.53,194.83) --
	( 90.65,193.95) --
	( 90.79,193.07) --
	( 90.95,192.19) --
	( 91.12,191.32) --
	( 91.30,190.45) --
	( 91.51,189.59) --
	( 91.72,188.72) --
	( 91.95,187.86) --
	( 92.20,187.01) --
	( 92.46,186.16) --
	( 92.73,185.31) --
	( 93.02,184.47) --
	( 93.33,183.64) --
	( 93.65,182.81) --
	( 93.98,181.98) --
	( 94.33,181.16) --
	( 94.69,180.35) --
	( 95.07,179.55) --
	( 95.46,178.75) --
	( 95.87,177.96) --
	( 96.28,177.17) --
	( 96.72,176.39) --
	( 97.16,175.62) --
	( 97.62,174.86) --
	( 98.09,174.11) --
	( 98.58,173.36) --
	( 99.08,172.63) --
	( 99.59,171.90) --
	(100.11,171.18) --
	(100.65,170.47) --
	(101.20,169.77) --
	(101.76,169.08) --
	(102.33,168.40) --
	(102.92,167.73) --
	(103.51,167.07) --
	(104.12,166.42) --
	(104.74,165.79) --
	(105.37,165.16) --
	(106.01,164.54) --
	(106.66,163.94) --
	(107.33,163.34) --
	(108.00,162.76) --
	(108.68,162.19) --
	(109.37,161.63) --
	(110.08,161.09) --
	(110.79,160.55) --
	(111.51,160.03) --
	(112.24,159.52) --
	(112.98,159.03) --
	(113.73,158.55) --
	(114.48,158.08) --
	(115.24,157.62) --
	(116.02,157.18) --
	(116.79,156.75) --
	(117.58,156.34) --
	(118.37,155.93) --
	(119.18,155.55) --
	(119.98,155.17) --
	(120.80,154.81) --
	(121.62,154.47) --
	(122.44,154.14) --
	(123.27,153.82) --
	(124.11,153.52) --
	(124.95,153.24) --
	(125.80,152.96) --
	(126.65,152.71) --
	(127.51,152.46) --
	(128.37,152.24) --
	(129.23,152.03) --
	(130.10,151.83) --
	(130.97,151.65) --
	(131.84,151.48) --
	(132.72,151.33) --
	(133.60,151.19) --
	(134.48,151.07) --
	(135.36,150.97) --
	(136.25,150.88) --
	(137.13,150.81) --
	(138.02,150.75) --
	(138.91,150.70) --
	(139.80,150.68) --
	(140.69,150.67) --
	(141.57,150.67) --
	(142.46,150.69) --
	(143.35,150.72) --
	(144.24,150.77) --
	(145.13,150.84) --
	(146.01,150.92) --
	(146.90,151.02) --
	(147.78,151.13) --
	(148.66,151.26) --
	(149.54,151.40) --
	(150.41,151.56) --
	(151.28,151.74) --
	(152.15,151.93) --
	(153.02,152.13) --
	(153.88,152.35) --
	(154.74,152.58) --
	(155.59,152.83) --
	(156.44,153.10) --
	(157.29,153.38) --
	(158.12,153.67) --
	(158.96,153.98) --
	(159.79,154.30) --
	(160.61,154.64) --
	(161.43,154.99) --
	(162.24,155.36) --
	(163.04,155.74) --
	(163.84,156.13) --
	(164.63,156.54) --
	(165.41,156.96) --
	(166.19,157.40) --
	(166.95,157.85) --
	(167.71,158.31) --
	(168.47,158.79) --
	(169.21,159.28) --
	(169.94,159.78) --
	(170.67,160.29) --
	(171.38,160.82) --
	(172.09,161.36) --
	(172.79,161.91) --
	(173.48,162.47) --
	(174.15,163.05) --
	(174.82,163.64) --
	(175.48,164.24) --
	(176.12,164.85) --
	(176.76,165.47) --
	(177.38,166.10) --
	(178.00,166.75) --
	(178.60,167.40) --
	(179.19,168.07) --
	(179.77,168.74) --
	(180.34,169.43) --
	(180.89,170.12) --
	(181.43,170.83) --
	(181.97,171.54) --
	(182.48,172.26) --
	(182.99,173.00) --
	(183.48,173.74) --
	(183.96,174.49) --
	(184.42,175.24) --
	(184.88,176.01) --
	(185.32,176.78) --
	(185.74,177.56) --
	(186.15,178.35) --
	(186.55,179.15) --
	(186.94,179.95) --
	(187.30,180.76) --
	(187.66,181.57) --
	(188.00,182.39) --
	(188.33,183.22) --
	(188.64,184.05) --
	(188.94,184.89) --
	(189.22,185.74) --
	(189.49,186.58) --
	(189.74,187.44) --
	(189.98,188.29) --
	(190.20,189.15) --
	(190.41,190.02) --
	(190.61,190.89) --
	(190.78,191.76) --
	(190.95,192.63) --
	(191.09,193.51) --
	(191.22,194.39) --
	(191.34,195.27) --
	(191.44,196.15) --
	(191.53,197.04) --
	(191.60,197.93) --
	(191.65,198.81) --
	(191.69,199.70) --
	(191.71,200.59) --
	(191.72,201.48);
\definecolor[named]{drawColor}{rgb}{0.00,0.00,1.00}

\path[draw=drawColor,line width= 1.2pt,line join=round,line cap=round] ( 57.86,252.94) --
	( 58.25,252.29) --
	( 58.68,251.61) --
	( 59.12,250.93) --
	( 59.58,250.26) --
	( 60.04,249.60) --
	( 60.52,248.94) --
	( 61.00,248.30) --
	( 61.50,247.66) --
	( 62.01,247.04) --
	( 62.53,246.42) --
	( 63.07,245.81) --
	( 63.61,245.21) --
	( 64.16,244.62) --
	( 64.72,244.04) --
	( 65.30,243.47) --
	( 65.88,242.91) --
	( 66.47,242.36) --
	( 67.07,241.82) --
	( 67.69,241.29) --
	( 68.31,240.77) --
	( 68.94,240.26) --
	( 69.57,239.77) --
	( 70.22,239.28) --
	( 70.88,238.81) --
	( 71.54,238.35) --
	( 72.21,237.90) --
	( 72.89,237.46) --
	( 73.58,237.03) --
	( 74.27,236.62) --
	( 74.97,236.21) --
	( 75.68,235.82) --
	( 76.40,235.45) --
	( 77.12,235.08) --
	( 77.85,234.73) --
	( 78.58,234.39) --
	( 79.32,234.06) --
	( 80.06,233.75) --
	( 80.81,233.45) --
	( 81.57,233.16) --
	( 82.33,232.89) --
	( 83.10,232.63) --
	( 83.87,232.38) --
	( 84.64,232.15) --
	( 85.42,231.93) --
	( 86.20,231.72) --
	( 86.99,231.53) --
	( 87.77,231.35) --
	( 88.57,231.19) --
	( 89.36,231.03) --
	( 90.16,230.90) --
	( 90.96,230.77) --
	( 91.76,230.66) --
	( 92.56,230.57) --
	( 93.36,230.49) --
	( 94.17,230.42) --
	( 94.98,230.37) --
	( 95.78,230.33) --
	( 96.59,230.30) --
	( 97.40,230.29) --
	( 98.21,230.30) --
	( 99.02,230.31) --
	( 99.82,230.35) --
	(100.63,230.39) --
	(101.44,230.45) --
	(102.24,230.53) --
	(103.05,230.61) --
	(103.85,230.72) --
	(104.65,230.83) --
	(105.45,230.96) --
	(106.24,231.11) --
	(107.04,231.27) --
	(107.83,231.44) --
	(108.61,231.62) --
	(109.40,231.82) --
	(110.18,232.04) --
	(110.95,232.26) --
	(111.72,232.50) --
	(112.49,232.76) --
	(113.25,233.02) --
	(114.01,233.31) --
	(114.77,233.60) --
	(115.51,233.91) --
	(116.26,234.23) --
	(116.99,234.56) --
	(117.72,234.90) --
	(118.45,235.26) --
	(119.17,235.63) --
	(119.88,236.02) --
	(120.58,236.41) --
	(121.28,236.82) --
	(121.97,237.24) --
	(122.65,237.68) --
	(123.33,238.12) --
	(124.00,238.58) --
	(124.66,239.04) --
	(125.31,239.52) --
	(125.95,240.01) --
	(126.58,240.52) --
	(127.21,241.03) --
	(127.83,241.55) --
	(128.43,242.09) --
	(129.03,242.63) --
	(129.62,243.19) --
	(130.20,243.75) --
	(130.76,244.33) --
	(131.32,244.91) --
	(131.87,245.51) --
	(132.41,246.11) --
	(132.93,246.73) --
	(133.45,247.35) --
	(133.95,247.98) --
	(134.44,248.62) --
	(134.93,249.27) --
	(135.40,249.93) --
	(135.86,250.59) --
	(136.30,251.27) --
	(136.74,251.95) --
	(137.16,252.64) --
	(137.35,252.94);
\definecolor[named]{drawColor}{rgb}{0.00,0.00,0.00}

\path[draw=drawColor,line width= 0.4pt,dash pattern=on 1pt off 3pt ,line join=round,line cap=round] ( 52.57,252.94) --
	( 52.98,252.18) --
	( 53.41,251.40) --
	( 53.86,250.63) --
	( 54.32,249.87) --
	( 54.79,249.12) --
	( 55.27,248.37) --
	( 55.77,247.64) --
	( 56.28,246.91) --
	( 56.81,246.19) --
	( 57.34,245.48) --
	( 57.89,244.78) --
	( 58.45,244.09) --
	( 59.03,243.41) --
	( 59.61,242.74) --
	( 60.21,242.08) --
	( 60.82,241.43) --
	( 61.44,240.79) --
	( 62.07,240.17) --
	( 62.71,239.55) --
	( 63.36,238.94) --
	( 64.02,238.35) --
	( 64.69,237.77) --
	( 65.38,237.20) --
	( 66.07,236.64) --
	( 66.77,236.09) --
	( 67.48,235.56) --
	( 68.20,235.04) --
	( 68.93,234.53) --
	( 69.67,234.04) --
	( 70.42,233.55) --
	( 71.17,233.09) --
	( 71.94,232.63) --
	( 72.71,232.19) --
	( 73.49,231.76) --
	( 74.28,231.34) --
	( 75.07,230.94) --
	( 75.87,230.55) --
	( 76.68,230.18) --
	( 77.49,229.82) --
	( 78.31,229.48) --
	( 79.14,229.15) --
	( 79.97,228.83) --
	( 80.80,228.53) --
	( 81.65,228.24) --
	( 82.49,227.97) --
	( 83.34,227.71) --
	( 84.20,227.47) --
	( 85.06,227.25) --
	( 85.92,227.03) --
	( 86.79,226.84) --
	( 87.66,226.65) --
	( 88.54,226.49) --
	( 89.41,226.34) --
	( 90.29,226.20) --
	( 91.17,226.08) --
	( 92.05,225.98) --
	( 92.94,225.89) --
	( 93.83,225.81) --
	( 94.71,225.76) --
	( 95.60,225.71) --
	( 96.49,225.69) --
	( 97.38,225.67) --
	( 98.27,225.68) --
	( 99.16,225.70) --
	(100.05,225.73) --
	(100.93,225.78) --
	(101.82,225.85) --
	(102.71,225.93) --
	(103.59,226.03) --
	(104.47,226.14) --
	(105.35,226.27) --
	(106.23,226.41) --
	(107.11,226.57) --
	(107.98,226.74) --
	(108.85,226.93) --
	(109.71,227.14) --
	(110.57,227.36) --
	(111.43,227.59) --
	(112.29,227.84) --
	(113.14,228.11) --
	(113.98,228.38) --
	(114.82,228.68) --
	(115.65,228.99) --
	(116.48,229.31) --
	(117.30,229.65) --
	(118.12,230.00) --
	(118.93,230.37) --
	(119.74,230.75) --
	(120.53,231.14) --
	(121.32,231.55) --
	(122.11,231.97) --
	(122.88,232.41) --
	(123.65,232.86) --
	(124.41,233.32) --
	(125.16,233.79) --
	(125.90,234.28) --
	(126.64,234.78) --
	(127.36,235.30) --
	(128.08,235.83) --
	(128.79,236.37) --
	(129.48,236.92) --
	(130.17,237.48) --
	(130.85,238.06) --
	(131.52,238.65) --
	(132.17,239.25) --
	(132.82,239.86) --
	(133.45,240.48) --
	(134.08,241.11) --
	(134.69,241.75) --
	(135.30,242.41) --
	(135.89,243.07) --
	(136.46,243.75) --
	(137.03,244.43) --
	(137.59,245.13) --
	(138.13,245.83) --
	(138.66,246.55) --
	(139.18,247.27) --
	(139.68,248.00) --
	(140.17,248.74) --
	(140.65,249.49) --
	(141.12,250.25) --
	(141.57,251.02) --
	(142.01,251.79) --
	(142.44,252.57) --
	(142.63,252.94);
\definecolor[named]{drawColor}{rgb}{0.00,0.00,1.00}

\path[draw=drawColor,line width= 1.2pt,line join=round,line cap=round] (252.94, 84.48) --
	(252.38, 83.91) --
	(251.82, 83.33) --
	(251.27, 82.74) --
	(250.72, 82.14) --
	(250.19, 81.54) --
	(249.67, 80.92) --
	(249.16, 80.29) --
	(248.66, 79.65) --
	(248.17, 79.01) --
	(247.70, 78.36) --
	(247.23, 77.69) --
	(246.78, 77.02) --
	(246.34, 76.35) --
	(245.91, 75.66) --
	(245.49, 74.97) --
	(245.09, 74.27) --
	(244.69, 73.56) --
	(244.31, 72.85) --
	(243.94, 72.13) --
	(243.59, 71.40) --
	(243.25, 70.67) --
	(242.92, 69.93) --
	(242.60, 69.19) --
	(242.30, 68.44) --
	(242.01, 67.69) --
	(241.73, 66.93) --
	(241.47, 66.16) --
	(241.21, 65.39) --
	(240.98, 64.62) --
	(240.75, 63.84) --
	(240.54, 63.06) --
	(240.35, 62.28) --
	(240.17, 61.49) --
	(240.00, 60.70) --
	(239.84, 59.91) --
	(239.70, 59.11) --
	(239.57, 58.31) --
	(239.46, 57.51) --
	(239.36, 56.71) --
	(239.28, 55.90) --
	(239.21, 55.10) --
	(239.15, 54.29) --
	(239.11, 53.49) --
	(239.08, 52.68) --
	(239.07, 51.87) --
	(239.07, 51.06) --
	(239.08, 50.25) --
	(239.11, 49.44) --
	(239.15, 48.64) --
	(239.21, 47.83) --
	(239.28, 47.03) --
	(239.36, 46.22) --
	(239.46, 45.42) --
	(239.57, 44.62) --
	(239.70, 43.82) --
	(239.84, 43.02) --
	(240.00, 42.23) --
	(240.17, 41.44) --
	(240.35, 40.65) --
	(240.54, 39.87) --
	(240.75, 39.09) --
	(240.98, 38.31) --
	(241.21, 37.54) --
	(241.47, 36.77) --
	(241.73, 36.00) --
	(242.01, 35.24) --
	(242.30, 34.49) --
	(242.60, 33.74) --
	(242.92, 33.00) --
	(243.25, 32.26) --
	(243.59, 31.53) --
	(243.94, 30.80) --
	(244.31, 30.08) --
	(244.69, 29.37) --
	(245.09, 28.66) --
	(245.49, 27.96) --
	(245.91, 27.27) --
	(246.34, 26.58) --
	(246.78, 25.91) --
	(247.23, 25.24) --
	(247.70, 24.57) --
	(248.17, 23.92) --
	(248.66, 23.28) --
	(249.16, 22.64) --
	(249.67, 22.01) --
	(250.19, 21.39) --
	(250.72, 20.79) --
	(251.27, 20.19) --
	(251.82, 19.60) --
	(252.38, 19.02) --
	(252.94, 18.45);
\definecolor[named]{drawColor}{rgb}{0.00,0.00,0.00}

\path[draw=drawColor,line width= 0.4pt,dash pattern=on 1pt off 3pt ,line join=round,line cap=round] (252.94, 90.68) --
	(252.35, 90.18) --
	(251.68, 89.60) --
	(251.02, 89.01) --
	(250.36, 88.40) --
	(249.72, 87.79) --
	(249.09, 87.16) --
	(248.47, 86.52) --
	(247.87, 85.87) --
	(247.27, 85.21) --
	(246.69, 84.54) --
	(246.11, 83.86) --
	(245.55, 83.17) --
	(245.00, 82.47) --
	(244.47, 81.76) --
	(243.94, 81.04) --
	(243.43, 80.32) --
	(242.93, 79.58) --
	(242.45, 78.84) --
	(241.97, 78.08) --
	(241.51, 77.32) --
	(241.07, 76.55) --
	(240.64, 75.77) --
	(240.22, 74.99) --
	(239.81, 74.20) --
	(239.42, 73.40) --
	(239.05, 72.59) --
	(238.68, 71.78) --
	(238.33, 70.96) --
	(238.00, 70.14) --
	(237.68, 69.31) --
	(237.38, 68.47) --
	(237.09, 67.63) --
	(236.81, 66.79) --
	(236.55, 65.94) --
	(236.30, 65.08) --
	(236.07, 64.22) --
	(235.86, 63.36) --
	(235.66, 62.49) --
	(235.47, 61.62) --
	(235.30, 60.75) --
	(235.15, 59.87) --
	(235.01, 59.00) --
	(234.88, 58.12) --
	(234.77, 57.23) --
	(234.68, 56.35) --
	(234.60, 55.46) --
	(234.54, 54.58) --
	(234.49, 53.69) --
	(234.46, 52.80) --
	(234.45, 51.91) --
	(234.45, 51.02) --
	(234.46, 50.13) --
	(234.49, 49.24) --
	(234.54, 48.35) --
	(234.60, 47.47) --
	(234.68, 46.58) --
	(234.77, 45.70) --
	(234.88, 44.81) --
	(235.01, 43.93) --
	(235.15, 43.06) --
	(235.30, 42.18) --
	(235.47, 41.31) --
	(235.66, 40.44) --
	(235.86, 39.57) --
	(236.07, 38.71) --
	(236.30, 37.85) --
	(236.55, 36.99) --
	(236.81, 36.14) --
	(237.09, 35.30) --
	(237.38, 34.46) --
	(237.68, 33.62) --
	(238.00, 32.79) --
	(238.33, 31.97) --
	(238.68, 31.15) --
	(239.05, 30.34) --
	(239.42, 29.53) --
	(239.81, 28.73) --
	(240.22, 27.94) --
	(240.64, 27.16) --
	(241.07, 26.38) --
	(241.51, 25.61) --
	(241.97, 24.85) --
	(242.45, 24.09) --
	(242.93, 23.35) --
	(243.43, 22.61) --
	(243.94, 21.89) --
	(244.47, 21.17) --
	(245.00, 20.46) --
	(245.55, 19.76) --
	(246.11, 19.07) --
	(246.69, 18.39) --
	(247.27, 17.72) --
	(247.87, 17.06) --
	(248.47, 16.41) --
	(249.09, 15.77) --
	(249.72, 15.14) --
	(250.36, 14.53) --
	(251.02, 13.92) --
	(251.68, 13.33) --
	(252.35, 12.75) --
	(252.94, 12.25);

\path[fill=fillColor] (241.95,126.47) circle (  2.25);
\definecolor[named]{drawColor}{rgb}{0.00,0.00,1.00}

\path[draw=drawColor,line width= 1.2pt,line join=round,line cap=round] (252.94,171.34) --
	(252.18,171.52) --
	(251.39,171.69) --
	(250.59,171.85) --
	(249.80,172.00) --
	(249.00,172.13) --
	(248.20,172.24) --
	(247.40,172.35) --
	(246.60,172.43) --
	(245.79,172.51) --
	(244.98,172.57) --
	(244.18,172.61) --
	(243.37,172.65) --
	(242.56,172.66) --
	(241.75,172.67) --
	(240.94,172.66) --
	(240.14,172.63) --
	(239.33,172.59) --
	(238.52,172.54) --
	(237.72,172.47) --
	(236.91,172.39) --
	(236.11,172.30) --
	(235.31,172.19) --
	(234.51,172.06) --
	(233.71,171.93) --
	(232.92,171.77) --
	(232.13,171.61) --
	(231.34,171.43) --
	(230.55,171.24) --
	(229.77,171.03) --
	(228.99,170.81) --
	(228.22,170.58) --
	(227.45,170.33) --
	(226.68,170.07) --
	(225.92,169.80) --
	(225.17,169.51) --
	(224.42,169.21) --
	(223.67,168.90) --
	(222.93,168.57) --
	(222.20,168.23) --
	(221.47,167.88) --
	(220.75,167.51) --
	(220.03,167.14) --
	(219.32,166.75) --
	(218.62,166.34) --
	(217.93,165.93) --
	(217.24,165.50) --
	(216.56,165.06) --
	(215.89,164.61) --
	(215.23,164.15) --
	(214.57,163.68) --
	(213.93,163.19) --
	(213.29,162.70) --
	(212.66,162.19) --
	(212.04,161.67) --
	(211.43,161.14) --
	(210.82,160.60) --
	(210.23,160.05) --
	(209.65,159.49) --
	(209.08,158.92) --
	(208.51,158.34) --
	(207.96,157.75) --
	(207.42,157.15) --
	(206.89,156.54) --
	(206.37,155.92) --
	(205.86,155.30) --
	(205.36,154.66) --
	(204.87,154.02) --
	(204.39,153.36) --
	(203.93,152.70) --
	(203.47,152.03) --
	(203.03,151.35) --
	(202.60,150.67) --
	(202.19,149.98) --
	(201.78,149.28) --
	(201.39,148.57) --
	(201.01,147.86) --
	(200.64,147.14) --
	(200.28,146.41) --
	(199.94,145.68) --
	(199.61,144.94) --
	(199.29,144.20) --
	(198.99,143.45) --
	(198.70,142.69) --
	(198.42,141.93) --
	(198.16,141.17) --
	(197.91,140.40) --
	(197.67,139.63) --
	(197.45,138.85) --
	(197.24,138.07) --
	(197.04,137.29) --
	(196.86,136.50) --
	(196.69,135.71) --
	(196.54,134.91) --
	(196.40,134.12) --
	(196.27,133.32) --
	(196.16,132.52) --
	(196.06,131.72) --
	(195.97,130.91) --
	(195.90,130.11) --
	(195.85,129.30) --
	(195.80,128.49) --
	(195.78,127.69) --
	(195.76,126.88) --
	(195.76,126.07) --
	(195.78,125.26) --
	(195.80,124.45) --
	(195.85,123.64) --
	(195.90,122.84) --
	(195.97,122.03) --
	(196.06,121.23) --
	(196.16,120.43) --
	(196.27,119.63) --
	(196.40,118.83) --
	(196.54,118.03) --
	(196.69,117.24) --
	(196.86,116.45) --
	(197.04,115.66) --
	(197.24,114.87) --
	(197.45,114.09) --
	(197.67,113.32) --
	(197.91,112.54) --
	(198.16,111.78) --
	(198.42,111.01) --
	(198.70,110.25) --
	(198.99,109.50) --
	(199.29,108.75) --
	(199.61,108.00) --
	(199.94,107.27) --
	(200.28,106.53) --
	(200.64,105.81) --
	(201.01,105.09) --
	(201.39,104.37) --
	(201.78,103.67) --
	(202.19,102.97) --
	(202.60,102.28) --
	(203.03,101.59) --
	(203.47,100.91) --
	(203.93,100.24) --
	(204.39, 99.58) --
	(204.87, 98.93) --
	(205.36, 98.28) --
	(205.86, 97.65) --
	(206.37, 97.02) --
	(206.89, 96.40) --
	(207.42, 95.79) --
	(207.96, 95.19) --
	(208.51, 94.60) --
	(209.08, 94.02) --
	(209.65, 93.45) --
	(210.23, 92.89) --
	(210.82, 92.34) --
	(211.43, 91.80) --
	(212.04, 91.27) --
	(212.66, 90.76) --
	(213.29, 90.25) --
	(213.93, 89.75) --
	(214.57, 89.27) --
	(215.23, 88.79) --
	(215.89, 88.33) --
	(216.56, 87.88) --
	(217.24, 87.44) --
	(217.93, 87.02) --
	(218.62, 86.60) --
	(219.32, 86.20) --
	(220.03, 85.81) --
	(220.75, 85.43) --
	(221.47, 85.07) --
	(222.20, 84.72) --
	(222.93, 84.38) --
	(223.67, 84.05) --
	(224.42, 83.74) --
	(225.17, 83.44) --
	(225.92, 83.15) --
	(226.68, 82.87) --
	(227.45, 82.61) --
	(228.22, 82.37) --
	(228.99, 82.13) --
	(229.77, 81.91) --
	(230.55, 81.71) --
	(231.34, 81.51) --
	(232.13, 81.34) --
	(232.92, 81.17) --
	(233.71, 81.02) --
	(234.51, 80.88) --
	(235.31, 80.76) --
	(236.11, 80.65) --
	(236.91, 80.55) --
	(237.72, 80.47) --
	(238.52, 80.41) --
	(239.33, 80.35) --
	(240.14, 80.31) --
	(240.94, 80.29) --
	(241.75, 80.28) --
	(242.56, 80.28) --
	(243.37, 80.30) --
	(244.18, 80.33) --
	(244.98, 80.38) --
	(245.79, 80.44) --
	(246.60, 80.51) --
	(247.40, 80.60) --
	(248.20, 80.70) --
	(249.00, 80.82) --
	(249.80, 80.95) --
	(250.59, 81.09) --
	(251.39, 81.25) --
	(252.18, 81.42) --
	(252.94, 81.60);
\definecolor[named]{drawColor}{rgb}{0.00,0.00,0.00}

\path[draw=drawColor,line width= 0.4pt,dash pattern=on 1pt off 3pt ,line join=round,line cap=round] (252.94,176.08) --
	(252.33,176.22) --
	(251.46,176.39) --
	(250.58,176.55) --
	(249.71,176.69) --
	(248.83,176.82) --
	(247.94,176.93) --
	(247.06,177.03) --
	(246.17,177.11) --
	(245.29,177.18) --
	(244.40,177.23) --
	(243.51,177.26) --
	(242.62,177.28) --
	(241.73,177.29) --
	(240.84,177.27) --
	(239.95,177.25) --
	(239.07,177.20) --
	(238.18,177.15) --
	(237.29,177.07) --
	(236.41,176.98) --
	(235.52,176.88) --
	(234.64,176.76) --
	(233.76,176.62) --
	(232.89,176.47) --
	(232.01,176.31) --
	(231.14,176.12) --
	(230.28,175.93) --
	(229.41,175.71) --
	(228.55,175.49) --
	(227.70,175.25) --
	(226.84,174.99) --
	(226.00,174.72) --
	(225.16,174.43) --
	(224.32,174.13) --
	(223.49,173.81) --
	(222.66,173.48) --
	(221.84,173.14) --
	(221.03,172.78) --
	(220.22,172.41) --
	(219.42,172.02) --
	(218.63,171.62) --
	(217.84,171.20) --
	(217.06,170.77) --
	(216.29,170.33) --
	(215.53,169.87) --
	(214.77,169.41) --
	(214.02,168.92) --
	(213.29,168.43) --
	(212.56,167.92) --
	(211.83,167.40) --
	(211.12,166.87) --
	(210.42,166.32) --
	(209.73,165.76) --
	(209.05,165.19) --
	(208.37,164.61) --
	(207.71,164.02) --
	(207.06,163.41) --
	(206.42,162.79) --
	(205.79,162.17) --
	(205.17,161.53) --
	(204.56,160.88) --
	(203.96,160.22) --
	(203.38,159.55) --
	(202.81,158.87) --
	(202.25,158.18) --
	(201.70,157.48) --
	(201.16,156.77) --
	(200.64,156.05) --
	(200.12,155.32) --
	(199.63,154.59) --
	(199.14,153.84) --
	(198.67,153.09) --
	(198.21,152.33) --
	(197.76,151.56) --
	(197.33,150.78) --
	(196.91,150.00) --
	(196.51,149.20) --
	(196.12,148.41) --
	(195.74,147.60) --
	(195.38,146.79) --
	(195.03,145.97) --
	(194.69,145.15) --
	(194.38,144.32) --
	(194.07,143.48) --
	(193.78,142.64) --
	(193.50,141.79) --
	(193.24,140.94) --
	(193.00,140.09) --
	(192.77,139.23) --
	(192.55,138.37) --
	(192.35,137.50) --
	(192.17,136.63) --
	(192.00,135.76) --
	(191.84,134.88) --
	(191.70,134.00) --
	(191.58,133.12) --
	(191.47,132.24) --
	(191.37,131.36) --
	(191.30,130.47) --
	(191.23,129.58) --
	(191.19,128.70) --
	(191.16,127.81) --
	(191.14,126.92) --
	(191.14,126.03) --
	(191.16,125.14) --
	(191.19,124.25) --
	(191.23,123.36) --
	(191.30,122.47) --
	(191.37,121.59) --
	(191.47,120.70) --
	(191.58,119.82) --
	(191.70,118.94) --
	(191.84,118.06) --
	(192.00,117.19) --
	(192.17,116.31) --
	(192.35,115.44) --
	(192.55,114.58) --
	(192.77,113.71) --
	(193.00,112.86) --
	(193.24,112.00) --
	(193.50,111.15) --
	(193.78,110.31) --
	(194.07,109.46) --
	(194.38,108.63) --
	(194.69,107.80) --
	(195.03,106.98) --
	(195.38,106.16) --
	(195.74,105.35) --
	(196.12,104.54) --
	(196.51,103.74) --
	(196.91,102.95) --
	(197.33,102.16) --
	(197.76,101.39) --
	(198.21,100.62) --
	(198.67, 99.86) --
	(199.14, 99.10) --
	(199.63, 98.36) --
	(200.12, 97.62) --
	(200.64, 96.89) --
	(201.16, 96.17) --
	(201.70, 95.47) --
	(202.25, 94.77) --
	(202.81, 94.08) --
	(203.38, 93.40) --
	(203.96, 92.73) --
	(204.56, 92.07) --
	(205.17, 91.42) --
	(205.79, 90.78) --
	(206.42, 90.15) --
	(207.06, 89.53) --
	(207.71, 88.93) --
	(208.37, 88.34) --
	(209.05, 87.75) --
	(209.73, 87.18) --
	(210.42, 86.63) --
	(211.12, 86.08) --
	(211.83, 85.55) --
	(212.56, 85.03) --
	(213.29, 84.52) --
	(214.02, 84.02) --
	(214.77, 83.54) --
	(215.53, 83.07) --
	(216.29, 82.61) --
	(217.06, 82.17) --
	(217.84, 81.74) --
	(218.63, 81.33) --
	(219.42, 80.93) --
	(220.22, 80.54) --
	(221.03, 80.17) --
	(221.84, 79.81) --
	(222.66, 79.46) --
	(223.49, 79.13) --
	(224.32, 78.82) --
	(225.16, 78.51) --
	(226.00, 78.23) --
	(226.84, 77.96) --
	(227.70, 77.70) --
	(228.55, 77.46) --
	(229.41, 77.23) --
	(230.28, 77.02) --
	(231.14, 76.82) --
	(232.01, 76.64) --
	(232.89, 76.47) --
	(233.76, 76.32) --
	(234.64, 76.19) --
	(235.52, 76.07) --
	(236.41, 75.96) --
	(237.29, 75.87) --
	(238.18, 75.80) --
	(239.07, 75.74) --
	(239.95, 75.70) --
	(240.84, 75.67) --
	(241.73, 75.66) --
	(242.62, 75.66) --
	(243.51, 75.68) --
	(244.40, 75.72) --
	(245.29, 75.77) --
	(246.17, 75.83) --
	(247.06, 75.92) --
	(247.94, 76.01) --
	(248.83, 76.12) --
	(249.71, 76.25) --
	(250.58, 76.40) --
	(251.46, 76.55) --
	(252.33, 76.73) --
	(252.94, 76.86);

\path[fill=fillColor] (198.65,201.48) circle (  2.25);
\definecolor[named]{drawColor}{rgb}{0.00,0.00,1.00}

\path[draw=drawColor,line width= 1.2pt,line join=round,line cap=round] (244.84,201.48) --
	(244.84,202.29) --
	(244.82,203.10) --
	(244.78,203.90) --
	(244.73,204.71) --
	(244.67,205.52) --
	(244.59,206.32) --
	(244.50,207.13) --
	(244.39,207.93) --
	(244.27,208.73) --
	(244.14,209.52) --
	(243.99,210.32) --
	(243.83,211.11) --
	(243.65,211.90) --
	(243.46,212.69) --
	(243.26,213.47) --
	(243.04,214.25) --
	(242.81,215.02) --
	(242.57,215.79) --
	(242.31,216.56) --
	(242.04,217.32) --
	(241.76,218.08) --
	(241.46,218.83) --
	(241.15,219.58) --
	(240.83,220.32) --
	(240.49,221.05) --
	(240.14,221.78) --
	(239.78,222.51) --
	(239.41,223.22) --
	(239.02,223.93) --
	(238.62,224.64) --
	(238.21,225.33) --
	(237.79,226.02) --
	(237.35,226.70) --
	(236.90,227.38) --
	(236.44,228.04) --
	(235.97,228.70) --
	(235.49,229.35) --
	(235.00,229.99) --
	(234.49,230.62) --
	(233.98,231.24) --
	(233.45,231.86) --
	(232.91,232.46) --
	(232.37,233.06) --
	(231.81,233.64) --
	(231.24,234.22) --
	(230.66,234.78) --
	(230.08,235.34) --
	(229.48,235.88) --
	(228.87,236.42) --
	(228.26,236.94) --
	(227.63,237.45) --
	(227.00,237.95) --
	(226.35,238.44) --
	(225.70,238.92) --
	(225.04,239.39) --
	(224.38,239.85) --
	(223.70,240.29) --
	(223.02,240.72) --
	(222.33,241.14) --
	(221.63,241.55) --
	(220.92,241.95) --
	(220.21,242.33) --
	(219.49,242.70) --
	(218.77,243.06) --
	(218.04,243.41) --
	(217.30,243.74) --
	(216.56,244.06) --
	(215.81,244.37) --
	(215.06,244.66) --
	(214.30,244.94) --
	(213.54,245.21) --
	(212.77,245.46) --
	(212.00,245.70) --
	(211.22,245.93) --
	(210.44,246.14) --
	(209.66,246.34) --
	(208.87,246.53) --
	(208.08,246.70) --
	(207.29,246.86) --
	(206.49,247.00) --
	(205.70,247.13) --
	(204.90,247.25) --
	(204.09,247.35) --
	(203.29,247.44) --
	(202.48,247.52) --
	(201.68,247.58) --
	(200.87,247.62) --
	(200.06,247.65) --
	(199.25,247.67) --
	(198.45,247.67) --
	(197.64,247.66) --
	(196.83,247.64) --
	(196.02,247.60) --
	(195.22,247.55) --
	(194.41,247.48) --
	(193.61,247.40) --
	(192.80,247.30) --
	(192.00,247.19) --
	(191.20,247.07) --
	(190.41,246.93) --
	(189.61,246.78) --
	(188.82,246.62) --
	(188.03,246.44) --
	(187.25,246.25) --
	(186.46,246.04) --
	(185.69,245.82) --
	(184.91,245.59) --
	(184.14,245.34) --
	(183.38,245.08) --
	(182.62,244.80) --
	(181.86,244.52) --
	(181.11,244.22) --
	(180.37,243.90) --
	(179.63,243.58) --
	(178.89,243.24) --
	(178.16,242.88) --
	(177.44,242.52) --
	(176.73,242.14) --
	(176.02,241.75) --
	(175.32,241.35) --
	(174.62,240.94) --
	(173.94,240.51) --
	(173.26,240.07) --
	(172.59,239.62) --
	(171.92,239.16) --
	(171.27,238.69) --
	(170.62,238.20) --
	(169.98,237.70) --
	(169.35,237.20) --
	(168.73,236.68) --
	(168.12,236.15) --
	(167.52,235.61) --
	(166.93,235.06) --
	(166.34,234.50) --
	(165.77,233.93) --
	(165.21,233.35) --
	(164.65,232.76) --
	(164.11,232.16) --
	(163.58,231.55) --
	(163.06,230.93) --
	(162.55,230.30) --
	(162.05,229.67) --
	(161.56,229.02) --
	(161.09,228.37) --
	(160.62,227.71) --
	(160.17,227.04) --
	(159.73,226.36) --
	(159.30,225.68) --
	(158.88,224.98) --
	(158.48,224.29) --
	(158.08,223.58) --
	(157.70,222.87) --
	(157.33,222.15) --
	(156.98,221.42) --
	(156.64,220.69) --
	(156.31,219.95) --
	(155.99,219.20) --
	(155.69,218.46) --
	(155.40,217.70) --
	(155.12,216.94) --
	(154.85,216.18) --
	(154.60,215.41) --
	(154.37,214.64) --
	(154.14,213.86) --
	(153.93,213.08) --
	(153.74,212.29) --
	(153.55,211.51) --
	(153.39,210.72) --
	(153.23,209.92) --
	(153.09,209.13) --
	(152.96,208.33) --
	(152.85,207.53) --
	(152.75,206.72) --
	(152.67,205.92) --
	(152.60,205.11) --
	(152.54,204.31) --
	(152.50,203.50) --
	(152.47,202.69) --
	(152.46,201.88) --
	(152.46,201.08) --
	(152.47,200.27) --
	(152.50,199.46) --
	(152.54,198.65) --
	(152.60,197.85) --
	(152.67,197.04) --
	(152.75,196.24) --
	(152.85,195.43) --
	(152.96,194.63) --
	(153.09,193.83) --
	(153.23,193.04) --
	(153.39,192.24) --
	(153.55,191.45) --
	(153.74,190.67) --
	(153.93,189.88) --
	(154.14,189.10) --
	(154.37,188.32) --
	(154.60,187.55) --
	(154.85,186.78) --
	(155.12,186.02) --
	(155.40,185.26) --
	(155.69,184.50) --
	(155.99,183.76) --
	(156.31,183.01) --
	(156.64,182.27) --
	(156.98,181.54) --
	(157.33,180.81) --
	(157.70,180.09) --
	(158.08,179.38) --
	(158.48,178.67) --
	(158.88,177.98) --
	(159.30,177.28) --
	(159.73,176.60) --
	(160.17,175.92) --
	(160.62,175.25) --
	(161.09,174.59) --
	(161.56,173.94) --
	(162.05,173.29) --
	(162.55,172.66) --
	(163.06,172.03) --
	(163.58,171.41) --
	(164.11,170.80) --
	(164.65,170.20) --
	(165.21,169.61) --
	(165.77,169.03) --
	(166.34,168.46) --
	(166.93,167.90) --
	(167.52,167.35) --
	(168.12,166.81) --
	(168.73,166.28) --
	(169.35,165.76) --
	(169.98,165.26) --
	(170.62,164.76) --
	(171.27,164.27) --
	(171.92,163.80) --
	(172.59,163.34) --
	(173.26,162.89) --
	(173.94,162.45) --
	(174.62,162.02) --
	(175.32,161.61) --
	(176.02,161.21) --
	(176.73,160.82) --
	(177.44,160.44) --
	(178.16,160.08) --
	(178.89,159.72) --
	(179.63,159.38) --
	(180.37,159.06) --
	(181.11,158.74) --
	(181.86,158.44) --
	(182.62,158.16) --
	(183.38,157.88) --
	(184.14,157.62) --
	(184.91,157.37) --
	(185.69,157.14) --
	(186.46,156.92) --
	(187.25,156.71) --
	(188.03,156.52) --
	(188.82,156.34) --
	(189.61,156.18) --
	(190.41,156.03) --
	(191.20,155.89) --
	(192.00,155.77) --
	(192.80,155.66) --
	(193.61,155.56) --
	(194.41,155.48) --
	(195.22,155.41) --
	(196.02,155.36) --
	(196.83,155.32) --
	(197.64,155.30) --
	(198.45,155.29) --
	(199.25,155.29) --
	(200.06,155.31) --
	(200.87,155.34) --
	(201.68,155.38) --
	(202.48,155.44) --
	(203.29,155.52) --
	(204.09,155.61) --
	(204.90,155.71) --
	(205.70,155.83) --
	(206.49,155.96) --
	(207.29,156.10) --
	(208.08,156.26) --
	(208.87,156.43) --
	(209.66,156.62) --
	(210.44,156.82) --
	(211.22,157.03) --
	(212.00,157.26) --
	(212.77,157.50) --
	(213.54,157.75) --
	(214.30,158.02) --
	(215.06,158.30) --
	(215.81,158.59) --
	(216.56,158.90) --
	(217.30,159.22) --
	(218.04,159.55) --
	(218.77,159.90) --
	(219.49,160.26) --
	(220.21,160.63) --
	(220.92,161.01) --
	(221.63,161.41) --
	(222.33,161.82) --
	(223.02,162.24) --
	(223.70,162.67) --
	(224.38,163.11) --
	(225.04,163.57) --
	(225.70,164.04) --
	(226.35,164.52) --
	(227.00,165.01) --
	(227.63,165.51) --
	(228.26,166.02) --
	(228.87,166.54) --
	(229.48,167.08) --
	(230.08,167.62) --
	(230.66,168.18) --
	(231.24,168.74) --
	(231.81,169.32) --
	(232.37,169.90) --
	(232.91,170.50) --
	(233.45,171.10) --
	(233.98,171.72) --
	(234.49,172.34) --
	(235.00,172.97) --
	(235.49,173.61) --
	(235.97,174.26) --
	(236.44,174.92) --
	(236.90,175.58) --
	(237.35,176.26) --
	(237.79,176.94) --
	(238.21,177.63) --
	(238.62,178.32) --
	(239.02,179.03) --
	(239.41,179.74) --
	(239.78,180.45) --
	(240.14,181.18) --
	(240.49,181.91) --
	(240.83,182.64) --
	(241.15,183.38) --
	(241.46,184.13) --
	(241.76,184.88) --
	(242.04,185.64) --
	(242.31,186.40) --
	(242.57,187.17) --
	(242.81,187.94) --
	(243.04,188.71) --
	(243.26,189.49) --
	(243.46,190.27) --
	(243.65,191.06) --
	(243.83,191.85) --
	(243.99,192.64) --
	(244.14,193.44) --
	(244.27,194.23) --
	(244.39,195.03) --
	(244.50,195.83) --
	(244.59,196.64) --
	(244.67,197.44) --
	(244.73,198.25) --
	(244.78,199.06) --
	(244.82,199.86) --
	(244.84,200.67) --
	(244.84,201.48);
\definecolor[named]{drawColor}{rgb}{0.00,0.00,0.00}

\path[draw=drawColor,line width= 0.4pt,dash pattern=on 1pt off 3pt ,line join=round,line cap=round] (249.46,201.48) --
	(249.46,202.37) --
	(249.43,203.26) --
	(249.39,204.15) --
	(249.34,205.03) --
	(249.27,205.92) --
	(249.18,206.81) --
	(249.08,207.69) --
	(248.97,208.57) --
	(248.83,209.45) --
	(248.69,210.33) --
	(248.52,211.20) --
	(248.35,212.07) --
	(248.15,212.94) --
	(247.95,213.81) --
	(247.72,214.67) --
	(247.48,215.52) --
	(247.23,216.38) --
	(246.96,217.22) --
	(246.68,218.07) --
	(246.38,218.91) --
	(246.07,219.74) --
	(245.74,220.57) --
	(245.40,221.39) --
	(245.05,222.20) --
	(244.68,223.01) --
	(244.29,223.81) --
	(243.89,224.61) --
	(243.48,225.40) --
	(243.06,226.18) --
	(242.62,226.95) --
	(242.17,227.72) --
	(241.70,228.47) --
	(241.22,229.22) --
	(240.73,229.96) --
	(240.22,230.70) --
	(239.71,231.42) --
	(239.18,232.13) --
	(238.63,232.84) --
	(238.08,233.53) --
	(237.51,234.22) --
	(236.93,234.89) --
	(236.34,235.56) --
	(235.74,236.21) --
	(235.13,236.86) --
	(234.50,237.49) --
	(233.87,238.11) --
	(233.22,238.72) --
	(232.56,239.32) --
	(231.89,239.91) --
	(231.22,240.49) --
	(230.53,241.05) --
	(229.83,241.60) --
	(229.12,242.14) --
	(228.41,242.67) --
	(227.68,243.18) --
	(226.95,243.68) --
	(226.21,244.17) --
	(225.45,244.65) --
	(224.69,245.11) --
	(223.93,245.56) --
	(223.15,246.00) --
	(222.37,246.42) --
	(221.58,246.83) --
	(220.78,247.22) --
	(219.98,247.60) --
	(219.17,247.97) --
	(218.35,248.32) --
	(217.53,248.66) --
	(216.70,248.98) --
	(215.87,249.29) --
	(215.03,249.58) --
	(214.18,249.86) --
	(213.33,250.13) --
	(212.48,250.38) --
	(211.62,250.61) --
	(210.76,250.83) --
	(209.89,251.03) --
	(209.02,251.22) --
	(208.15,251.40) --
	(207.28,251.56) --
	(206.40,251.70) --
	(205.52,251.83) --
	(204.64,251.94) --
	(203.75,252.04) --
	(202.87,252.12) --
	(201.98,252.19) --
	(201.09,252.24) --
	(200.20,252.27) --
	(199.32,252.29) --
	(198.43,252.29) --
	(197.54,252.28) --
	(196.65,252.26) --
	(195.76,252.21) --
	(194.87,252.15) --
	(193.99,252.08) --
	(193.10,251.99) --
	(192.22,251.89) --
	(191.34,251.77) --
	(190.46,251.63) --
	(189.58,251.48) --
	(188.71,251.31) --
	(187.84,251.13) --
	(186.97,250.93) --
	(186.11,250.72) --
	(185.25,250.50) --
	(184.39,250.25) --
	(183.54,250.00) --
	(182.69,249.72) --
	(181.85,249.44) --
	(181.01,249.14) --
	(180.18,248.82) --
	(179.36,248.49) --
	(178.54,248.15) --
	(177.72,247.79) --
	(176.92,247.41) --
	(176.12,247.03) --
	(175.32,246.62) --
	(174.54,246.21) --
	(173.76,245.78) --
	(172.98,245.34) --
	(172.22,244.88) --
	(171.47,244.41) --
	(170.72,243.93) --
	(169.98,243.44) --
	(169.25,242.93) --
	(168.53,242.41) --
	(167.82,241.87) --
	(167.12,241.33) --
	(166.42,240.77) --
	(165.74,240.20) --
	(165.07,239.62) --
	(164.41,239.02) --
	(163.75,238.42) --
	(163.11,237.80) --
	(162.48,237.17) --
	(161.86,236.54) --
	(161.26,235.89) --
	(160.66,235.23) --
	(160.07,234.56) --
	(159.50,233.88) --
	(158.94,233.19) --
	(158.39,232.49) --
	(157.85,231.78) --
	(157.33,231.06) --
	(156.82,230.33) --
	(156.32,229.60) --
	(155.84,228.85) --
	(155.36,228.10) --
	(154.90,227.34) --
	(154.46,226.57) --
	(154.03,225.79) --
	(153.61,225.00) --
	(153.20,224.21) --
	(152.81,223.41) --
	(152.43,222.61) --
	(152.07,221.80) --
	(151.72,220.98) --
	(151.39,220.15) --
	(151.07,219.32) --
	(150.76,218.49) --
	(150.47,217.65) --
	(150.20,216.80) --
	(149.94,215.95) --
	(149.69,215.10) --
	(149.46,214.24) --
	(149.25,213.37) --
	(149.05,212.51) --
	(148.86,211.64) --
	(148.69,210.77) --
	(148.53,209.89) --
	(148.40,209.01) --
	(148.27,208.13) --
	(148.16,207.25) --
	(148.07,206.36) --
	(147.99,205.48) --
	(147.93,204.59) --
	(147.88,203.70) --
	(147.85,202.81) --
	(147.84,201.92) --
	(147.84,201.04) --
	(147.85,200.15) --
	(147.88,199.26) --
	(147.93,198.37) --
	(147.99,197.48) --
	(148.07,196.60) --
	(148.16,195.71) --
	(148.27,194.83) --
	(148.40,193.95) --
	(148.53,193.07) --
	(148.69,192.19) --
	(148.86,191.32) --
	(149.05,190.45) --
	(149.25,189.59) --
	(149.46,188.72) --
	(149.69,187.86) --
	(149.94,187.01) --
	(150.20,186.16) --
	(150.47,185.31) --
	(150.76,184.47) --
	(151.07,183.64) --
	(151.39,182.81) --
	(151.72,181.98) --
	(152.07,181.16) --
	(152.43,180.35) --
	(152.81,179.55) --
	(153.20,178.75) --
	(153.61,177.96) --
	(154.03,177.17) --
	(154.46,176.39) --
	(154.90,175.62) --
	(155.36,174.86) --
	(155.84,174.11) --
	(156.32,173.36) --
	(156.82,172.63) --
	(157.33,171.90) --
	(157.85,171.18) --
	(158.39,170.47) --
	(158.94,169.77) --
	(159.50,169.08) --
	(160.07,168.40) --
	(160.66,167.73) --
	(161.26,167.07) --
	(161.86,166.42) --
	(162.48,165.79) --
	(163.11,165.16) --
	(163.75,164.54) --
	(164.41,163.94) --
	(165.07,163.34) --
	(165.74,162.76) --
	(166.42,162.19) --
	(167.12,161.63) --
	(167.82,161.09) --
	(168.53,160.55) --
	(169.25,160.03) --
	(169.98,159.52) --
	(170.72,159.03) --
	(171.47,158.55) --
	(172.22,158.08) --
	(172.98,157.62) --
	(173.76,157.18) --
	(174.54,156.75) --
	(175.32,156.34) --
	(176.12,155.93) --
	(176.92,155.55) --
	(177.72,155.17) --
	(178.54,154.81) --
	(179.36,154.47) --
	(180.18,154.14) --
	(181.01,153.82) --
	(181.85,153.52) --
	(182.69,153.24) --
	(183.54,152.96) --
	(184.39,152.71) --
	(185.25,152.46) --
	(186.11,152.24) --
	(186.97,152.03) --
	(187.84,151.83) --
	(188.71,151.65) --
	(189.58,151.48) --
	(190.46,151.33) --
	(191.34,151.19) --
	(192.22,151.07) --
	(193.10,150.97) --
	(193.99,150.88) --
	(194.87,150.81) --
	(195.76,150.75) --
	(196.65,150.70) --
	(197.54,150.68) --
	(198.43,150.67) --
	(199.32,150.67) --
	(200.20,150.69) --
	(201.09,150.72) --
	(201.98,150.77) --
	(202.87,150.84) --
	(203.75,150.92) --
	(204.64,151.02) --
	(205.52,151.13) --
	(206.40,151.26) --
	(207.28,151.40) --
	(208.15,151.56) --
	(209.02,151.74) --
	(209.89,151.93) --
	(210.76,152.13) --
	(211.62,152.35) --
	(212.48,152.58) --
	(213.33,152.83) --
	(214.18,153.10) --
	(215.03,153.38) --
	(215.87,153.67) --
	(216.70,153.98) --
	(217.53,154.30) --
	(218.35,154.64) --
	(219.17,154.99) --
	(219.98,155.36) --
	(220.78,155.74) --
	(221.58,156.13) --
	(222.37,156.54) --
	(223.15,156.96) --
	(223.93,157.40) --
	(224.69,157.85) --
	(225.45,158.31) --
	(226.21,158.79) --
	(226.95,159.28) --
	(227.68,159.78) --
	(228.41,160.29) --
	(229.12,160.82) --
	(229.83,161.36) --
	(230.53,161.91) --
	(231.22,162.47) --
	(231.89,163.05) --
	(232.56,163.64) --
	(233.22,164.24) --
	(233.87,164.85) --
	(234.50,165.47) --
	(235.13,166.10) --
	(235.74,166.75) --
	(236.34,167.40) --
	(236.93,168.07) --
	(237.51,168.74) --
	(238.08,169.43) --
	(238.63,170.12) --
	(239.18,170.83) --
	(239.71,171.54) --
	(240.22,172.26) --
	(240.73,173.00) --
	(241.22,173.74) --
	(241.70,174.49) --
	(242.17,175.24) --
	(242.62,176.01) --
	(243.06,176.78) --
	(243.48,177.56) --
	(243.89,178.35) --
	(244.29,179.15) --
	(244.68,179.95) --
	(245.05,180.76) --
	(245.40,181.57) --
	(245.74,182.39) --
	(246.07,183.22) --
	(246.38,184.05) --
	(246.68,184.89) --
	(246.96,185.74) --
	(247.23,186.58) --
	(247.48,187.44) --
	(247.72,188.29) --
	(247.95,189.15) --
	(248.15,190.02) --
	(248.35,190.89) --
	(248.52,191.76) --
	(248.69,192.63) --
	(248.83,193.51) --
	(248.97,194.39) --
	(249.08,195.27) --
	(249.18,196.15) --
	(249.27,197.04) --
	(249.34,197.93) --
	(249.39,198.81) --
	(249.43,199.70) --
	(249.46,200.59) --
	(249.46,201.48);
\definecolor[named]{drawColor}{rgb}{0.00,0.00,1.00}

\path[draw=drawColor,line width= 1.2pt,line join=round,line cap=round] (115.60,252.94) --
	(115.99,252.29) --
	(116.42,251.61) --
	(116.86,250.93) --
	(117.32,250.26) --
	(117.78,249.60) --
	(118.26,248.94) --
	(118.75,248.30) --
	(119.24,247.66) --
	(119.75,247.04) --
	(120.28,246.42) --
	(120.81,245.81) --
	(121.35,245.21) --
	(121.90,244.62) --
	(122.46,244.04) --
	(123.04,243.47) --
	(123.62,242.91) --
	(124.21,242.36) --
	(124.81,241.82) --
	(125.43,241.29) --
	(126.05,240.77) --
	(126.68,240.26) --
	(127.31,239.77) --
	(127.96,239.28) --
	(128.62,238.81) --
	(129.28,238.35) --
	(129.95,237.90) --
	(130.63,237.46) --
	(131.32,237.03) --
	(132.01,236.62) --
	(132.71,236.21) --
	(133.42,235.82) --
	(134.14,235.45) --
	(134.86,235.08) --
	(135.59,234.73) --
	(136.32,234.39) --
	(137.06,234.06) --
	(137.81,233.75) --
	(138.56,233.45) --
	(139.31,233.16) --
	(140.07,232.89) --
	(140.84,232.63) --
	(141.61,232.38) --
	(142.38,232.15) --
	(143.16,231.93) --
	(143.94,231.72) --
	(144.73,231.53) --
	(145.51,231.35) --
	(146.31,231.19) --
	(147.10,231.03) --
	(147.90,230.90) --
	(148.70,230.77) --
	(149.50,230.66) --
	(150.30,230.57) --
	(151.10,230.49) --
	(151.91,230.42) --
	(152.72,230.37) --
	(153.52,230.33) --
	(154.33,230.30) --
	(155.14,230.29) --
	(155.95,230.30) --
	(156.76,230.31) --
	(157.57,230.35) --
	(158.37,230.39) --
	(159.18,230.45) --
	(159.98,230.53) --
	(160.79,230.61) --
	(161.59,230.72) --
	(162.39,230.83) --
	(163.19,230.96) --
	(163.98,231.11) --
	(164.78,231.27) --
	(165.57,231.44) --
	(166.35,231.62) --
	(167.14,231.82) --
	(167.92,232.04) --
	(168.69,232.26) --
	(169.46,232.50) --
	(170.23,232.76) --
	(170.99,233.02) --
	(171.75,233.31) --
	(172.51,233.60) --
	(173.25,233.91) --
	(174.00,234.23) --
	(174.73,234.56) --
	(175.46,234.90) --
	(176.19,235.26) --
	(176.91,235.63) --
	(177.62,236.02) --
	(178.32,236.41) --
	(179.02,236.82) --
	(179.71,237.24) --
	(180.40,237.68) --
	(181.07,238.12) --
	(181.74,238.58) --
	(182.40,239.04) --
	(183.05,239.52) --
	(183.69,240.01) --
	(184.33,240.52) --
	(184.95,241.03) --
	(185.57,241.55) --
	(186.17,242.09) --
	(186.77,242.63) --
	(187.36,243.19) --
	(187.94,243.75) --
	(188.50,244.33) --
	(189.06,244.91) --
	(189.61,245.51) --
	(190.15,246.11) --
	(190.67,246.73) --
	(191.19,247.35) --
	(191.69,247.98) --
	(192.19,248.62) --
	(192.67,249.27) --
	(193.14,249.93) --
	(193.60,250.59) --
	(194.04,251.27) --
	(194.48,251.95) --
	(194.90,252.64) --
	(195.09,252.94);
\definecolor[named]{drawColor}{rgb}{0.00,0.00,0.00}

\path[draw=drawColor,line width= 0.4pt,dash pattern=on 1pt off 3pt ,line join=round,line cap=round] (110.31,252.94) --
	(110.72,252.18) --
	(111.15,251.40) --
	(111.60,250.63) --
	(112.06,249.87) --
	(112.53,249.12) --
	(113.02,248.37) --
	(113.51,247.64) --
	(114.03,246.91) --
	(114.55,246.19) --
	(115.09,245.48) --
	(115.63,244.78) --
	(116.20,244.09) --
	(116.77,243.41) --
	(117.35,242.74) --
	(117.95,242.08) --
	(118.56,241.43) --
	(119.18,240.79) --
	(119.81,240.17) --
	(120.45,239.55) --
	(121.10,238.94) --
	(121.76,238.35) --
	(122.43,237.77) --
	(123.12,237.20) --
	(123.81,236.64) --
	(124.51,236.09) --
	(125.22,235.56) --
	(125.94,235.04) --
	(126.67,234.53) --
	(127.41,234.04) --
	(128.16,233.55) --
	(128.92,233.09) --
	(129.68,232.63) --
	(130.45,232.19) --
	(131.23,231.76) --
	(132.02,231.34) --
	(132.81,230.94) --
	(133.61,230.55) --
	(134.42,230.18) --
	(135.23,229.82) --
	(136.05,229.48) --
	(136.88,229.15) --
	(137.71,228.83) --
	(138.54,228.53) --
	(139.39,228.24) --
	(140.23,227.97) --
	(141.09,227.71) --
	(141.94,227.47) --
	(142.80,227.25) --
	(143.66,227.03) --
	(144.53,226.84) --
	(145.40,226.65) --
	(146.28,226.49) --
	(147.15,226.34) --
	(148.03,226.20) --
	(148.91,226.08) --
	(149.80,225.98) --
	(150.68,225.89) --
	(151.57,225.81) --
	(152.45,225.76) --
	(153.34,225.71) --
	(154.23,225.69) --
	(155.12,225.67) --
	(156.01,225.68) --
	(156.90,225.70) --
	(157.79,225.73) --
	(158.68,225.78) --
	(159.56,225.85) --
	(160.45,225.93) --
	(161.33,226.03) --
	(162.21,226.14) --
	(163.09,226.27) --
	(163.97,226.41) --
	(164.85,226.57) --
	(165.72,226.74) --
	(166.59,226.93) --
	(167.45,227.14) --
	(168.32,227.36) --
	(169.17,227.59) --
	(170.03,227.84) --
	(170.88,228.11) --
	(171.72,228.38) --
	(172.56,228.68) --
	(173.39,228.99) --
	(174.22,229.31) --
	(175.05,229.65) --
	(175.86,230.00) --
	(176.67,230.37) --
	(177.48,230.75) --
	(178.27,231.14) --
	(179.06,231.55) --
	(179.85,231.97) --
	(180.62,232.41) --
	(181.39,232.86) --
	(182.15,233.32) --
	(182.90,233.79) --
	(183.64,234.28) --
	(184.38,234.78) --
	(185.10,235.30) --
	(185.82,235.83) --
	(186.53,236.37) --
	(187.22,236.92) --
	(187.91,237.48) --
	(188.59,238.06) --
	(189.26,238.65) --
	(189.91,239.25) --
	(190.56,239.86) --
	(191.20,240.48) --
	(191.82,241.11) --
	(192.43,241.75) --
	(193.04,242.41) --
	(193.63,243.07) --
	(194.21,243.75) --
	(194.77,244.43) --
	(195.33,245.13) --
	(195.87,245.83) --
	(196.40,246.55) --
	(196.92,247.27) --
	(197.42,248.00) --
	(197.92,248.74) --
	(198.39,249.49) --
	(198.86,250.25) --
	(199.31,251.02) --
	(199.75,251.79) --
	(200.18,252.57) --
	(200.37,252.94);

\path[draw=drawColor,line width= 0.4pt,dash pattern=on 1pt off 3pt ,line join=round,line cap=round] (252.94,146.38) --
	(252.77,145.97) --
	(252.44,145.15) --
	(252.12,144.32) --
	(251.81,143.48) --
	(251.52,142.64) --
	(251.25,141.79) --
	(250.98,140.94) --
	(250.74,140.09) --
	(250.51,139.23) --
	(250.29,138.37) --
	(250.09,137.50) --
	(249.91,136.63) --
	(249.74,135.76) --
	(249.58,134.88) --
	(249.44,134.00) --
	(249.32,133.12) --
	(249.21,132.24) --
	(249.12,131.36) --
	(249.04,130.47) --
	(248.98,129.58) --
	(248.93,128.70) --
	(248.90,127.81) --
	(248.88,126.92) --
	(248.88,126.03) --
	(248.90,125.14) --
	(248.93,124.25) --
	(248.98,123.36) --
	(249.04,122.47) --
	(249.12,121.59) --
	(249.21,120.70) --
	(249.32,119.82) --
	(249.44,118.94) --
	(249.58,118.06) --
	(249.74,117.19) --
	(249.91,116.31) --
	(250.09,115.44) --
	(250.29,114.58) --
	(250.51,113.71) --
	(250.74,112.86) --
	(250.98,112.00) --
	(251.25,111.15) --
	(251.52,110.31) --
	(251.81,109.46) --
	(252.12,108.63) --
	(252.44,107.80) --
	(252.77,106.98) --
	(252.94,106.56);
\definecolor[named]{drawColor}{rgb}{0.00,0.00,1.00}

\path[draw=drawColor,line width= 1.2pt,line join=round,line cap=round] (252.94,247.55) --
	(252.15,247.48) --
	(251.35,247.40) --
	(250.54,247.30) --
	(249.74,247.19) --
	(248.94,247.07) --
	(248.15,246.93) --
	(247.35,246.78) --
	(246.56,246.62) --
	(245.77,246.44) --
	(244.99,246.25) --
	(244.21,246.04) --
	(243.43,245.82) --
	(242.65,245.59) --
	(241.88,245.34) --
	(241.12,245.08) --
	(240.36,244.80) --
	(239.60,244.52) --
	(238.85,244.22) --
	(238.11,243.90) --
	(237.37,243.58) --
	(236.63,243.24) --
	(235.90,242.88) --
	(235.18,242.52) --
	(234.47,242.14) --
	(233.76,241.75) --
	(233.06,241.35) --
	(232.36,240.94) --
	(231.68,240.51) --
	(231.00,240.07) --
	(230.33,239.62) --
	(229.66,239.16) --
	(229.01,238.69) --
	(228.36,238.20) --
	(227.72,237.70) --
	(227.09,237.20) --
	(226.47,236.68) --
	(225.86,236.15) --
	(225.26,235.61) --
	(224.67,235.06) --
	(224.08,234.50) --
	(223.51,233.93) --
	(222.95,233.35) --
	(222.40,232.76) --
	(221.85,232.16) --
	(221.32,231.55) --
	(220.80,230.93) --
	(220.29,230.30) --
	(219.79,229.67) --
	(219.30,229.02) --
	(218.83,228.37) --
	(218.36,227.71) --
	(217.91,227.04) --
	(217.47,226.36) --
	(217.04,225.68) --
	(216.62,224.98) --
	(216.22,224.29) --
	(215.82,223.58) --
	(215.44,222.87) --
	(215.07,222.15) --
	(214.72,221.42) --
	(214.38,220.69) --
	(214.05,219.95) --
	(213.73,219.20) --
	(213.43,218.46) --
	(213.14,217.70) --
	(212.86,216.94) --
	(212.59,216.18) --
	(212.34,215.41) --
	(212.11,214.64) --
	(211.88,213.86) --
	(211.67,213.08) --
	(211.48,212.29) --
	(211.30,211.51) --
	(211.13,210.72) --
	(210.97,209.92) --
	(210.83,209.13) --
	(210.70,208.33) --
	(210.59,207.53) --
	(210.49,206.72) --
	(210.41,205.92) --
	(210.34,205.11) --
	(210.28,204.31) --
	(210.24,203.50) --
	(210.21,202.69) --
	(210.20,201.88) --
	(210.20,201.08) --
	(210.21,200.27) --
	(210.24,199.46) --
	(210.28,198.65) --
	(210.34,197.85) --
	(210.41,197.04) --
	(210.49,196.24) --
	(210.59,195.43) --
	(210.70,194.63) --
	(210.83,193.83) --
	(210.97,193.04) --
	(211.13,192.24) --
	(211.30,191.45) --
	(211.48,190.67) --
	(211.67,189.88) --
	(211.88,189.10) --
	(212.11,188.32) --
	(212.34,187.55) --
	(212.59,186.78) --
	(212.86,186.02) --
	(213.14,185.26) --
	(213.43,184.50) --
	(213.73,183.76) --
	(214.05,183.01) --
	(214.38,182.27) --
	(214.72,181.54) --
	(215.07,180.81) --
	(215.44,180.09) --
	(215.82,179.38) --
	(216.22,178.67) --
	(216.62,177.98) --
	(217.04,177.28) --
	(217.47,176.60) --
	(217.91,175.92) --
	(218.36,175.25) --
	(218.83,174.59) --
	(219.30,173.94) --
	(219.79,173.29) --
	(220.29,172.66) --
	(220.80,172.03) --
	(221.32,171.41) --
	(221.85,170.80) --
	(222.40,170.20) --
	(222.95,169.61) --
	(223.51,169.03) --
	(224.08,168.46) --
	(224.67,167.90) --
	(225.26,167.35) --
	(225.86,166.81) --
	(226.47,166.28) --
	(227.09,165.76) --
	(227.72,165.26) --
	(228.36,164.76) --
	(229.01,164.27) --
	(229.66,163.80) --
	(230.33,163.34) --
	(231.00,162.89) --
	(231.68,162.45) --
	(232.36,162.02) --
	(233.06,161.61) --
	(233.76,161.21) --
	(234.47,160.82) --
	(235.18,160.44) --
	(235.90,160.08) --
	(236.63,159.72) --
	(237.37,159.38) --
	(238.11,159.06) --
	(238.85,158.74) --
	(239.60,158.44) --
	(240.36,158.16) --
	(241.12,157.88) --
	(241.88,157.62) --
	(242.65,157.37) --
	(243.43,157.14) --
	(244.21,156.92) --
	(244.99,156.71) --
	(245.77,156.52) --
	(246.56,156.34) --
	(247.35,156.18) --
	(248.15,156.03) --
	(248.94,155.89) --
	(249.74,155.77) --
	(250.54,155.66) --
	(251.35,155.56) --
	(252.15,155.48) --
	(252.94,155.41);
\definecolor[named]{drawColor}{rgb}{0.00,0.00,0.00}

\path[draw=drawColor,line width= 0.4pt,dash pattern=on 1pt off 3pt ,line join=round,line cap=round] (252.94,252.18) --
	(252.61,252.15) --
	(251.73,252.08) --
	(250.84,251.99) --
	(249.96,251.89) --
	(249.08,251.77) --
	(248.20,251.63) --
	(247.32,251.48) --
	(246.45,251.31) --
	(245.58,251.13) --
	(244.71,250.93) --
	(243.85,250.72) --
	(242.99,250.50) --
	(242.13,250.25) --
	(241.28,250.00) --
	(240.43,249.72) --
	(239.59,249.44) --
	(238.75,249.14) --
	(237.92,248.82) --
	(237.10,248.49) --
	(236.28,248.15) --
	(235.46,247.79) --
	(234.66,247.41) --
	(233.86,247.03) --
	(233.06,246.62) --
	(232.28,246.21) --
	(231.50,245.78) --
	(230.73,245.34) --
	(229.96,244.88) --
	(229.21,244.41) --
	(228.46,243.93) --
	(227.72,243.44) --
	(226.99,242.93) --
	(226.27,242.41) --
	(225.56,241.87) --
	(224.86,241.33) --
	(224.16,240.77) --
	(223.48,240.20) --
	(222.81,239.62) --
	(222.15,239.02) --
	(221.49,238.42) --
	(220.85,237.80) --
	(220.22,237.17) --
	(219.60,236.54) --
	(219.00,235.89) --
	(218.40,235.23) --
	(217.81,234.56) --
	(217.24,233.88) --
	(216.68,233.19) --
	(216.13,232.49) --
	(215.60,231.78) --
	(215.07,231.06) --
	(214.56,230.33) --
	(214.06,229.60) --
	(213.58,228.85) --
	(213.10,228.10) --
	(212.64,227.34) --
	(212.20,226.57) --
	(211.77,225.79) --
	(211.35,225.00) --
	(210.94,224.21) --
	(210.55,223.41) --
	(210.18,222.61) --
	(209.81,221.80) --
	(209.46,220.98) --
	(209.13,220.15) --
	(208.81,219.32) --
	(208.51,218.49) --
	(208.22,217.65) --
	(207.94,216.80) --
	(207.68,215.95) --
	(207.43,215.10) --
	(207.20,214.24) --
	(206.99,213.37) --
	(206.79,212.51) --
	(206.60,211.64) --
	(206.43,210.77) --
	(206.28,209.89) --
	(206.14,209.01) --
	(206.01,208.13) --
	(205.90,207.25) --
	(205.81,206.36) --
	(205.73,205.48) --
	(205.67,204.59) --
	(205.62,203.70) --
	(205.59,202.81) --
	(205.58,201.92) --
	(205.58,201.04) --
	(205.59,200.15) --
	(205.62,199.26) --
	(205.67,198.37) --
	(205.73,197.48) --
	(205.81,196.60) --
	(205.90,195.71) --
	(206.01,194.83) --
	(206.14,193.95) --
	(206.28,193.07) --
	(206.43,192.19) --
	(206.60,191.32) --
	(206.79,190.45) --
	(206.99,189.59) --
	(207.20,188.72) --
	(207.43,187.86) --
	(207.68,187.01) --
	(207.94,186.16) --
	(208.22,185.31) --
	(208.51,184.47) --
	(208.81,183.64) --
	(209.13,182.81) --
	(209.46,181.98) --
	(209.81,181.16) --
	(210.18,180.35) --
	(210.55,179.55) --
	(210.94,178.75) --
	(211.35,177.96) --
	(211.77,177.17) --
	(212.20,176.39) --
	(212.64,175.62) --
	(213.10,174.86) --
	(213.58,174.11) --
	(214.06,173.36) --
	(214.56,172.63) --
	(215.07,171.90) --
	(215.60,171.18) --
	(216.13,170.47) --
	(216.68,169.77) --
	(217.24,169.08) --
	(217.81,168.40) --
	(218.40,167.73) --
	(219.00,167.07) --
	(219.60,166.42) --
	(220.22,165.79) --
	(220.85,165.16) --
	(221.49,164.54) --
	(222.15,163.94) --
	(222.81,163.34) --
	(223.48,162.76) --
	(224.16,162.19) --
	(224.86,161.63) --
	(225.56,161.09) --
	(226.27,160.55) --
	(226.99,160.03) --
	(227.72,159.52) --
	(228.46,159.03) --
	(229.21,158.55) --
	(229.96,158.08) --
	(230.73,157.62) --
	(231.50,157.18) --
	(232.28,156.75) --
	(233.06,156.34) --
	(233.86,155.93) --
	(234.66,155.55) --
	(235.46,155.17) --
	(236.28,154.81) --
	(237.10,154.47) --
	(237.92,154.14) --
	(238.75,153.82) --
	(239.59,153.52) --
	(240.43,153.24) --
	(241.28,152.96) --
	(242.13,152.71) --
	(242.99,152.46) --
	(243.85,152.24) --
	(244.71,152.03) --
	(245.58,151.83) --
	(246.45,151.65) --
	(247.32,151.48) --
	(248.20,151.33) --
	(249.08,151.19) --
	(249.96,151.07) --
	(250.84,150.97) --
	(251.73,150.88) --
	(252.61,150.81) --
	(252.94,150.78);
\definecolor[named]{drawColor}{rgb}{0.00,0.00,1.00}

\path[draw=drawColor,line width= 1.2pt,line join=round,line cap=round] (173.34,252.94) --
	(173.73,252.29) --
	(174.16,251.61) --
	(174.60,250.93) --
	(175.06,250.26) --
	(175.52,249.60) --
	(176.00,248.94) --
	(176.49,248.30) --
	(176.99,247.66) --
	(177.50,247.04) --
	(178.02,246.42) --
	(178.55,245.81) --
	(179.09,245.21) --
	(179.64,244.62) --
	(180.21,244.04) --
	(180.78,243.47) --
	(181.36,242.91) --
	(181.95,242.36) --
	(182.56,241.82) --
	(183.17,241.29) --
	(183.79,240.77) --
	(184.42,240.26) --
	(185.06,239.77) --
	(185.70,239.28) --
	(186.36,238.81) --
	(187.02,238.35) --
	(187.69,237.90) --
	(188.37,237.46) --
	(189.06,237.03) --
	(189.75,236.62) --
	(190.45,236.21) --
	(191.16,235.82) --
	(191.88,235.45) --
	(192.60,235.08) --
	(193.33,234.73) --
	(194.06,234.39) --
	(194.80,234.06) --
	(195.55,233.75) --
	(196.30,233.45) --
	(197.05,233.16) --
	(197.81,232.89) --
	(198.58,232.63) --
	(199.35,232.38) --
	(200.12,232.15) --
	(200.90,231.93) --
	(201.68,231.72) --
	(202.47,231.53) --
	(203.26,231.35) --
	(204.05,231.19) --
	(204.84,231.03) --
	(205.64,230.90) --
	(206.44,230.77) --
	(207.24,230.66) --
	(208.04,230.57) --
	(208.85,230.49) --
	(209.65,230.42) --
	(210.46,230.37) --
	(211.27,230.33) --
	(212.07,230.30) --
	(212.88,230.29) --
	(213.69,230.30) --
	(214.50,230.31) --
	(215.31,230.35) --
	(216.11,230.39) --
	(216.92,230.45) --
	(217.72,230.53) --
	(218.53,230.61) --
	(219.33,230.72) --
	(220.13,230.83) --
	(220.93,230.96) --
	(221.72,231.11) --
	(222.52,231.27) --
	(223.31,231.44) --
	(224.09,231.62) --
	(224.88,231.82) --
	(225.66,232.04) --
	(226.43,232.26) --
	(227.20,232.50) --
	(227.97,232.76) --
	(228.74,233.02) --
	(229.49,233.31) --
	(230.25,233.60) --
	(230.99,233.91) --
	(231.74,234.23) --
	(232.47,234.56) --
	(233.20,234.90) --
	(233.93,235.26) --
	(234.65,235.63) --
	(235.36,236.02) --
	(236.06,236.41) --
	(236.76,236.82) --
	(237.45,237.24) --
	(238.14,237.68) --
	(238.81,238.12) --
	(239.48,238.58) --
	(240.14,239.04) --
	(240.79,239.52) --
	(241.43,240.01) --
	(242.07,240.52) --
	(242.69,241.03) --
	(243.31,241.55) --
	(243.91,242.09) --
	(244.51,242.63) --
	(245.10,243.19) --
	(245.68,243.75) --
	(246.24,244.33) --
	(246.80,244.91) --
	(247.35,245.51) --
	(247.89,246.11) --
	(248.41,246.73) --
	(248.93,247.35) --
	(249.43,247.98) --
	(249.93,248.62) --
	(250.41,249.27) --
	(250.88,249.93) --
	(251.34,250.59) --
	(251.79,251.27) --
	(252.22,251.95) --
	(252.64,252.64) --
	(252.83,252.94);
\definecolor[named]{drawColor}{rgb}{0.00,0.00,0.00}

\path[draw=drawColor,line width= 0.4pt,dash pattern=on 1pt off 3pt ,line join=round,line cap=round] (168.05,252.94) --
	(168.46,252.18) --
	(168.89,251.40) --
	(169.34,250.63) --
	(169.80,249.87) --
	(170.27,249.12) --
	(170.76,248.37) --
	(171.25,247.64) --
	(171.77,246.91) --
	(172.29,246.19) --
	(172.83,245.48) --
	(173.38,244.78) --
	(173.94,244.09) --
	(174.51,243.41) --
	(175.09,242.74) --
	(175.69,242.08) --
	(176.30,241.43) --
	(176.92,240.79) --
	(177.55,240.17) --
	(178.19,239.55) --
	(178.84,238.94) --
	(179.50,238.35) --
	(180.18,237.77) --
	(180.86,237.20) --
	(181.55,236.64) --
	(182.25,236.09) --
	(182.96,235.56) --
	(183.69,235.04) --
	(184.42,234.53) --
	(185.15,234.04) --
	(185.90,233.55) --
	(186.66,233.09) --
	(187.42,232.63) --
	(188.19,232.19) --
	(188.97,231.76) --
	(189.76,231.34) --
	(190.55,230.94) --
	(191.35,230.55) --
	(192.16,230.18) --
	(192.97,229.82) --
	(193.79,229.48) --
	(194.62,229.15) --
	(195.45,228.83) --
	(196.29,228.53) --
	(197.13,228.24) --
	(197.97,227.97) --
	(198.83,227.71) --
	(199.68,227.47) --
	(200.54,227.25) --
	(201.41,227.03) --
	(202.27,226.84) --
	(203.14,226.65) --
	(204.02,226.49) --
	(204.89,226.34) --
	(205.77,226.20) --
	(206.65,226.08) --
	(207.54,225.98) --
	(208.42,225.89) --
	(209.31,225.81) --
	(210.19,225.76) --
	(211.08,225.71) --
	(211.97,225.69) --
	(212.86,225.67) --
	(213.75,225.68) --
	(214.64,225.70) --
	(215.53,225.73) --
	(216.42,225.78) --
	(217.30,225.85) --
	(218.19,225.93) --
	(219.07,226.03) --
	(219.96,226.14) --
	(220.84,226.27) --
	(221.71,226.41) --
	(222.59,226.57) --
	(223.46,226.74) --
	(224.33,226.93) --
	(225.19,227.14) --
	(226.06,227.36) --
	(226.91,227.59) --
	(227.77,227.84) --
	(228.62,228.11) --
	(229.46,228.38) --
	(230.30,228.68) --
	(231.13,228.99) --
	(231.96,229.31) --
	(232.79,229.65) --
	(233.60,230.00) --
	(234.41,230.37) --
	(235.22,230.75) --
	(236.01,231.14) --
	(236.80,231.55) --
	(237.59,231.97) --
	(238.36,232.41) --
	(239.13,232.86) --
	(239.89,233.32) --
	(240.64,233.79) --
	(241.38,234.28) --
	(242.12,234.78) --
	(242.84,235.30) --
	(243.56,235.83) --
	(244.27,236.37) --
	(244.96,236.92) --
	(245.65,237.48) --
	(246.33,238.06) --
	(247.00,238.65) --
	(247.65,239.25) --
	(248.30,239.86) --
	(248.94,240.48) --
	(249.56,241.11) --
	(250.17,241.75) --
	(250.78,242.41) --
	(251.37,243.07) --
	(251.95,243.75) --
	(252.51,244.43) --
	(252.94,244.97);
\definecolor[named]{drawColor}{rgb}{0.00,0.00,1.00}

\path[draw=drawColor,line width= 1.2pt,line join=round,line cap=round] (231.08,252.94) --
	(231.47,252.29) --
	(231.90,251.61) --
	(232.34,250.93) --
	(232.80,250.26) --
	(233.26,249.60) --
	(233.74,248.94) --
	(234.23,248.30) --
	(234.73,247.66) --
	(235.24,247.04) --
	(235.76,246.42) --
	(236.29,245.81) --
	(236.83,245.21) --
	(237.38,244.62) --
	(237.95,244.04) --
	(238.52,243.47) --
	(239.10,242.91) --
	(239.69,242.36) --
	(240.30,241.82) --
	(240.91,241.29) --
	(241.53,240.77) --
	(242.16,240.26) --
	(242.80,239.77) --
	(243.44,239.28) --
	(244.10,238.81) --
	(244.76,238.35) --
	(245.43,237.90) --
	(246.11,237.46) --
	(246.80,237.03) --
	(247.49,236.62) --
	(248.20,236.21) --
	(248.90,235.82) --
	(249.62,235.45) --
	(250.34,235.08) --
	(251.07,234.73) --
	(251.80,234.39) --
	(252.54,234.06) --
	(252.94,233.89);
\definecolor[named]{drawColor}{rgb}{0.00,0.00,0.00}

\path[draw=drawColor,line width= 0.4pt,dash pattern=on 1pt off 3pt ,line join=round,line cap=round] (225.79,252.94) --
	(226.20,252.18) --
	(226.63,251.40) --
	(227.08,250.63) --
	(227.54,249.87) --
	(228.01,249.12) --
	(228.50,248.37) --
	(229.00,247.64) --
	(229.51,246.91) --
	(230.03,246.19) --
	(230.57,245.48) --
	(231.12,244.78) --
	(231.68,244.09) --
	(232.25,243.41) --
	(232.83,242.74) --
	(233.43,242.08) --
	(234.04,241.43) --
	(234.66,240.79) --
	(235.29,240.17) --
	(235.93,239.55) --
	(236.58,238.94) --
	(237.24,238.35) --
	(237.92,237.77) --
	(238.60,237.20) --
	(239.29,236.64) --
	(239.99,236.09) --
	(240.71,235.56) --
	(241.43,235.04) --
	(242.16,234.53) --
	(242.89,234.04) --
	(243.64,233.55) --
	(244.40,233.09) --
	(245.16,232.63) --
	(245.93,232.19) --
	(246.71,231.76) --
	(247.50,231.34) --
	(248.29,230.94) --
	(249.09,230.55) --
	(249.90,230.18) --
	(250.71,229.82) --
	(251.53,229.48) --
	(252.36,229.15) --
	(252.94,228.92);

\path[draw=drawColor,line width= 1.2pt,line join=round,line cap=round] (172.67,126.47) --
	(172.66,127.28) --
	(172.64,128.09) --
	(172.60,128.90) --
	(172.55,129.70) --
	(172.49,130.51) --
	(172.41,131.31) --
	(172.32,132.12) --
	(172.22,132.92) --
	(172.10,133.72) --
	(171.96,134.52) --
	(171.81,135.31) --
	(171.65,136.10) --
	(171.48,136.89) --
	(171.29,137.68) --
	(171.08,138.46) --
	(170.87,139.24) --
	(170.64,140.02) --
	(170.39,140.79) --
	(170.14,141.55) --
	(169.87,142.31) --
	(169.58,143.07) --
	(169.29,143.82) --
	(168.97,144.57) --
	(168.65,145.31) --
	(168.32,146.05) --
	(167.97,146.78) --
	(167.60,147.50) --
	(167.23,148.22) --
	(166.84,148.93) --
	(166.44,149.63) --
	(166.03,150.32) --
	(165.61,151.01) --
	(165.17,151.69) --
	(164.73,152.37) --
	(164.27,153.03) --
	(163.80,153.69) --
	(163.32,154.34) --
	(162.82,154.98) --
	(162.32,155.61) --
	(161.80,156.23) --
	(161.28,156.85) --
	(160.74,157.45) --
	(160.19,158.05) --
	(159.63,158.63) --
	(159.07,159.21) --
	(158.49,159.77) --
	(157.90,160.33) --
	(157.30,160.87) --
	(156.70,161.41) --
	(156.08,161.93) --
	(155.45,162.44) --
	(154.82,162.95) --
	(154.18,163.44) --
	(153.53,163.92) --
	(152.87,164.38) --
	(152.20,164.84) --
	(151.52,165.28) --
	(150.84,165.72) --
	(150.15,166.14) --
	(149.45,166.55) --
	(148.75,166.94) --
	(148.04,167.33) --
	(147.32,167.70) --
	(146.59,168.06) --
	(145.86,168.40) --
	(145.13,168.73) --
	(144.38,169.05) --
	(143.64,169.36) --
	(142.88,169.65) --
	(142.12,169.94) --
	(141.36,170.20) --
	(140.59,170.46) --
	(139.82,170.70) --
	(139.05,170.92) --
	(138.27,171.14) --
	(137.48,171.34) --
	(136.70,171.52) --
	(135.91,171.69) --
	(135.11,171.85) --
	(134.32,172.00) --
	(133.52,172.13) --
	(132.72,172.24) --
	(131.92,172.35) --
	(131.11,172.43) --
	(130.31,172.51) --
	(129.50,172.57) --
	(128.70,172.61) --
	(127.89,172.65) --
	(127.08,172.66) --
	(126.27,172.67) --
	(125.46,172.66) --
	(124.65,172.63) --
	(123.85,172.59) --
	(123.04,172.54) --
	(122.23,172.47) --
	(121.43,172.39) --
	(120.63,172.30) --
	(119.83,172.19) --
	(119.03,172.06) --
	(118.23,171.93) --
	(117.44,171.77) --
	(116.64,171.61) --
	(115.86,171.43) --
	(115.07,171.24) --
	(114.29,171.03) --
	(113.51,170.81) --
	(112.74,170.58) --
	(111.97,170.33) --
	(111.20,170.07) --
	(110.44,169.80) --
	(109.69,169.51) --
	(108.93,169.21) --
	(108.19,168.90) --
	(107.45,168.57) --
	(106.72,168.23) --
	(105.99,167.88) --
	(105.27,167.51) --
	(104.55,167.14) --
	(103.84,166.75) --
	(103.14,166.34) --
	(102.45,165.93) --
	(101.76,165.50) --
	(101.08,165.06) --
	(100.41,164.61) --
	( 99.75,164.15) --
	( 99.09,163.68) --
	( 98.44,163.19) --
	( 97.81,162.70) --
	( 97.18,162.19) --
	( 96.56,161.67) --
	( 95.94,161.14) --
	( 95.34,160.60) --
	( 94.75,160.05) --
	( 94.17,159.49) --
	( 93.59,158.92) --
	( 93.03,158.34) --
	( 92.48,157.75) --
	( 91.94,157.15) --
	( 91.40,156.54) --
	( 90.88,155.92) --
	( 90.37,155.30) --
	( 89.87,154.66) --
	( 89.39,154.02) --
	( 88.91,153.36) --
	( 88.45,152.70) --
	( 87.99,152.03) --
	( 87.55,151.35) --
	( 87.12,150.67) --
	( 86.70,149.98) --
	( 86.30,149.28) --
	( 85.91,148.57) --
	( 85.53,147.86) --
	( 85.16,147.14) --
	( 84.80,146.41) --
	( 84.46,145.68) --
	( 84.13,144.94) --
	( 83.81,144.20) --
	( 83.51,143.45) --
	( 83.22,142.69) --
	( 82.94,141.93) --
	( 82.68,141.17) --
	( 82.43,140.40) --
	( 82.19,139.63) --
	( 81.97,138.85) --
	( 81.76,138.07) --
	( 81.56,137.29) --
	( 81.38,136.50) --
	( 81.21,135.71) --
	( 81.06,134.91) --
	( 80.91,134.12) --
	( 80.79,133.32) --
	( 80.67,132.52) --
	( 80.58,131.72) --
	( 80.49,130.91) --
	( 80.42,130.11) --
	( 80.36,129.30) --
	( 80.32,128.49) --
	( 80.29,127.69) --
	( 80.28,126.88) --
	( 80.28,126.07) --
	( 80.29,125.26) --
	( 80.32,124.45) --
	( 80.36,123.64) --
	( 80.42,122.84) --
	( 80.49,122.03) --
	( 80.58,121.23) --
	( 80.67,120.43) --
	( 80.79,119.63) --
	( 80.91,118.83) --
	( 81.06,118.03) --
	( 81.21,117.24) --
	( 81.38,116.45) --
	( 81.56,115.66) --
	( 81.76,114.87) --
	( 81.97,114.09) --
	( 82.19,113.32) --
	( 82.43,112.54) --
	( 82.68,111.78) --
	( 82.94,111.01) --
	( 83.22,110.25) --
	( 83.51,109.50) --
	( 83.81,108.75) --
	( 84.13,108.00) --
	( 84.46,107.27) --
	( 84.80,106.53) --
	( 85.16,105.81) --
	( 85.53,105.09) --
	( 85.91,104.37) --
	( 86.30,103.67) --
	( 86.70,102.97) --
	( 87.12,102.28) --
	( 87.55,101.59) --
	( 87.99,100.91) --
	( 88.45,100.24) --
	( 88.91, 99.58) --
	( 89.39, 98.93) --
	( 89.87, 98.28) --
	( 90.37, 97.65) --
	( 90.88, 97.02) --
	( 91.40, 96.40) --
	( 91.94, 95.79) --
	( 92.48, 95.19) --
	( 93.03, 94.60) --
	( 93.59, 94.02) --
	( 94.17, 93.45) --
	( 94.75, 92.89) --
	( 95.34, 92.34) --
	( 95.94, 91.80) --
	( 96.56, 91.27) --
	( 97.18, 90.76) --
	( 97.81, 90.25) --
	( 98.44, 89.75) --
	( 99.09, 89.27) --
	( 99.75, 88.79) --
	(100.41, 88.33) --
	(101.08, 87.88) --
	(101.76, 87.44) --
	(102.45, 87.02) --
	(103.14, 86.60) --
	(103.84, 86.20) --
	(104.55, 85.81) --
	(105.27, 85.43) --
	(105.99, 85.07) --
	(106.72, 84.72) --
	(107.45, 84.38) --
	(108.19, 84.05) --
	(108.93, 83.74) --
	(109.69, 83.44) --
	(110.44, 83.15) --
	(111.20, 82.87) --
	(111.97, 82.61) --
	(112.74, 82.37) --
	(113.51, 82.13) --
	(114.29, 81.91) --
	(115.07, 81.71) --
	(115.86, 81.51) --
	(116.64, 81.34) --
	(117.44, 81.17) --
	(118.23, 81.02) --
	(119.03, 80.88) --
	(119.83, 80.76) --
	(120.63, 80.65) --
	(121.43, 80.55) --
	(122.23, 80.47) --
	(123.04, 80.41) --
	(123.85, 80.35) --
	(124.65, 80.31) --
	(125.46, 80.29) --
	(126.27, 80.28) --
	(127.08, 80.28) --
	(127.89, 80.30) --
	(128.70, 80.33) --
	(129.50, 80.38) --
	(130.31, 80.44) --
	(131.11, 80.51) --
	(131.92, 80.60) --
	(132.72, 80.70) --
	(133.52, 80.82) --
	(134.32, 80.95) --
	(135.11, 81.09) --
	(135.91, 81.25) --
	(136.70, 81.42) --
	(137.48, 81.61) --
	(138.27, 81.81) --
	(139.05, 82.02) --
	(139.82, 82.25) --
	(140.59, 82.49) --
	(141.36, 82.74) --
	(142.12, 83.01) --
	(142.88, 83.29) --
	(143.64, 83.58) --
	(144.38, 83.89) --
	(145.13, 84.21) --
	(145.86, 84.54) --
	(146.59, 84.89) --
	(147.32, 85.25) --
	(148.04, 85.62) --
	(148.75, 86.00) --
	(149.45, 86.40) --
	(150.15, 86.81) --
	(150.84, 87.23) --
	(151.52, 87.66) --
	(152.20, 88.10) --
	(152.87, 88.56) --
	(153.53, 89.03) --
	(154.18, 89.51) --
	(154.82, 90.00) --
	(155.45, 90.50) --
	(156.08, 91.01) --
	(156.70, 91.54) --
	(157.30, 92.07) --
	(157.90, 92.62) --
	(158.49, 93.17) --
	(159.07, 93.74) --
	(159.63, 94.31) --
	(160.19, 94.90) --
	(160.74, 95.49) --
	(161.28, 96.10) --
	(161.80, 96.71) --
	(162.32, 97.33) --
	(162.82, 97.96) --
	(163.32, 98.61) --
	(163.80, 99.25) --
	(164.27, 99.91) --
	(164.73,100.58) --
	(165.17,101.25) --
	(165.61,101.93) --
	(166.03,102.62) --
	(166.44,103.32) --
	(166.84,104.02) --
	(167.23,104.73) --
	(167.60,105.45) --
	(167.97,106.17) --
	(168.32,106.90) --
	(168.65,107.63) --
	(168.97,108.38) --
	(169.29,109.12) --
	(169.58,109.87) --
	(169.87,110.63) --
	(170.14,111.39) --
	(170.39,112.16) --
	(170.64,112.93) --
	(170.87,113.70) --
	(171.08,114.48) --
	(171.29,115.27) --
	(171.48,116.05) --
	(171.65,116.84) --
	(171.81,117.63) --
	(171.96,118.43) --
	(172.10,119.23) --
	(172.22,120.03) --
	(172.32,120.83) --
	(172.41,121.63) --
	(172.49,122.44) --
	(172.55,123.24) --
	(172.60,124.05) --
	(172.64,124.86) --
	(172.66,125.66) --
	(172.67,126.47);
\definecolor[named]{drawColor}{rgb}{1.00,0.00,0.00}

\node[text=drawColor,rotate=-60.00,anchor=base,inner sep=0pt, outer sep=0pt, scale=  1.00] at (102.25,180.36) {$\mathbf{k}_{gv}$};

\node[text=drawColor,anchor=base,inner sep=0pt, outer sep=0pt, scale=  1.00] at (154.04,129.49) {$\mathbf{k}_{gx}$};
\end{scope}
\end{tikzpicture}

\caption{Example reciprocal lattices for two oblique gratings. $\lambda_{gx}= 600$ nm, $\lambda_{gv}=400$ nm and the angles between the two periodicities are (a) $\alpha = 75^\circ$ and (b) $\alpha= 60^\circ$. }
\end{figure} 
Consequently, the SPP propagation on such a grating is not constrained by many symmetry considerations which dictated the characteristics of SPPs on highly symmetric gratings. The modes exhibit strong polarization conversion and the interaction of the modes to form band-gaps do not occur at points of any significance other than that they exist where counter-propagating modes meet.
\section{coupling conditions and mode interaction for square-profile gratings}
A square profile grating such as those fabricated using EBL contain little or no even-Fourier components in their surface profile. Consequently, a plane wave  incident on any such grating may be modified by values of $\pm m\mathbf{k}_g$ , where $m$ is odd, by direct scattering events, as the plane wave scatters from the surface profile containing odd-Fourier components. A direct scattering event, where the plane wave is modified by $\pm n\mathbf{k}_g$, where $n$ is even, are not possible. There does exist, however, the oppertunity for the plane wave to be modified by a total value of $\pm n\mathbf{k}_g$ through multiple-scattering processes, summing the contributions from  $\pm m\mathbf{k}_g$ scattering events. These multiple scattering events are found in experiment to be so weak, they are rarely observed. 

These scattering efficiencies also relate to the strength of interaction between counter-propagating SPP modes. Modes that are separated by a direct scattering event tend to interact strongly, while little, if any, interaction is observed for modes sperated by a multiple-scattering process.  The coupling of light does not effect the eigenmodes of the scattered SPPs, and so the band-gap character of a uncoupled mode can be evident if the mode interacts strongly with awell coupled SPP mode. IN this instance, it is not uncommon for the well coupled SPP to excite the uncoupled (to light)  SPP, and baggaps can be fully realized.

\section{Identifying zero group velocity at BZ boundaries using equi-energy contours}
The formation of photonic bandgaps on an oblique bigrating are of interest, as on a surface with such broken symmetry the locations in k-space of  SPP standing wavesneed not occur at the BZ boundary. To show this clearly, we must identify how photonic bandgaps are illustrated in the recorded equi-energy contours scatterometry results.
The equi-energy image obtained using scatterometry maps k-space at a single frequency, with SPP bands seen as a anomalous discontinuity of the reflected light. 
The group velocity of a general propagating wave is defined as,
\begin{equation}
\mathbf{v}_g = \nabla_k\: \omega(\mathbf{k})
\end{equation}
Where $\mathbf{v}_g$ is the group velocity, $\omega(\mathbf{k})$ is the angular frequency of the wave as a function of wavevector, $\mathbf{k}$, and $\nabla_k$ is the gradient operator with respect to k. For a small change in frequency $d\omega$, the corresponding small movement in k-space, $d\mathbf{k}$, is related to this group velocity simply by,
\begin{equation}
d\omega = \nabla_k\: \omega(\mathbf{k}) \cdot d\mathbf{k}
\end{equation}
For an equi-energy contour, there must be no change in frequency, $d\omega=0$. Setting $d\omega=0$ restricts the values of $d\mathbf{k}$ to those values that move along a contour of equal frequency. It is then apparent that in the resultant expression,
\begin{equation}
\mathbf{v}_g \cdot d\mathbf{k} = 0
\end{equation}
That $\mathbf{v}_g \perp d\mathbf{k}$.
This is true for any general contour of constant frequency. If the group velocity in one direction falls to zero at a boundary, such as at the BZ boundary,  the iso-frequency SPP contour will intersect that boundary perpendicularly.

Plasmonic bandgaps on surface relief gratings are covered in section \ref{}. To briefly recap, when a SPP meets an equivalent, counter-propagating SPP a standing wave forms. There are generally two possible arrangements of the electric field  for a SPP standing wave on a grating, which will generally differ in energy. This leads to a upper and lower energy band, with an energy range between where there are no possible solutions for a SPP, and SPP propagation is forbidden. This is a plasmonic band-gap
The energy gap size is dependant on the two possible field distributions, and so is linked intimately to the surface geometry. The surface profile also provides the scattering mechanism by which SPPs Bragg scatter to meet counter-propagating SPPs and form these standing waves. The strength of this scattering, and so the amplitude of the Bragg scattered SPP, was discussed earlier \ref{} and also effects the band-gap size.
\begin{figure}
\caption{equi-energy results for rectangular grating}
\end{figure}
An example equi-energy surface for a rectangular bigrating is shown in figure \ref{}, obtained using imaging scatterometry. Using the established notation (section \ref{}), the grating parameters are designed as $\alpha=90^\circ$, $\lambda_{gx}=600 $nm, $\lambda_{gv}\equiv\lambda_{gy}=400 $nm  The wavelength of the illuminating light is $550 \pm 5$ nm, with the blue circles indicating the calculated positions of diffracted light cones. The polarization is set as to couple strongly to the ($\pm 1,0$) SPPs, which are measured as dark contours following the ($\pm 1,0$) diffracted light circles. In this scattergram, the ($\pm 2,0$) SPP are also weakly coupled, while the ($0,\pm 1$) require a rotation of the polarization to be observed. The ($\pm 1, \pm 1$) and the ($\pm 1, \mp 1$) SPPs require multiple scattering events to couple to the incident field, and are so weakly coupled they are not observed experimentally.
The figure shows bandgaps occurring at the boundary of the first BZ in the $\mathbf{k}_{gy}$ direction. The SPP contour intersects the BZ (red lines) perpendicularly, which as discussed previously, requires that the group velocity in the $\mathbf{k}_{gy}$ direction is zero and standing waves in that direction have formed at the boundary.  This is where the ($\pm 1, 0$) SPPs has crossed and interacted with a ($\pm 1, \pm 1$) or ($\pm 1, \mp 1$) SPP, forming two standing waves with a high and low energy solution. While these SPPs are not coupled to strongly by the incident field, the SPPs themselves only require a single scattering event to Bragg scatter cross, leading to the discontinuities observed in the ($\pm 1,0$) SPP contours. 
Figure \ref{} also shows the FEM modelled results of the bandgap area in k-space, showing good agreement between the two. The two field distributions for the two possible standingwaves are shown also. 
 

\section{symmetry considerations: BZ boundaries}
A Brillioun Zone boundary in reciprocal space outlines a unit cell in the reciprocal lattice and contains in the boundary points of high-symmetry. One way to determine a boundary that contains the maximum amount of high-symmetry points is to connect the perpendicular bisectors of the vectors connecting the nearest neighbours to one lattice point, a method known and the Wigner-Seitz method. The area mapped out by this method, for the highly-symmetric cases of square, rectangular or hexagonal lattices, is exactly equivalent to the area we call the Brillioun Zone, and no other choice of unit cell contains as many points of high-symmetry in the boundary.

In the oblique case, we may again use the Wigner-Seitz method to construct a perfectly valid Brillioun Zone. However, this zone is not a unique choice, we can easily contain the points of high-symmetry using instead a trapezoidal shape, as shown in figure \ref{}.

\begin{figure}
\centering %LaTeX with PSTricks extensions
%%Creator: inkscape 0.48.2
%%Please note this file requires PSTricks extensions
\psset{xunit=.5pt,yunit=.5pt,runit=.5pt}
\begin{pspicture}(359.92419434,324.88348389)
{
\newrgbcolor{curcolor}{0 0 0}
\pscustom[linewidth=1.84079344,linecolor=curcolor]
{
\newpath
\moveto(179.95596336,162.43775739)
\lineto(179.95596336,162.43775739)
}
}
{
\newrgbcolor{curcolor}{0 0 0}
\pscustom[linestyle=none,fillstyle=solid,fillcolor=curcolor]
{
\newpath
\moveto(13.25861844,6.62318327)
\curveto(13.25861844,-2.20771645)(-0.00000309,-2.20771645)(-0.00000309,6.62318327)
\curveto(-0.00000309,15.46144616)(13.25861844,15.46144616)(13.25861844,6.62318327)
}
}
{
\newrgbcolor{curcolor}{0 0 0}
\pscustom[linestyle=none,fillstyle=solid,fillcolor=curcolor]
{
\newpath
\moveto(37.29447196,318.25478589)
\curveto(37.29447196,309.42388618)(24.04566799,309.42388618)(24.04566799,318.25478589)
\curveto(24.04566799,327.09304878)(37.29447196,327.09304878)(37.29447196,318.25478589)
}
}
{
\newrgbcolor{curcolor}{0 0 0}
\pscustom[linestyle=none,fillstyle=solid,fillcolor=curcolor]
{
\newpath
\moveto(335.87853069,6.62318327)
\curveto(335.87853069,-2.20771645)(322.62727233,-2.20771645)(322.62727233,6.62318327)
\curveto(322.62727233,15.46144616)(335.87853069,15.46144616)(335.87853069,6.62318327)
}
}
{
\newrgbcolor{curcolor}{0 0 0}
\pscustom[linestyle=none,fillstyle=solid,fillcolor=curcolor]
{
\newpath
\moveto(294.12442675,162.43775739)
\curveto(294.12442675,153.60931206)(280.87562278,153.60931206)(280.87562278,162.43775739)
\curveto(280.87562278,171.27847467)(294.12442675,171.27847467)(294.12442675,162.43775739)
}
}
{
\newrgbcolor{curcolor}{0 0 0}
\pscustom[linestyle=none,fillstyle=solid,fillcolor=curcolor]
{
\newpath
\moveto(359.92420177,318.25478589)
\curveto(359.92420177,309.42388618)(346.66312584,309.42388618)(346.66312584,318.25478589)
\curveto(346.66312584,327.09304878)(359.92420177,327.09304878)(359.92420177,318.25478589)
}
}
{
\newrgbcolor{curcolor}{0.88235295 0 0}
\pscustom[linewidth=1.963513,linecolor=curcolor,strokeopacity=0.29411766,linestyle=dashed,dash=2 1]
{
\newpath
\moveto(114.1684603,6.62595182)
\lineto(245.7508296,318.25754659)
\lineto(245.7508296,318.25754659)
}
}
{
\newrgbcolor{curcolor}{0.88235295 0 0}
\pscustom[linewidth=1.963513,linecolor=curcolor,strokeopacity=0.29411766,linestyle=dashed,dash=2 1]
{
\newpath
\moveto(138.20922259,318.25754659)
\lineto(221.71006731,6.62595182)
}
}
{
\newrgbcolor{curcolor}{0.88235295 0 0}
\pscustom[linewidth=1.963513,linecolor=curcolor,strokeopacity=0.29411766,linestyle=dashed,dash=2 2]
{
\newpath
\moveto(72.41926513,162.44235201)
\lineto(287.50002476,162.44235201)
}
}
{
\newrgbcolor{curcolor}{0.88235295 0 0}
\pscustom[linewidth=1.963513,linecolor=curcolor]
{
\newpath
\moveto(125.77635645,93.30028516)
\lineto(125.77635645,232.17918658)
\lineto(191.82653828,249.58429596)
\lineto(233.41558933,232.19628878)
\lineto(233.41558933,93.17454178)
\lineto(167.57061424,75.35403159)
\closepath
}
}
{
\newrgbcolor{curcolor}{0 0 0.88235295}
\pscustom[linewidth=1.963513,linecolor=curcolor]
{
\newpath
\moveto(92.81013165,84.42063141)
\lineto(158.80144737,240.88175109)
\lineto(265.71575124,240.84208812)
\lineto(201.07079676,84.42053324)
\closepath
}
}
{
\newrgbcolor{curcolor}{0 1 0}
\pscustom[linestyle=none,fillstyle=solid,fillcolor=curcolor]
{
\newpath
\moveto(104.42550261,192.58960702)
\lineto(106.54241506,192.58960702)
\lineto(106.54241506,179.70405296)
\lineto(104.42550261,179.70405296)
\lineto(104.42550261,192.58960702)
\moveto(104.42550261,197.60576914)
\lineto(106.54241506,197.60576914)
\lineto(106.54241506,194.92511369)
\lineto(104.42550261,194.92511369)
\lineto(104.42550261,197.60576914)
}
}
{
\newrgbcolor{curcolor}{0 1 0}
\pscustom[linestyle=none,fillstyle=solid,fillcolor=curcolor]
{
\newpath
\moveto(110.63935771,176.26293744)
\lineto(113.10720617,176.26293744)
\lineto(113.10720617,184.7807538)
\lineto(110.42248618,184.24231413)
\lineto(110.42248618,185.61832661)
\lineto(113.09224951,186.15676627)
\lineto(114.6028719,186.15676627)
\lineto(114.6028719,176.26293744)
\lineto(117.07072037,176.26293744)
\lineto(117.07072037,174.99162157)
\lineto(110.63935771,174.99162157)
\lineto(110.63935771,176.26293744)
}
}
{
\newrgbcolor{curcolor}{0 1 0}
\pscustom[linestyle=none,fillstyle=solid,fillcolor=curcolor]
{
\newpath
\moveto(248.92344313,144.88453428)
\lineto(251.04035558,144.88453428)
\lineto(251.04035558,131.99898021)
\lineto(248.92344313,131.99898021)
\lineto(248.92344313,144.88453428)
\moveto(248.92344313,149.90069639)
\lineto(251.04035558,149.90069639)
\lineto(251.04035558,147.22004095)
\lineto(248.92344313,147.22004095)
\lineto(248.92344313,149.90069639)
}
}
{
\newrgbcolor{curcolor}{0 1 0}
\pscustom[linestyle=none,fillstyle=solid,fillcolor=curcolor]
{
\newpath
\moveto(257.4601227,149.17588398)
\lineto(257.4601227,142.79063174)
\lineto(255.50427967,142.79063174)
\lineto(255.50427967,149.17588398)
\lineto(257.4601227,149.17588398)
}
}
{
\newrgbcolor{curcolor}{0 1 0}
\pscustom[linestyle=none,fillstyle=solid,fillcolor=curcolor]
{
\newpath
\moveto(261.6260951,128.5578647)
\lineto(264.09394356,128.5578647)
\lineto(264.09394356,137.07568106)
\lineto(261.40922357,136.53724139)
\lineto(261.40922357,137.91325387)
\lineto(264.0789869,138.45169353)
\lineto(265.58960929,138.45169353)
\lineto(265.58960929,128.5578647)
\lineto(268.05745776,128.5578647)
\lineto(268.05745776,127.28654883)
\lineto(261.6260951,127.28654883)
\lineto(261.6260951,128.5578647)
}
}
{
\newrgbcolor{curcolor}{0 1 0}
\pscustom[linestyle=none,fillstyle=solid,fillcolor=curcolor]
{
\newpath
\moveto(136.69903978,266.09694715)
\lineto(138.81595223,266.09694715)
\lineto(138.81595223,253.21139309)
\lineto(136.69903978,253.21139309)
\lineto(136.69903978,266.09694715)
\moveto(136.69903978,271.11310927)
\lineto(138.81595223,271.11310927)
\lineto(138.81595223,268.43245382)
\lineto(136.69903978,268.43245382)
\lineto(136.69903978,271.11310927)
}
}
{
\newrgbcolor{curcolor}{0 1 0}
\pscustom[linestyle=none,fillstyle=solid,fillcolor=curcolor]
{
\newpath
\moveto(143.95238256,249.77027757)
\lineto(149.22460428,249.77027757)
\lineto(149.22460428,248.4989617)
\lineto(142.1351487,248.4989617)
\lineto(142.1351487,249.77027757)
\curveto(142.70848553,250.36355645)(143.48872371,251.15875127)(144.47586557,252.15586442)
\curveto(145.46798605,253.1579558)(146.09117949,253.8035842)(146.34544774,254.09275155)
\curveto(146.82904051,254.63617062)(147.16556496,255.09484099)(147.35502211,255.46876402)
\curveto(147.54945212,255.84765866)(147.64667029,256.21908194)(147.64667693,256.58303499)
\curveto(147.64667029,257.17630706)(147.4372773,257.65990516)(147.01849732,258.03383075)
\curveto(146.60469087,258.40773728)(146.06375898,258.59469531)(145.3957,258.5947054)
\curveto(144.92206861,258.59469531)(144.42102109,258.51243378)(143.89255593,258.34792056)
\curveto(143.36907057,258.18338764)(142.80819648,257.93411027)(142.20993198,257.60008769)
\lineto(142.20993198,259.12566674)
\curveto(142.81816758,259.36994794)(143.38651999,259.5544132)(143.91499092,259.67906306)
\curveto(144.44345605,259.80369057)(144.92705415,259.86600991)(145.36578668,259.86602128)
\curveto(146.52242934,259.86600991)(147.44475562,259.57684816)(148.13276829,258.99853515)
\curveto(148.82076672,258.42020115)(149.1647695,257.64744129)(149.16477765,256.68025327)
\curveto(149.1647695,256.22157472)(149.07752242,255.78533931)(148.90303614,255.37154575)
\curveto(148.73351964,254.96272398)(148.42192292,254.47912588)(147.96824506,253.92074999)
\curveto(147.84359942,253.77616369)(147.4472484,253.3573777)(146.7791908,252.66439077)
\curveto(146.11112168,251.97638105)(145.16885321,251.01167762)(143.95238256,249.77027757)
}
}
{
\newrgbcolor{curcolor}{0 1 0}
\pscustom[linestyle=none,fillstyle=solid,fillcolor=curcolor]
{
\newpath
\moveto(217.12008128,72.28992089)
\lineto(219.23699373,72.28992089)
\lineto(219.23699373,59.40436683)
\lineto(217.12008128,59.40436683)
\lineto(217.12008128,72.28992089)
\moveto(217.12008128,77.30608301)
\lineto(219.23699373,77.30608301)
\lineto(219.23699373,74.62542756)
\lineto(217.12008128,74.62542756)
\lineto(217.12008128,77.30608301)
}
}
{
\newrgbcolor{curcolor}{0 1 0}
\pscustom[linestyle=none,fillstyle=solid,fillcolor=curcolor]
{
\newpath
\moveto(225.65676084,76.58127059)
\lineto(225.65676084,70.19601836)
\lineto(223.70091782,70.19601836)
\lineto(223.70091782,76.58127059)
\lineto(225.65676084,76.58127059)
}
}
{
\newrgbcolor{curcolor}{0 1 0}
\pscustom[linestyle=none,fillstyle=solid,fillcolor=curcolor]
{
\newpath
\moveto(230.86222093,55.96325132)
\lineto(236.13444264,55.96325132)
\lineto(236.13444264,54.69193544)
\lineto(229.04498706,54.69193544)
\lineto(229.04498706,55.96325132)
\curveto(229.6183239,56.55653019)(230.39856208,57.35172501)(231.38570394,58.34883816)
\curveto(232.37782442,59.35092954)(233.00101785,59.99655794)(233.25528611,60.28572529)
\curveto(233.73887888,60.82914437)(234.07540333,61.28781473)(234.26486048,61.66173776)
\curveto(234.45929049,62.0406324)(234.55650866,62.41205569)(234.55651529,62.77600873)
\curveto(234.55650866,63.3692808)(234.34711567,63.8528789)(233.92833569,64.2268045)
\curveto(233.51452924,64.60071102)(232.97359734,64.78766905)(232.30553836,64.78767915)
\curveto(231.83190697,64.78766905)(231.33085945,64.70540752)(230.8023943,64.5408943)
\curveto(230.27890894,64.37636139)(229.71803485,64.12708401)(229.11977035,63.79306143)
\lineto(229.11977035,65.31864048)
\curveto(229.72800594,65.56292168)(230.29635835,65.74738694)(230.82482929,65.8720368)
\curveto(231.35329442,65.99666431)(231.83689252,66.05898365)(232.27562505,66.05899502)
\curveto(233.43226771,66.05898365)(234.35459399,65.7698219)(235.04260666,65.1915089)
\curveto(235.73060509,64.61317489)(236.07460786,63.84041503)(236.07461601,62.87322701)
\curveto(236.07460786,62.41454846)(235.98736078,61.97831306)(235.81287451,61.56451949)
\curveto(235.64335801,61.15569773)(235.33176129,60.67209962)(234.87808343,60.11372373)
\curveto(234.75343779,59.96913743)(234.35708676,59.55035144)(233.68902917,58.85736451)
\curveto(233.02096004,58.1693548)(232.07869157,57.20465136)(230.86222093,55.96325132)
}
}
{
\newrgbcolor{curcolor}{0 1 0}
\pscustom[linestyle=none,fillstyle=solid,fillcolor=curcolor]
{
\newpath
\moveto(231.19843593,265.5612475)
\lineto(233.31534838,265.5612475)
\lineto(233.31534838,252.67569343)
\lineto(231.19843593,252.67569343)
\lineto(231.19843593,265.5612475)
\moveto(231.19843593,270.57740961)
\lineto(233.31534838,270.57740961)
\lineto(233.31534838,267.89675417)
\lineto(231.19843593,267.89675417)
\lineto(231.19843593,270.57740961)
}
}
{
\newrgbcolor{curcolor}{0 1 0}
\pscustom[linestyle=none,fillstyle=solid,fillcolor=curcolor]
{
\newpath
\moveto(241.72728667,253.98331663)
\curveto(242.45018484,253.82875864)(243.01355171,253.50719082)(243.41738895,253.01861223)
\curveto(243.82619594,252.53002352)(244.03060339,251.92677228)(244.03061191,251.20885669)
\curveto(244.03060339,250.10704746)(243.65170178,249.25451884)(242.89390595,248.65126828)
\curveto(242.13609535,248.04801635)(241.0592171,247.74639073)(239.66326796,247.74639052)
\curveto(239.19462235,247.74639073)(238.71102425,247.79375343)(238.2124722,247.88847876)
\curveto(237.7189003,247.97821869)(237.20788169,248.11532124)(236.67941482,248.29978684)
\lineto(236.67941482,249.75806093)
\curveto(237.09819964,249.51376731)(237.55687001,249.32930205)(238.0554273,249.20466461)
\curveto(238.5539795,249.08002468)(239.07496921,249.01770533)(239.61839799,249.01770639)
\curveto(240.5656479,249.01770533)(241.28605951,249.20466336)(241.77963498,249.57858104)
\curveto(242.27818345,249.95249548)(242.52746083,250.49592016)(242.52746784,251.20885669)
\curveto(242.52746083,251.86694571)(242.29563287,252.3804571)(241.83198328,252.7493924)
\curveto(241.37330659,253.12330367)(240.73266374,253.3102617)(239.91005281,253.31026705)
\lineto(238.60882362,253.31026705)
\lineto(238.60882362,254.55166961)
\lineto(239.96987944,254.55166961)
\curveto(240.71272155,254.55166302)(241.28107396,254.69873667)(241.67493837,254.992891)
\curveto(242.06879046,255.29201682)(242.26571959,255.7207739)(242.26572634,256.27916353)
\curveto(242.26571959,256.85249317)(242.06131214,257.29122135)(241.65250339,257.59534937)
\curveto(241.2486679,257.90444369)(240.66785162,258.05899566)(239.91005281,258.05900575)
\curveto(239.49624797,258.05899566)(239.05253425,258.01412573)(238.5789103,257.92439584)
\curveto(238.10528023,257.83464602)(237.58429052,257.69505069)(237.01593961,257.50560943)
\lineto(237.01593961,258.85170859)
\curveto(237.58927607,259.01123522)(238.12522242,259.13088836)(238.62378028,259.21066837)
\curveto(239.12731746,259.29042588)(239.60094447,259.33031026)(240.04466272,259.33032163)
\curveto(241.19133411,259.33031026)(242.09870375,259.06856902)(242.76677436,258.54509712)
\curveto(243.43483047,258.0265896)(243.76886215,257.32362741)(243.7688704,256.43620843)
\curveto(243.76886215,255.81799207)(243.59187521,255.29450959)(243.23790907,254.86575941)
\curveto(242.88392747,254.44198097)(242.38038718,254.14783367)(241.72728667,253.98331663)
}
}
{
\newrgbcolor{curcolor}{0 1 0}
\pscustom[linestyle=none,fillstyle=solid,fillcolor=curcolor]
{
\newpath
\moveto(111.56055459,75.05542541)
\lineto(113.67746705,75.05542541)
\lineto(113.67746705,62.16987134)
\lineto(111.56055459,62.16987134)
\lineto(111.56055459,75.05542541)
\moveto(111.56055459,80.07158752)
\lineto(113.67746705,80.07158752)
\lineto(113.67746705,77.39093208)
\lineto(111.56055459,77.39093208)
\lineto(111.56055459,80.07158752)
}
}
{
\newrgbcolor{curcolor}{0 1 0}
\pscustom[linestyle=none,fillstyle=solid,fillcolor=curcolor]
{
\newpath
\moveto(120.09723416,79.34677511)
\lineto(120.09723416,72.96152287)
\lineto(118.14139113,72.96152287)
\lineto(118.14139113,79.34677511)
\lineto(120.09723416,79.34677511)
}
}
{
\newrgbcolor{curcolor}{0 1 0}
\pscustom[linestyle=none,fillstyle=solid,fillcolor=curcolor]
{
\newpath
\moveto(128.5782022,63.47749454)
\curveto(129.30110037,63.32293655)(129.86446723,63.00136873)(130.26830448,62.51279014)
\curveto(130.67711147,62.02420143)(130.88151892,61.42095019)(130.88152743,60.7030346)
\curveto(130.88151892,59.60122537)(130.50261731,58.74869675)(129.74482148,58.14544619)
\curveto(128.98701088,57.54219426)(127.91013263,57.24056864)(126.51418349,57.24056843)
\curveto(126.04553788,57.24056864)(125.56193978,57.28793134)(125.06338773,57.38265667)
\curveto(124.56981583,57.4723966)(124.05879722,57.60949915)(123.53033035,57.79396475)
\lineto(123.53033035,59.25223884)
\curveto(123.94911517,59.00794522)(124.40778554,58.82347996)(124.90634283,58.69884252)
\curveto(125.40489503,58.57420259)(125.92588474,58.51188325)(126.46931352,58.5118843)
\curveto(127.41656343,58.51188325)(128.13697504,58.69884128)(128.6305505,59.07275895)
\curveto(129.12909898,59.44667339)(129.37837636,59.99009807)(129.37838337,60.7030346)
\curveto(129.37837636,61.36112362)(129.1465484,61.87463501)(128.68289881,62.24357031)
\curveto(128.22422212,62.61748158)(127.58357927,62.80443961)(126.76096834,62.80444496)
\lineto(125.45973915,62.80444496)
\lineto(125.45973915,64.04584752)
\lineto(126.82079497,64.04584752)
\curveto(127.56363708,64.04584093)(128.13198949,64.19291458)(128.5258539,64.48706891)
\curveto(128.91970599,64.78619473)(129.11663512,65.21495181)(129.11664187,65.77334144)
\curveto(129.11663512,66.34667108)(128.91222767,66.78539926)(128.50341892,67.08952729)
\curveto(128.09958343,67.3986216)(127.51876715,67.55317357)(126.76096834,67.55318366)
\curveto(126.3471635,67.55317357)(125.90344978,67.50830364)(125.42982583,67.41857375)
\curveto(124.95619576,67.32882393)(124.43520605,67.1892286)(123.86685514,66.99978734)
\lineto(123.86685514,68.3458865)
\curveto(124.4401916,68.50541313)(124.97613795,68.62506627)(125.47469581,68.70484628)
\curveto(125.97823299,68.78460379)(126.45186,68.82448817)(126.89557825,68.82449954)
\curveto(128.04224964,68.82448817)(128.94961928,68.56274693)(129.61768989,68.03927503)
\curveto(130.285746,67.52076751)(130.61977768,66.81780532)(130.61978593,65.93038634)
\curveto(130.61977768,65.31216998)(130.44279074,64.7886875)(130.0888246,64.35993732)
\curveto(129.734843,63.93615888)(129.23130271,63.64201158)(128.5782022,63.47749454)
}
}
{
\newrgbcolor{curcolor}{0 1 0}
\pscustom[linestyle=none,fillstyle=solid,fillcolor=curcolor]
{
\newpath
\moveto(129.14847849,162.65118356)
\curveto(129.14847849,160.80506763)(127.65190583,159.30849496)(125.80578989,159.30849496)
\curveto(123.95967395,159.30849496)(122.46310129,160.80506763)(122.46310129,162.65118356)
\curveto(122.46310129,164.4972995)(123.95967395,165.99387217)(125.80578989,165.99387217)
\curveto(127.65190583,165.99387217)(129.14847849,164.4972995)(129.14847849,162.65118356)
\closepath
}
}
{
\newrgbcolor{curcolor}{0 1 0}
\pscustom[linewidth=1.38241851,linecolor=curcolor]
{
\newpath
\moveto(129.14847849,162.65118356)
\curveto(129.14847849,160.80506763)(127.65190583,159.30849496)(125.80578989,159.30849496)
\curveto(123.95967395,159.30849496)(122.46310129,160.80506763)(122.46310129,162.65118356)
\curveto(122.46310129,164.4972995)(123.95967395,165.99387217)(125.80578989,165.99387217)
\curveto(127.65190583,165.99387217)(129.14847849,164.4972995)(129.14847849,162.65118356)
\closepath
}
}
{
\newrgbcolor{curcolor}{0 0 0}
\pscustom[linestyle=none,fillstyle=solid,fillcolor=curcolor]
{
\newpath
\moveto(186.5926373,162.43775739)
\curveto(186.5926373,153.60931206)(173.33156138,153.60931206)(173.33156138,162.43775739)
\curveto(173.33156138,171.27847467)(186.5926373,171.27847467)(186.5926373,162.43775739)
}
}
{
\newrgbcolor{curcolor}{0 0 0}
\pscustom[linestyle=none,fillstyle=solid,fillcolor=curcolor]
{
\newpath
\moveto(79.0485759,162.43775739)
\curveto(79.0485759,153.60931206)(65.78995437,153.60931206)(65.78995437,162.43775739)
\curveto(65.78995437,171.27847467)(79.0485759,171.27847467)(79.0485759,162.43775739)
}
}
{
\newrgbcolor{curcolor}{0 1 0}
\pscustom[linestyle=none,fillstyle=solid,fillcolor=curcolor]
{
\newpath
\moveto(162.14413635,240.88174144)
\curveto(162.14413635,239.0356255)(160.64756369,237.53905284)(158.80144775,237.53905284)
\curveto(156.95533181,237.53905284)(155.45875914,239.0356255)(155.45875914,240.88174144)
\curveto(155.45875914,242.72785738)(156.95533181,244.22443004)(158.80144775,244.22443004)
\curveto(160.64756369,244.22443004)(162.14413635,242.72785738)(162.14413635,240.88174144)
\closepath
}
}
{
\newrgbcolor{curcolor}{0 1 0}
\pscustom[linewidth=1.38241851,linecolor=curcolor]
{
\newpath
\moveto(162.14413635,240.88174144)
\curveto(162.14413635,239.0356255)(160.64756369,237.53905284)(158.80144775,237.53905284)
\curveto(156.95533181,237.53905284)(155.45875914,239.0356255)(155.45875914,240.88174144)
\curveto(155.45875914,242.72785738)(156.95533181,244.22443004)(158.80144775,244.22443004)
\curveto(160.64756369,244.22443004)(162.14413635,242.72785738)(162.14413635,240.88174144)
\closepath
}
}
{
\newrgbcolor{curcolor}{0 1 0}
\pscustom[linestyle=none,fillstyle=solid,fillcolor=curcolor]
{
\newpath
\moveto(215.96375279,240.89029236)
\curveto(215.96375279,239.04417642)(214.46718012,237.54760376)(212.62106418,237.54760376)
\curveto(210.77494825,237.54760376)(209.27837558,239.04417642)(209.27837558,240.89029236)
\curveto(209.27837558,242.7364083)(210.77494825,244.23298096)(212.62106418,244.23298096)
\curveto(214.46718012,244.23298096)(215.96375279,242.7364083)(215.96375279,240.89029236)
\closepath
}
}
{
\newrgbcolor{curcolor}{0 1 0}
\pscustom[linewidth=1.38241851,linecolor=curcolor]
{
\newpath
\moveto(215.96375279,240.89029236)
\curveto(215.96375279,239.04417642)(214.46718012,237.54760376)(212.62106418,237.54760376)
\curveto(210.77494825,237.54760376)(209.27837558,239.04417642)(209.27837558,240.89029236)
\curveto(209.27837558,242.7364083)(210.77494825,244.23298096)(212.62106418,244.23298096)
\curveto(214.46718012,244.23298096)(215.96375279,242.7364083)(215.96375279,240.89029236)
\closepath
}
}
{
\newrgbcolor{curcolor}{0 1 0}
\pscustom[linestyle=none,fillstyle=solid,fillcolor=curcolor]
{
\newpath
\moveto(236.75827831,162.68541527)
\curveto(236.75827831,160.83929933)(235.26170564,159.34272667)(233.41558971,159.34272667)
\curveto(231.56947377,159.34272667)(230.0729011,160.83929933)(230.0729011,162.68541527)
\curveto(230.0729011,164.53153121)(231.56947377,166.02810388)(233.41558971,166.02810388)
\curveto(235.26170564,166.02810388)(236.75827831,164.53153121)(236.75827831,162.68541527)
\closepath
}
}
{
\newrgbcolor{curcolor}{0 1 0}
\pscustom[linewidth=1.38241851,linecolor=curcolor]
{
\newpath
\moveto(236.75827831,162.68541527)
\curveto(236.75827831,160.83929933)(235.26170564,159.34272667)(233.41558971,159.34272667)
\curveto(231.56947377,159.34272667)(230.0729011,160.83929933)(230.0729011,162.68541527)
\curveto(230.0729011,164.53153121)(231.56947377,166.02810388)(233.41558971,166.02810388)
\curveto(235.26170564,166.02810388)(236.75827831,164.53153121)(236.75827831,162.68541527)
\closepath
}
}
{
\newrgbcolor{curcolor}{0 1 0}
\pscustom[linestyle=none,fillstyle=solid,fillcolor=curcolor]
{
\newpath
\moveto(203.83579076,84.26428668)
\curveto(203.83579076,82.41817074)(202.3392181,80.92159808)(200.49310216,80.92159808)
\curveto(198.64698622,80.92159808)(197.15041356,82.41817074)(197.15041356,84.26428668)
\curveto(197.15041356,86.11040262)(198.64698622,87.60697528)(200.49310216,87.60697528)
\curveto(202.3392181,87.60697528)(203.83579076,86.11040262)(203.83579076,84.26428668)
\closepath
}
}
{
\newrgbcolor{curcolor}{0 1 0}
\pscustom[linewidth=1.38241851,linecolor=curcolor]
{
\newpath
\moveto(203.83579076,84.26428668)
\curveto(203.83579076,82.41817074)(202.3392181,80.92159808)(200.49310216,80.92159808)
\curveto(198.64698622,80.92159808)(197.15041356,82.41817074)(197.15041356,84.26428668)
\curveto(197.15041356,86.11040262)(198.64698622,87.60697528)(200.49310216,87.60697528)
\curveto(202.3392181,87.60697528)(203.83579076,86.11040262)(203.83579076,84.26428668)
\closepath
}
}
{
\newrgbcolor{curcolor}{0 1 0}
\pscustom[linestyle=none,fillstyle=solid,fillcolor=curcolor]
{
\newpath
\moveto(150.0668467,84.44411529)
\curveto(150.0668467,82.59799935)(148.57027404,81.10142669)(146.7241581,81.10142669)
\curveto(144.87804216,81.10142669)(143.3814695,82.59799935)(143.3814695,84.44411529)
\curveto(143.3814695,86.29023123)(144.87804216,87.7868039)(146.7241581,87.7868039)
\curveto(148.57027404,87.7868039)(150.0668467,86.29023123)(150.0668467,84.44411529)
\closepath
}
}
{
\newrgbcolor{curcolor}{0 1 0}
\pscustom[linewidth=1.38241851,linecolor=curcolor]
{
\newpath
\moveto(150.0668467,84.44411529)
\curveto(150.0668467,82.59799935)(148.57027404,81.10142669)(146.7241581,81.10142669)
\curveto(144.87804216,81.10142669)(143.3814695,82.59799935)(143.3814695,84.44411529)
\curveto(143.3814695,86.29023123)(144.87804216,87.7868039)(146.7241581,87.7868039)
\curveto(148.57027404,87.7868039)(150.0668467,86.29023123)(150.0668467,84.44411529)
\closepath
}
}
{
\newrgbcolor{curcolor}{0 0 0}
\pscustom[linestyle=none,fillstyle=solid,fillcolor=curcolor]
{
\newpath
\moveto(120.79286228,6.62318327)
\curveto(120.79286228,-2.20771645)(107.54405831,-2.20771645)(107.54405831,6.62318327)
\curveto(107.54405831,15.46144616)(120.79286228,15.46144616)(120.79286228,6.62318327)
}
}
{
\newrgbcolor{curcolor}{0 0 0}
\pscustom[linestyle=none,fillstyle=solid,fillcolor=curcolor]
{
\newpath
\moveto(228.33446929,6.62318327)
\curveto(228.33446929,-2.20771645)(215.08566532,-2.20771645)(215.08566532,6.62318327)
\curveto(215.08566532,15.46144616)(228.33446929,15.46144616)(228.33446929,6.62318327)
}
}
{
\newrgbcolor{curcolor}{0 0 0}
\pscustom[linestyle=none,fillstyle=solid,fillcolor=curcolor]
{
\newpath
\moveto(144.83853336,318.25478589)
\curveto(144.83853336,309.42388618)(131.57991182,309.42388618)(131.57991182,318.25478589)
\curveto(131.57991182,327.09304878)(144.83853336,327.09304878)(144.83853336,318.25478589)
}
}
{
\newrgbcolor{curcolor}{0 0 0}
\pscustom[linestyle=none,fillstyle=solid,fillcolor=curcolor]
{
\newpath
\moveto(252.38014037,318.25478589)
\curveto(252.38014037,309.42388618)(239.12151883,309.42388618)(239.12151883,318.25478589)
\curveto(239.12151883,327.09304878)(252.38014037,327.09304878)(252.38014037,318.25478589)
}
}
\end{pspicture}

\caption{Two possible Brillioun Zones for an oblique lattice. The Wigner-Seitz cell (\color{red}red\color{black}) and a simple trapezium (\color{blue}blue\color{black}) contain in their boundaries the points of high symmetry labelled $i_1,i_2,i_3$.}
\end{figure}

The arbitrary choice of BZ in the oblique case highlights that the properties of physical phenomena on the lattice are determined, not by the BZ, but by the symmetry operations of the lattice.

Neumann’s principle with respect to our system requires that the physical properties of phenomena associated with the grating possess the same symmetry as the point symmetry group of the grating [reference]. While we could discuss specific propagation properties of the surface modes with respect to its lattice, let us instead generalize these concepts to some arbitrary vector field and see what restrictions the symmetry of the grating places upon it.
In the case of a rectangular grating, the mirror and translational symmetry results in any point on the BZ boundary hosting vectors of equal magnitude but opposite direction (figure 4c). These vectors sum to give a magnitude of zero in the direction perpendicular to the zone boundary. Whether this vector field represents the SPP’s momentum, group velocity or Poyniting vector, the conclusion is the same: a standing wave forms perpendicular to the BZ boundary.
\begin{figure}
\centering %LaTeX with PSTricks extensions
%%Creator: inkscape 0.48.2
%%Please note this file requires PSTricks extensions
\psset{xunit=.5pt,yunit=.5pt,runit=.5pt}
\begin{pspicture}(526.44989014,265.96920776)
{
\newrgbcolor{curcolor}{0 0 0}
\pscustom[linewidth=0.51812367,linecolor=curcolor,linestyle=dashed,dash=24 1]
{
\newpath
\moveto(119.09817811,249.63932999)
\lineto(119.09817811,6.48505484)
}
}
{
\newrgbcolor{curcolor}{0 0 0}
\pscustom[linewidth=0.51812367,linecolor=curcolor,linestyle=dashed,dash=24 1]
{
\newpath
\moveto(28.3434024,128.06219371)
\lineto(209.28216755,128.06219371)
}
}
{
\newrgbcolor{curcolor}{0 0 0}
\pscustom[linewidth=1.55437101,linecolor=curcolor]
{
\newpath
\moveto(340.59332844,229.66186207)
\lineto(436.48515993,255.91795521)
\lineto(498.12990801,230.23264265)
\lineto(498.12990801,26.46249944)
\lineto(402.23807652,0.77718688)
\lineto(340.02254268,26.46249944)
\closepath
}
}
{
\newrgbcolor{curcolor}{0 0 0}
\pscustom[linewidth=1.55437101,linecolor=curcolor]
{
\newpath
\moveto(39.75909709,241.64834394)
\lineto(198.43725392,241.64834394)
\lineto(198.43725392,14.47603417)
\lineto(39.75909709,14.47603417)
\closepath
}
}
{
\newrgbcolor{curcolor}{0 0 0}
\pscustom[linewidth=1.03624734,linecolor=curcolor]
{
\newpath
\moveto(39.37149944,203.97654951)
\lineto(12.54461743,225.66636693)
}
}
{
\newrgbcolor{curcolor}{0 0 0}
\pscustom[linestyle=none,fillstyle=solid,fillcolor=curcolor]
{
\newpath
\moveto(18.26833397,215.12591881)
\lineto(11.46300138,226.51739225)
\lineto(24.02703741,222.24852529)
\curveto(20.779862,221.39686713)(18.46292611,218.51294858)(18.26833397,215.12591881)
\closepath
}
}
{
\newrgbcolor{curcolor}{0 0 0}
\pscustom[linewidth=1.03624734,linecolor=curcolor]
{
\newpath
\moveto(197.73415621,203.97654951)
\lineto(170.90727731,225.66636693)
}
}
{
\newrgbcolor{curcolor}{0 0 0}
\pscustom[linestyle=none,fillstyle=solid,fillcolor=curcolor]
{
\newpath
\moveto(176.63099325,215.12591849)
\lineto(169.8256613,226.51739231)
\lineto(182.3896971,222.24852464)
\curveto(179.14252163,221.39686666)(176.82558558,218.51294825)(176.63099325,215.12591849)
\closepath
}
}
{
\newrgbcolor{curcolor}{0 0 0}
\pscustom[linewidth=1.03624734,linecolor=curcolor]
{
\newpath
\moveto(40.20780234,204.54733527)
\lineto(67.03468434,226.23715269)
}
}
{
\newrgbcolor{curcolor}{0 0 0}
\pscustom[linestyle=none,fillstyle=solid,fillcolor=curcolor]
{
\newpath
\moveto(55.52931205,222.84770053)
\lineto(68.09334818,227.11656627)
\lineto(61.28801443,215.72509319)
\curveto(61.13525433,219.07861882)(58.80048835,221.94812155)(55.52931205,222.84770053)
\closepath
}
}
{
\newrgbcolor{curcolor}{0 0 0}
\pscustom[linewidth=1.03624734,linecolor=curcolor]
{
\newpath
\moveto(198.31516973,203.97654951)
\lineto(225.14204863,225.66636693)
}
}
{
\newrgbcolor{curcolor}{0 0 0}
\pscustom[linestyle=none,fillstyle=solid,fillcolor=curcolor]
{
\newpath
\moveto(213.63667652,222.27691412)
\lineto(226.20071242,226.54578057)
\lineto(219.39537931,215.1543071)
\curveto(219.24261902,218.50783273)(216.90785288,221.37733533)(213.63667652,222.27691412)
\closepath
}
}
{
\newrgbcolor{curcolor}{0 0 0}
\pscustom[linewidth=1.03624734,linecolor=curcolor]
{
\newpath
\moveto(340.12786531,203.97654951)
\lineto(313.3009833,225.66636693)
}
}
{
\newrgbcolor{curcolor}{0 0 0}
\pscustom[linestyle=none,fillstyle=solid,fillcolor=curcolor]
{
\newpath
\moveto(319.02469984,215.12591881)
\lineto(312.21936725,226.51739225)
\lineto(324.78340329,222.24852529)
\curveto(321.53622787,221.39686713)(319.21929198,218.51294858)(319.02469984,215.12591881)
\closepath
}
}
{
\newrgbcolor{curcolor}{0 0 0}
\pscustom[linewidth=1.03624734,linecolor=curcolor]
{
\newpath
\moveto(497.97061088,204.44201109)
\lineto(471.14373198,226.13182851)
}
}
{
\newrgbcolor{curcolor}{0 0 0}
\pscustom[linestyle=none,fillstyle=solid,fillcolor=curcolor]
{
\newpath
\moveto(476.86744793,215.59138007)
\lineto(470.06211598,226.98285389)
\lineto(482.62615177,222.71398622)
\curveto(479.37897631,221.86232824)(477.06204026,218.97840983)(476.86744793,215.59138007)
\closepath
}
}
{
\newrgbcolor{curcolor}{0 0 0}
\pscustom[linewidth=1.03624734,linecolor=curcolor]
{
\newpath
\moveto(498.54140701,54.597423)
\lineto(525.3682859,32.90760557)
}
}
{
\newrgbcolor{curcolor}{0 0 0}
\pscustom[linestyle=none,fillstyle=solid,fillcolor=curcolor]
{
\newpath
\moveto(519.64456996,43.44805402)
\lineto(526.44990191,32.0565802)
\lineto(513.88586611,36.32544786)
\curveto(517.13304158,37.17710584)(519.44997763,40.06102426)(519.64456996,43.44805402)
\closepath
}
}
{
\newrgbcolor{curcolor}{0 0 0}
\pscustom[linewidth=1.03624734,linecolor=curcolor]
{
\newpath
\moveto(340.56638083,54.32563086)
\lineto(367.39325973,32.63581344)
}
}
{
\newrgbcolor{curcolor}{0 0 0}
\pscustom[linestyle=none,fillstyle=solid,fillcolor=curcolor]
{
\newpath
\moveto(361.66954379,43.17626189)
\lineto(368.47487574,31.78478806)
\lineto(355.91083994,36.05365573)
\curveto(359.15801541,36.90531371)(361.47495146,39.78923213)(361.66954379,43.17626189)
\closepath
}
}
{
\newrgbcolor{curcolor}{0 0 0}
\pscustom[linewidth=1.03624734,linecolor=curcolor]
{
\newpath
\moveto(198.88595549,56.44230077)
\lineto(225.71283439,34.75248334)
}
}
{
\newrgbcolor{curcolor}{0 0 0}
\pscustom[linestyle=none,fillstyle=solid,fillcolor=curcolor]
{
\newpath
\moveto(219.98911844,45.29293179)
\lineto(226.79445039,33.90145797)
\lineto(214.2304146,38.17032563)
\curveto(217.47759006,39.02198361)(219.79452612,41.90590203)(219.98911844,45.29293179)
\closepath
}
}
{
\newrgbcolor{curcolor}{0 0 0}
\pscustom[linewidth=1.03624734,linecolor=curcolor]
{
\newpath
\moveto(40.34011247,56.44230077)
\lineto(67.16699137,34.75248334)
}
}
{
\newrgbcolor{curcolor}{0 0 0}
\pscustom[linestyle=none,fillstyle=solid,fillcolor=curcolor]
{
\newpath
\moveto(61.44327542,45.29293179)
\lineto(68.24860737,33.90145797)
\lineto(55.68457158,38.17032563)
\curveto(58.93174704,39.02198361)(61.2486831,41.90590203)(61.44327542,45.29293179)
\closepath
}
}
{
\newrgbcolor{curcolor}{0 0 0}
\pscustom[linewidth=1.03624734,linecolor=curcolor]
{
\newpath
\moveto(197.86646427,55.87151501)
\lineto(171.03958537,34.18169758)
}
}
{
\newrgbcolor{curcolor}{0 0 0}
\pscustom[linestyle=none,fillstyle=solid,fillcolor=curcolor]
{
\newpath
\moveto(182.54495747,37.5711504)
\lineto(169.98092158,33.30228394)
\lineto(176.78625469,44.69375741)
\curveto(176.93901498,41.34023179)(179.27378112,38.47072919)(182.54495747,37.5711504)
\closepath
}
}
{
\newrgbcolor{curcolor}{0 0 0}
\pscustom[linewidth=1.03624734,linecolor=curcolor]
{
\newpath
\moveto(39.75909895,56.44230077)
\lineto(12.93222005,34.75248334)
}
}
{
\newrgbcolor{curcolor}{0 0 0}
\pscustom[linestyle=none,fillstyle=solid,fillcolor=curcolor]
{
\newpath
\moveto(24.43759215,38.14193616)
\lineto(11.87355626,33.8730697)
\lineto(18.67888937,45.26454317)
\curveto(18.83164966,41.91101755)(21.1664158,39.04151495)(24.43759215,38.14193616)
\closepath
}
}
{
\newrgbcolor{curcolor}{1 1 1}
\pscustom[linestyle=none,fillstyle=solid,fillcolor=curcolor,opacity=0]
{
\newpath
\moveto(128.54705593,123.15011788)
\curveto(125.7896308,118.01597954)(119.32388975,116.05314901)(114.1054191,118.76601319)
\curveto(108.88694844,121.47887737)(106.89187686,127.84012979)(109.64930199,132.97426813)
\curveto(112.40672713,138.10840647)(118.87246818,140.071237)(124.09093883,137.35837282)
\curveto(126.21934314,136.25190463)(127.90570429,134.47086069)(128.87683309,132.30376873)
}
}
{
\newrgbcolor{curcolor}{0 0 0}
\pscustom[linewidth=1.03624736,linecolor=curcolor]
{
\newpath
\moveto(128.54705593,123.15011788)
\curveto(125.7896308,118.01597954)(119.32388975,116.05314901)(114.1054191,118.76601319)
\curveto(108.88694844,121.47887737)(106.89187686,127.84012979)(109.64930199,132.97426813)
\curveto(112.40672713,138.10840647)(118.87246818,140.071237)(124.09093883,137.35837282)
\curveto(126.21934314,136.25190463)(127.90570429,134.47086069)(128.87683309,132.30376873)
}
}
{
\newrgbcolor{curcolor}{0 0 0}
\pscustom[linewidth=0.51812367,linecolor=curcolor]
{
\newpath
\moveto(125.23759107,125.11337585)
\closepath
}
}
{
\newrgbcolor{curcolor}{0 0 0}
\pscustom[linewidth=1.03624734,linecolor=curcolor]
{
\newpath
\moveto(128.51688386,123.09534634)
\lineto(124.22857631,123.09534634)
\lineto(124.22857631,123.09534634)
\lineto(124.22857631,123.09534634)
}
}
{
\newrgbcolor{curcolor}{0 0 0}
\pscustom[linewidth=1.03624734,linecolor=curcolor]
{
\newpath
\moveto(128.64301226,118.7818129)
\lineto(128.64301226,123.07012045)
\lineto(128.64301226,123.07012045)
\lineto(128.64301226,123.07012045)
}
}
{
\newrgbcolor{curcolor}{1 1 1}
\pscustom[linestyle=none,fillstyle=solid,fillcolor=curcolor,opacity=0]
{
\newpath
\moveto(428.52510161,123.43549335)
\curveto(425.76767648,118.301355)(419.30193543,116.33852447)(414.08346477,119.05138865)
\curveto(408.86499412,121.76425283)(406.86992253,128.12550525)(409.62734767,133.25964359)
\curveto(412.3847728,138.39378193)(418.85051385,140.35661246)(424.06898451,137.64374828)
\curveto(426.19738881,136.53728009)(427.88374996,134.75623615)(428.85487877,132.58914419)
}
}
{
\newrgbcolor{curcolor}{0 0 0}
\pscustom[linewidth=1.03624736,linecolor=curcolor]
{
\newpath
\moveto(428.52510161,123.43549335)
\curveto(425.76767648,118.301355)(419.30193543,116.33852447)(414.08346477,119.05138865)
\curveto(408.86499412,121.76425283)(406.86992253,128.12550525)(409.62734767,133.25964359)
\curveto(412.3847728,138.39378193)(418.85051385,140.35661246)(424.06898451,137.64374828)
\curveto(426.19738881,136.53728009)(427.88374996,134.75623615)(428.85487877,132.58914419)
}
}
{
\newrgbcolor{curcolor}{0 0 0}
\pscustom[linewidth=0.51812367,linecolor=curcolor]
{
\newpath
\moveto(425.21563675,125.39875132)
\closepath
}
}
{
\newrgbcolor{curcolor}{0 0 0}
\pscustom[linewidth=1.03624734,linecolor=curcolor]
{
\newpath
\moveto(428.49492954,123.3807218)
\lineto(424.20662199,123.3807218)
\lineto(424.20662199,123.3807218)
\lineto(424.20662199,123.3807218)
}
}
{
\newrgbcolor{curcolor}{0 0 0}
\pscustom[linewidth=1.03624734,linecolor=curcolor]
{
\newpath
\moveto(428.62105794,119.06718836)
\lineto(428.62105794,123.35549591)
\lineto(428.62105794,123.35549591)
\lineto(428.62105794,123.35549591)
}
}
{
\newrgbcolor{curcolor}{0 0 0}
\pscustom[linestyle=none,fillstyle=solid,fillcolor=curcolor]
{
\newpath
\moveto(445.00448535,120.11334373)
\lineto(444.58958163,117.9578683)
\curveto(443.9419137,118.61225685)(443.25378139,119.0979973)(442.52518263,119.41509113)
\curveto(441.80330642,119.73215845)(441.02072458,119.89069874)(440.17743475,119.89071246)
\curveto(439.02379159,119.89069874)(438.00845912,119.61409654)(437.13143427,119.06090502)
\curveto(436.26114276,118.50768772)(435.50891969,117.66438832)(434.87476281,116.53100429)
\curveto(434.46997483,115.80238325)(434.15964065,115.02992099)(433.94375934,114.21361522)
\curveto(433.73461775,113.40403975)(433.63004863,112.58772593)(433.63005165,111.76467131)
\curveto(433.63004863,110.38840109)(433.98423438,109.33933664)(434.69260996,108.6174748)
\curveto(435.40772376,107.89560806)(436.44329543,107.53467592)(437.79932806,107.53467729)
\curveto(438.7370698,107.53467592)(439.63771356,107.68309661)(440.50126204,107.97993981)
\curveto(441.36479073,108.28352579)(442.20471693,108.73553427)(443.02104317,109.33596661)
\lineto(442.54542183,106.91738151)
\curveto(441.74258886,106.5733146)(440.92964824,106.31357839)(440.10659753,106.13817208)
\curveto(439.29027421,105.96276584)(438.47058719,105.8750627)(437.64753402,105.87506241)
\curveto(435.71131155,105.8750627)(434.19674582,106.40128152)(433.1038323,107.45372046)
\curveto(432.01766017,108.50615683)(431.47457536,109.97012459)(431.47457623,111.84562813)
\curveto(431.47457536,113.0464808)(431.68034041,114.22709996)(432.09187201,115.38748916)
\curveto(432.51014702,116.54785992)(433.107203,117.60704396)(433.88304173,118.56504448)
\curveto(434.70609866,119.58373776)(435.6438476,120.33596082)(436.69629134,120.82171593)
\curveto(437.7554693,121.30744173)(438.98668642,121.55031196)(440.38994641,121.55032734)
\curveto(441.25347521,121.55031196)(442.06978903,121.42887685)(442.83889032,121.18602164)
\curveto(443.61471353,120.94988279)(444.33657782,120.59232384)(445.00448535,120.11334373)
}
}
{
\newrgbcolor{curcolor}{0 0 0}
\pscustom[linestyle=none,fillstyle=solid,fillcolor=curcolor]
{
\newpath
\moveto(447.68065075,103.1417734)
\lineto(452.31802411,103.1417734)
\lineto(452.31802411,102.02354153)
\lineto(446.08223695,102.02354153)
\lineto(446.08223695,103.1417734)
\curveto(446.58653611,103.66361331)(447.27282283,104.36305568)(448.14109917,105.24010263)
\curveto(449.01375464,106.12152836)(449.56190697,106.68941417)(449.78555781,106.94376178)
\curveto(450.21091933,107.42174569)(450.50692159,107.8251858)(450.67356547,108.15408333)
\curveto(450.84458342,108.48735382)(450.93009519,108.81405261)(450.93010102,109.13418068)
\curveto(450.93009519,109.65601459)(450.745916,110.0813808)(450.37756292,110.41028058)
\curveto(450.01358449,110.7391636)(449.53778826,110.9036093)(448.95017282,110.90361818)
\curveto(448.53357319,110.9036093)(448.09285872,110.83125319)(447.62802807,110.68654964)
\curveto(447.16757758,110.54182876)(446.67424049,110.32256782)(446.1480153,110.02876618)
\lineto(446.1480153,111.37064443)
\curveto(446.68301092,111.5855108)(447.18292585,111.74776389)(447.64776158,111.85740419)
\curveto(448.1125922,111.96702482)(448.53795841,112.02184005)(448.92386148,112.02185005)
\curveto(449.94122838,112.02184005)(450.75249383,111.76749737)(451.35766027,111.25882124)
\curveto(451.96281418,110.75012664)(452.26539426,110.07041775)(452.26540143,109.21969253)
\curveto(452.26539426,108.81624522)(452.18865294,108.43253859)(452.03517722,108.06857148)
\curveto(451.88607285,107.70897751)(451.61199669,107.2836113)(451.21294791,106.79247158)
\curveto(451.10331132,106.66529547)(450.75468644,106.2969371)(450.16707221,105.68739538)
\curveto(449.57944784,105.08223154)(448.75064152,104.23369173)(447.68065075,103.1417734)
}
}
{
\newrgbcolor{curcolor}{0 0 0}
\pscustom[linestyle=none,fillstyle=solid,fillcolor=curcolor]
{
\newpath
\moveto(145.01088042,121.66771474)
\lineto(144.5959767,119.51223931)
\curveto(143.94830877,120.16662786)(143.26017646,120.65236831)(142.5315777,120.96946214)
\curveto(141.80970149,121.28652946)(141.02711965,121.44506975)(140.18382982,121.44508347)
\curveto(139.03018666,121.44506975)(138.01485419,121.16846755)(137.13782934,120.61527603)
\curveto(136.26753783,120.06205873)(135.51531476,119.21875933)(134.88115788,118.0853753)
\curveto(134.4763699,117.35675426)(134.16603572,116.584292)(133.95015441,115.76798623)
\curveto(133.74101282,114.95841076)(133.6364437,114.14209694)(133.63644672,113.31904232)
\curveto(133.6364437,111.9427721)(133.99062945,110.89370765)(134.69900503,110.17184581)
\curveto(135.41411883,109.44997907)(136.4496905,109.08904693)(137.80572313,109.0890483)
\curveto(138.74346487,109.08904693)(139.64410863,109.23746762)(140.50765711,109.53431082)
\curveto(141.3711858,109.8378968)(142.211112,110.28990528)(143.02743824,110.89033762)
\lineto(142.5518169,108.47175252)
\curveto(141.74898393,108.12768561)(140.93604331,107.8679494)(140.1129926,107.69254309)
\curveto(139.29666928,107.51713685)(138.47698226,107.42943371)(137.65392909,107.42943342)
\curveto(135.71770662,107.42943371)(134.20314089,107.95565253)(133.11022737,109.00809147)
\curveto(132.02405524,110.06052784)(131.48097043,111.5244956)(131.4809713,113.39999914)
\curveto(131.48097043,114.60085181)(131.68673548,115.78147097)(132.09826708,116.94186017)
\curveto(132.51654209,118.10223093)(133.11359807,119.16141497)(133.8894368,120.11941549)
\curveto(134.71249373,121.13810877)(135.65024267,121.89033183)(136.70268641,122.37608694)
\curveto(137.76186437,122.86181274)(138.99308149,123.10468297)(140.39634148,123.10469835)
\curveto(141.25987028,123.10468297)(142.0761841,122.98324786)(142.84528539,122.74039265)
\curveto(143.6211086,122.5042538)(144.34297289,122.14669485)(145.01088042,121.66771474)
}
}
{
\newrgbcolor{curcolor}{0 0 0}
\pscustom[linestyle=none,fillstyle=solid,fillcolor=curcolor]
{
\newpath
\moveto(147.68704582,104.69614441)
\lineto(152.32441918,104.69614441)
\lineto(152.32441918,103.57791254)
\lineto(146.08863202,103.57791254)
\lineto(146.08863202,104.69614441)
\curveto(146.59293118,105.21798432)(147.2792179,105.91742669)(148.14749424,106.79447364)
\curveto(149.02014971,107.67589937)(149.56830204,108.24378518)(149.79195288,108.49813279)
\curveto(150.2173144,108.9761167)(150.51331666,109.37955681)(150.67996054,109.70845434)
\curveto(150.85097849,110.04172483)(150.93649026,110.36842362)(150.93649609,110.68855169)
\curveto(150.93649026,111.2103856)(150.75231107,111.63575181)(150.38395799,111.96465159)
\curveto(150.01997956,112.29353461)(149.54418333,112.45798031)(148.95656789,112.45798919)
\curveto(148.53996826,112.45798031)(148.09925379,112.3856242)(147.63442314,112.24092065)
\curveto(147.17397265,112.09619977)(146.68063556,111.87693883)(146.15441037,111.58313719)
\lineto(146.15441037,112.92501544)
\curveto(146.68940599,113.13988181)(147.18932092,113.3021349)(147.65415665,113.4117752)
\curveto(148.11898727,113.52139583)(148.54435348,113.57621106)(148.93025655,113.57622106)
\curveto(149.94762345,113.57621106)(150.7588889,113.32186838)(151.36405534,112.81319225)
\curveto(151.96920925,112.30449765)(152.27178933,111.62478876)(152.2717965,110.77406354)
\curveto(152.27178933,110.37061623)(152.19504801,109.9869096)(152.04157229,109.62294249)
\curveto(151.89246792,109.26334852)(151.61839176,108.83798231)(151.21934298,108.34684259)
\curveto(151.10970639,108.21966648)(150.76108151,107.85130811)(150.17346728,107.24176639)
\curveto(149.58584291,106.63660255)(148.75703659,105.78806274)(147.68704582,104.69614441)
}
}
{
\newrgbcolor{curcolor}{0 0 0}
\pscustom[linestyle=none,fillstyle=solid,fillcolor=curcolor]
{
\newpath
\moveto(155.25813314,105.24868252)
\lineto(156.64605623,105.24868252)
\lineto(156.64605623,104.11729497)
\lineto(155.56729137,102.01238792)
\lineto(154.71875071,102.01238792)
\lineto(155.25813314,104.11729497)
\lineto(155.25813314,105.24868252)
}
}
{
\newrgbcolor{curcolor}{0 0 0}
\pscustom[linestyle=none,fillstyle=solid,fillcolor=curcolor]
{
\newpath
\moveto(163.05944579,112.24092065)
\lineto(159.70475017,106.99838651)
\lineto(163.05944579,106.99838651)
\lineto(163.05944579,112.24092065)
\moveto(162.71082056,113.39861953)
\lineto(164.38159053,113.39861953)
\lineto(164.38159053,106.99838651)
\lineto(165.78266929,106.99838651)
\lineto(165.78266929,105.8933103)
\lineto(164.38159053,105.8933103)
\lineto(164.38159053,103.57791254)
\lineto(163.05944579,103.57791254)
\lineto(163.05944579,105.8933103)
\lineto(158.6259853,105.8933103)
\lineto(158.6259853,107.17598804)
\lineto(162.71082056,113.39861953)
}
}
{
\newrgbcolor{curcolor}{0 0 0}
\pscustom[linestyle=none,fillstyle=solid,fillcolor=curcolor]
{
\newpath
\moveto(121.56375252,261.65825383)
\curveto(121.56374084,260.64628724)(121.40857375,259.67480633)(121.09825079,258.74380818)
\curveto(120.78790539,257.81280125)(120.34938971,257.00323383)(119.78270241,256.31510348)
\curveto(119.19575613,255.59998362)(118.51774341,255.04003282)(117.74866222,254.63524939)
\curveto(116.9795653,254.23721179)(116.08229474,254.03819313)(115.05684784,254.03819282)
\curveto(113.71431002,254.03819313)(112.66524557,254.41936446)(111.90965133,255.18170795)
\curveto(111.16079943,255.95079617)(110.7863745,257.00998022)(110.7863754,258.35926327)
\curveto(110.7863745,259.37121854)(110.93816839,260.33595306)(111.24175753,261.25346971)
\curveto(111.55209036,262.17097255)(111.99735244,262.98728637)(112.57754512,263.70241362)
\curveto(113.13749323,264.39053658)(113.82225234,264.94036779)(114.63182451,265.35190889)
\curveto(115.44813359,265.763428)(116.33865776,265.96919306)(117.30339969,265.96920467)
\curveto(117.64745843,265.96919306)(117.97465859,265.94220748)(118.28500117,265.88824785)
\curveto(118.59532695,265.83426515)(118.88204875,265.75668161)(119.14516742,265.65549698)
\lineto(124.29604531,265.65549698)
\lineto(123.94185921,264.13755654)
\lineto(120.71370588,264.13755654)
\curveto(120.96331167,263.84745176)(121.16570353,263.48314642)(121.32088205,263.04463943)
\curveto(121.4827841,262.60611505)(121.56374084,262.14398697)(121.56375252,261.65825383)
\moveto(118.46715403,257.38778139)
\curveto(118.83145078,257.92748997)(119.10805299,258.55153153)(119.29696147,259.25990793)
\curveto(119.49259752,259.96827452)(119.59042025,260.71712439)(119.59042995,261.50645978)
\curveto(119.59042025,262.45094796)(119.36441601,263.16606585)(118.91241655,263.6518156)
\curveto(118.46039905,264.13754676)(117.8228647,264.38041699)(116.9998116,264.38042701)
\curveto(116.34540415,264.38041699)(115.75846777,264.2218767)(115.23900069,263.90480568)
\curveto(114.71952291,263.59446194)(114.27088763,263.16606585)(113.8930935,262.6196161)
\curveto(113.52878415,262.08664261)(113.24880875,261.46260106)(113.05316646,260.74748956)
\curveto(112.85751783,260.03236527)(112.7596951,259.2902618)(112.75969797,258.52117692)
\curveto(112.7596951,257.59017021)(112.98569934,256.87505232)(113.43771137,256.3758211)
\curveto(113.8897163,255.87658583)(114.53062384,255.6269692)(115.36043593,255.62697048)
\curveto(116.00808439,255.6269692)(116.59502077,255.78213629)(117.12124684,256.09247221)
\curveto(117.64745843,256.40280465)(118.09609371,256.83457395)(118.46715403,257.38778139)
}
}
{
\newrgbcolor{curcolor}{0 0 0}
\pscustom[linestyle=none,fillstyle=solid,fillcolor=curcolor]
{
\newpath
\moveto(123.36150563,257.57408585)
\lineto(124.64418337,257.57408585)
\lineto(126.94642546,251.39092137)
\lineto(129.24866756,257.57408585)
\lineto(130.5313453,257.57408585)
\lineto(127.76865478,250.20691115)
\lineto(126.12419615,250.20691115)
\lineto(123.36150563,257.57408585)
}
}
{
\newrgbcolor{curcolor}{0 0 0}
\pscustom[linewidth=1.03624734,linecolor=curcolor]
{
\newpath
\moveto(340.12786168,203.97654951)
\lineto(313.30098278,225.66636693)
}
}
{
\newrgbcolor{curcolor}{0 0 0}
\pscustom[linestyle=none,fillstyle=solid,fillcolor=curcolor]
{
\newpath
\moveto(319.02469873,215.12591849)
\lineto(312.21936678,226.51739231)
\lineto(324.78340257,222.24852464)
\curveto(321.53622711,221.39686666)(319.21929106,218.51294825)(319.02469873,215.12591849)
\closepath
}
}
{
\newrgbcolor{curcolor}{0 0 0}
\pscustom[linewidth=0.51812367,linecolor=curcolor]
{
\newpath
\moveto(39.5008744,204.02057966)
\lineto(11.93162511,204.02057966)
}
}
{
\newrgbcolor{curcolor}{0 0 0}
\pscustom[linestyle=none,fillstyle=solid,fillcolor=curcolor]
{
\newpath
\moveto(17.11286181,204.02057966)
\lineto(19.18535649,206.09307434)
\lineto(11.93162511,204.02057966)
\lineto(19.18535649,201.94808498)
\lineto(17.11286181,204.02057966)
\closepath
}
}
{
\newrgbcolor{curcolor}{0 0 0}
\pscustom[linewidth=0.51812367,linecolor=curcolor]
{
\newpath
\moveto(17.11286181,204.02057966)
\lineto(19.18535649,206.09307434)
\lineto(11.93162511,204.02057966)
\lineto(19.18535649,201.94808498)
\lineto(17.11286181,204.02057966)
\closepath
}
}
{
\newrgbcolor{curcolor}{0 0 0}
\pscustom[linewidth=0.51812367,linecolor=curcolor]
{
\newpath
\moveto(39.6840591,203.90926829)
\lineto(67.25330839,203.90926829)
}
}
{
\newrgbcolor{curcolor}{0 0 0}
\pscustom[linestyle=none,fillstyle=solid,fillcolor=curcolor]
{
\newpath
\moveto(62.07207169,203.90926829)
\lineto(59.99957701,205.98176297)
\lineto(67.25330839,203.90926829)
\lineto(59.99957701,201.83677361)
\lineto(62.07207169,203.90926829)
\closepath
}
}
{
\newrgbcolor{curcolor}{0 0 0}
\pscustom[linewidth=0.51812367,linecolor=curcolor]
{
\newpath
\moveto(62.07207169,203.90926829)
\lineto(59.99957701,205.98176297)
\lineto(67.25330839,203.90926829)
\lineto(59.99957701,201.83677361)
\lineto(62.07207169,203.90926829)
\closepath
}
}
{
\newrgbcolor{curcolor}{0 0 0}
\pscustom[linestyle=none,fillstyle=solid,fillcolor=curcolor]
{
\newpath
\moveto(8.41951002,234.09816353)
\curveto(8.10241929,234.24657598)(7.78533871,234.3545183)(7.46826735,234.42199083)
\curveto(7.15792396,234.4961926)(6.84421658,234.53329778)(6.52714427,234.53330646)
\curveto(5.59613347,234.53329778)(4.87764238,234.23308319)(4.37166885,233.6326618)
\curveto(3.87242949,233.03897124)(3.62281287,232.18555224)(3.62281823,231.07240226)
\lineto(3.62281823,225.85068715)
\lineto(0.00000038,225.85068715)
\lineto(0.00000038,237.18464243)
\lineto(3.62281823,237.18464243)
\lineto(3.62281823,235.32263549)
\curveto(4.08831414,236.06472949)(4.62127936,236.60444111)(5.2217155,236.94177196)
\curveto(5.8288841,237.28582702)(6.55412159,237.4578601)(7.39743013,237.45787171)
\curveto(7.5188561,237.4578601)(7.65041081,237.4511137)(7.79209464,237.4376325)
\curveto(7.93375941,237.43087452)(8.13952446,237.41063533)(8.40939042,237.37691488)
\lineto(8.41951002,234.09816353)
}
}
{
\newrgbcolor{curcolor}{0 0 0}
\pscustom[linestyle=none,fillstyle=solid,fillcolor=curcolor]
{
\newpath
\moveto(309.96748596,234.09816353)
\curveto(309.65039523,234.24657598)(309.33331465,234.3545183)(309.01624329,234.42199083)
\curveto(308.7058999,234.4961926)(308.39219252,234.53329778)(308.07512021,234.53330646)
\curveto(307.14410941,234.53329778)(306.42561832,234.23308319)(305.91964479,233.6326618)
\curveto(305.42040543,233.03897124)(305.17078881,232.18555224)(305.17079417,231.07240226)
\lineto(305.17079417,225.85068715)
\lineto(301.54797632,225.85068715)
\lineto(301.54797632,237.18464243)
\lineto(305.17079417,237.18464243)
\lineto(305.17079417,235.32263549)
\curveto(305.63629008,236.06472949)(306.1692553,236.60444111)(306.76969144,236.94177196)
\curveto(307.37686004,237.28582702)(308.10209753,237.4578601)(308.94540607,237.45787171)
\curveto(309.06683204,237.4578601)(309.19838675,237.4511137)(309.34007058,237.4376325)
\curveto(309.48173535,237.43087452)(309.6875004,237.41063533)(309.95736636,237.37691488)
\lineto(309.96748596,234.09816353)
}
}
{
\newrgbcolor{curcolor}{0 0 0}
\pscustom[linestyle=none,fillstyle=solid,fillcolor=curcolor]
{
\newpath
\moveto(120.39483454,128.06219195)
\curveto(120.39483454,127.3460686)(119.81430186,126.76553592)(119.09817851,126.76553592)
\curveto(118.38205516,126.76553592)(117.80152248,127.3460686)(117.80152248,128.06219195)
\curveto(117.80152248,128.7783153)(118.38205516,129.35884798)(119.09817851,129.35884798)
\curveto(119.81430186,129.35884798)(120.39483454,128.7783153)(120.39483454,128.06219195)
\closepath
}
}
{
\newrgbcolor{curcolor}{0 0 0}
\pscustom[linewidth=0.35192806,linecolor=curcolor]
{
\newpath
\moveto(120.39483454,128.06219195)
\curveto(120.39483454,127.3460686)(119.81430186,126.76553592)(119.09817851,126.76553592)
\curveto(118.38205516,126.76553592)(117.80152248,127.3460686)(117.80152248,128.06219195)
\curveto(117.80152248,128.7783153)(118.38205516,129.35884798)(119.09817851,129.35884798)
\curveto(119.81430186,129.35884798)(120.39483454,128.7783153)(120.39483454,128.06219195)
\closepath
}
}
{
\newrgbcolor{curcolor}{0 0 0}
\pscustom[linestyle=none,fillstyle=solid,fillcolor=curcolor]
{
\newpath
\moveto(420.37288022,128.34757105)
\curveto(420.37288022,127.63144769)(419.79234754,127.05091501)(419.07622419,127.05091501)
\curveto(418.36010083,127.05091501)(417.77956815,127.63144769)(417.77956815,128.34757105)
\curveto(417.77956815,129.0636944)(418.36010083,129.64422708)(419.07622419,129.64422708)
\curveto(419.79234754,129.64422708)(420.37288022,129.0636944)(420.37288022,128.34757105)
\closepath
}
}
{
\newrgbcolor{curcolor}{0 0 0}
\pscustom[linewidth=0.35192806,linecolor=curcolor]
{
\newpath
\moveto(420.37288022,128.34757105)
\curveto(420.37288022,127.63144769)(419.79234754,127.05091501)(419.07622419,127.05091501)
\curveto(418.36010083,127.05091501)(417.77956815,127.63144769)(417.77956815,128.34757105)
\curveto(417.77956815,129.0636944)(418.36010083,129.64422708)(419.07622419,129.64422708)
\curveto(419.79234754,129.64422708)(420.37288022,129.0636944)(420.37288022,128.34757105)
\closepath
}
}
{
\newrgbcolor{curcolor}{0 0 0}
\pscustom[linestyle=none,fillstyle=solid,fillcolor=curcolor]
{
\newpath
\moveto(225.25042956,130.05484165)
\curveto(225.25041788,129.04287506)(225.09525079,128.07139415)(224.78492783,127.140396)
\curveto(224.47458243,126.20938907)(224.03606674,125.39982165)(223.46937944,124.7116913)
\curveto(222.88243316,123.99657144)(222.20442045,123.43662064)(221.43533926,123.03183721)
\curveto(220.66624234,122.63379961)(219.76897178,122.43478095)(218.74352488,122.43478064)
\curveto(217.40098706,122.43478095)(216.3519226,122.81595228)(215.59632836,123.57829577)
\curveto(214.84747647,124.34738399)(214.47305154,125.40656804)(214.47305244,126.75585109)
\curveto(214.47305154,127.76780636)(214.62484543,128.73254088)(214.92843457,129.65005753)
\curveto(215.2387674,130.56756037)(215.68402948,131.38387419)(216.26422216,132.09900144)
\curveto(216.82417027,132.7871244)(217.50892938,133.33695561)(218.31850155,133.74849671)
\curveto(219.13481063,134.16001582)(220.02533479,134.36578088)(220.99007673,134.36579249)
\curveto(221.33413546,134.36578088)(221.66133563,134.3387953)(221.97167821,134.28483567)
\curveto(222.28200399,134.23085297)(222.56872579,134.15326943)(222.83184446,134.0520848)
\lineto(227.98272235,134.0520848)
\lineto(227.62853625,132.53414436)
\lineto(224.40038291,132.53414436)
\curveto(224.64998871,132.24403958)(224.85238056,131.87973424)(225.00755909,131.44122725)
\curveto(225.16946114,131.00270287)(225.25041788,130.54057479)(225.25042956,130.05484165)
\moveto(222.15383106,125.78436921)
\curveto(222.51812782,126.32407779)(222.79473003,126.94811935)(222.9836385,127.65649575)
\curveto(223.17927455,128.36486234)(223.27709728,129.11371221)(223.27710699,129.9030476)
\curveto(223.27709728,130.84753578)(223.05109304,131.56265367)(222.59909359,132.04840342)
\curveto(222.14707609,132.53413458)(221.50954174,132.77700481)(220.68648864,132.77701483)
\curveto(220.03208119,132.77700481)(219.44514481,132.61846452)(218.92567773,132.3013935)
\curveto(218.40619994,131.99104976)(217.95756466,131.56265367)(217.57977054,131.01620392)
\curveto(217.21546119,130.48323043)(216.93548579,129.85918888)(216.7398435,129.14407738)
\curveto(216.54419487,128.42895309)(216.44637214,127.68684962)(216.44637501,126.91776474)
\curveto(216.44637214,125.98675803)(216.67237638,125.27164014)(217.12438841,124.77240892)
\curveto(217.57639333,124.27317365)(218.21730088,124.02355702)(219.04711296,124.0235583)
\curveto(219.69476143,124.02355702)(220.28169781,124.17872411)(220.80792387,124.48906003)
\curveto(221.33413546,124.79939247)(221.78277075,125.23116177)(222.15383106,125.78436921)
}
}
{
\newrgbcolor{curcolor}{0 0 0}
\pscustom[linestyle=none,fillstyle=solid,fillcolor=curcolor]
{
\newpath
\moveto(234.0404208,123.05011513)
\lineto(234.0404208,118.60349897)
\lineto(232.83009924,118.60349897)
\lineto(232.83009924,123.01064812)
\curveto(232.83009306,123.70789348)(232.69415128,124.2297345)(232.4222735,124.57617274)
\curveto(232.15038417,124.92259904)(231.74255883,125.09581518)(231.19879627,125.09582167)
\curveto(230.54539414,125.09581518)(230.03013095,124.88751729)(229.65300515,124.47092739)
\curveto(229.27587334,124.05432575)(229.08730894,123.48643994)(229.08731138,122.76726824)
\lineto(229.08731138,118.60349897)
\lineto(227.87041199,118.60349897)
\lineto(227.87041199,128.83860953)
\lineto(229.08731138,128.83860953)
\lineto(229.08731138,124.82613046)
\curveto(229.37673337,125.26903132)(229.71658782,125.60011533)(230.10687574,125.81938347)
\curveto(230.50154196,126.03863719)(230.95541209,126.14826766)(231.46848749,126.1482752)
\curveto(232.31482987,126.14826766)(232.95507179,125.88515454)(233.38921518,125.35893506)
\curveto(233.82334508,124.83708728)(234.04041341,124.06748141)(234.0404208,123.05011513)
}
}
\end{pspicture}

\caption{Applying the symmetry operations of the rectangular BZ, an arbitrary vector, r (lying on the BZ boundary) corresponds to seven other vectors of known magnitude and direction. The summation of these vectors leads to no perpendicular component of the vector at the BZ boundary. (d) In the oblique case, there is no such condition, the C2 rotation operation only placing constraints on three additional vectors.}
\end{figure}
In the plasmonic case, a SPP has Bragg-scattered and interfered with a counter-propagating SPP. The interference between these two SPPs, which by symmetry can be equal in magnitude and opposite in direction, forms a standing wave. The symmetry considerations we have outlined show us that this is possible, with no restrictions on a ‘mirrored’ SPP occupying the same position and, importantly, that their vector properties will have equal magnitudes and so interfere completely. The two possible field arrangements for SPP standing waves on gratings lead to two solutions of different energies with a forbidden band of no propagating waves between the two: a photonic band gap [reference]. These are observed experimentally as discontinuities of the SPP curves at the BZ boundary, and are well understood [reference]. 

We observe such a band gap using in the imaging scatterometry data for a rectangular bi-grating (figure 4a). The SPP equi-energy contour in figure 1 is for a wavelength of 550 nm, and intersects the BZ boundary perpendicularly, indicating zero component of group velocity across the BZ boundary (The group velocity vector’s direction on a equi-energy contour is normal to the SPP equi-energy curve. A SPP curve hitting a BZ boundary perpendicularly shows that there is no group velocity across the boundary). The SPP associated with the (0,+-1) diffracted light lines becomes discontinuous across the BZ boundary, as expected.
\begin{figure}
\caption{Bandgaps}
\end{figure}
Using the same approach, we apply the symmetry operations of the oblique lattice to an arbitrary vector field (figure 4d). With no mirror symmetry, the oblique lattice possess only translational and a two-fold rotation symmetry operations (a two fold rotation operation is equivalent in two dimensions to an inversion operation). As shown in figure 4d, there are no special conditions on the vectors lying along the BZ formed using the Weigner-Setiz method, and no condition for the vectors to cancel perfectly. Standing waves do not necessarily occur at the Brillion zone boundary. 

This leads to the observation that the band-gaps observed do not form at the BZ boundaries, but simply where the SPP meets a Bragg scattered counter-propagating SPP. Figure 4b shows the equi-energy surface for an oblique bi-grating obtained experimentally unsing imaging scatterometry. The SPP equi-energy curves clearly do not band gap at the BZ boundaries, passing through the boundary unperturbed. The SPP equi-energy curves do, however, display band gaps in other locations in momentum space. Highlighted in figure 2, a band gap has formed as a result of the interaction between the (-1,-1) and the (-1,0) SPPs. In the special case of an oblique lattice, these intersections are not required to occur at the same momenta as the BZ.
The mid-points of BZ boundaries are of high symmetry in both rectangular and oblique cases, and do allow the total cancellation of our vector field. However, as isolated singularities these effects are not seen in our oblique grating results.


 

\end{document}
