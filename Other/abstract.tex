
% Thesis Abstract -----------------------------------------------------

%\begin{abstractslong}    %uncommenting this line, gives a different abstract heading
\begin{abstracts}        %this creates the heading for the abstract page
This thesis details original experimental investigations in to the interaction of light with the mobile electrons at the surface of metallic diffraction gratings. The gratings used in this work to support the resultant trapped surface waves (\textit{surface plasmon polaritons}), may be divided into two classes: `crossed' bigratings and `zigzag' gratings. 

Crossed bigratings are composed of two diffraction gratings formed of periodic grooves in a metal surface, which are crossed at an angle relative to one another. While crossed bigratings have been studied previously, this work focuses on symmetries which have received comparatively little attention in the literature. The gratings explored in this work possesses two different underlying Bravais lattices: \textit{rectangular} and \textit{oblique}.

Control over the surface plasmon polariton (SPP) dispersion on a rectangular bigrating is demonstrated by the deepening of one of the two constituent gratings. The resulting change in the diffraction efficiency of the surface waves leads to large SPP band-gaps in one direction across the grating, leaving the SPP propagation in the orthogonal direction largely unperturbed. This provides a mechanism to design surfaces that support highly anisotropic propagation of SPPs.

SPPs on the oblique grating are found to mediate polarisation conversion of the incident light field. Additionally, the SPP band-gaps that form on such a surface are shown to not necessarily occur at the Brillouin Zone boundaries of this lattice, as the BZ boundary for an oblique lattice is not a continuous contour of high-symmetry points.

The second class of diffraction grating investigated in this thesis is the new \textit{zigzag} grating geometry. This grating is formed of sub-wavelength (non-diffracting) grooves that are `zigzagged' along their length to provide a diffractive periodicity for visible frequency radiation. The excitation and propagation of SPPs on such gratings is investigated and found to be highly polarisation selective. 

The first type of zigzag grating investigated possesses a single mirror plane. SPP excitation to found to be dependant on which diffracted order of SPP is under polarised illumination. The formation of SPP band-gaps is also investigated, finding that the band-gap at the first Brillouin Zone boundary is forbidden by the grating's symmetry.

The final grating considered is a zigzag grating which possesses no mirror symmetry. Using this grating, it is demonstrated that any polarisation of incident light may resonantly drive the same SPP modes. SPP propagation on this grating is found to be forbidden in all directions for a range of frequencies, forming a full SPP band-gap.

\end{abstracts}
%\end{abstractlong}


% ----------------------------------------------------------------------


%%% Local Variables: 
%%% mode: latex
%%% TeX-master: "../thesis"
%%% End: 
